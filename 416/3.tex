\documentclass{article}
\usepackage{../m}

\begin{document}
\noindent Paul Gustafson\\
\noindent Texas A\&M University - Math 416\\
\noindent Instructor: Dr. Papanikolas

\subsection*{HW 3, due 2/14}
%Problems, Sec. 30, pp. 280-282: 16, 17, 19, 21*, 23, 24*, 25*, 27H
%Problems, Sec. 31, pp. 291-293: 2, 3, 5, 6*, 10*, 13*, 22, 23*, 27*, 30*, 34H

\p{30.21} Prove that if $V$ is a finite-dimensional vector space over a field $F$, then a subset $\{\beta_1, \ldots, \beta_n\}$ of $V$ is a basis for $V$ over $F$ if and only if every vector in $V$ can be expressed uniquely as a linear combination of the $\beta_i$.
\begin{proof}
Suppose $B : = (\beta_i)$ is a basis for $V$. Let $v \in V$. Since $B$ spans $V$, there exist $(a_i)$ such that $v = \sum_i a_i \beta_i$. To see that the $(a_i)$ are unique, suppose $(b_i)$ also satisfy $v =\sum_i b_i \beta_i$. Subtracting,
$0 = \sum_i (a_i - b_i) \beta_i$, which implies $a_i = b_i \forall i$ by the linear independence of $B$.

Conversely, suppose every vector in $V$ can be expressed uniquely as a linear combination of $B: = (\beta_i)$. Then $B$ trivially spans. Also, if $0 = \sum_i a_i \beta_i$ for some $a_i$, then $a_i = 0$ for all $i$ by the uniqueness.
\end{proof}
\p{30.24} Let $V$ and $V^\prime$ be vector spaces over the same field $F$. 
\begin{enumerate}[a.]
\item If $\{\beta_i : i\in I\}$ is a basis for $V$ over $F$, show that a linear transformation $\phi:V\to V^\prime$ is completely determined by the vectors $\phi(\beta_i) \in V^\prime$.
\begin{proof}
Let $v \in V$. Then $v = \sum_i v_i \beta_i$, so $\phi(v) = \sum_i v_i \phi(\beta_i)$.
\end{proof}
\item Let $\{\beta_i : i \in I\}$ be a basis for $V$, and let $\{\beta_i^\prime : i \in I\}$ be any set of vectors, not necessarily distinct, of $V^\prime$. Show that there exists exactly one linear transformation $\phi:V\to V^\prime$ such that $\phi(\beta_i) = \beta_i^\prime$.
\begin{proof}
Let $v \in V$. Then $v = \sum_i v_i \beta_i$ for unique $v_i$. Define $\phi(v):= \sum_i v_i \beta_i^\prime$. $\phi$ is obviously linear. The uniqueness follows from (a).
\end{proof}
\end{enumerate}
\p{30.25} Let $\phi: V \to V^\prime$ be a linear transformation.
\begin{enumerate}[a.]
\item Linear transformation is to vector space as what is to groups/rings?
\\\emph{Answer:} Homomorphism. 
\item Define the $kernel$ of $\phi$, and show that it is a subspace of $V$.
\begin{proof}
$\ker(\phi):= \phi^{-1}(0)$. Suppose $v,w \in \ker(\phi)$, then $\phi(\alpha v + \beta w) = 0$ by linearity. 
\end{proof}
\item Describe when $\phi$ is an isomorphism of $V$ with $V^\prime$.
\\\emph{Answer:} $\phi$ must be bijective linear transformation. That is, $\ker(\phi) = \{0\}$ and $\phi(V) = V^\prime$.
\end{enumerate}
\p{30.27} Let $\phi: V\to V^\prime$ be $F$-linear with $V$ finite dimensional.
\begin{enumerate}[a.]
\item Show that $\phi(V)$ is a subspace.
\begin{proof}
Let $v,w \in \phi(V)$. Note that $\{\alpha v + \beta w\} = \phi(\alpha \phi^{-1}(v) + \beta \phi^{-1}(w))$.
\end{proof}
\item Show that $\dim(\phi(V)) = \dim(V) - \dim(\ker(\phi))$.
\begin{proof}
Let $A := (\alpha_i)$ be a basis for $\ker(\phi)$. Extend it to a basis for $V$ by adding the vectors in $B := (\beta_i)$. It is easy to check that $(\phi(\beta_i))_i$ forms a basis for $\phi(V)$. Indeed, by a previous problem on this homework, $\phi(B) = \phi(A \cup B)$ spans $\phi(V)$. Linear independence follows from the linearity of $\phi$ and linearly independence of $B$.
\end{proof}
\end{enumerate}
\p{31.6} Find the degree and a basis for $\Q(\sqrt{2} + \sqrt{3}) / \Q$.
\begin{proof}
I claim $f(x) := x^4 -10x^2 +1 = irr(\sqrt 2 + \sqrt 3, \Q)$. Note that $\sqrt 2\pm \sqrt 3$ and $-\sqrt 2 \pm \sqrt 3$ are the roots of $f$ over $\C$. It is easy to check that every product involving a proper subset of the linear factors of $f$ has an irrational coefficient. For example, to see $\sqrt{2} + \sqrt 3$ is irrational, suppose $\sqrt 2 + \sqrt 3 = r$ for $r\in \Q$. Square both sides to reduce to the case that $\sqrt 6$ is irrational.

Hence, $\Q(\sqrt{2} + \sqrt{3}) / \Q$ is of degree 4, and a basis is $\{1, (\sqrt{2} + \sqrt{3}), (\sqrt{2} + \sqrt{3})^2, (\sqrt{2} + \sqrt{3})^3\}$.
\end{proof}
\p{31.10} Find the degree and a basis for $\Q(\sqrt{2}, \sqrt{6}) / \Q(\sqrt{3})$.
\begin{proof}
The degree is $2$, a basis is $\{1, \sqrt 2\}$.  This follows from the fact that $\sqrt 2 = a + b\sqrt 3$ has no solutions over $\Q$ (square both sides, etc.).
\end{proof}
\p{31.13} Find the degree and a basis for $\Q(\sqrt{2}, \sqrt{6} + \sqrt{10}) / \Q(\sqrt{3} + \sqrt{5})$.
\begin{proof}
The degree is $2$, a basis is $\{1, \sqrt 2\}$. The proof that $\sqrt 2$ is irreducible over $\Q(\sqrt{3} + \sqrt{5})$ is straightforward, but tedious case work.
\end{proof}
\p{31.23} Show that if $E$ is a finite extension of a field $F$ and $[E : F]$ is a prime number, then $E$ is a simple extension of $F$ and $E = F(\alpha)$ for every $\alpha \in E\setminus F$.
\begin{proof}
Let $\alpha \in E\setminus F$. Suppose $F(\alpha) \neq E$.  But then we are in trouble since $[E : F] = [E : F(\alpha)] [F(\alpha) : F]$ which contradicts the assumption that $[E:F]$ is prime.

\end{proof}
\p{31.27} Prove in detail that $\Q(\sqrt{3} + \sqrt{7}) = \Q(\sqrt{3}, \sqrt{7})$.
\begin{proof}
It is obvious that $\Q(\sqrt{3} + \sqrt{7}) \subset \Q(\sqrt{3}, \sqrt{7})$.
%%%%FIXME: dimension counting
For the opposite inclusion, let $f := irr(\sqrt 3 + \sqrt 7 , \Q)$.  It is easy to check that the roots of $f$ over $\C$ are $\sqrt 3 \pm \sqrt 7$ and $-\sqrt 3 \pm \sqrt 7$, and that every product of proper subsets of the linear factors of $f$ has an irrational coefficient. Hence, $((\sqrt 3 + \sqrt 7)^i)_{i=0}^3$ forms a basis for $\Q(\sqrt 3 + \sqrt 7)$. Note that $(\sqrt 3 + \sqrt 7)^3 = 14 \sqrt 3 + 16 \sqrt 7$. Thus, $\sqrt 3$ and $\sqrt 7$ are in the span of $(\sqrt 3 + \sqrt 7)^3$ and $\sqrt 3 + \sqrt 7$.
\end{proof}
\p{31.30} Let $E$ be an extension field of $F$. Let $\alpha \in E$ be algebraic of odd degree over $F$. Show that $\alpha^2$ is algebraic of odd degree over $F$, and $F(\alpha) = F(\alpha^2)$.
\begin{proof}
We have $[F(\alpha) : F] = [F(\alpha):F(\alpha^2)] [F(\alpha^2): F]$. Note that if the first factor is $1$, then we are done.  If the second factor is $1$,  then $[F(\alpha) : F] \leq 2$ which implies $F(\alpha) = F(\alpha^2) = F$ since $[F(\alpha) : F]$ is odd.

The remaining case is that both factors are greater than 1, hence greater than 2 since their product is odd. Let $m:= [F(\alpha^2): F]$. There exists a $F$-linear dependence involving ${1, \alpha^2, \ldots, \alpha^{2m}}$. But then $[F(\alpha) : F] \leq 2m$, a contradiction.

\end{proof}
\p{30.34} Show that if $E$ is an algebraic extension of a field $F$ and contains all zeros in $\bar{F}$ of every $f(x) \in F[x]$, then $E$ is an algebraically closed field.
\begin{proof}
Let $g(x) \in E[x]$ with $g(x) = \sum_{i=1}^n a_i x^i$ with $a_n \ne 0$. Let $K = F(a_1, \ldots, a_n)$.  Since each $a_i$ is algebraic over $F$, $K/F$ is a finite extension. Since $g$ lies in $K[x]$, any root $\alpha$ of $g$ must lie in a finite extension of $K$. By the product of degrees in towers theorem, then, $\alpha$ lies in a finite extension of $F$. In particular, there must be a finite linear dependence relation among the powers of $\alpha$. That is, $\alpha$ is a root of a polynomial over $F$.
\end{proof}
\end{document}