\documentclass{article}
\usepackage{../m}

\begin{document}
\noindent Paul Gustafson\\
\noindent Texas A\&M University - Math 416\\
\noindent Instructor: Dr. Papanikolas

\subsection*{HW 5, due 3/21}

%Problems, Sec. 16, pp. 159-161: 1, 2, 3*, 12*, 13*
%Problems, Sec. 17, pp. 164-165: 1, 2*, 3, 4*, 5*, 7H, 9H

\p{16.3} Orbits: $\{1,2,3,4\}, \{s_1, s_2, s_3, s_4\}, \{m_1, m_2\}, \{d_1, d_2\},
\{C\}, \{P_1, P_2, P_3, P_4\}$.
\p{16.12}  Let $g,h \in G_Y$. If $y \in Y$, then $ghy = gy = y$ and $g^{-1}y = g^{-1}(gy) = y$.
\p{16.13} \textbf{a.} If $x \in \R^2$, then the action of $0$ on $x$ leaves $x$ fixed by definition.  For the associative property, write $x = (r,\theta)$ in polar form.  Then $\alpha(\beta * x) = (r, \theta + \beta + \alpha) 
= (\alpha +\beta) * x$.

\textbf{b.} The orbit containing $P$ is the circle centered at the origin passing through $P$.

\textbf{c.} $G_P = 2\pi(\Z,+)$, since $\theta$ fixes $P$ iff $\theta = 2\pi n$ for some $n \in \Z$.

\p{17.2} $G$ is the direct product of $\langle (1 3) \rangle$ and $\langle (2 4 7) \rangle$ since these cycles are disjoint.  Hence, $|G| = (2)(3) = 6$.
By Corollary 17.2, the number of orbits is $(1/6)(2*3 + 3*2 + 3*6) = 5$. An easier way would be to just count the orbits directly.

\p{17.4} As mentioned in the section, the group of rotations of the cube has 24 elements. Indeed, if we fix an orientation of the cube with a distinguished face and neighboring face, each rotation can be described by the 6 choices for the first face and then 4 choices for the neighboring face.

Hence, there are ${8 \choose 6} 6!/24 = 840$ distinguishable colorings.

\p{5} We can refine the hint, by noting that there are 6 quarter-turns that leave opposite faces fixed and 3 half-turns. The 8 vertex rotations are third-turns, and the 6 edge rotations are half-turns. Thus, the Polya enumeration theorem implies that the number of distinguishable colorings is
$$ \frac 1 {|G|} \sum_{g \in G} 8^\lambda(g) = (1/24) ((8)^6 + 6(8)^3 + 3(8)^5 + 8(8)^2 + 6(8)^3) = 15296.$$

\end{document}
