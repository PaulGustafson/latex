\documentclass{article}

\usepackage{../m}

\begin{document}

\noindent Paul Gustafson\\
\noindent Texas A\&M University - Math 416\\
\noindent Instructor: Dr. Papanikolas

\subsection*{HW 2}
%Problems, Sec. 27, pp. 252-254: 18*, 19*, 30*, 31, 33H
%Problems, Sec. 29, pp. 272-274: 6, 8, 18, 30*, 31*, 34*

\p{27.18} Is $\Q[x]/(x^2 - 5x + 6)$ a field? Why?
\begin{proof}
No, $x-2$ and $x-3$ are zero divisors.
\end{proof}
\p{27.19} Is $K := \Q[x]/(x^2 - 6x + 6)$ a field? Why?
\begin{proof}
Yes, $x^2 -6x + 6$ is irreducible over $\Q$ by Eisenstein's criterion ($p=2$). Hence, $\langle p(x)\rangle$ is maximal by Theorem 27.25. Hence, $K$ is a field.
\end{proof}
\p{27.30} Prove that if $F$ is a field, every proper nontrivial prime ideal of $F[x]$ is maximal.
\begin{proof} 
Let $N \subset F[x]$ be a proper nontrivial prime ideal of $F[x]$. By Theorem 27.24, $N$ is principal; let $p(x)$ be a generator of $N$. 

To see that $p(x)$ is irreducible, suppose $ p(x) = q(x) r(x)$ for $q,r$ of degree less than $p$.  Then since $N = \langle p(x) \rangle$ is a prime ideal, either $q$ or $r$ is a multiple of $p$. This contradicts the assumption that both $q$ and $r$ have degree less than $p$. Hence, $p(x)$ is irreducible. Hence, since $N \ne 0$, Theorem 27.25 implies $N$ is maximal.
\end{proof}
\p{27.33} Use the fact that, for any field $F$, $F[x]$ is a PID to show TFAE:
\begin{enumerate}
\item Every nonconstant polynomial in $\C[x]$ has a zero in $\C$.
\item Let $f_1(x), \ldots, f_r(x) \in \C[x]$ and suppose that every $\alpha \in \C$ that is a zero of all $r$ of these polynomials is also a zero of a polynomial $g(x)$ in $\C[x]$. Then some power of $g(x)$ is in the smallest ideal of $\C[x]$ that contains the $r$ polynomials $f_1(x), \ldots, f_r(x)$.
\end{enumerate}
\begin{proof}
Suppose 1 holds. Let $N \subset \C[x]$ be the ideal generated by $\{f_i\}_{i=1}^r$. Let $p(x)$ generate $N$. If $p$ is constant, then $N = \C[x]$, so the conclusion holds. Otherwise, by 1, $p$ has a root $\beta \in \C$. Since $p$ generates $N$, $\beta$ is a root of every element of $N$. In particular, $\beta$ is a root of every $f_i$. Conversely, suppose $\alpha$ is a root of $f_i$ for all $i$. Then since the $f_i$ generate $N$, $\alpha$ must be a root of $p$. Thus, the roots of $p$ are precisely the simultaneous roots of the $f_i$.

By repeated application of 1 and the division algorithm, $p(x) = \prod_{i=1}^n(x-\alpha_i)^{e_i}$ where the $\alpha_i$ are distince. Let $e = \max_i e_i$. By repeated application of the division algorithm, $\prod_{i=1}^n (x - \alpha_i) \mid g(x)$. Hence, $p(x) \mid \prod_{i=1}^n (x - \alpha_i)^e \mid (g(x))^n$.

For the converse, suppose 2 holds. Suppose $f \in \C[x]$ has no zeros in $\C$. Then, by 2, if $g(x) = 1$, some power of $g$ is in $N := \langle f \rangle$, i.e. $1 \in N$. But then $f(x) \mid 1$, so $\deg(f) = 0$.
\end{proof}
\p{29.30} Let $E$ be an extension field of a finite field $F$, where $F$ has $q$ elements. Let $\alpha \in E$ be algebraic over $F$ of degree $n$. Prove that $F(\alpha)$ has $q^n$ elements.
\begin{proof}
This follows immediately from Theorem 29.18, which says that $F(\alpha)$ is an $n$ dimensional vector space over $F$. Each coordinate has $q$ choices, so there are $q^n$ total elements in $F(\alpha)$.
\end{proof}
\p{29.31}
\begin{enumerate}
\item Show that there exists an irreducible polynomial of degree $3$ in $\Z_3[x]$.
\item Show from part (1) that there exists a finite field of 27 elements. (Hint: use 30)
\end{enumerate}
\begin{proof}
Let $p(x) = x^3 - x + 1$. Since $p$ has no roots in $\Z_3$, it cannot have linear factors, so must be irreducible. Part 2 follows directly from 30.
\end{proof}
\p{29.34} Show that $S := \{a + b(\sqrt[3] 2) + c(\sqrt[3] 2)^2 | a,b,c \in \Q\}$ is a subfield of $\R$ by using Theorem 29.18.
\begin{proof}
By Eisenstein's criterion with $p=2$, $x^3 - 2$ is irreducible over $\Q$. Hence, since $\sqrt[3] 2$ is of degree 3 over $\Q$, Theorem 29.18 states that the elements of $Q[\sqrt[3] 2]$ are precisely the elements of $S$.
\end{proof}

\end{document}
