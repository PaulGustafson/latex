\documentclass{article}
\usepackage{../m}

\begin{document}
\noindent Paul Gustafson\\
\noindent Texas A\&M University - Math 416\\
\noindent Instructor: Dr. Papanikolas

\subsection*{HW 6, due April 4}

%Problems, Sec. 35, pp. 319-321: 2, 3, 8, 9, 18*, 19*
%Problems, Sec. 36, pp. 326-327: 5*, 13*, 15H, 17*, 18*, 19H
%Problems, Sec. 37, pp. 332-333: 4*, 5*, 6*, 8H

%check this
\p{35.18} Consider the subnormal series
$0 \to A_3 \times 0 \to S_3 \times 0 \to S_3 \times A_3 \to S_3 \times S_3$.
All the factor groups have prime order so are simple, abelian. Thus, $S_3 \times S_3$ is solvable.

\p{19} Yes, let $\sigma$ be a 90-degree rotation and $\tau$ a reflection. Note that $\langle \sigma \rangle$  is cyclic of order 4 and normal in $D_4$.  Hence, we have the subnormal series $0 \to C_2 \to C_4 \to D_4$, which is a composition series since the orders of all the factor groups are prime (2, actually).

\p{36.5} Each Sylow 3-subgroup of $S_4$ are generated by one of the following 3-cycles: $(1, 2, 3), (1, 2, 4), (1, 3, 4), (2, 3, 4)$. The fact that they are conjugate is a consequence of the Sylow theorems, but you could just conjugate by transpositions if you want to be explicit.  For example, $(3, 4) (1, 2, 3) (3, 4) = (1, 2, 4)$, so the corresponding 3-Sylow subgroups are conjugate.

\p{13} The only divisor of 45 that is congruent to 1 mod 3 is 1. Thus, the 3-Sylow subgroup (of order 9) is normal in the whole group.

\p{15} $P$ is obviously a p-Sylow subgroup of $N[N[P]]$. Suppose $Q$ is a p-Sylow subgroup of $N[N[P]]$.  Then $Q = gPg^{-1}$ for some $g \in N[N[P]]$. Since $N[P]$ is normal in $N[N[P]]$, this implies $Q \subset N[P]$. Hence, $Q$ and $P$ are p-Sylow subgroups of $N[P]$, so $Q = P$ since $P$ is normal in $N[P]$.  Thus, $P$ is the unique p-Sylow subgroup of $N[N[P]]$, so is normal in $N[N[P]]$.

\p{18} Note that 3, 5, and 15 are not congruent to 1 mod 17.  Hence, the only divisor of 255 that is congruent to 1 mod 17 is 1.  Thus, the 17-Sylow subgroup is normal in the whole group.

\p{19} Presumably $m \neq 1$  or else we have the counterexample $C_p$.  Since $n_p  \equiv 1 \pmod p$, $n_p \mid m$. This implies $n_p = 1$ since $1 < m < p$. Thus, the $p$-Sylow subgroup is normal in the whole group.

\p{37.4} Call the group $G$. By the Sylow theorems, $n_5 = 1$, $n_7 = 1$, and $n_{47} = 1$. Hence, the corresponding Sylow subgroups are normal in $G$.  Since they have prime order, they are cyclic and have trivial intersection.   Hence, using the trick from class (proved below), each pair of Sylow subgroups commutes pointwise. 

Trick from class: If $H, K \lhd G$ with $H \cap K = \{e\}$ and $h \in H$, $k \in K$; then $hk = kh$.  Proof of trick: $hkh^{-1}k^{-1} = k' k^{-1} \in K$ and $hkh^{-1}k^{-1} = h h' \in H$, so $hkh^{-1}k^{-1} = e$.

Let $x,y,z \in G$ have orders 5,7, and 47, respectively.  Since $x,y$ commute, $xy$ has order 35 ($x^k$ and $y^{k}$ only have the same order for $35 \mid k$).  Similarly, $xyz$ has order $(5)(7)(47)$.


\p{5} Call the group $G$. $96 = (32)(3)$, so the possibilities are $n_2 = 1$ or $n_2 = 3$. WLOG $n_2 = 3$ since $G$ is not simple if $n_2 = 1$.  But $(n_2)! = 6  < 96 = |G|$. Hence, by a theorem proved in class, $G$ is not simple (consider the transitive action of $G$ on the set of 3-Sylow subgroups by conjugation).

\p{6} $160 = (32)(5)$, so $n_2 = 1$ or $n_2 = 5$.  WLOG, $n_2 = 5$.  But $5! = 120 < 160$, so $G$ is not simple.

\p{8} 
\vspace{-1em}
\begin{enumerate}[a.] 
\item Note that $\tau\sigma \tau^{-1} (\tau a_i) = \tau \sigma a_i = \tau a_{i'}$ where $i' = i + 1 \pmod m$.  If $x \not\in (\tau a_i)_i$ for any $i$, then $\tau\sigma \tau^{-1}(x) = \tau\tau^{-1}(x) = x$ since $\tau^{-1} x \not\in (a_i)_i$.
\begin{comment}
Since every permutation is a product of transpositions, WLOG $\tau$ is a transposition. By looking at $S_{n+1}$, WLOG $n > m$ (if the identity is true for $m<n$ in $S_{n+1}$, then it must be true for $m \le n$ in $S_n$). If $\tau$ and $\sigma$ have disjoint support, we're done.  If $\tau$ transposes two of the $a_i$, note that, since $m<n$, $\tau$ can be decomposed into three transpositions each of which transposes an $a_i$ with a non-$a_i$ (consider the equality $(1, 2) = (2, 3) (1, 3) (2, 3)$ and generalize).  Thus, WLOG $\tau$ transposes an $a_i$ with a non-$a_i$.  By cyclically rearranging the $a_i$, WLOG $\tau$ transposes $a_1$ with $\tau(a_1) \not\in (a_i)_i$. To finish, consider $(1,4) (1,2,3) (1,4) = (4, 2, 3)$ and generalize.
\end{comment}

\item It suffices to show that  $(1, 2, \ldots, m)$ is conjugate to each $(a_1, a_2, \ldots, a_m)$.  By part(a), this is obvious: just define $\tau$ by $i \mapsto a_i$ for $1 \le i \le m$ and extend this to a bijection of $[n]$ however you like.

\item Let $\sigma = \prod_i \sigma_i$ and $\eta = \prod_i \eta_i$ denote two such products of disjoint cycles with each $\sigma_i = (\sigma_{i1}, \ldots, \sigma_{i,r_i})$ and $eta_i = (\eta_{i1}, \ldots, \eta_{i, r_i})$. Since the $sigma_{ij}$ are distinct and the $\eta_{ij}$ are distinct, there exists $\tau \in S_n$ such that $\tau(\sigma_{ij}) = \eta_{ij}$ for all $i,j$.

By the fact that conjugation by $\tau$ is an homomorphism and by part (a),
$\tau \sigma \tau^{-1} = \prod_i \tau\sigma_i \tau_{-1} = \prod_i \eta_i = \eta$.

\item Let $\sigma \in S_n$.  Let $(x_i)_{1\le i \le s$} be a complete set of representatives of the orbits of $[n]$ under  $\sigma$. Let $r_i$ be the size of the orbit containing $x_i$. Then
$\sigma = \prod_{i=1}^s (x_i, \sigma(x_i), \ldots, \sigma^{r_i} x_i)$.
Moreover, by reindexing we can assume that $0 < r_1 \le r_2 \le \ldots \le r_s$.

Suppose $\sigma'$ has orbit sizes $0 < r_1' \le r_2' \le \ldots \le r_t'$.
By the decomposition above and part (b), $\sigma$ is conjugate to $\sigma'$ if $t = s$ and $r_i = r_i'$ for all $i$.

\end{enumerate}

\end{document}
