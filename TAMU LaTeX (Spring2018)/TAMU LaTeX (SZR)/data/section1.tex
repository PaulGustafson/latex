%%%%%%%%%%%%%%%%%%%%%%%%%%%%%%%%%%%%%%%%%%%%%%%%%%%
%
%  New template code for TAMU Theses and Dissertations starting Fall 2016.  
%
%
%  Author: Sean Zachary Roberson
%  Version 3.17.09
%  Last Updated: 9/21/2017
%
%%%%%%%%%%%%%%%%%%%%%%%%%%%%%%%%%%%%%%%%%%%%%%%%%%%

%%%%%%%%%%%%%%%%%%%%%%%%%%%%%%%%%%%%%%%%%%%%%%%%%%%%%%%%%%%%%%%%%%%%%%
%%                           SECTION I
%%%%%%%%%%%%%%%%%%%%%%%%%%%%%%%%%%%%%%%%%%%%%%%%%%%%%%%%%%%%%%%%%%%%%


\pagestyle{plain} % No headers, just page numbers
\pagenumbering{arabic} % Arabic numerals
\setcounter{page}{1}


\chapter{\uppercase {Introduction}}
%% What is the problem? 

%% Why is it hard? 
%% What have others done? 
%% What have I done?

\section{Topological quantum computation}

The term ``topological quantum computation'' refers to a variety of proposals for building a quantum computer using topological phases of matter.  In the usual setup, one creates $n$ quasiparticle excitations (anyons) in a 2-dimensional disk.  The motion group of $n$ points on a disk is the braid group $B_n$.  Physically braiding quasiparticle excitations (FIXME: INSERT PICTURE) corresponds to an unitary action of the braid group on the Hilbert space of possible states of the system.  These unitary braid group representations are completely determined by the anyon types of the $n$ quasiparticles.   

\subsection{Mapping class groups}

More generally, one can consider a system of quasiparticle excitations on a closed surface of arbitrary genu.  In this case, there may be nontrivial self-homeomorphisms of the underlying surface in addition to motions of the quasiparticle excitions on the surface.  Both types of actions correspond to elements of the mapping class group of a compact surface with boundary, where a labeled boundary components replaces each quasiparticle excitation. The previously mentioned braid group example corresponds to the mapping class group of a disk with $n$ open disks removed relative to the outer, vacuum-labeled boundary. 

\subsection{Spherical fusion categories}

The algebraic aspects of a two dimensional topological phase of matter are governed by the theory of tensor categories.    


\subsection{Topological quantum field theories}


\section{The Property F conjecture}
\subsection{Universality}
\subsection{Weak integrality}
\subsection{Current progress}

\section{Twisted Dijkgraaf-Witten Theory}
\subsection{The spherical fusion category $\Vect^\omega_G$}
\subsection{Colored graphs}
\subsection{Birman's mapping class group generators}




