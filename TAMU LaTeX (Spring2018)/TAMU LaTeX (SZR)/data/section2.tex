%%%%%%%%%%%%%%%%%%%%%%%%%%%%%%%%%%%%%%%%%%%%%%%%%%%
%
%  New template code for TAMU Theses and Dissertations starting Fall 2016.  
%
%
%  Author: Sean Zachary Roberson
%  Version 3.17.09
%  Last Updated: 9/21/2017
%
%%%%%%%%%%%%%%%%%%%%%%%%%%%%%%%%%%%%%%%%%%%%%%%%%%%

%%%%%%%%%%%%%%%%%%%%%%%%%%%%%%%%%%%%%%%%%%%%%%%%%%%%%%%%%%%%%%%%%%%%%%%
%%%                           SECTION II
%%%%%%%%%%%%%%%%%%%%%%%%%%%%%%%%%%%%%%%%%%%%%%%%%%%%%%%%%%%%%%%%%%%%%%


\chapter{\uppercase{Related Work*}}

The closest related work is a result of Etingof, Rowell, and Witherspoon who showed purely algebraically that the braid group representations associated to the modular category $\Mod(D^\omega(G))$ have finite images \cite{erw}.   The braid group $B_n$ is the mapping class group of a disk with $n$ marked points relative to its boundary, so they asked whether their result generalizes to arbitrary mapping class group representations associated to $\Mod(D^\omega(G))$. This paper answers their question affirmatively, using a different, more geometric approach. \let\thefootnote\relax\footnote{* Reprinted with permission from ``Finiteness for mapping class group representations from twisted dijkgraaf--witten theory,'' P.\ P.\ Gustafson, \emph{Journal of Knot Theory and Its Ramifications}, vol. 27, no. 06, p. 1850043, Copyright 2018 by World Scientific Publishing Company.}



Prior to the current work, certain specific cases had already been solved. In the case of the torus, Ng and Schauenburg's Congruence Subgroup Theorem implies the much stronger result that any Reshitikhin-Turaev representation of the mapping class group of the torus has finite image \cite{0806.2493}.   Another related result is due to Fjelstad and Fuchs \cite{fjfu}.  They showed that, given a surface with at most one boundary component, the mapping class group representations corresponding to the untwisted (i.e. $\omega = 1$) Dijkgraaf-Witten theory have finite image.  Their paper uses an algebraic method of Lyubashenko \cite{Lyubashenko1996} that gives a projective mapping class group representation to any factorizable ribbon Hopf algebra, in their case, the double $D(G)$. In our case, we instead consider the mapping class group action on a vector space of $\Vect_G^\omega$-colored embedded graphs defined by Kirillov \cite{kirillovStringNets}, yielding a simpler, geometric proof of the more general twisted case.

Bantay also calculated the images of certain representations of mapping class groups on the Hilbert space of an orbifold model associated to $D^\omega(G)$ \cite{bantay}.  These representations appear to coincide with the twisted Dijkgraaf-Witten representations. However, due to lack of proof, the precise connection is unclear. 


