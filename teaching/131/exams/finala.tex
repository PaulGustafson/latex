\documentclass[11pt]{article}

%opening
\usepackage{caption}
\usepackage{mcexam}
\usepackage{amsmath}
\usepackage{amsfonts}
\usepackage{graphicx}
\usepackage{enumerate}
\usepackage{tikz}
\usepackage{float}
\usepgflibrary{arrows}
\usepackage{todo}
\everymath{\displaystyle}
\pagestyle{empty}
\usepackage{lastpage} % this calculates the page number of the last page
\usepackage{fancyhdr}
\pagestyle{fancy}
\lhead{\textsf{Spring 2017}}
\chead{\textsf{Math 131 -- Final Exam A}}
\rhead{\textsf{Page \thepage\ of \pageref{LastPage}}}
\cfoot{}
\rfoot{\textsf{\thepage}}

\newcommand{\series}[3]{\displaystyle \sum_{{#1}={#2}}^{#3} }
\newcommand{\limit}[2]{\displaystyle \lim_{{#1} \rightarrow {#2}} }
\newcommand{\din}[2]{\displaystyle \int_{#1}^{#2}}
\newcommand{\Int}{\displaystyle \int}
\newcommand{\Q}{\ensuremath \mathbb{Q}}
\newcommand{\R}{\ensuremath \mathbb{R}}
\newcommand{\C}{\ensuremath \mathbb{C}}
\newcommand{\Z}{\ensuremath \mathbb{Z}}
\newcommand{\N}{\ensuremath \mathbb{N}}
\newcommand{\isom}{\ensuremath \cong}
\newcommand{\inv}{\ensuremath ^{-1}}
\newcommand{\ot}{\ensuremath \otimes}
\newcommand{\op}{\ensuremath \oplus}

\Course{Math 131}{Principles of Calculus}
\Instructor{Paul Gustafson}
\TestName{Final Exam A\hfill{\bfseries\Huge RED}}
\Date{}
%\Section{}


\begin{document}
\Head
\begin{instructions}
\item For questions which require a written answer, show all your work.  Full credit will be given only if the necessary work is shown justifying your answer.
\item Simplify your answers.
\item Calculators are allowed.
\item Should you have need for more space than is allocated to answer a question, use the back of the exam.
\item Please do not talk about the test with other students until exams are handed back.
\item \textbf{Honor Code:}

\vspace{0.1in}
An Aggie does not lie, cheat, or steal or tolerate those who do.
\vspace{0.3in}

\par\noindent\makebox[2.5in]{\hrulefill} 
\par\noindent\makebox[2.5in][l]{Signature}     
\end{instructions}
%\PointTable{2}
%\hrule width \linewidth height 2pt\vspace{2pt}%
%\hrule width \linewidth height 1pt\vspace{2pt}%
%\hrule width \linewidth height 1pt%
%\vspace{4mm}%
%\noindent {\bf For Instructor use only.}\\ \vspace{-.2in}%
%\begin{center}
%{\Large
%\begin{tabular}{|p{0.75in}|p{0.4in}|p{0.4in}|p{0.4in}|p{0.4in}|p{0.4in}|p{0.4in}|p{0.4in}|p{0.4in}|p{0.4in}|}
%\hline
%Question&MC&11&12&13&14&15&16&17&Total\\\hline
%Points&30&10&10&10&10&10&10&10&100\\\hline
%Earned&&&&&&&&&\\\hline
%\end{tabular}
%}
%\end{center}
\newpage

\vspace{.2in}

\noindent \emph{{\bf Multiple Choice (5 points each)} Mark the correct
answer on the bubble sheet.}


For questions 1-4, use the following graph of $f(x)$:\\


\begin{minipage}{\linewidth}% to keep image and caption on one page
\centering
\makebox[\linewidth]{}
\includegraphics{finalgraph1.pdf}
\captionof{figure}{$f(x)$}\label{graph1exam1}%      only if needed  
\end{minipage}


\begin{questions}
\begin{multiplechoice}{5}
%8.5,8.6

\question According to the graph of $f(x)$, the $\lim_{x\to 3}f(x) $ equals which of the following.
\begin{answers}{3}
\ans $8$
\ans $2$
\ans $-1$
\ans $-3$
\ans The limit does not exist.
\end{answers}

%\textbf{Solution: d}


\question According to the graph of $f(x)$, the $\lim_{x\to -2^-}f(x) $ equals which of the following.
\begin{answers}{3}
\ans $7$
\ans $0$
\ans $-2$
\ans $-5$
\ans The limit does not exist.
\end{answers}

%\textbf{Solution: a}\\

\question According to the graph of $f(x)$, the $\lim_{x\to 5}f(x) $ equals which of the following.
\begin{answers}{3}
\ans $8$
\ans $4$
\ans $0$
\ans $-1$
\ans The limit does not exist.
\end{answers}

%\textbf{Solution: b}\\




\question According to the graph of $f(x)$, the function $f(x)$ is not continuous at $x=3$ because
\begin{answers}{3}
\ans $f(x)$ is not defined at $x=3$.
\ans there is a removable discontinuity at $x=3$
\ans $\lim_{x\to 3}f(x)$ does not exist.
\ans there is a horizontal asymptote at $x=3$. 
\ans there is a vertical asymptote at $x=3$.
\end{answers}

%\textbf{Solution: e}

%11.1,2

\newpage

\question The graph of $g(x)$ is given below.\\

\begin{minipage}{\linewidth}% to keep image and caption on one page
\centering
\makebox[\linewidth]{}
\includegraphics{exam1graph2.pdf}
\captionof{figure}{$g(x)$}\label{graph2exam1}%      only if needed  
\end{minipage}
According to the graph above, the domain and range of $g(x)$ are
\begin{answers}{2}
\ans Domain: $[-4,4]$, Range: $[-4,2]$
\ans Domain: $[-6,4]$, Range: $[-2,6]$
\ans Domain: $[-6,2]$, Range: $[-2,4]$
\ans Domain: $[-4,4]$, Range: $[-6,2]$
\ans Domain: $[-2,-6]$, Range: $[-2,4]$
\end{answers}
% Answer: 232/99

\question Find the domain of $f(x) = \frac{1}{x^2 - 16}$.
\begin{answers}{2}
\ans $(-2,2) \cup (2, \infty)$
\ans $(-\infty, -4) \cup (-4,4) \cup (4, \infty) $
\ans  $(-\infty, -4) \cup (0, \infty)$
\ans  $(-\infty, -2) \cup (-2,2) \cup (2, \infty)$
\ans $[-2,2)$
\end{answers}


\question Let $f(x) = \sqrt{4-x^2}$ and $g(x)=\ln(x)$.  What is the domain of $f(x)g(x)$?
\begin{answers}{2}
\ans $(0,2]$
\ans $[-1,2)$
\ans $[-2,\infty)$
\ans $(0,\infty)$
\ans $[-2,2]$
\end{answers}

\newpage

%(25 - x^2) - 16 / 3 - x
% 9 - x^2 / 3 -x
% 3 + x
% 6

\question Evaluate $\displaystyle \lim_{x \to 3} \frac{\sqrt{25-x^2} - 4}{3-x}$.
\begin{answers}{3}
\ans $1$
\ans $\frac{\sqrt{23} - 4}{2}$
\ans $-2$
\ans $6$
\ans $9$
\end{answers}



\question Given a function $f(x)$, then the graph of $2f\left(3 - x\right)$ will be
\begin{answers}{2}
\ans the graph of $f(x)$ shrunk horizontally by a factor of 2, shifted 4 units up, then reflected across the $x$-axis.
\ans the graph of $f(x)$ stretched vertically by a factor 3, shifted 2 units up, then reflected across the $y$-axis.
\ans the graph of $f(x)$ stretched vertically by a factor of 2, shifted 3 units to the right, then reflected across the $y$-axis.
\ans the graph of $f(x)$ stretched vertically by a factor of 2, shifted 3 units to the left, then reflected across the $y$-axis.
\ans the graph of $f(x)$ shrunk horizontally by a factor of 3, shifted 4 units to the right, then reflected across the $x$-axis.
\end{answers}



\question A bacteria population doubles every 47 minutes.  If the initial population is 1000 bacteria, how many bacteria will there be after 5 hours?
\begin{answers}{3}
\ans $4.2 \times 10^4$
\ans $3.2 \times 10^3$
\ans $4.3 \times 10^3$
\ans $1.2 \times 10^3$
\ans $8.3 \times 10^3$
\end{answers}


%% EXAM 2

\question Find the derivative of the function $f(x) = \frac{3}{x^3} - 4x^2 + 3$.
\begin{answers}{2}
\ans $-\frac{15}{x^2} - 8x$
\ans $-\frac{15}{x^2} + 3$
\ans $-\frac{9}{x^4} -8x$
\ans $-\frac{9}{x^4} - 4$
\ans $-\frac{10}{x} - 3$
\end{answers}

\newpage


For the next two questions, use the following graph of $f'(x)$:\\


\begin{minipage}{\linewidth}% to keep image and caption on one page
\centering
\makebox[\linewidth]{}
\includegraphics{finalgraph3.pdf}
\captionof{figure}{$f'(x)$}\label{graph1exam1}%      only if needed  
\end{minipage}


\question According to the graph of $f'(x)$, the original function $f(x)$ has a local maximum at
\begin{answers}{3}
\ans $-5$
\ans $-3$
\ans $0$
\ans $1$
\ans $5$
\end{answers}


\question According to the graph of $f'(x)$, the original function $f(x)$ is concave upward in which interval(s)?
\begin{answers}{2}
\ans $(-\infty, \infty)$
\ans $(-\infty, -3) \cup (1,5)$
\ans $(1, \infty)$
\ans $(-\infty, -3)$
\ans The original function $f(x)$ is never concave up.
\end{answers}



\question A vertical spring is released at time $t = 0$ seconds and begins to oscillate in a straight vertical line.
The height of its endpoint above the ground in meters is given by the function
$$h(t) = 3 - 0.2\cos(2t)$$
To two decimal places, what is the velocity (in meters/second) of the spring's endpoint at time $t = 3$ ?
\begin{answers}{3}
\ans -0.11
\ans -0.06
\ans 0.12
\ans 2.12
\ans 2.84
\end{answers}

\newpage

\question Find the linear approximation to $\sqrt{x^2+8}$ at $x = 1$
\begin{answers}{2}
\ans $3x + \sqrt{8}$
\ans $x + \sqrt{8}$
\ans $\frac{1}{6} x + 3$
\ans $\frac{2}{3}x + 3$
\ans $\frac{1}{3} x + 3$
\end{answers}


\question We are given an unknown function $f(x)$ such that $f'(2) > 0$ and $f''(2) < 0$.
We can conclude that at $x = 2$, the function $f(x)$ has 
\begin{answers}{2}
\ans a local min.
\ans a local max.
\ans an inflection point.
\ans an undefined derivative.
\ans none of the above.
\end{answers}

\question Calculate the equation of the tangent line to $y = \frac{1}{x}$ at $x = 2$
\begin{answers}{2}
\ans $y = -\frac{1}{4}x + \frac{1}{2}$
\ans $y = x + \frac{1}{2}$
\ans $y = -\frac{1}{2}x + 2$
\ans $y = -\frac{1}{2}x + \frac{1}{2}$
\ans $y = -\frac{1}{4}x + 2$
\end{answers}





%% EXAM 3


\question Find the absolute maximum and minimum values for the function 
$f(x) = \ln(x^2 + 1)$ on the interval $[-1, 3]$
\begin{answers}{1}
\ans maximum value $= 2.30$, minimum value $ = 1.1$
\ans maximum value $= 3.62$, minimum value $ = 1.1$
\ans maximum value $= 2.30$, minimum value $ = 0$
\ans maximum value $= 3.62$, minimum value $ = 0$
\ans maximum value $= 1.32$, minimum value $ = 1$
\end{answers}

\newpage

\question Find the derivative of the function $f(x) = \tan(x e^x)$.
\begin{answers}{2}
\ans $(1+x)e^x \sec^2(xe^x) $
\ans $e^x \sec^2(xe^x) $
\ans $(1+x)e^x \tan(xe^x) $
\ans $e^x \cos^2(xe^x) $
\ans $\sec^2(xe^x) $
\end{answers}


\question If $f'(x) = \frac{1}{2\sqrt{x}}$ and $f(9) = 5$
\begin{answers}{2}
\ans $f(x) = \frac{3}{4}x^{-3/2} +  \frac{11}{4}$
\ans $f(x) = \sqrt{x} + \frac{7}{2}$
\ans $f(x) = \frac{1}{2}\sqrt{x} + 3$
\ans $f(x) = \frac{1}{2}\sqrt{x} + \frac{7}{2}$
\ans $f(x) = \sqrt{x} + 2$
\end{answers}


\question A particle moves along a wire with velocity $v(t) = 4\cos(2t)$.  Find the
net change in position between time $t = 0$ and $t = \pi$
\begin{answers}{2}
\ans $1  + \pi$
\ans $2\pi$
\ans $4\pi$
\ans $0$
\ans $\frac{\pi}{2}$
\end{answers}


% 600 = 2s^2 + 4sl
% V = s^2l
% 4sl = 600 -2s^2
% l = 600 -2 s^2 / 4 s
% V =  (600s - 2 s^3)/ 4
% V' = 150 - (3s^2/2)
% s = 10

\question Alex wants to make a box with a square base, closed on all sides.  He has 600 square inches of cardboard.  What is the maximum volume of the box in cubic inches?
\begin{answers}{3}
\ans $598.32$
\ans $643.60$
\ans $1000$
\ans $1284.81$
\ans $1500$
\end{answers}


\question Calculate the indefinite integral 
$\displaystyle \int \frac{1}{x} + \sec(3x) \tan(3x) \, dx$
\begin{answers}{2}
\ans $ \ln |x| + 3 \sec(3x) + C$
\ans $\frac{2}{x^2} + \frac{1}{3}\sec(3x) + C$
\ans $\ln|x|  + \frac{1}{3}\sec(3x) + C$
\ans $-\frac{2}{x^2}  + \frac{1}{3}\tan(3x) + C$
\ans $\ln |x| + \frac{1}{3}\cot(3x) + C$
\end{answers}


\newpage

\question Use the fundamental theorem of calculus to find the derivative of 
$\displaystyle f(x) = \int_1^{x^2} \sin(\cos(t)) \, dt$
\begin{answers}{2}
\ans $\sin(\cos(x^2))$
\ans $x^2 \sin(\cos(x^2))$
\ans $(x^2 - 1) \sin(\cos(x^2))$
\ans $2x \sin(\cos(x^2))$
\ans $(2x - 1) \sin(\cos(x)$
\end{answers}

\question Use the geometric shape of the graph to find the integral 
$\displaystyle \int_{-3}^2 f(x)$ where 
$$ f(x) = 
\begin{cases}
5, & x \le 0 \\
\sqrt{4 - x^2}, & x > 0
\end{cases}
$$
\begin{answers}{2}
\ans $2\pi$
\ans $\frac{15}{2} + \frac{1}{4}\pi$
\ans $15 + \pi$
\ans $15 + 2\pi$
\ans $10  + \frac{\pi}{2}$
\end{answers}

\question The acceleration of a particle is given by $a(t) = 6\sin(t)$.  The position
of the particle at times $t = 0$ is $s(0) = 3$.  The initial velocity of the particle is $v(0) = -7$.
The position function for the particle is
\begin{answers}{2}
\ans $s(t) = -3t^2 + 5t + 3$ 
\ans $s(t) = -6 \sin(t) -t + 3$
\ans $s(t) = -6 \cos(t) -t + 6$
\ans $s(t) = -6 \cos(t) - 13t + 3$
\ans $s(t) = -6 \sin(t) - 13t + 3$
\end{answers}

\question Calculate $\int_1^{e^3} \frac{(\ln(x))^2}{x} \, dx$.
\begin{answers}{2}
\ans $9$ 
\ans $\frac{1}{3} e^{3} - 1$
\ans $2e^{-3}$
\ans $\frac{1}{3}$
\ans $8$
\end{answers}

\question Calculate the area between the curves $y = x$ and $y = x^2$.
\begin{answers}{3}
\ans $\frac 1 3$ 
\ans $\frac 1 6$ 
\ans $\frac 2 3$ 
\ans $1$ 
\ans $- \frac 1 2$ 
\end{answers}

\question What is the average value of the function $f(x) = \sin(x)$ on $[0, \pi]$
\begin{answers}{3}
\ans $2$
\ans $-2\pi$
\ans $-\frac{\pi}{2}$
\ans $\pi$
\ans $\frac{2}{\pi}$
\end{answers}


% ln(y+2/(y-3)) = x - 7
% \frac{y+2}{y-3} = e^{x - 7}
% y +2 = (y - 3)e^{x-7}
% y(1 - e^{x-7}) = -3e^{x-7} - 2
% C
\question Find the inverse function to $f(x) = \ln(x+2) - \ln(x-3) + 7$.
\begin{answers}{2}
\ans $\frac{-2e^{x-7} - 3}{-e^{x-7} - 1}$
\ans $\frac{-3e^{x-7} - 2}{-e^{x-7} - 1}$
\ans $\frac{-3e^{x-7} - 2}{-e^{x-7} + 1}$
\ans $\frac{-2e^{x-1} - 3}{-e^{x-1} + 7}$
\ans $\frac{-2e^{x-3} - 2}{-e^{x-3} - 7}$
\end{answers}

\end{multiplechoice}
\end{questions}
\end{document}

*************************************************************************
*************************************************************************
