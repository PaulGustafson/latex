\documentclass[t]{beamer}

\mode<handout>
{
  \usepackage{pgf}
  \usepackage{pgfpages}

\pgfpagesdeclarelayout{4 on 1 boxed}
{
  \edef\pgfpageoptionheight{\the\paperheight} 
  \edef\pgfpageoptionwidth{\the\paperwidth}
  \edef\pgfpageoptionborder{0pt}
}
{
  \pgfpagesphysicalpageoptions
  {%
    logical pages=4,%
    physical height=\pgfpageoptionheight,%
    physical width=\pgfpageoptionwidth%
  }
  \pgfpageslogicalpageoptions{1}
  {%
    border code=\pgfsetlinewidth{2pt}\pgfstroke,%
    border shrink=\pgfpageoptionborder,%
    resized width=.5\pgfphysicalwidth,%
    resized height=.5\pgfphysicalheight,%
    center=\pgfpoint{.25\pgfphysicalwidth}{.75\pgfphysicalheight}%
  }%
  \pgfpageslogicalpageoptions{2}
  {%
    border code=\pgfsetlinewidth{2pt}\pgfstroke,%
    border shrink=\pgfpageoptionborder,%
    resized width=.5\pgfphysicalwidth,%
    resized height=.5\pgfphysicalheight,%
    center=\pgfpoint{.75\pgfphysicalwidth}{.75\pgfphysicalheight}%
  }%
  \pgfpageslogicalpageoptions{3}
  {%
    border code=\pgfsetlinewidth{2pt}\pgfstroke,%
    border shrink=\pgfpageoptionborder,%
    resized width=.5\pgfphysicalwidth,%
    resized height=.5\pgfphysicalheight,%
    center=\pgfpoint{.25\pgfphysicalwidth}{.25\pgfphysicalheight}%
  }%
  \pgfpageslogicalpageoptions{4}
  {%
    border code=\pgfsetlinewidth{2pt}\pgfstroke,%
    border shrink=\pgfpageoptionborder,%
    resized width=.5\pgfphysicalwidth,%
    resized height=.5\pgfphysicalheight,%
    center=\pgfpoint{.75\pgfphysicalwidth}{.25\pgfphysicalheight}%
  }%
}


  \pgfpagesuselayout{4 on 1 boxed}[a4paper, border shrink=5mm, landscape]
  \nofiles
}

%% Language and font encodings
\usepackage[english]{babel}
\usepackage[utf8x]{inputenc}
\usepackage[T1]{fontenc}

\usetheme{Madrid}
\usecolortheme{beaver}

%% Useful packages
\usepackage{amsmath}
\usepackage{graphicx}

\usepackage{enumitem}

% full page itemieze
\newenvironment{fpi}
  {\itemize[nolistsep,itemsep=\fill]}
  {\vfill\enditemize}

\newcommand{\img}[1]{
\vfill
\includegraphics[width=\textwidth,height=0.5\textheight,keepaspectratio]{#1}
\vfill
} 


\title{Derivatives and Rates of Change (Section 2.6)}

\date{Feb 9, 2017 \\ 9:35 - 10:50 AM}

\begin{document}
\frame{\titlepage}

\begin{frame}{Outline}
\begin{fpi}
\item Infinity in the answer
\item Infinity in the problem
\end{fpi}
\end{frame}

\begin{frame}{Infinity in the answer}
Plug a number into an expression and get infinity out
\end{frame}

\begin{frame}{Infinity in the answer}
When we write 
$$ \lim_{x \to a} f(x) = \infty$$
or 
$$ \lim_{x \to a} f(x) = -\infty,$$
we are saying that the limit \textbf{does not exist}, and 
we are \textbf{describing} the way it does not exist.
\end{frame}

\begin{frame}{Example}
\begin{fpi}
\item $\displaystyle f(x) = \frac{1}{x}$
\item You can a calculator to see the behavior near 0.
\end{fpi}
\end{frame}

\begin{frame}{Example}
$$f(x) = \frac{1}{(x-1)^2}$$
\end{frame}

\begin{frame}{Example}
$$f(x) = \frac{x}{x+5}$$
\end{frame}


\begin{frame}{Example}
$$f(x) = \frac{(x+5)(x-3)}{(x+2)(x-5)}$$
\end{frame}

\begin{frame}{Procedure for determining the sign of an infinite limit}
\begin{fpi}
\item Plug the limit $x$-value into all non-zero factors
\item Use a number line to determine signs of zero factors ($0^+$ and $0^-$)
\item Count the number of negatives
\item Even -> $+\infty$,  odd -> $-\infty$
\end{fpi}
\end{frame}

\begin{frame}{Example}
$$f(x) = \frac{(x+4)(x-8)}{(x+1)(x-6)}$$
\end{frame}

\begin{frame}{Vertical asymptotes}
Whenever there is an infinite discontinuity in a graph at $x=a$, we
say that there is a vertical asymptote at $x=a$.
\end{frame}

\begin{frame}{Log limit}
Note: 
$$ \lim_{x \to 0^+} \ln(x) = -\infty $$
\end{frame}

\begin{frame}{Infinity in the problem}
\begin{fpi}
\item $ \displaystyle \lim_{x \to \infty} \frac{x+1}{x-2}$
\item You can use a calculator to understand the behavior as $x \to \infty$
\end{fpi}
\end{frame}

\begin{frame}{Procedure for limits of fractions as $x \to \pm \infty$}
\begin{fpi}
\item Identify the highest power term in the fraction
\item Multiply the top and bottom with $\frac{1}{x^n}$ for that term
\item Plug in $\infty$, everything with $\infty$ in the denominator goes away
\item What's left is the answer
\end{fpi}
\end{frame}

\begin{frame}{Horizontal asymptotes}
If $$\lim_{x \to \infty} f(x) = c$$ for some number $c$
or $$\lim_{x \to -\infty} f(x) = c$$ for some number $c$,
then we say $f(x)$ has a \textbf{horizontal asymptote} at
$y = c$.
\end{frame}

\begin{frame}{Example}
$$\lim_{x \to -\infty} \frac{x^2 -4x +1}{3x^2 -5}$$
\end{frame}

\begin{frame}{Example}
$$\lim_{x \to \infty} \frac{5x^2 - 4x +1}{2x -5}$$
\end{frame}

\begin{frame}{Example}
$$\lim_{x \to \infty} \frac{x^2 -3x +4}{x^3 +1}$$
\end{frame}


\begin{frame}{Example}
$$\lim_{x \to -\infty} \frac{10x^3 -4x + 1}{4x^3 + 8 x^2 +x -5}$$
\end{frame}

\begin{frame}{Fractions involving exponential functions}
$$\lim_{x \to \infty} e^x = \infty$$
and
$$\lim_{x \to -\infty} e^x = 0.$$

The function $e^x$ grows faster than any polynomial as $x \to \infty$ .
\end{frame}


\begin{frame}{Example}
$$\lim_{x \to \infty} \frac{3e^x -2x + 1}{2e^x +x -5}$$
\end{frame}

\begin{frame}{Example}
$$\lim_{x \to -\infty} \frac{3e^x -2x + 1}{2e^x +x -5}$$
\end{frame}


\begin{frame}{Summary}
We learned two different ways of dealing with limits with infinities,
depending on whether the infinities appear in the question or the answer.
\end{frame}

\end{document}
