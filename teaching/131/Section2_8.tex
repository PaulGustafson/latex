\documentclass[t]{beamer}

\mode<handout>
{
  \usepackage{pgf}
  \usepackage{pgfpages}

\pgfpagesdeclarelayout{4 on 1 boxed}
{
  \edef\pgfpageoptionheight{\the\paperheight} 
  \edef\pgfpageoptionwidth{\the\paperwidth}
  \edef\pgfpageoptionborder{0pt}
}
{
  \pgfpagesphysicalpageoptions
  {%
    logical pages=4,%
    physical height=\pgfpageoptionheight,%
    physical width=\pgfpageoptionwidth%
  }
  \pgfpageslogicalpageoptions{1}
  {%
    border code=\pgfsetlinewidth{2pt}\pgfstroke,%
    border shrink=\pgfpageoptionborder,%
    resized width=.5\pgfphysicalwidth,%
    resized height=.5\pgfphysicalheight,%
    center=\pgfpoint{.25\pgfphysicalwidth}{.75\pgfphysicalheight}%
  }%
  \pgfpageslogicalpageoptions{2}
  {%
    border code=\pgfsetlinewidth{2pt}\pgfstroke,%
    border shrink=\pgfpageoptionborder,%
    resized width=.5\pgfphysicalwidth,%
    resized height=.5\pgfphysicalheight,%
    center=\pgfpoint{.75\pgfphysicalwidth}{.75\pgfphysicalheight}%
  }%
  \pgfpageslogicalpageoptions{3}
  {%
    border code=\pgfsetlinewidth{2pt}\pgfstroke,%
    border shrink=\pgfpageoptionborder,%
    resized width=.5\pgfphysicalwidth,%
    resized height=.5\pgfphysicalheight,%
    center=\pgfpoint{.25\pgfphysicalwidth}{.25\pgfphysicalheight}%
  }%
  \pgfpageslogicalpageoptions{4}
  {%
    border code=\pgfsetlinewidth{2pt}\pgfstroke,%
    border shrink=\pgfpageoptionborder,%
    resized width=.5\pgfphysicalwidth,%
    resized height=.5\pgfphysicalheight,%
    center=\pgfpoint{.75\pgfphysicalwidth}{.25\pgfphysicalheight}%
  }%
}


  \pgfpagesuselayout{4 on 1 boxed}[a4paper, border shrink=5mm, landscape]
  \nofiles
}

%% Language and font encodings
\usepackage[english]{babel}
\usepackage[utf8x]{inputenc}
\usepackage[T1]{fontenc}

\usetheme{Madrid}
\usecolortheme{beaver}

%% Useful packages
\usepackage{amsmath}
\usepackage{graphicx}

\usepackage{enumitem}

% full page itemieze
\newenvironment{fpi}
  {\itemize[nolistsep,itemsep=\fill]}
  {\vfill\enditemize}

\newcommand{\img}[1]{
\vfill
\includegraphics[width=\textwidth,height=0.5\textheight,keepaspectratio]{#1}
\vfill
} 


\title{What does $f'$ say about the graph of $f$?  (Section 2.8) \\
and Derivatives of Polynomials and Exponential Functions (Section 3.1)}
\date{}

\begin{document}
\frame{\titlepage}

\begin{frame}{Introduction}
\begin{fpi}
\item The main point of this lecture is to plug the variable $x$ into the derivative, instead of a constant $a$.
\item This means we are thinking of the derivative as a function of $x$, instead of just a number.
\end{fpi}
\end{frame}

\begin{frame}{Overview}
\begin{fpi}
\item Finish derivative examples from last time
\item Derivative as a function
\item Differentiability
\end{fpi}
\end{frame}

\begin{frame}{Derivative as Number}
Last lecture we saw the formula for the derivative at $x = a$:
$$ f'(a) = \lim_{h \to 0} \frac{f(a + h) - f(a)}{h} $$
We plugged in a number for $a$ and got a number.
\end{frame}

\begin{frame}{Derivative as Function}
Today we replace $a$ with the variable $x$:
$$ f'(x) = \lim_{h \to 0} \frac{f(x + h) - f(x)}{h}$$
The answer now is a function of $x$.  The $h$ will disappear after evaluating the limit.
\end{frame}

\begin{frame}{Derivative as Function}
Connection:

If we get $f'(x)$ as a function, we can plug in any number $a$ to get the same
answer as last time.  
\end{frame}

\begin{frame}{Derivative as Function}
In general, when we say ``calculate the derivative of $f(x)$'' we mean calculate
its derivative as a function.
\end{frame}

\begin{frame}{Example}
Find the derivative of
$$f(x) = 2x^2 - 3$$
\end{frame}

\begin{frame}{Comparing the graphs from the last example}

\end{frame}

\begin{frame}{Example}
Find the derivative of
$$f(x) = \frac{1}{1-x} + 2$$
\end{frame}

\begin{frame}{Notation}
Usually we write
$$f'(x) = derivative$$
Sometimes we write
$$\frac{d}{dx} f(x) = \frac{df}{dx} = \frac{dy}{dx} = derivative$$
\end{frame}

\begin{frame}{Differentiability}
A function $f(x)$ is \textbf{differentiable} at $x = a$ if the derivative exists at $x = a.$

Ways to check:

The limits match:
$$\lim_{x \to a^-} \frac{f(x) - f(a)}{x - a} = \lim_{x \to a^+} \frac{f(x) - f(a)}{x-a}.$$

OR

Calculate the derivative $f'(x)$ and check that you can plug in $x = a$.
\end{frame}

\begin{frame}{Not Differentiable}
Example of a function which is not differentiable at $x = 0$: 
$$f(x) = |x|$$
\end{frame}

\begin{frame}{Not Differentiable}
\img{notdiff}
\end{frame}

\begin{frame}{Differentiabity requires Continuity}
If $f(x)$ is differentiable at $x=a$, the $f(x)$ must be continuous at $x = a$.

Contrapositive:
\end{frame}

\begin{frame}{Higher derivatives}
Doing the derivative twice or more:
$$f''(x) = \text{ second derivative}$$
$$ f'''(x) = \text{ third derivative}$$
The second derivative of position wrt time is the instantaneous acceleration.

The third derivative of position wrt time  is the ``jerk''.
\end{frame}

\end{document}
