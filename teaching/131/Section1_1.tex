\documentclass[t]{beamer}

%% Language and font encodings
\usepackage[english]{babel}
\usepackage[utf8x]{inputenc}
\usepackage[T1]{fontenc}

\usetheme{Madrid}
\usecolortheme{beaver}

%% Useful packages
\usepackage{amsmath}
\usepackage{graphicx}

\usepackage{enumitem}

% full page itemieze
\newenvironment{fpi}
  {\itemize[nolistsep,itemsep=\fill]}
  {\vfill\enditemize}

\title{Introduction and Functions}
\author{Math 131, Section 501}
\date{January 17, 2017}

\begin{document}
\frame{\titlepage}


\begin{frame}{Introduction}
\begin{fpi}
\item Paul Gustafson
\item 4th year PhD student
\item Topological phases of matter 
\item Functional programming
\item Piano (favorite musician: marasy8)
\end{fpi}
\end{frame}

\begin{frame}{Motivation for the course}
\begin{fpi}
\item Work ethic
\item Critical thinking skills
\item Attention to detail
\item Mathematical maturity 
\item Signalling 
\end{fpi}
\end{frame}

\begin{frame}{Course information}
\begin{fpi}
\item Course webpage: http://math.tamu.edu/~pgustafs/math131
\item Office hours: 2:00-3:00 PM Mon, 11:00-12:30 AM Thurs 
\item Exam dates: Feb 16, Mar 23, Apr 20, May 4
\item Lowest exam grade
\item Take-home quizzes (must turn them in yourself!)
\end{fpi}
\end{frame}

\begin{frame}{Book and Webassign}
\begin{fpi}
\item Stewart Calculus 4.0
\item Hard copy or ebook
\item Must pay for webassign
\begin{itemize}
\item Hard copy purchase includes webassign
\item Can just buy webassign/ebook
\item 2 week free trial 
\end{itemize}
\end{fpi}
\end{frame}

\begin{frame}{Teaching Philosopy}
\begin{fpi}
\item Respect
\item I'm here to help you
\item No such thing as a stupid question
\end{fpi}
\end{frame}

\begin{frame}
\frametitle{Functions}
\begin{block}{Definition}
A \textbf{function} $f$ is a rule that assigns to each element in a set $D$ exactly one element, called $f(x)$
in a set $E$.
\end{block}
\end{frame}

\begin{frame}
\frametitle{Ways to define a function}
\begin{fpi}
\item Words
\item A table
\item An algebraic rule
\item A graph
\end{fpi}
\end{frame}

\begin{frame}{Applying functions}
\begin{fpi}
\item Let $\displaystyle f(x) = \frac{x^2 + 1}{x+3}$.
\item $f(1)$
\item $f(a)$
\item $f(2z -1)$
\item $f(g(x))$ where $g(x) = x^2 - 1$
\end{fpi}
\end{frame}

\begin{frame}{Domain and range}
\begin{block}{Definition}
The \textbf{domain} of $f$ is the set of values $x$ for which $f(x)$ is defined.
\end{block}
\begin{block}{Definition}
The \textbf{range} of $f$ is the set of all possible values $f(x)$.
\end{block}
\end{frame}

\begin{frame}{Finding domains}
Rules:
\begin{itemize}
\item Cannot divide by 0
\item Cannot take even roots of negative numbers  
\item Cannot take logarithms of numbers $\le 0$
\end{itemize}
\end{frame}

\begin{frame}
\frametitle{Even and odd functions}
\begin{block}{Definition}
A function $f$ is \textbf{even} if the values $f(-x) = f(x)$ for all $x$.
\end{block}
The graph of an even function is symmetric about the $y$-axis.
\begin{block}{Definition}
A function $f$ is \textbf{odd} if the values $f(-x) = -f(-x)$ for all $x$.
\end{block}
The graph of an odd function is symmetric about the origin.
\end{frame}

\begin{frame}
\frametitle{Piecewise functions}
\begin{block}{Definition}
A \textbf{piecewise function} is a function that has different rules for different parts of its domain.
\end{block}
\begin{block}{Example}
$$
|x| = \begin{cases}
x, & x \ge 0 \\
-x, &  x < 0
\end{cases}
$$
\end{block}
\end{frame}


\begin{frame}
\frametitle{Increasing and decreasing functions}
\begin{block}{Definition}
A function $f$ is \textbf{increasing} if $f(x)$ increases as $x$ increases.
\end{block}
\begin{block}{Definition}
A function $f$ is \textbf{decreasing} if $f(x)$ increases as $x$ increases.
\end{block}
\end{frame}

\end{document}
