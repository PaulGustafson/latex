\documentclass[t]{beamer}

%% Language and font encodings
\usepackage[english]{babel}
%\usepackage[utf8x]{inputenc}
%\usepackage[T1]{fontenc}

\usetheme{Madrid}
% \usecolortheme{beaver}

%% Useful packages
\usepackage{amsmath}
\usepackage{graphicx}

%\usepackage{enumitem}

% full page itemieze
%% \newenvironment{fpi}
%%   {\itemize[nolistsep,itemsep=\fill]}
%%   {\vfill\enditemize}

\newcommand{\img}[1]{
\vfill
\begin{center}
\includegraphics[width=\textwidth,height=0.5\textheight,keepaspectratio]{#1}
\end{center}
\vfill
} 


\title{Average Value of a Function (Section 6.5) and \\
Applications to Biology (Section 6.7)}
\date{}

\begin{document}
\frame{\titlepage}

\begin{frame}{Intro}
Computing the average value of a list of numbers is a 
natural way to understand the data.  Computing the
average value of a function is similar, except we need
a new formula.
\end{frame}

\begin{frame}{Averages}
We know how to calculate the average value of a list of numbers.
Data: 
$$(1,2,5,5,2,3,4)$$
Average:
$$\frac{1 + 2 + 5 + 5 + 2 + 3 + 4}{7}$$
\end{frame}

\begin{frame}{Average value of a function}
For a function, there are infinitely many $y$-values. Instead of a 
sum, we use an integral.  On the interval $[a,b]$, the average 
value of a function $f$ is:
$$f_{ave} = \frac{1}{b - a} \int_a^b f(x) \, dx$$
\end{frame}

\begin{frame}{Average value of a function}
On the interval $[a,b]$, the average 
value of a function $f$ is:
$$f_{ave} = \frac{1}{b - a} \int_a^b f(x) \, dx$$

The denominator $b - a$ is the length of the interval -- analogous to the
number of terms in a list.

The integral is analogous to the sum of all the terms in the list.
\end{frame}

\begin{frame}{Example}
Find the average value of the function $f(x)$ on the interval $[2,8]$.
$$f(x) = x^2 - 3x$$
\end{frame}

\begin{frame}{Example}
Find the average value of the function $f(x)$ on the interval $[0,2]$.
$$f(x) = \sqrt{12x + 1}$$
\end{frame}

\begin{frame}{Try it!}
Find the average value of $f(x)$ on the interval $[-2,2]$.
$$f(x) = 6x (x^2-5)^3$$
\end{frame}

\begin{frame}{Example}
Find the $x$-value $c$ such that the average value of $f(x)$
on the interval $[0,6]$ equals $f(c)$.
$$f(x) = x^2 - 5x + 1 $$
\end{frame}

\begin{frame}{Try it!}
Find the $x$-value $c$ such that the average value of $f(x)$
on the interval $[0,8]$ equals $f(c)$.
$$f(x) = 3x^2 - 4x - 7$$
\end{frame}

\begin{frame}{Example}
Let 
$$f(x) = 2x - 3 $$
Find a value of $b$ such that the average value of $f(x)$ on the interval $[0,b]$
is equal to $10$.
\end{frame}

\begin{frame}{Course evaluations}
Please fill out your course evaluation form at

http://www.math.tamu.edu
\end{frame}

\begin{frame}{Section 6.7: Poiseuille's Law}
You need this formula which gives the blood flow 
in a blood vessel for the homework:
$$F = \frac{\pi P R^4}{8 \eta l}$$
Just plug the numbers in.
\end{frame}




\end{document}
