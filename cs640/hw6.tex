\documentclass{article}

\usepackage{amsmath,amssymb,amsthm,latexsym,paralist}
%\DeclareGraphicsRule{.1}{mps}{*}{}

\theoremstyle{definition}
\newtheorem{problem}{Problem}
\newtheorem*{solution}{Solution}
\newtheorem*{resources}{Resources}

\newcommand{\ket}[1]{|#1\rangle} 
\newcommand{\bra}[1]{\langle#1|} 
\newcommand{\C}{\mathbf{C}}
\DeclareMathOperator{\tr}{tr}

\newcommand{\name}[1]{\noindent\textbf{Name: #1}}
\newcommand{\honor}{\noindent On my honor, as an Aggie, I have neither
  given nor received any unauthorized aid on any portion of the
  academic work included in this assignment. Furthermore, I have
  disclosed all resources (people, books, web sites, etc.) that have
  been used to prepare this homework. \\[1ex]
 \textbf{Signature:} \underline{\hspace*{5cm}} }

 
\newcommand{\checklist}{\noindent\textbf{Checklist:}
\begin{compactitem}[$\Box$] 
\item Did you add your name? 
\item Did you disclose all resources that you have used? \\
(This includes all people, books, websites, etc. that you have consulted)
\item Did you sign that you followed the Aggie honor code? 
\item Did you solve all problems? 
\item Did you submit the pdf file resulting from your latex source
  file on ecampus? 
\item Did you submit a hardcopy of the pdf file in class? 
\end{compactitem}
}

\newcommand{\problemset}[1]{\begin{center}\textbf{Problem Set #1}\\ 
CSCE 440/640\end{center}}
\newcommand{\duedate}[2]{\begin{quote}\textbf{Due dates:} Electronic
    submission of the pdf file of this homework is due on \textbf{#1} on ecampus.tamu.edu, a signed paper copy
    of the pdf file is due on \textbf{#2} at the beginning of
    class. \end{quote} }

\newcommand{\N}{\mathbf{N}}
\newcommand{\R}{\mathbf{R}}
\newcommand{\Z}{\mathbf{Z}}


\begin{document}


\problemset{6}
\duedate{11/2/2016 before 2:50pm}{11/2/2016}
\name{ Paul Gustafson }
\begin{resources} 
I used Mathematica to calculate a couple singular value decompositions.
\end{resources}
\honor

\newpage

\noindent 
\begin{problem}{(20 points)}
Consider the mixed state 
$$ M= \left\{ \left(\ket{0}, \frac{1}{3}\right), \left(\ket{1}, \frac{2}{3}\right)\right\}.$$
\begin{compactenum}[(a)]
\item Determine the density matrix $\rho$ of the mixed state $M$. 
\item Derive a different mixed state $M'$ (which should not consist of
  computational basis states) that has the same density matrix
  $\rho$ as $M$. 
\end{compactenum}
[This problem shows that density matrices are not in one-to-one
correspondence with mixed states.]
\end{problem}
\begin{solution}
\begin{compactenum}[(a)]
\item
\begin{align*}
\rho & = \frac{1}{3} \ket{0}\bra{0} + \frac{2}{3} \ket{1}\bra{1} 
\\ & = \frac{1}{3} \begin{pmatrix}
                                1  \\
                                0 
                              \end{pmatrix}
                              \begin{pmatrix} 
                                1  & 0
                              \end{pmatrix}
  + \frac{2}{3} \begin{pmatrix}
                                 0 \\ 
                                 1
                         \end{pmatrix} 
                         \begin{pmatrix}
                           0 & 1
                         \end{pmatrix}
\\ & = \frac{1}{3} \begin{pmatrix}
                                1 & 0 \\
                                0 & 0
                              \end{pmatrix}
     + \frac{2}{3} \begin{pmatrix}
                             0 & 0 \\
                             0 & 1
                          \end{pmatrix} 
\\ & = \begin{pmatrix}
               \frac{1}{3} & 0 \\
               0 & \frac{2}{3}
           \end{pmatrix}
\end{align*}

\item  First let's guess that there's a solution with real coefficients, so the derivation is slightly simpler. We're trying to find a mixed state
$$M' = \left\{ \left(a' \ket{0} + b' \ket{1}, p \right), \left(c' \ket{0} + d' \ket{1}, q \right)\right\}$$
with density matrix $\rho'$.
Let $a = \sqrt{p} a'$, $b = \sqrt{p} b'$, $c = \sqrt{q} c'$, and $d = \sqrt{q} d'$.
Then the condition $\rho = \rho'$ becomes
\begin{align*}
a^2 + c^2 & = 1/3
\\ b^2 + d^2 & = 2/3
\\ ab + cd & = 0
\end{align*}
The last equation implies that there exists a $k$ such that $(a,c) = k (b, -d)$.  Solving for $k$,
\begin{align*}
1/3 & = a^2 + c^2
\\ & = k^2 (b^2 + d^2)
\\ & = k^2 (2/3),
\end{align*}

so $k = \frac{1}{\sqrt{2}}$.  A solution to the first system of equations is $a = c = \frac{1}{\sqrt{6}}$ and $b = -d = \frac{1}{\sqrt{3}}$.  This corresponds to the mixed state
$$M' = \left\{ \left( \frac{1}{\sqrt{3}}\ket{0} + \sqrt{\frac{2}{3}} \ket{1}, \frac{1}{2} \right), \left(\frac{1}{\sqrt{3}} \ket{0} - \sqrt{\frac{2}{3}} \ket{1}, \frac{1}{2} \right)\right\}.$$                   
\end{compactenum}
\end{solution}

\begin{problem}{(20 points)}
\begin{compactenum}[(a)]
\item Do Exercise 3.5.1 (b) on page 55 of our textbook KLM.
\item Do Exercise 3.5.1 (c) on page 55 of our textbook KLM.
\end{compactenum}
\end{problem}
\begin{solution}
\begin{compactenum}[(a)]
\item $$\begin{pmatrix}
           \frac{1}{2} & \frac{1}{2}
           \\ \frac{1}{2} & \frac{1}{2}
        \end{pmatrix}
      $$
\item 
\begin{align*}
\frac{1}{2} \begin{pmatrix}
           \frac{1}{2} & \frac{1}{2}
           \\ \frac{1}{2} & \frac{1}{2}
        \end{pmatrix}
+ 
\frac{1}{2} \begin{pmatrix}
           \frac{1}{2} & -\frac{1}{2}
           \\ -\frac{1}{2} & \frac{1}{2}
           \end{pmatrix}
& = 
\begin{pmatrix}
\frac{1}{2} & 0
\\ 0 & \frac{1}{2}
\end{pmatrix}
\end{align*}
\end{compactenum}
\end{solution}

\begin{problem}{(20 points)}
Find the Schmidt decomposition of the states 
\begin{compactenum}[(a)]
\item $ \frac{1}{2}\left( \ket{00} + \ket{01} + \ket{10} + \ket{11}
  \right)$.
\item $ \frac{1}{\sqrt{3}}\left( \ket{00} + \ket{01} + \ket{10}
  \right)$.
\end{compactenum}
[Students of CSCE 440 only need to solve (a), and students of CSCE 640
should solve both (a) and (b).] 
\end{problem}
\begin{solution}
I used Mathematica to find the SVD of the corresponding matrices.
\begin{compactenum}[(a)]
\item 
$(\frac{\sqrt{2}}{2} \ket{0} + \frac{\sqrt{2}}{2} \ket{1}) \otimes (\frac{\sqrt{2}}{2} \ket{0} + \frac{\sqrt{2}}{2} \ket{1})$
\item 
\begin{align*}
&  \sqrt{\frac{1}{6} \left(\sqrt{5}+3\right)} \left(\frac{\sqrt{5}+3}{2 \sqrt{2 \sqrt{5}+5}} \ket{0} + \frac{3-\sqrt{5}}{2 \sqrt{5-2 \sqrt{5}}} \ket{1}\right) 
       \otimes \left(\frac{\sqrt{5}+1}{\sqrt{2 \left(\sqrt{5}+5\right)}}\ket{0} + \frac{1-\sqrt{5}}{\sqrt{10-2 \sqrt{5}}} \ket{1}\right) 
\\ +  &  \sqrt{\frac{1}{6} \left(\sqrt{5}-3\right)} \left(\frac{\sqrt{5}+1}{2 \sqrt{2 \sqrt{5}+5}} \ket{0} + \frac{1-\sqrt{5}}{2 \sqrt{5-2 \sqrt{5}}} \ket{1} \right) 
       \otimes \left(\frac{2}{\sqrt{2 \left(\sqrt{5}+5\right)}}\ket{0} + \sqrt{\frac{1}{10} \left(\sqrt{5}+5\right)} \ket{1}\right)  
\end{align*}
which is approximately
\begin{align*}
& 0.934(0.851 \ket{0} + 0.526 \ket{1}) \otimes (0.526 \ket{0} -0.851 \ket{1})
\\ + &  0.357(0.851 \ket{0} + 0.526 \ket{1}) \otimes (-0.526 \ket{0} + 0.851 \ket{1})
\end{align*}
\end{compactenum}
\end{solution}

\begin{problem}{(20 points)}
Exercise 3.5.4 (a) on page 57 in our textbook KLM. 
\end{problem}
\begin{solution}
For any pure state $\ket{a} \otimes \ket{b} \in \mathcal{H}_A \otimes \mathcal{H}_B$,
\begin{align*}
\tr_B((U \otimes I) \ket{a} \bra{a} \otimes \ket{b} \bra{b} (U^\dagger \otimes I)) 
& = \tr_B(U \ket{a} \bra{a} U^\dagger \otimes \ket{b} \bra{b}) 
\\  & = U \ket{a} \bra{a} U^\dagger) \langle b | b \rangle 
\\  & = U \ket{a} \bra{a} \langle b | b \rangle  U^\dagger)  
\\  & = U \tr_B(\ket{a} \bra{a} \otimes \ket{b} \bra{b})   U^\dagger)  
\end{align*}
Since the trace $\tr_B$ is linear, the same identity holds for any linear
combination of density matrices of pure states.  Hence, it holds for all states.
\end{solution}

\begin{problem}{(20 points)}
Choi has shown that for all matrices $V_j\in \C^{n\times m}$, the map 
$T\colon M_n(\C) \rightarrow M_m(\C)$ given by 
$$ T(\rho)  = \sum_{j=1}^\ell V_j^* \rho V_j$$
is completely positive. Show that if the matrices $V_j$ satisfy the
condition
$$ \sum_{j=1}^\ell V_jV_j^* = I,$$
where $I$ denotes the identity matrix, then $T$ is trace preserving,
so $\tr T(A) = \tr A$. [Hint: the matrix trace satisfies $\tr(ABC)=\tr(CAB)$.]
\end{problem}
\begin{solution}
\begin{align*}
\tr T(A) & = \tr \left( \sum_{j=1}^\ell V_j^* A V_j \right)
\\ & =  \sum_{j=1}^\ell \tr (V_j^* A V_j)
\\ & =  \sum_{j=1}^\ell \tr (V_j V_j^* A )
\\ & =  \tr \left( \sum_{j=1}^\ell  V_j V_j^* A \right)
\\ & =  \tr  A 
\end{align*}

\end{solution}


\goodbreak
\checklist

\end{document}
