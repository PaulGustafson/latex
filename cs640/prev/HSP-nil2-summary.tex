\documentclass{article}

\newcommand{\ket}[1]{\left| #1 \right\rangle} 

%\title{Summary of Nil-2 HSP Algorithm}

\begin{document}

%\maketitle

\section{Summary of Nil-2 HSP Algorithm}

\textbf{Problem:} Solve the HSP for $H \leq G$, where $G$ is a nil-2 group.

\subsection{Reduction Steps}

\begin{enumerate}
\item Calculate the refined polycyclic representation of $G$.
\item Reduce to HSP in nil-2 $p$-groups
\item Reduce to case where $H$ is either trivial or order $p$
\item Reduce to case where $G$ has exponent $p$.
\item Reduce to finding a quantum hiding function for  $HG'$
\item Reduce to finding an appropriate triple
\item Reduce to solving a large system of linear and quadratic equations
\end{enumerate}

\subsection{Quantum Algorithm}

All of the above steps have efficient classical algorithms, except for the step of generating a quantum hiding function for $HG'$ given an appropriate triple, so  I'll focus on this step.

\textbf{Summary:}
\begin{enumerate}
\item Compute the superposition $\sum_{u \in G'} \ket{u} \ket{aHG'_u}$ for random $a \in G$, where $G_u = DFT(\ket{u})$.
\item Do the last step $n$ times in parallel for some large $n$
\item Solve the system of equations to get $\bar j \in (\mathbf{Z}_p)^n$
\item $\ket{\Psi_g^{\bar a, \bar u, \bar j}} = \bigotimes_{i=1}^n \ket {a_i H G'_{u_i} \phi_{j_i}(g)}$ as a function of $g \in G$ is a hiding function for $H G'$, where $\phi_j$ are nice automorphisms of $G$.
\end{enumerate}

The following properties of the automorphisms $\phi_j$ are used in the proof:
\begin{enumerate}
\item $\ket{aHG'_u}$ is an eigenvector for right multiplication by $\phi_j(g)$
\item $\phi_j$ maps $HG'$ to $HG'$
\end{enumerate}

\end{document}
