\documentclass[12pt,a4paper]{article}
\usepackage{endnotes,emp,amsmath,amsthm}
\usepackage[pdftex]{graphicx}
\DeclareGraphicsRule{*}{mps}{*}{}

\theoremstyle{definition}
\newtheorem{problem}{Problem}
\newtheorem*{solution}{Solution}
\newtheorem*{resources}{Resources}

\newcommand{\name}[1]{\noindent\textbf{Name: Paul Gustafson} \\}
\newcommand{\honor}{\noindent On my honor, as an Aggie, I have neither
  given nor received any unauthorized aid on any portion of the
  academic work included in this assignment. Furthermore, I have
  disclosed all resources (people, books, web sites, etc.) that have
  been used to prepare this homework. \\[1ex]
 \textbf{Signature:} \underline{\hspace*{5cm}} }

 
\newcommand{\checklist}{\noindent\textbf{Checklist:}
\begin{compactitem}[$\Box$] 
\item Did you add your name? 
\item Did you disclose all resources that you have used? \\
(This includes all people, books, websites, etc. that you have consulted)
\item Did you sign that you followed the Aggie honor code? 
\item Did you solve all problems? 
\item Did you submit the pdf file resulting from your latex source
  file on ecampus? 
\item Did you submit a hardcopy of the pdf file in class? 
\end{compactitem}
}


\def\display #1:#2\par{\par\hangindent 30pt \noindent
        \hbox to 30pt{\hfill #1 \hskip .1em}\ignorespaces#2\par\medskip}

\newcommand{\ds}{\displaystyle}
\newcommand{\F}{\mathbf{F}}




\begin{document}
\thispagestyle{empty}
\begin{center}
\textbf{Problem Set 4}\\
CPSC 440/640 Quantum Algorithms\\ 
Andreas Klappenecker
\end{center}

\name

\honor


\begin{center}
\bf The deadline for this assignment has been extended to Thursday,
Oct 5, 11:59pm. The signed hardcopy is due on Friday, Oct 6, in
class. 
\end{center}


\noindent The goals of this assignment are (a) to make you familiar
with the simulation of quantum circuits on a classical computer, and
(b) to give you the opportunity to get familiar with lex and yacc (use 
flex and bison if possible).

You are given a considerable amount of time for this assignment so
that this exercise does not provide a conflict with your research. I
recommend that you start early. You can earn 150 points instead of the
usual 100 points.  Some source code is provided in the
tar-ball~\texttt{alfred.tgz}.
\medskip

\noindent 0) Modify the source code so that it will compile on the
system of your choice. For instance, on Windows systems, you might
need to include \verb|windows.h|. If your system does not support the
\verb|drand48| random number generator, then you might want to supply
your own pseudo-random number generator. 


\noindent 1) The core simulator is contained in the file \texttt{sim.c}. 
Supplement the missing code in the procedures \texttt{measure\_state}
and \texttt{applygate}. 
\smallskip

\noindent 2) Write a small test file that uses the procedures given in 
\texttt{sim.c} to create the state $0.707|00)+0.707|11)$ from input 
state $|00)$. Do this by simulating the action of one Hadamard gate 
and one controlled-not gate with \texttt{applygate}. This should be 
followed by measuring the state with \texttt{measure\_state}. 
Monitor the evolution of the state after each step by \texttt{print\_state}. 
\smallskip

\noindent 3) Get familiar with lex and yacc, or, rather, with flex 
and bison. Read the manuals and implement some small example that
allows you to grasp the main concept of the interaction between lex
and yacc.
\smallskip

\noindent 4) Correct all errors so that you get a fully functional simulator 
for the language Alfred. 
\medskip

You will receive a little Alfred program and you need to demonstrate
that your simulator works. The details will be discussed in class. 
\medskip

\noindent\textbf{Remarks.} i) The simulator assumes that you will 
not simulate more than 20-30 quantum bits. For efficiency reasons,
information about the position of control and target qubits are
encoded by setting the corresponding bits in an integer.  For example,
if the target qubit position \texttt{pos} is the least significant
bit, then this is represented by \texttt{pos = 1<<0}, if the target
qubit is the most significant qubit in a system of 3 qubits, then this
is encoded by \texttt{pos = 1<<2}. Review the bit operations of the
language C to see the benefit of this convention.

ii) Write the code for the procedure \texttt{measure\_state}.
The input for this procedure is an integer \texttt{pos}, which has a
bit set at the position of the quantum bit, which will be observed.
The measurement is done with respect to the computational basis. Your
procedure should directly modify the input state vector
\texttt{state}. You can address the content of this state vector by
\texttt{state[\textit{i}]}, where \texttt{0 $\le \texttt{\textit{i}} <
$ 1<<Nbits}.  Use the random number generator \texttt{rand} in your
implementation. Make sure that your implementation will reflect the
behaviour of quantum mechanics. You might need to change the
initialization of the random number generator on you system to obtain
the desired results.

iii) Write the code for the procedure \texttt{applygate}.
Your code should realize an implementation of a multiply conditioned
gate, as explained in the lecture. The conditions are provided in
terms of integers. A bit set in \texttt{ocnd} means that this bit must
be 0, a bit set in \texttt{icnd} means that this bit must be set to~1.
The integer \texttt{gpos} has a single bit set at the target bit
position of the gate.

iv) Before writing the code for \texttt{applygate}, I suggest that you
re-read the lecture notes on multiple control quantum gates. 

v) Update: Fill in the verbatim output of the five challenges below. 

vi) Use the GNU gcc compiler. The source code uses some language
extensions provided by gcc. If you use some Windows operating system,
then I suggest that you have a look at cygwin, which provides you with
all the Unix utilities such as lex and yacc (or rather flex and
bison),...

vii) You are welcome to solve this problem set in a different
language such as Ruby or C++. 

viii) You are allowed to discuss problems related to compilation of
the code and other technical difficulties. You should avoid sharing
code or showing your solution to others. 

\begin{problem}
Include a verbatim copy of your implementation \verb|measure_state|. 
Include a rational for your implementation that documents your code
well. 
\end{problem}
\begin{solution}
My \verb|measure_state| code is given by 
\begin{verbatim}
/* input: a single bit set at the position of the gate; 
   this bitposition is measured and the state vector  
   is changed according to the postulates of quantum 
   mechanics. 
   NOTE: the state must be normalized to length 1 */
void measure_state(int pos, cplx *state) { 
  int i;  
  double ran; 
  double prob; 
  
  srand(clock());
  ran = ((double)rand())/RAND_MAX;
  /* complete the missing code */ 

  /* compute the probability to observe 0,
   * determine whether 0 or 1 should be observed in this measurement
   * then modify state to produce the post measurement state 
   * of the quantum computer
   */ 

  // the probability of getting a zero in the ith position
  prob = 0;

  // calculate prob by adding up the squared moduli of coefficients
  // with 0 in the position "pos"
  for (i = 0; i < (1 << Nbits); i++) {
    if (((i / pos) % 2) == 0) {
      prob += sq(state[i].re)+sq(state[i].im);
    }
  }

  // do the measurement
  int measuredBit;

  if(ran < prob) 
    measuredBit = 0;
  else
    measuredBit = 1;
  
  // Apply the projection operator
  for (i = 0; i < (1 << Nbits); i++) {
    if (((i / pos) % 2) != measuredBit) {
      state[i].re = 0;
      state[i].im = 0;
    }
  }


  normalize_state();  
}
\end{verbatim}

At the beginning of the method, I pick a random number, uniformly distributed
between 0 and 1.  Then I calculate the probability of measuring a zero in the 
position $pos$ by adding up the squared moduli of all coefficients of basis
elements with 0 in position $pos$.  If the random number is less than this probability, 
I measure a 0.  Otherwise, I measure a 1.  Then I orthogonally project the state onto the 
subspace compatible with the measurement, followed by normalizing the state.
\end{solution}

\begin{problem}
Include the verbatim output of \verb|./alfred < chlg1.alf|
\end{problem}
\begin{solution}
\begin{verbatim}
If the state is of the form
0.707|0)+0.707|1)
Then one expects that
0 will be observed with prob. 1/2
1 will be observed with prob. 1/2
1|1)
1|0)
1|0)
1|0)
1|0)
1|1)
1|1)
1|1)
1|0)
1|1)
\end{verbatim}
\end{solution}

\begin{problem}
Include the verbatim output of \verb|./alfred < chlg2.alf|
\end{problem}
\begin{solution}
\begin{verbatim}
If the state is of the form
0.577|00)+0.816|10)
Then one expects that
0 will be observed with prob. 1/3
1 will be observed with prob. 2/3
1|10)
1|10)
1|10)
1|00)
1|00)
1|00)
1|10)
1|00)
1|10)
1|10)
\end{verbatim}
\end{solution}

\begin{problem}
Include a verbatim copy of your implementation \verb|applygate|. 
Include a rational for your implementation that documents your code
well. 
\end{problem}
\begin{solution}
My \verb|applygate| code is given by 
\begin{verbatim}
/* Applying a conditioned g-gate to a state vector */ 
/* ASSUME: ocnd, icnd, and cpos have disjoint support */
void applygate(int Nbits, int ocnd, int icnd, int gpos, 	       
               mat g, cplx *state  ) { 
  cplx tmp0, tmp1; 
  int i;
  int acnd = ( ocnd | icnd ); 
  

  /* complete the missing code */

  /* gpos is a integer of the form 2^k, where 
   * k is the position of the target qubit. 
   * The bits set in ocnd correspond to the 
   * non-filled circles, and the bits set in icnd
   * correspond to the filled circles
   */ 

  for (i = 0; i < (1 << Nbits); i++) {
    if(((i & acnd) ^ icnd) == 0) {
      if(((i / gpos) % 2) == 0) {
        tmp0 = state[i];
        tmp1 = state[i + gpos];
        state[i] =  add(mult(g[0], tmp0), mult(g[1], tmp1));
        state[i + gpos] = add(mult(g[2], tmp0), mult(g[3], tmp1));
      }
    }
  }
}
\end{verbatim}
To apply the gate, I loop through all the basis elements.  The basis elements
that satisfy the control bit (i.e. $0$'s on $ocnd$ and $1$'s on $acnd$) form
pairs based on whether they have a $0$ or a $1$ in position $gpos$.  Each such 
pair forms an invariant subspace for the gate, so it's enough to just apply the
gate to the pair when you hit the first of each pair in the loop.  
\end{solution}


\begin{problem}
Include the verbatim output of \verb|./alfred < chlg3.alf|
\end{problem}
\begin{solution}
\begin{verbatim}
 
A Hadamard gate that acts on the 
least significant bit
------------------------
|00) -->
0.707|00)+0.707|01)
------------------------
|01) -->
0.707|00)-0.707|01)
------------------------
|10) -->
0.707|10)+0.707|11)
------------------------
|11) -->
0.707|10)-0.707|11)
\end{verbatim}
\end{solution}

\begin{problem}
Include the verbatim output of \verb|./alfred < chlg4.alf|
\end{problem}
\begin{solution}
\begin{verbatim}
 
A Hadamard gate that acts on the 
most significant bit
------------------------
|00) -->
0.707|00)+0.707|10)
------------------------
|01) -->
0.707|01)+0.707|11)
------------------------
|10) -->
0.707|00)-0.707|10)
------------------------
|11) -->
0.707|01)-0.707|11)
\end{verbatim}
\end{solution}


\begin{problem}
Include the verbatim output of \verb|./alfred < chlg5.alf|
\end{problem}
\begin{solution}
\begin{verbatim}
 
A controlled Hadamard gate that acts on the 
most significant bit when the least significant
is 0 and the middle bit is 1:
------------------------
|000) -->
1|000)
------------------------
|001) -->
1|001)
------------------------
|010) -->
0.707|010)+0.707|110)
------------------------
|011) -->
1|011)
------------------------
|100) -->
1|100)
------------------------
|101) -->
1|101)
------------------------
|110) -->
0.707|010)-0.707|110)
------------------------
|111) -->
1|111)
\end{verbatim}
\end{solution}

\end{document}





