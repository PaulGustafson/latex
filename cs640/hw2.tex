\documentclass{article}
\usepackage{amsmath,amssymb,amsthm,latexsym,paralist}

\theoremstyle{definition}
\newtheorem{problem}{Problem}
\newtheorem*{solution}{Solution}
\newtheorem*{resources}{Resources}

\newcommand{\name}[1]{\noindent\textbf{Name: #1}}
\newcommand{\honor}{\noindent On my honor, as an Aggie, I have neither
  given nor received any unauthorized aid on any portion of the
  academic work included in this assignment. Furthermore, I have
  disclosed all resources (people, books, web sites, etc.) that have
  been used to prepare this homework. \\[1ex]
 \textbf{Signature:} \underline{\hspace*{5cm}} }

 
\newcommand{\checklist}{\noindent\textbf{Checklist:}
\begin{compactitem}[$\Box$] 
\item Did you add your name? 
\item Did you disclose all resources that you have used? \\
(This includes all people, books, websites, etc. that you have consulted)
\item Did you sign that you followed the Aggie honor code? 
\item Did you solve all problems? 
\item Did you submit the pdf file resulting from your latex source
  file on ecampus? 
\item Did you submit a hardcopy of the pdf file in class? 
\end{compactitem}
}

\newcommand{\problemset}[1]{\begin{center}\textbf{Problem Set #1}\\ 
CSCE 440/640\end{center}}
\newcommand{\duedate}[2]{\begin{quote}\textbf{Due dates:} Electronic
    submission of the pdf file of this homework is due on \textbf{#1} on ecampus.tamu.edu, a signed paper copy
    of the pdf file is due on \textbf{#2} at the beginning of
    class. \end{quote} }

\newcommand{\N}{\mathbf{N}}
\newcommand{\R}{\mathbf{R}}
\newcommand{\Z}{\mathbf{Z}}


\begin{document}
\problemset{2}
\duedate{9/14/2016 before 2:50pm}{9/14/2016}
\name{ Paul Gustafson }
\begin{resources} I talked to Andrew Kimball about problem 3.3.
\end{resources}
\honor

\newpage

\noindent Read chapters 2 and 3 in the lecture notes and chapter 5 in the textbook. \medskip

\noindent\textbf{Quantum Circuits}
\begin{problem} (10 points)
Exercise 2.11 in the lecture notes. Hint: $X=HZH$. 
\end{problem}
\begin{solution}
Diagonalize and take the square root. The result is:
$$R = 
\frac{1}{2} \begin{pmatrix}
 1 + i & 1 - i \\
1 - i & 1 + i
\end{pmatrix}
$$
\end{solution}

\begin{problem} (10 points)
Exercise 2.16 in the lecture notes. 
\end{problem}
\begin{solution}
Write down all the transformations in permutation cycle notation,
where $0 = \left|00 \right\rangle$, $1 = \left|01 \right\rangle$,
$2 = \left|10 \right\rangle$, and $3 = \left| 11 \right\rangle$.

The given map is $(12)$.  The CNOT gates are 
$\Lambda_{1,0} = (23)$ and $\Lambda_{0,1} = (13)$.

Since $(12) = (13) (23) (13)$, the given map is
$\Lambda_{0,1}\Lambda_{1,0}\Lambda_{0,1}$.
\end{solution}

\begin{problem} (15 points)
Exercise 2.22 in the lecture notes.
\end{problem}
\begin{solution}
$$
H \otimes 1_2 = \frac{1}{\sqrt 2} 
\begin{pmatrix}
1 & 1 \\
1 & -1
\end{pmatrix} \otimes  
\begin{pmatrix}
1 & 0 \\
0 & 1
\end{pmatrix}
= \frac{1}{\sqrt 2}
\begin{pmatrix}
1 & 0 & 1 & 0 \\
0 & 1 & 0 & 1 \\
1 & 0 & -1 & 0 \\
0 & 1 & 0 & -1 
\end{pmatrix}
$$ 
\end{solution}

\begin{problem} (15 points)
Exercise 2.23 in the lecture notes.
\end{problem}
\begin{solution}
\begin{align*}
\Lambda_{1,0} \circ (H \otimes 1_2) & = 
\begin{pmatrix}
1 & 0 & 0 & 0 \\
0 & 1 & 0 & 0 \\
0 & 0 & 0 & 1 \\
0 & 0 & 1 & 0
\end{pmatrix}
\frac{1}{\sqrt 2}
\begin{pmatrix}
1 & 0 & 1 & 0 \\
0 & 1 & 0 & 1 \\
1 & 0 & -1 & 0 \\
0 & 1 & 0 & -1 
\end{pmatrix}
\\
& = \frac{1}{\sqrt 2} 
\begin{pmatrix}
1 & 0 & 1 & 0 \\
0 & 1 & 0 & 1 \\
0 & 1 & 0 & -1 \\
1 & 0 & -1 & 0
\end{pmatrix},
\end{align*}
so 
\begin{align*}
 \left| 00 \right\rangle & \mapsto \frac{1}{\sqrt 2} \left( \left| 00 \right\rangle +  \left| 10 \right\rangle \right)  \\
 \left| 01 \right\rangle & \mapsto \frac{1}{\sqrt 2} \left( \left| 01 \right\rangle +  \left| 11 \right\rangle \right)  \\
 \left| 10 \right\rangle & \mapsto \frac{1}{\sqrt 2} \left( \left| 01 \right\rangle -  \left| 11 \right\rangle \right)  \\
 \left| 11 \right\rangle & \mapsto \frac{1}{\sqrt 2} \left( \left| 00 \right\rangle -  \left| 10 \right\rangle \right)  
\end{align*}
\end{solution}


\noindent\textbf{Entangled States and Teleportation.} 
\begin{problem} (15 points)
Exercise 3.1 in the lecture notes.
\end{problem}
\begin{solution}
It follows from the result in the next exercise.
\end{solution}

\begin{problem} (15 points)
Exercise 3.2 in the lecture notes.
\end{problem}
\begin{solution}
Let $\left| \psi \right\rangle = \alpha \left| 00 \right\rangle + \beta \left| 01 \right\rangle + \gamma \left| 10 \right\rangle + \delta \left| 11 \right\rangle.$.
Now suppose $\left| \psi \right\rangle = v \otimes w$, where $v = a \left| 0 \right\rangle + b \left| 1 \right\rangle$ and $w = c\left| 0 \right\rangle + d \left| 1 \right\rangle$.  Then we have $\alpha = ac$, $\beta = ad$, $\gamma = bc$, and $\delta = bd$.  Hence, $\alpha\delta - \beta\gamma = (ac)(bd) - (ad)(bc) = 0$.

Conversely, suppose $\alpha\delta - \beta\gamma = 0$. WLOG suppose $\alpha \neq 0$ (if all four constants are 0, then $\psi$ is decomposable). Then, from linear algebra, there exists a constant $m$ such that $m(\alpha \gamma) = (\beta \delta)$.  
Let $v = \alpha \left| 0 \right\rangle +  \gamma\left| 1 \right\rangle$ and $w =  \left| 0 \right\rangle + m \left| 1 \right\rangle$.
Then 
\begin{align*}
v \otimes w & = (\alpha \left| 0 \right\rangle +  \gamma\left| 1 \right\rangle) \otimes (\left| 0 \right\rangle + m \left| 1 \right\rangle) \\ 
& = \alpha \left| 00 \right\rangle + m\alpha \left| 01 \right\rangle + \gamma \left| 10 \right\rangle + m\gamma \left| 1 1 \right\rangle) \\
& = \alpha \left| 00 \right\rangle + \beta \left| 01 \right\rangle + \gamma \left| 10 \right\rangle + \delta \left| 1 1 \right\rangle) \\
& = \left| \psi \right\rangle
\end{align*}
\end{solution}

\begin{problem} (20 points)
Exercise 3.3 in the lecture notes. 
\end{problem}
\begin{solution}
Do the exact same protocol as for the usual teleportation, including applying the same gate at the end by BOb.  At the end you have following pairs of observations with results:
\begin{align*}
observation & \otimes result \\
\left| 00 \right\rangle & \otimes (a \left| 0 \right\rangle + e^{i\theta} b \left| 1 \right\rangle ) \\
\left| 01 \right\rangle & \otimes (a e^{i\theta} \left| 0 \right\rangle + b \left| 1 \right\rangle ) \\
\left| 10 \right\rangle & \otimes (a \left| 0 \right\rangle + e^{i\theta} b \left| 1 \right\rangle ) \\
\left| 11 \right\rangle & \otimes (a e^{i\theta} \left| 0 \right\rangle + e^{i\theta} b \left| 1 \right\rangle ) 
\end{align*}
In the $\left| 00 \right\rangle$ and $\left| 10 \right\rangle$ cases, pass the resulting qubit through a circuit corresponding to the 
unitary matrix
$$
\begin{pmatrix}
1 & 0 \\
0 & e^{-i\theta}
\end{pmatrix}.
$$


In the $\left| 01 \right\rangle$ and $\left| 11 \right\rangle$ cases, pass the resulting qubit through a circuit corresponding to the 
unitary matrix
$$
\begin{pmatrix}
e^{-i\theta} & 0 \\
0 & 1
\end{pmatrix}.
$$
\end{solution}









\goodbreak
\checklist
\end{document}
