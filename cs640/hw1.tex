\documentclass{article}
\usepackage{amsmath,amssymb,amsthm,latexsym,paralist,url,tikz}

\theoremstyle{definition}
\newtheorem{problem}{Problem}
\newtheorem*{solution}{Solution}
\newtheorem*{resources}{Resources}

\newcommand{\name}[1]{\noindent\textbf{Name: #1}}
\newcommand{\honor}{\noindent On my honor, as an Aggie, I have neither
  given nor received any unauthorized aid on any portion of the
  academic work included in this assignment. Furthermore, I have
  disclosed all resources (people, books, web sites, etc.) that have
  been used to prepare this homework. \\[1ex]
 \textbf{Signature:} \underline{\hspace*{5cm}} }

 
\newcommand{\checklist}{\noindent\textbf{Checklist:}
\begin{compactitem}[$\Box$] 
\item Did you add your name? 
\item Did you disclose all resources that you have used? \\
(This includes all people, books, websites, etc. that you have consulted)
\item Did you sign that you followed the Aggie honor code? 
\item Did you solve all problems? 
\item Did you submit the pdf file resulting from your latex file
  of your homework?
\item Did you submit a hardcopy of the pdf file in class? 
\end{compactitem}
}

\newcommand{\problemset}[1]{\begin{center}\textbf{Problem Set #1}\\ 
CSCE 440/640 Fall 2012\end{center}}
\newcommand{\duedate}[2]{\begin{quote}\textbf{Due dates:} Electronic submission of .tex
    and .pdf files of this homework is due on \textbf{#1} on ecampus.tamu.edu, a signed paper copy
    of the pdf file is due on \textbf{#2} at the beginning of
    class. \end{quote} }

\newcommand{\N}{\mathbf{N}}
\newcommand{\R}{\mathbf{R}}
\newcommand{\Z}{\mathbf{Z}}


\begin{document}
\problemset{1}
\duedate{9/7/2016 before 2:40pm}{9/7/2016}
\name{Paul Gustafson}
\begin{resources} TikZ wikibook.
\end{resources}
\honor

\newpage

\textbf{Important: } 
Read Chapter 1 in our textbook (Kaye, Laflamme, Mosca). Read Chapter 1
and Appendix A in the lecture notes (Quantum Algorithms). 

\begin{problem} (20 points) Get familiar with \LaTeX. Let $a+ib$ be a
  complex number, where $a$ and $b$ are real numbers and
  $i^2=-1$. Nicely typeset how to convert the representation $(a,b)$
  to polar coordinates $(r,\theta)$. Find out how you can include a
  helpful graph or picture (tikz is my current personal favorite). 

  If you need to refresh your memory how the conversion to polar
  coordinates is done in practice, then watch the Khan Academy
  videos. 
\end{problem}
\begin{solution}
The polar representation of the complex number $z = a + bi$ is given by
$z = r e^{i \theta}$, where $r = \sqrt{a^2 + b^2}$ is the modulus and
$\theta$ is the angle from $(a,b)$ to the positive $x$-axis.  See Figure \ref{fig:help}
for a depiction.
\begin{figure}
\centering
\begin{tikzpicture}[scale=3]
 %\draw[step=.5cm, gray, very thin] (-1.2,-1.2) grid (1.2,1.2); 
 \filldraw[fill=green!20,draw=green!50!black] (0,0) -- (3mm,0mm) arc  (0:30:3mm) -- node[right=4pt] {$\theta$} cycle ; 
 \draw[->] (-1.25,0) -- (1.25,0) coordinate (x axis);
 \draw[->] (0,-1.25) -- (0,1.25) coordinate (y axis);
 \draw[very thick,red] (30:1cm) -- node[left,fill=white] {$a$} (30:1cm |- x axis);
 \draw[very thick,blue] (30:1cm |- x axis) -- node[below=2pt,fill=white] {$b$} (0,0);
 \draw (0,0) -- node[above=2pt] {$r$} (30:1cm);
  \end{tikzpicture}
\caption{Helpful graph comparing polar to Cartesian coordinates (Modified from Tikz Wikibook)}
\label{fig:help}
\end{figure}

\end{solution}

\begin{problem} (15 points) 
Find the real and imaginary part of the following complex numbers
\begin{compactenum}[(a)]
\item $(i-1)/(i+1)$.
\item $(3+4i)/(1-2i)$. 
\item $i^n$ for any integer $n$.
\end{compactenum}
\end{problem} 
\begin{solution}
\begin{compactenum}[(a)]
\item 
\begin{align*}
(i-1)/(i+1) & = \frac{(i-1)^2}{-2} \\
& = i 
\end{align*}
\item 
\begin{align*}
(3+4i)/(1-2i) & = \frac{(3+4i)(1+2i)}{1 + 4}\\
& = \frac{-5 + 10i}{5}\\
& = -1 + 2i
\end{align*}
\item 
\[
i^n = \begin{cases} 
      1 & n \equiv 0 \pmod 4 \\
      i & n \equiv 1 \pmod 4 \\
      -1 & n \equiv 2 \pmod 4 \\
      -i & n \equiv 3 \pmod 4 \\
   \end{cases}
\]
\end{compactenum}
\end{solution}

\begin{problem} (15 points) 
Calculate the modulus (=absolute value) of the following complex numbers: 
\begin{compactenum}[(a)]
\item $-3+i$.
\item $2+3i$.
\item $i^n$ for all integers $n$. 
\end{compactenum}
\end{problem}
\begin{solution}
\begin{compactenum}[(a)]
\item $|-3+i| = \sqrt{9 + 1} = \sqrt{10}$
\item $|2+3i| = \sqrt{4 + 9} = \sqrt{13}$
\item $|i^n| = 1$
\end{compactenum}
\end{solution}



\begin{problem} (10 points) 
Exercise 1.2 in the lecture notes.
\end{problem}
\begin{solution}
It is called circularly polarized because the elecromagnetic wave looks like
$ e^{it} | \leftrightarrow \rangle + e^{i(t + \pi/2)} |\updownarrow \rangle$.
Both the real and complex parts of this factor parametrize circles.  For example, the 
real part is $\cos(t) | \leftrightarrow \rangle - \sin(t) |\updownarrow \rangle $, which turns to the right.

An example of left hand polarization would be a real part of the form $ -\sin(t) | \leftrightarrow \rangle + \cos(t) |\updownarrow \rangle $.  This
corresponds to a complex amplitude of the form $ e^{i(t + \pi/2)}| \leftrightarrow \rangle + e^{it} |\updownarrow \rangle $.  This
corresponds to the state $\frac{1}{\sqrt 2} \left( i | \leftrightarrow \rangle + |\updownarrow \rangle \right)$.
\end{solution}

\begin{problem} (10 points) 
Exercise 2.1 in the lecture notes.
\end{problem}
\begin{solution}
The probability of observing $0$ is $1/10$.  The probability of $1$ is $9/10$.
\end{solution}

\begin{problem} (10 points) 
Exercise 2.2 in the lecture notes.
\end{problem}
\begin{solution}
The probability of observing $0$ is $1/2$.  The probability of $1$ is also $1/2$.
\end{solution}

\begin{problem} (10 points) 
Exercise 2.3 in the lecture notes.
\end{problem}
\begin{solution}
The probability of observing $11$ is $1/2$.  After observing $11$, the system collapses to the state $|11\rangle$.
\end{solution}

\begin{problem} (10 points) 
Exercise 2.4 in the lecture notes.
\end{problem}
\begin{solution}
Any such state is of the form $\frac{\alpha_1}{\sqrt 2}|00 \rangle + \frac{\alpha_2}{2} |01\rangle + \frac{\alpha_3}{2} |11\rangle$, where the $\alpha_i$ are phases.
\end{solution}







\goodbreak
\checklist
\end{document}
