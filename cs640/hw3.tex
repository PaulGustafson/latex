\documentclass{article}

\usepackage{amsmath,amssymb,amsthm,latexsym,paralist}
%\DeclareGraphicsRule{.1}{mps}{*}{}

\theoremstyle{definition}
\newtheorem{problem}{Problem}
\newtheorem*{solution}{Solution}
\newtheorem*{resources}{Resources}

\newcommand{\ket}[1]{|#1\rangle} 

\newcommand{\name}[1]{\noindent\textbf{Name: #1}}
\newcommand{\honor}{\noindent On my honor, as an Aggie, I have neither
  given nor received any unauthorized aid on any portion of the
  academic work included in this assignment. Furthermore, I have
  disclosed all resources (people, books, web sites, etc.) that have
  been used to prepare this homework. \\[1ex]
 \textbf{Signature:} \underline{\hspace*{5cm}} }

 
\newcommand{\checklist}{\noindent\textbf{Checklist:}
\begin{compactitem}[$\Box$] 
\item Did you add your name? 
\item Did you disclose all resources that you have used? \\
(This includes all people, books, websites, etc. that you have consulted)
\item Did you sign that you followed the Aggie honor code? 
\item Did you solve all problems? 
\item Did you submit the pdf file resulting from your latex source
  file on ecampus? 
\item Did you submit a hardcopy of the pdf file in class? 
\end{compactitem}
}

\newcommand{\problemset}[1]{\begin{center}\textbf{Problem Set #1}\\ 
CSCE 440/640\end{center}}
\newcommand{\duedate}[2]{\begin{quote}\textbf{Due dates:} Electronic
    submission of the pdf file of this homework is due on \textbf{#1} on ecampus.tamu.edu, a signed paper copy
    of the pdf file is due on \textbf{#2} at the beginning of
    class. \end{quote} }

\newcommand{\N}{\mathbf{N}}
\newcommand{\R}{\mathbf{R}}
\newcommand{\Z}{\mathbf{Z}}


\begin{document}


\problemset{3}
\duedate{9/21/2016 before 2:50pm}{9/21/2014}
\name{ (put your name here)}
\begin{resources} I used the program ``julia'' to do some matrix multiplication.
\end{resources}
\honor

\newpage

\noindent Read chapter 4 in the lecture notes and make five insightful
comments on perusall. Read chapter 6 in the textbook. \medskip


\begin{problem}(10 points) 
Exercise 2.24 in the lecture notes. 
\end{problem}
\begin{solution}
$$P(0) = \left(\frac{1}{3}\right)^2 + \left(\frac{\sqrt 5}{3}\right)^2 =  \frac{1}{9} + \frac{5}{9} = \frac{2}{3},$$
and
$$P(1) = \left(\frac{\sqrt 3}{3}\right)^2 = \frac{1}{3}.$$
The resulting states are
$$v_0 = \frac{\sqrt 6}{6} \left| 00 \right\rangle + \frac{\sqrt{30}}{6} \left| 10 \right\rangle,$$
and
$$v_1 = \left| 01 \right\rangle.$$
\end{solution}

\begin{problem}(20 points) 
Exercise 2.26 in the lecture notes. 
\end{problem}
\begin{solution}

%Can we use additional single qubits a la pg 77 of textbook?

%(H \otimes 1 \otimes 1)(\left| 000 \right\rangle) & = \frac{1}{\sqrt 2}(\left| 000 \right\rangle + \left| 100 \right\rangle) \\
%& = \\
\end{solution}

\begin{problem}(20 points)
Exercise 2.27 in the lecture notes. (a) Design the circuit, 
(b) prove the correctness of the
circuit and (c) show how to create the state. 
\end{problem}
%Same as 26
\begin{solution}

\end{solution}

\begin{problem}(20 points)
\begin{compactenum}[(a)]
\item Exercise 6.1.1 (a) in the textbook KLM (should read Figure 6.1)
\item Exercise 6.1.1 (b) in the textbook KLM
\end{compactenum}
\end{problem}
\begin{solution}
\begin{compactenum}[(a)]
\item Since $U$ is unitary, $\overline U^T U = I$.  This means $\delta_{jk} = \sum_i \overline u_{ij}  u_{ik}$, i.e. that the
columns of $U$ are orthormal.  Setting $k = j$, we get $1 = \sum_i |u_{ij}|^2$.
\item Any example given from (a) will have the rows summing to zero as well because the rows of a unitary matrix are orthonormal, too.  But this
doesn't have to be the case for a stochastic matrix of this type.  A counterexample is
$$
\begin{pmatrix}
1 & 1 \\
0 & 0
\end{pmatrix}
$$
\end{compactenum}

\end{solution}


\begin{problem} (10 points)
Exercise 3.4 in the lecture notes. 
\end{problem}
\begin{solution}
Just write out where each basis vector goes and eyeball it.
\begin{compactenum}[(a)]
\item If $f(x) = 0$, then the corresponding circuit is just the identity.
\item If $f(x) = 1$, then the corresponding circuit is $1 \otimes X$.
\item If $f(x) = x$, then the corresponding circuit is $\Lambda_{1,0}$.
\item If $f(x) = 1+x$, then the corresponding circuit is $(X \otimes 1) \Lambda_{1,0} (X \otimes 1)$.
\end{compactenum}
\end{solution}


\begin{problem}(20 points) 
  Consider a system of two quantum bits and a controlled-not gate
  $\lambda_{0,1}(X)$ that has the least significant bit as a control
  bit and acts on the most significant quantum bit. Dispel the myth
  that the control bit of the controlled-not gate remains
  unaffected. Specifically, describe the action of the controlled-not gate on the
  following four states:
$$ 
\ket{0_H} \otimes \ket{0_H}, \quad 
\ket{0_H} \otimes \ket{1_H}, \quad 
\ket{1_H} \otimes \ket{0_H}, \quad 
\ket{1_H} \otimes \ket{1_H},
$$
where 
$$ \ket{0_H} = \frac{1}{\sqrt{2}}\ket{0} +
\frac{1}{\sqrt{2}}\ket{1}\quad\text{and}\quad
\ket{1_H} = \frac{1}{\sqrt{2}}\ket{0} -
\frac{1}{\sqrt{2}}\ket{1}.$$
Express the result in terms of the $\ket{0_H}$ and $\ket{1_H}$ basis. 
\end{problem}
\begin{solution}
The transition matrices from the $H$-basis to the standard basis and vice versa are both
$$S := \frac{1}{2}
\begin{pmatrix}
1 & 1 & 1 & 1 \\
1 & -1 & 1 & -1 \\
1 & 1 & -1 & -1 \\
1 & -1 & -1 & 1
\end{pmatrix}.
$$

In the standard basis,
$$\Lambda_{0,1} = 
\begin{pmatrix}
1 & 0 & 0 & 0 \\
0 & 0 & 0 & 1 \\
0 & 0 & 1 & 0 \\
0 & 1 & 0 & 0
\end{pmatrix}.
$$

Conjugating by the transition matrix gives 
$$
\begin{pmatrix}
1 & 0 & 0 & 0 \\
0 & 1 & 0 & 0 \\
0 & 0 & 0 & 1 \\
0 & 0 & 1 & 0
\end{pmatrix}.
$$

Thus,
\begin{align*}
\Lambda_{0,1} (\ket{0_H} \otimes \ket{0_H}) & = \ket{0_H} \otimes \ket{0_H} \\ 
\Lambda_{0,1} (\ket{0_H} \otimes \ket{1_H}) & = \ket{0_H} \otimes \ket{1_H} \\
\Lambda_{0,1} (\ket{1_H} \otimes \ket{0_H}) & = \ket{1_H} \otimes \ket{1_H} \\
\Lambda_{0,1} (\ket{1_H} \otimes \ket{1_H}) & = \ket{1_H} \otimes \ket{0_H} 
\end{align*}

\end{solution}


















\goodbreak
\checklist

\end{document}
