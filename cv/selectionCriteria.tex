%%%%%%%%%%%%%%%%%%%%%%%%%%%%%%%%%%%%%%%%%
% Plain Cover Letter
% LaTeX Template
% Version 1.0 (28/5/13)
%
% This template has been downloaded from:
% http://www.LaTeXTemplates.com
%
% Original author:
% Rensselaer Polytechnic Institute 
% http://www.rpi.edu/dept/arc/training/latex/resumes/
%
% License:
% CC BY-NC-SA 3.0 (http://creativecommons.org/licenses/by-nc-sa/3.0/)
%
%%%%%%%%%%%%%%%%%%%%%%%%%%%%%%%%%%%%%%%%%

%----------------------------------------------------------------------------------------
%	PACKAGES AND OTHER DOCUMENT CONFIGURATIONS
%----------------------------------------------------------------------------------------

\documentclass[11pt]{letter} % Default font size of the document, change to 10pt to fit more text

\usepackage{newcent} % Default font is the New Century Schoolbook PostScript font 
%\usepackage{helvet} % Uncomment this (while commenting the above line) to use the Helvetica font

% Margins
\topmargin=-1in % Moves the top of the document 1 inch above the default
\textheight=8.5in % Total height of the text on the page before text goes on to the next page, this can be increased in a longer letter
\oddsidemargin=-10pt % Position of the left margin, can be negative or positive if you want more or less room
\textwidth=6.5in % Total width of the text, increase this if the left margin was decreased and vice-versa

\let\raggedleft\raggedright % Pushes the date (at the top) to the left, comment this line to have the date on the right

\begin{document}

%----------------------------------------------------------------------------------------
%	ADDRESSEE SECTION
%----------------------------------------------------------------------------------------

\begin{letter}{Hiring Committee \\
Department of Mathematics \\
Australian National University \\
Canberra ACT 0200}

%----------------------------------------------------------------------------------------
%	YOUR NAME & ADDRESS SECTION
%----------------------------------------------------------------------------------------

\begin{center}
\large\bf Paul Gustafson \\ % Your name
%\vspace{20pt} \hrule height 1pt % If you would like a horizontal line separating the name from the address, uncomment the line to the left of this text
Department of Mathematics \\ Texas A\&M University \\ College Station, TX 77843 \\ pgustafs@math.tamu.edu \\ (979)774-9184 % Your address and phone number
\end{center} 
\vfill

%\signature{John Smith} % Your name for the signature at the bottom

%----------------------------------------------------------------------------------------
%	LETTER CONTENT SECTION
%----------------------------------------------------------------------------------------

\opening{To the Hiring Committee:}

In regard to the selection critera for the MSI Fellowship, I expect to receive a PhD in Mathematics from Texas A\&M University by May 2018.  I have two recent mathematical publications submitted, one for which I am the sole author, and one with multiple collaborators. I have also given four conference presentations at AMS Sectional Meetings.

In regard to attracting external funding for collaborative research, I was funded to attend the Mathematical Research Community (MRC) in Homotopy Type Theory last summer.  I also recently received funding to attend the School and Workshop on Univalent Mathematics in Birmingham, UK.  This summer I co-organized a seminar on Quantum Information Theory with Michael Brannan and Julia Plavnik.

In regard to teaching, I have been the sole instructor of record for a Calculus course.  I have also mentored undergraduate students in an NSF-funded Research Experience for Undergraduates.  I have also been the teaching assistant for a variety of courses, from multivariable calculus to algebraic topology.

My collaborators include Eric Rowell and Julia Plavnik at Texas A\&M University, and Paul Bruillard at Pacific Northwest National Laboratory.  I have also worked with many people on (variants of) homotopy type theory, including Dan Licata and Guillaume Brunerie.

I am committed to recruiting and retaining underrepresented groups in mathematics, both in collaborative research and in teaching. Everyone should have the opportunity to learn and participate in mathematics, regardless of race, ethnicity, gender, personality, and socioeconomic and educational background.


\closing{Sincerely yours,\\ Paul Gustafson \\ Texas A\&M University}


%\encl{Curriculum vitae, employment form} % List your enclosed documents here, comment this out to get rid of the "encl:"

%----------------------------------------------------------------------------------------

\end{letter}

\end{document}
