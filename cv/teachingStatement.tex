\documentclass[12pt]{article}
\usepackage{amsmath}
\usepackage{amssymb}
\usepackage{amsthm}
\usepackage{amscd}
\usepackage{amsfonts}
\usepackage{graphicx}%
\usepackage{fancyhdr}


\newcommand{\ZZ}{\mathbb{Z}}
\DeclareMathOperator{\Mod}{Mod}

\theoremstyle{plain} \numberwithin{equation}{section}
\newtheorem{theorem}{Theorem}[section]
\newtheorem{corollary}[theorem]{Corollary}
\newtheorem{conjecture}{Conjecture}
\newtheorem{lemma}[theorem]{Lemma}
\newtheorem{proposition}[theorem]{Proposition}
\theoremstyle{definition}
\newtheorem{definition}[theorem]{Definition}
\newtheorem{finalremark}[theorem]{Final Remark}
\newtheorem{remark}[theorem]{Remark}
\newtheorem{example}[theorem]{Example}
\newtheorem{question}{Question}

\topmargin-2cm


\textwidth6in

\setlength{\topmargin}{0in} \addtolength{\topmargin}{-\headheight}
\addtolength{\topmargin}{-\headsep}

\setlength{\oddsidemargin}{0in}

\oddsidemargin  0.0in \evensidemargin 0.0in % \parindent0em

\pagestyle{fancy}
\lhead{Teaching Statement} \rhead{October 2017}
\chead{{\large{\bf Paul Gustafson}}} \lfoot{} \rfoot{\bf \thepage} \cfoot{}

\newcounter{list}

\begin{document}

\begin{center}
{\bf TEACHING STATEMENT}\\
\vspace*{0.1cm}
{\normalsize Paul Gustafson (pgustafs@math.tamu.edu)}
\end{center}


My teaching style is interactive.  I find that students learn best when they are actively participating.  When I teach a course, I invite each of my students into an ongoing conversation with me and with their peers.  I strive to promote a classroom inclusive to students of all races, ethnicities, genders, personalities, and socioeconomic and educational backgrounds.

During my five years of grad school at Texas A\&M, I have constantly been learning and gaining experience as a teacher.   I have had the opportunity to teach in a wide variety of settings, from calculus recitations to grading graduate level algebraic topology.  Last spring, I was the instructor of record for Math 131, a 60-student calculus course.  This summer, I mentored  students in an REU.  I bring this experience, as well as a commitment to ongoing learning, to my future classrooms.

When I led my first recitation, I naively believed that knowing how to solve calculus problems was the same as knowing how to teach calculus to engineering freshmen.  I could not have been more wrong.  I overemphasized proofs and logic, confusing the class.  I asked questions, but I did not pause long enough to give students the chance to answer them.  I was nervous and talked too fast.  Over the course of that semester and subsequent semesters at Texas A\&M,  I learned, when teaching calculus to engineering freshmen, to emphasize problem solving and heuristic visual explanations instead of rigorous proofs.  I also learned how to turn a class into a conversation by asking good questions and giving students time to think.  Most importantly, I learned how to enjoy teaching, to slow down my rate of speech, and to build rapport with my students by revealing how exciting math is to me.

More concretely, based on my experience as a student and as a teacher, I have found three factors necessary for good teaching: subject mastery, preparation, and interactivity.   

Subject mastery goes beyond the ability to simply answer questions found in a textbook. My best teachers knew their subjects like the backs of their hands, from the most basic questions to applications beyond the scope of the course.  In my own teaching, I always try to motivate the subject material by connecting it to real-world applications or more advanced mathematics.  For example, students in TAMU’s Math 151 Engineering Calculus class perennially wonder why they have to learn MATLAB.  I have done a fair amount of programming, in both industry and academia, so it’s easy for me to authoritatively say that, regardless of their future jobs, learning how to program and how to do basic data analysis will always be useful. 

Thoroughly preparing for each class is also essential.  My teaching role models employed a variety of teaching media -- blackboards, whiteboards, overhead projectors, slides -- but they all came to class well-prepared.  In my own teaching, I prefer to use Beamer slides for large classes and a whiteboard for small classes.    Beamer slides provide a great outline for a more traditional lecture class, and ensure that I cover all necessary material in an easily digestible manner.  The whiteboard is better for a small class or recitation because it’s more malleable: I can easily go into more depth or do more of a certain type of problem if students need it.  

Lastly, my best classroom experiences, both as a student and as a teacher, were heavily interactive.  A large part of why I pursued a PhD in math was an analysis class in which a professor asked challenging questions almost every class. I looked forward to every class because it was a chance to continue an ongoing, fascinating conversation.

As a teacher, I try to be just as interactive with each of my students. I recently led a review session in a recitation for an upcoming calculus exam.  Review sessions provide a great opportunity for interactivity because the students should have seen the material multiple times already (in lecture, in previous recitations, and in their homework).  I gave them sample problems to work on for a couple minutes each, and then I walked through the solutions to each problem, asking them what to do at each step.  This promotes active participation and makes sure that no one gets stuck for too long.   I encouraged everyone to speak at some point during the class, and I tried to be supportive towards every answer, correct or not.  One student, in particular, took this to heart and was completely unafraid of blurting out an incorrect answer.  This was really helpful: when he said something wrong, other students were quick to come up with alternative solutions, sometimes correct, sometimes not.  Within reason, I pursued their suggestions, pointing out what went right and what went wrong.  This built rapport within the class and fostered an environment where students were unafraid to ask questions.

%% CONCLUSION?
\end{document}
