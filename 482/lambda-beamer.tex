\documentclass{beamer}
\usepackage[latin1]{inputenc}
\usetheme{Warsaw}
\title{The Untyped Lambda Calculus: A Simple Functional Programming Language}
\author{Paul Gustafson}
\institute{Math 482 - Texas A\&M University}
\date{March 5, 2013}
\begin{document}

\begin{frame}
\titlepage
\end{frame}


\begin{frame}{Why is the $\lambda$-calculus important?}
\begin{itemize}
\item Computer Science 
\begin{itemize}
\item Variable binding in function declarations
\item Scope
\item Type sytems
\item Functional programming languages (Lisp, ML variants, Haskell)
\end{itemize}
\item Logic
\begin{itemize}
\item Recursion theory
\item Computability
\end{itemize}
\item Linguistics
\end{itemize}
\end{frame}

\begin{frame}{Why was the $\lambda$-calculus developed?}
\begin{itemize}
\item Formal system of logic developed by Alonzo Church in 1932
\item Used to solve Leibniz'\emph{Entscheidungsproblem} (``Decision problem'')
\begin{itemize}
\item ``Is every statement in first-order logic over a finite set of axioms decidable?''
\item No.
\item Solved independently by Turing.
\end{itemize}
\end{itemize}
\end{frame}

\begin{frame}{How does the $\lambda$-calculus work? (I): $\lambda$-terms}
\begin{itemize}
\item The set of $\lambda$-terms, $\Lambda$, is built from a countable set of variables $V = \{v, v^\prime, v^{\prime\prime}, \ldots\}$:
\begin{enumerate}
\item $x \in V \implies x \in \Lambda$
\item $M,N \in \Lambda \implies (MN) \in \Lambda$
\item $M \in \Lambda, x\in V \implies (\lambda x . M) \in \Lambda$
\end{enumerate}
\item{Examples of $\lambda$-terms}
\begin{itemize}
\item $v^\prime$
\item $(\lambda v. (v^\prime v))$
\item $(((\lambda v. (\lambda v^\prime . (v^\prime v))) v^{\prime\prime}) v^{\prime\prime\prime})$
\end{itemize}
\end{itemize}
\end{frame}

\begin{frame}{How does the $\lambda$-calculus work? (II): Conversion Rules}
\begin{itemize}
\item $\alpha$-conversion: $\lambda x.[...x...] = \lambda y.[...y...]$.  
  \begin{itemize}
  \item ``We can rename bound variables.''
  \item Example: $\lambda a.a = \lambda b.b$
  \end{itemize}
\item $\beta$-conversion: $\lambda x.[...x...] T = [...T...]$. 
  \begin{itemize}
    \item ``Evaluation by substitution.''
    \item Example: $(\lambda x.x)y = y.$
  \end{itemize}
\item $\eta$-conversion: $\lambda x.F(x) = F$. 
  \begin{itemize}
    \item ``Extensionality - a function is defined by what it does.''
    \item Example: $\lambda y.\lambda x.yx =\lambda y.y$
  \end{itemize}
\end{itemize}
\end{frame}

\begin{frame}{Church numerals}
\begin{itemize}
\item A representation of the natural numbers
\item $0 := \lambda f.\lambda x.x$
\item $1 := \lambda f.\lambda x.f x$
\item $2 := \lambda f.\lambda x.f (f x)$
\item $3 := \lambda f.\lambda x.f (f (f x))$
\item $\ldots$
\end{itemize}
\end{frame}

\begin{frame}{Arithmetic with Church numerals}
\begin{itemize}
\item Successor: $\lambda n.\lambda f.\lambda x. f (n f x)$
\item Addition: $\lambda m.\lambda n.\lambda f.\lambda x. m f (n f x)$
\item Multiplication: $\lambda m.\lambda n.\lambda f. m (n f)$
\item Exponentiation: $\lambda m.\lambda n. n m$
\item Predecessor: $\lambda n.\lambda f.\lambda x. n (\lambda g.\lambda h. h (g f)) (\lambda u. x) (\lambda u. u)$
\end{itemize}
\end{frame}

% infinite beta reduction $(\lambda x.xx)(\lambda x.xx)$

\begin{frame}{References}
\begin{itemize}
\item \emph{Introduction to Lambda Calculus.} Barendregt and Barensen.
ftp://ftp.cs.ru.nl/pub/CompMath.Found/lambda.pdf
\item \emph{Lambda Calculus, Then and Now.} Dana S. Scott. http://www.youtube.com/watch?v=7cPtCpyBPNI
\end{itemize}
\end{frame}

\end{document}