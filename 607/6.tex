\documentclass{article}
\usepackage{../m}

\begin{document}
\noindent Paul Gustafson\\
\noindent Texas A\&M University - Math 607\\ 
\noindent Instructor: Thomas Schlumprecht

\subsection*{HW 6}
\p{1} Assume $(X, \mM, \mu)$ is a complete measure space.

a) If $f : X \to \R$ is $\mM$-measurable and $f = g$ $\mu$-a.e., then $g$ is also measurable.

b) If $f_n : X \to \R$ is $\mM$-measurable and $\lim_{n \to \infty} f_n = f$ $\mu$-a.e., then $f$ is measurable.

\begin{proof}
For (a), let $N = \{x : f(x) \neq g(x)\}$. Since $\mM$ is complete, $N \in \mM$.  Let $a \in \R$.  We need to show that $A := g^{-1}((a,\infty))$ is in $\mM$.  To see this, note that $A \cap N^c = f^{-1}((a,\infty)) \cap N^c$, which is in $\mM$.  Hence $A = (A \cap N) \cup (A \cap N^c)$ is in $\mM$ since $A \cap N$ is a null set.

For (b),  let $N = \{x : f(x) \neq \lim_n f_n \} \in \mM$.  Then $f_n \chi_{N^c} \to f \chi_{N^c}$ pointwise. Hence $f \chi_{N^c}$ is measurable by the first exercise of the last homework set.  Thus, by (a), $f$ is measurable.
\end{proof}

\p{2} If $f \in \mL^+$ and $\int f \,d \mu < \infty$, then for any $\epsilon > 0$, there is an $E \in \mM$, so that $\mu(E) < \infty$ and $\int_E f \,d\mu > \int f\,d \mu - \epsilon$.

\begin{proof}
There exists a simple function $0 \le \phi \le f$ with $\int f - \int \phi \le \epsilon$.  Write $\phi$ in standard form as $\phi = \sum_{n=1}^N a_n \chi_{A_n}$, where $a_n \neq 0$.  Since $\phi$ is integrable,  $\mu(A_n) < \infty$ for all $n$.  Thus if $E := \bigcup_n A_n$, then $\mu(E) = \sum_n \mu(A_n) < \infty$. Also, we have $\int f - \int_E f \le \int f - \int_E \phi = \int f - \int \phi \le \epsilon$.
\end{proof}

\p{3} Let $f \in \mL^+(\mu)$. If $\int f \, d\mu < \infty$, then $\mu(\{x \in X : f(x) = \infty \}) = 0$.

\begin{proof}
Let $E := \{x \in X : f(x) = \infty \}$.  Suppose $\mu(E) \neq 0$.  Let $\phi_n = n \chi_E$.  Then $\int f \ge \int \phi_n \to \infty$, a contradiction.
\end{proof}

\p{4} Prove Fatou's Lemma without using the Monotone Convergence Theorem, and deduce the MCT from Fatou's Lemma. Fatou's lemma states that if $(f_n) \subset \mL^+$, then $\int \liminf f_n \le \liminf \int f_n$.
\begin{proof}
\emph{Step 1.} Suppose $\rho$ is a measure on $\mM$, and $(E_n) \subset \mM$. Then $\rho(\liminf E_n) = \rho(\bigcup_n \bigcap_{k \ge n} E_k)  = \lim_n \rho(\bigcap_{k \ge n} E_k)  \le  \lim_n \inf_{k \ge n} \rho(E_k) = \liminf_n \rho(E_n)$.

\emph{Step 2.} Let $\phi$ be a simple function such that $0 \le \phi \le \liminf f_n$.  Let $E_n = \{x : \phi(x) \le f_n(x) \}$.  Then 
\begin{align*}
\liminf E_n & = \bigcup_n \bigcap_{k \ge n} E_k
\\ &  = \{x : \exists n \forall k \ge n \quad \phi(x) \le f_k(x) \} 
\\ &  = \{x : \exists n \quad \phi(x) \le \inf_{k \ge n} f_k(x) \} 
\\ & \supset \{x : \phi(x) \le \sup_n \inf_{k \ge n} f_k(x)\} 
\\ & = \R.
\end{align*}

 By Step 1 applied to the measure $A \mapsto \int_A \phi$, we have $\int \phi = \int_{\liminf E_n} \phi \le \liminf \int_{E_n} \phi \le \liminf \int_{E_n} f_n \le \liminf \int f_n$. Hence $\int \liminf f_n \le \liminf \int f_n$.

For the proof of the MCT, let $(f_n) \subset \mL^+$ with $f_n \uparrow f$. First note that $\lim \int f_n$ must exist since $(\int f_n)$ is an increasing sequence.  Hence $\int f = \int \lim f_n \le \liminf \int f_n = \lim \int f_n$. For the reverse inequality, we have $f_n \le f$ so $\int f_n \le \int f$. Thus, $\lim \int f_n \le \int f$.
\end{proof}

\p{5} Let $f : [0,1] \to [0,1]$ be the Cantor function, and $C$ be the Cantor set.  Define $g(x) = f(x) + x$ for $x \in [0,1]$.

a) $g$ is a bijection from $[0,1]$ to $[0,2]$ and $g^{-1}$ is continuous.

b) $m(g(C)) = 1$.

c) Using Exercise 29/Chapter 1 show that for some nonmeasurable $A \subset g(C)$, $B = g^{-1}(A)$ is Lebesgue measurable but not Borel measurable.

d) There is a Lebesgue measurable function on $\mR$ which is not Borel measurable.

\begin{proof}
The Cantor function $f$ is clearly increasing. To see that it is continous, fix $\epsilon > 0$.  There exists $n$ such that $2^{-n} < \epsilon$.  Let $a, b \in C$ with expansions $a = \sum_j a_j 3^{-j}$ and $b = \sum_j b_j 3^{-j}$.  If $|a - b| < 3^{-n}$, then I claim $a_j = b_j$ for $j \le n$. Suppose not. Let $J < n$ denote the first index such that $a_J \neq b_J$. WLOG $2 = a_J > b_J = 0$.  Then $a - b = \sum_{j = J}^\infty (a_j - b_j) 3^{-j} \ge (2)3^{-J} - \sum_{j = J+1}^\infty (2) 3^{-j}  = (2)3^{-J} - (2)3^{-J - 1} \frac 1 {1 - 1/3} = (2)3^{-J} - (2)3^{-J - 1} \frac 3 2   =  3^{-J} > 3^{-n}$, a contradiction.  Hence, $a_j = b_j$ for $j \le n$, so $|f(a) - f(b)| \le \sum_{j=n+1}^\infty 2^{-j} = 2^{-n} < \epsilon$.  Thus $f$ is continuous.

Since $g$ is the sum of an increasing function and a strictly increasing function, $g$ is strictly increasing.  In particular, $g$ is injective.  Since $f$ is continuous, so is $g$.  Since $g(0) = 0$ and $g(1) = 2$, the intermediate value theorem implies that $g$ surjects onto $[0, 2]$.  

To see that $g^{-1}$ is continuous, suppose $F \subset [0,1]$ is closed.  Then $F$ is compact.  Let $(U_\alpha)_{\alpha \in A}$ be an open cover of $g(F)$. Then $(g^{-1}(U_\alpha))$ is an open cover of $F$. Let $(g^{-1}(U_\alpha))_{\alpha \in G}$ be a finite subcover of $F$.  Then $(U_\alpha)_{\alpha \in G}$ is a subcover of $g(F)$, since $g$ is a bijection.  Thus $g(F)$ is compact, hence closed.  Thus $g^{-1}$ is continuous.

For (b), from the definition of the Cantor set we have $C^c = \bigcup_n I_n$ for some countable collection of disjoint intervals $I_n$ with $\sum_n m(I_n) = 1$.  The construction of $f$ implies that, for each $n$,  $f(I_n) = \{x_n\}$ for some singleton $\{x_n\}$.  Hence $m(g(I_n)) = m(x_n + I_n) = m(I_n)$.  Since $g$ is a bijection, $(g(I_n))_n$ remain pairwise disjoint.  Thus, $m(g(C)) = 2 - m(g(C^c)) = 2 - \sum_n m(g(I_n)) = 2 - \sum_n m(I_n) = 1$.

For (c), Exercise 29/Chapter 1 implies that there exists a nonmeasurable set $A \subset g(C)$. The proof of this exercise is simple (pick an interval $[i, i+1]$ for which $E \cap [i,i+1]$ has positive measure and then do the Vitali construction using $E \cap N_r$ instead of $N_r$).  Since $m(C) = 0$ and the Lebesgue measure is complete, $B: = g^{-1}(A)$ is Lebesgue measurable.   However, if $B$ were Borel measurable, then $A = g(B)$ would be Borel since $g^{-1}$ is continuous hence Borel measurable.  Thus, $B$ is not Borel.

For (d), such a function is $\chi_B$.
\end{proof}

\end{document}
