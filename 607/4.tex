\documentclass{article}
\usepackage{../m}

\nc{\AQ}{\mA^{(\QQ)}}

\begin{document}
\noindent Paul Gustafson\\
\noindent Texas A\&M University - Math 607\\ 
\noindent Instructor: Thomas Schlumprecht

\subsection*{HW 4}
\p{1} Define 
$$ \mA^{(\QQ)}  = \left\{ \bigcup_{i = 1}^n [a_i, b_i) \cap \QQ : \begin{array}{c} n \in \NN, \{a_i, b_i : 1 \le i \le n\} \subset \QQ \cup \{ \pm \infty\}, 
\\ \text{ and } a_1 < b_1 < a_2 < \ldots < b_n \end{array} \right\}.$$

For $A = \bigcup_{i=1}^n [a_i, b_i) \cap \QQ$ with $-\infty \le a_1 < b_1 < a_2 < \ldots < b_n \le \infty$ put 
$$ \mu_0(A) = \sum_{i=1}^n b_i - a_i.$$

a) $\AQ$ is an algebra on $\QQ$ and $\mu_0$ is a finitely additive measure on $\AQ$.

b) Show that $\mu_0$ is not a premeasure.

\begin{proof}
For (a), suppose $E, F \in \AQ$ with $E = \bigcup_{i=1}^n [a_i, b_i) \subset \Q$ and $F = \bigcup_{i=1}^m [c_i, d_i) \subset \Q$ for $a_i, b_i, c_i, d_i \in \R$ for all $i$.  We have $\emptyset \in \AQ$, so to show that $\AQ$ is an algebra, we only need to show that $E^c$ and $E \cup F$ are in $\AQ$. For the former, we have $E^c = [-\infty, a_1) \cup [b_n, \infty) \cup \bigcup_{i=1}^{n-1} [b_i, a_{i+1})$, so $E^c \in \AQ$. 

For the latter, we have $E \cup F = \bigcup_{i=1}^{n+m} [e_i, f_i)$ for  $([e_i, f_i))_i$ a reordering of the concatenation of $([a_i, b_i))$ and $([c_i, d_i))$ such that $e_1 \le e_2 \le \ldots \le e_{n+m}$.  Suppose $f_i > e_{i+1}$ for some $i$.  Then $[e_i, f_i) \cup [e_{i+1}, f_{i+1}) = [e_i, f_{i+1})$.  Hence, $E \cap F = \bigcup_{i=1}^{n+m-1} [e_j', f_j')$ where $[e_j', f_j') = ([e_j, f_j))$  for $j < i$, $[e_i', f_i') =  [e_i, f_{i+1})$, and $[e_j', f_j') = [e_{j+1}, f_{j+1}$ for $j > i$.  Then $e_1' \le e_2' \le \ldots \le e_{n+m-1}'$.  We can continue this process until we get $E \cup F = \bigcup_{i=1}^l [g_i, h_i)$ for some $l$, with $g_i \le g_{i+1}$ for all $i$ and $h_i \le g_{i+1}$ for all $i$. Note that $g_i \le h_i$ by construction. This implies that $E \cup F \in \AQ$.

To see that $\mu_0$ is finitely additive on $\AQ$, we need to show that $\mu_0(E \cup F) = \mu_0(E) + \mu_0(F)$ if $E,F$ are disjoint. Using the same notation as above, we have $E \cup F = \bigcup_{i=1}^{n+m} [e_i, f_i)$ for  $([e_i, f_i))_i$ a reordering of the concatenation of $([a_i, b_i))$ and $([c_i, d_i))$ such that $e_1 \le e_2 \le \ldots \le e_{n+m}$.  If $f_i > e_{i+1}$ for some $i$, then we contradict $b_j \le a_{j+1}$, $d_j \le c_{j+1}$, or the disjointness of $E$ and $F$.  Hence $e_i \le f_i$ and $f_i \le e_{i+1}$ for all $i$, so $\mu_0(E \cup F) = \sum_{i=1}^{n+m} f_i - e_i = \sum_{i=1}^n b_i - a_i + \sum_{i=1}^m d_i - c_i = \mu_0(E \cup F)$.

For (b), suppose $\mu_0$ is a premeasure.  It extends to a measure $\mu$ on $\mM(\mA)$. Let $q \in \Q$.  Pick any real-valued sequences $a_n \uparrow q$ and $b_n \downarrow q$.  Then $q = \bigcap_n (a_n, b_n] \cap \Q$ and $\mu(b_1 - a_1) < \infty$, so $\mu(q) = \lim_{n\to\infty} b_n - a_n = 0$. Since every element in $\mA$ is the union of its countably many rational elements, this implies that every element of $\mA$ has measure 0, a contradiction.
\end{proof}

\p{2} Let $d \in \NN$ and 
$$\mE = \left\{ \prod_{i=1}^d [a_i, b_i) : -\infty \le a_i \le b_i \le \infty \text{ for } i = 1,2, \ldots n \right\}.$$
(if $a_i = \infty$, replace $[a_i, b_i)$ with $(a_i, b_i)$). Let $\mA$ be the algebra generated by $\mE$.

a) Show that 
$$\mA = \left\{ \bigcup_{i=1}^n E_i : n\in \NN, E_i \in \mE \text{ are pairwise disjoint } \right\}.$$

b) Show that there is a measure $\mu$ on $\mM(\mA)$ so that 
$$\mu(\prod_{i=1}^d [a_i, b_i)) = \prod_{j=1}^d (b_i - a_i) \text{ whenever } -\infty \le a_i \le b_i \le \infty \text{ for } i = 1,2, \ldots, n.$$

\begin{proof}
For (a), let $\mB = \left\{ \bigcup_{i=1}^n E_i : n\in \NN, E_i \in \mE \text{ are pairwise disjoint } \right\}$.  Clearly $\mB \subset \mA$, so it suffices to show that $\mB$ is an algebra. Since $\mE$ is nonempty, $\mB$ must be nonempty.  

To see that $\mB$ is closed under taking finite intersections,  let $B,C \in \mB$. Then $B = \bigcup_{i=1}^m B_i$ for some $m \in \N$ and disjoint $(B_i) \subset \mE$, and $C = \bigcup_{i=1}^n C_i$ for some $n \in \N$ and disjoint $(C_i) \subset \mE$. Then $B \cap C = \bigcup_{i,j} B_i \cap C_j$. To see that the sets $(B_i \cap C_j)_{i,j}$ are disjoint, suppose $(i,j) \neq (i',j')$.  WLOG $i \neq i'$. Then $(B_i \cap C_j) \cap (B_{i'} \cap C_{j'}) = (B_i \cap B_{i'}) \cap (C_j \cap C_{j'}) = \emptyset$ since the $(B_i)$ are disjoint.  Hence $(B_i \cap C_j)_{ij}$ are disjoint, so it suffices to break an arbitrary $B_i \cap C_j$ into disjoint elements of $\mE$. 

Write $B_i = \prod_{i=1}^d [a_i, b_i)$ and $C_i = \prod_{i=1}^d [c_i, d_i)$.  Then $B_i \cap C_i = \prod_{i=1}^d [a_i, b_i) \cap [c_i, d_i)$.  For each $i$, we have $[a_i, b_i) \cap [c_i, d_i) = [e_i, f_i)$ for some $-\infty \le e_i \le f_i \le \infty$ by case analysis on the order of $a_i, b_i, c_i, d_i$.  Hence, $B_i \cap C_i \in \mE$.

To see that $\mB$ is closed under taking complements, let $B \in \mB$. Then $B = \bigcup_{i=1}^n E_i$ for $E_i \in \mE$, and $B^c = \bigcap_i E_i^c$.  Since we know that $\mB$ is closed under finite intersections, it suffices to show that each $E_i^c \in \mB$.  Writing $E_i$ as $E_i = \prod_{j=1}^d [a_j, b_j)$, let $\mathcal U = \{\prod_{j=1}^d U_j : \forall j \quad U_j \in \{ (-\infty, a_j), [a_j, b_j), [b_j,\infty) \} \} \subset \mE$.  Then $\mathcal U$ is a  finite partition of $\R^d$, and $E_i^c = \bigcup (\mU \setminus E)$.  Hence $E_i^c \in \mB$.

For (b), first define $V:\mE \to [0, \infty]$ by $V( \prod_{i=1}^n [a_i, b_i) = \prod_{i=1}^n b_i - a_i$. 
By the extension of premeasures theorem, it suffices to show that $\mu_0: \mA \to [0, \infty]$ defined by $\mu_0(A) = \sum_{i=1}^n V(E_i)$, where $(E_i)_{i=1}^n$ is a $\mE$-partition of $A$, is a premeasure.

\emph{Step 1: $\mu_0$ is well-defined.} I first show that if $E \in \mE$ and $\mP \subset \mE$ is a partition of $E$, then $V(E) = \sum_{P \in \mP} V(P)$.
For $1 \le j \le d$, let $a_{j1} < a_{j2} < \ldots < a_{j m_j}$ enumerate all the endpoints of intervals that appear in the $j$-th coordinate of an element of $\mP$.  Define $\mU = \{ \prod_{i=1}^d U_i : \forall i \exists j \quad U_i = [a_{j,i}, a_{j,i+1}) \}$.  Then $\mU$ is a refinement of $\mP$.

Fix $P = \prod_{i=1}^d [p_i, p_i') \in \mP$.  The family $\mQ := \{Q \in \mP : Q \cap P \neq \emptyset \}$ is a partition of $P$. I claim that $V(P) = \sum_{Q \in \mQ} V(Q)$. If $Q = \prod_i [q_i, q_i') \in \mQ$

%$E = \prod_{i=1}^d [a_i, b_i) = \bigcup_{j=1}^J \prod_{i=1}^d [c_{ji}, d_{ji})$ then $\prod_{i=1}^d b_i - a_i = \sum_{j=1}^J \prod_{i=1}^d d_{ji} - c_{ji}$.  
  



suppose $A = \bigcup_{i=1}^I E_i = \bigcup_{j=1}^J F_i$ for disjoint $(E_i = \prod_{k=1}^d [a_{ik}, b_{ik}))_i \subset\mE$ and disjoint $((F_j = \prod_{k=1}^d [c_{jk}, d_{jk})) \subset \mE$. We need to show that $\sum_{i=1}^I \prod_{k=1}^d b_{ik} - a_{ik} = \sum_{j=1}^J \prod_{k=1}^d d_{jk} - c_{jk}$.

Note that $(E_i \cap F_k)_{i,k}$ is a partition of $A$. Let $(D_l)_{l=1}^L$ enumerate this partition with the empty sets removed.  The intersection of two boxes is a box, so we can write each $D_{l}$ as $\prod_{k=1}^d (e_{lk}, j_{lk})$



From the definition of $\mu_0$, we have $\mu_0(\emptyset) = 0$.

To check countable additivity, let $(A_n) \subset \mA$ be disjoint with $A := \bigcup_n A_n \in \mA$.  Since $A \in \mA$, we have  $A = \bigcup_{i=1}^m F_i$ for disjoint $(F_i) \subset \mE$.  Replacing each $A_n$ with $(A_n \cap F_i)_{i=1}^m$ and removing the empty sets, WLOG each $A_n$ intersects one $F_i$. By the definition of $\mu_0$, we have $\mu_0(A) = \sum_i \mu_0(F_i)$.  Hence, to show that $\sum_n \mu_0(A_n) = \mu_0(A)$, it suffices to show that $\sum_{A_n \cap F_i \neq \emptyset) \mu_0(A_n) = \mu_0(F_i)$ for each $i$.  Fix an $F_i$. By replacing $(A_n)$ with its subsequence intersecting $F_i$ and replacing $A$ with $F_i$,  WLOG $A \in \mE$. Since each $A_n = \bigcup_{l=1}^L E_l$ for disjoint $(E_l) \subset E$, by replacing each $A_n$ with $(E_l)_{l=1}^L$ we may assume each $A_n \in \mE$.

%Check finite additivity first?

\emph{Step 2: $\mu_0(A) \le \sum_n \mu_0(A_n)$.} 

%Might have to show that each B_n is bounded (easy)
\emph{Case 1: $A$ is bounded.} Since $A$ is bounded, each $A_n$ must be bounded. Let $\epsilon > 0$.  There is an open box $B_n$ of volume $\mu_0(A_n) + 2^{-n} \epsilon$ containing each $A_n$.  These boxes cover $\overline {A}$, so by the Heine-Borel theorem, there exists a finite set $F$ such that $(B_n)_{n\in F}$ covers $\overline A$. We can write each $B_n$ as $B_n = \prod_{i=1}^d (a_{ni}, b_{ni})$ for some $-\infty < a_i < b_i < \infty$.  Let $\mU = \{\prod_{j=1}^d (c_j, d_j] :  c_j < d_j, c_j = a_{ni}, d_j = b_{n'i'} \text{ for some } n,n',i, i' \}$.   Then $\mU$

\emph{Case 2: $A$ is unbounded.} 


\emph{Step 3:  $\sum_n \mu_0(A_n) \le \mu_0(A)$.} It suffices to show that $\sum_{n=1}^N \mu(A_n) \le \mu(A)$ for all $N$.  T





\end{proof}

\p{3} Let $\mu$ be a finite measure on $\mB_\R$. Show that for all $\epsilon > 0$ and all $A \in \mB_\R$, there is an open set $U$ and a closed set $F$ so that $F \subset A \subset U$ and $\mu(U \setminus F) < \epsilon$.  Prove this by showing that 
$$\widetilde \mM := \{ A \in \mB_\R : \forall \epsilon > 0 \exists U \text{ open } \exists C \text{ closed} \quad C \subset A \subset U \text{ and } \mu(U\setminus C) < \epsilon\}$$
is a $\sigma$-algebra.
\begin{proof}
I first show that $\wt \mM$ contains the open sets in $\R$. Let $U \subset \R$ be open, and $\epsilon > 0$.  We have $U = \bigcup_{n=1}^\infty I_n$ for disjoint open intervals $I_n$.  Thus $\mu(U) = \sum_n \mu(I_n)$. Since $\mu(U) < \infty$, we can pick $N$ such that $\mu(U) - \sum_{n=1}^N \mu(I_n) < \epsilon/2$. For each open interval $I_n$ we can pick an ascending sequence $(F_m)$ of closed intervals such that $\bigcup_m F_m = I_n$. Hence $\mu(I_n) = \lim_m \mu(F_m)$, so there exists $C_n \in (F_m)$ such that $\mu(I_n \setminus C_n) < \epsilon/(2N)$.  Hence $C = \bigcup_{n=1}^N C_n$ is closed, and $\mu(U \setminus C) = \mu(U \setminus \bigcap_{n=1}^N I_n) + \mu(\bigcap_{n=1}^N I_n \setminus C_n) < \epsilon/2 + N(\epsilon/(2N) = \epsilon$.

Thus $\wt \mM$ contains all the open sets in $\R$, so it suffices to show that $\wt \mM$ is a $\sigma$-algebra. Clearly $\wt \mM$ is nonempty.  Suppose $M \in \wt \mM$.  Let $\epsilon > 0$. There exist $F$ closed and $U$ open such that $F \subset M \subset U$ and $\mu( U \setminus F) < \epsilon$.  We have $U^c$ closed and $F^c$ open with $U^c \subset M^c \subset F^c$ and $\mu(F^c \setminus U^c) = \mu(F^c \cap U) = \mu(U \setminus F) < \epsilon$, so $M^c \in \wt \mM$.

For closure under countable unions, let $(M_n) \subset \wt \mM$. Let $\epsilon > 0$, $M = \bigcup_{n=1}^\infty M_n$ and $S_N = \bigcup_{n=1}^N M_n$.  Then $\mu(M) = \lim_N \mu(S_N)$, so we can pick $N$ such that $\mu(\bigcup_n M_n) - \mu(S_N) < \epsilon$.  For each $n\in\N$, pick closed $F_n$ and open $U_n$ such that $F_n \subset M_n \subset U_n$ and $\mu(U_n \setminus F_n) < 2^{-n} \epsilon$.  Let $U = \bigcup_{n=1}^\infty U_n$ and $F = \bigcup_{n=1}^N F_n$. Then $U$ is open and $F$ is closed with $F \subset S_N \subset M \subset U$.  Moreover, $\mu(U \setminus F) \le \mu(\bigcup_{n=N+1}^\infty U_n) + \mu((\bigcup_{n=1}^N U_n) \setminus \bigcup_{n=1}^N F_n) < 2^{-N} \epsilon + \mu(\bigcup_{n=1}^N U_n \setminus F_n) < 2^{-N} \epsilon + \epsilon < 2 \epsilon$.
\end{proof}

\p{4} If $E \in \mL$ (the Lebesgue sets) and $m(E) > 0$ then there is for any $\alpha < 1$ and open interval $I$ such that $m(E \cap I) > \alpha m(I)$.
\begin{proof}
\end{proof} 
Let $\epsilon > 0$. Since $m$ is outer regular, we can pick an open set $U \supset E$ with $m(U \setminus E) < \epsilon$. We can write $U = \bigcup_{n=1}^\infty I_n$ for disjoint open intervals $I_n$.  Suppose $m(E \cap I_n) \le \alpha \,m(I_n)$ for all $n$.  Then $m(E) = m(\bigcup_n E \cap I_n) = 
\sum_n E \cap I_n \le \alpha \sum_n I_n = \alpha \m(U) = \alpha(m(E) + m(U\setminus E)) < \alpha(m(E) + \epsilon)$. Letting $\epsilon \to 0$, we have $m(E) \le \alpha \,m(E)$, a contradiction. Hence, $m(E \cap I_n) > \alpha \, m(I_n)$ for some $n$.
\end{document}
