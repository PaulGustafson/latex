\documentclass{article}
\usepackage{../m}

\begin{document}
\noindent Paul Gustafson\\
\noindent Texas A\&M University - Math 607\\ 
\noindent Instructor: Thomas Schlumprecht

\subsection*{HW 5}
\p{1} Let $(X, \mM)$ be a measurable space and $f_n : X \to \R$ be measurable for $n \in \N$. Show that 

a) $\liminf_{n \to \infty} f_n$ is measurable.

b) $\{x \in X : \lim_{n \to \infty} f_n \text{ exists } \} \in \mM$

\begin{proof}
For (a), I first show that if $g_n : X \to \R$ are measurable for $n \in \N$, then $g := \sup_n g_n$ is measurable.  
We have
\begin{align*}
x \in g^{-1}((a, \infty)) & \iff g(x) > a
\\ & \iff \exists n \quad g_n(x) > a
\\ & \iff \exists n \quad x \in g_n^{-1}((a,\infty))
\\ & \iff x \in \bigcup_n g_n^{-1}((a, \infty)),
\end{align*}
where the second equivalence follows from the fact that if $g_n(x) \le a$ for all $n$, then $\sup_n g_n(x) \le a$.
Thus $g^{-1}((a, \infty)) = \bigcup_n g_n^{-1}((a, \infty))$, so $g$ is measurable.  Moreover note that, under the same conditions, $\inf_n g_n = - \sup_n -g_n$, so $\inf_n g_n$ is measurable also.

Hence $\inf_{k \ge n} f_k$ is measurable for each $k$, so $\liminf_{n\to \infty} f_n = \sup_n \inf_{k \ge n} f_k$ is measurable.

For (b), first note that the same argument applies to $\limsup_{n \to \infty} f_n$.  Thus $h := \liminf_{n \to \infty} f_n - \limsup_{n \to \infty} f_n$ is measurable.  Thus $h^{-1}(0) \in \mM$, and this is precisely the set of points where the limit exists.
\end{proof}

\p{2} Let $(X, \mM)$ and $(Y, \mN)$ be measure spaces. Assume that $\mu$ is a measure on $(X, \mM)$ and that $\phi: X \to Y$ is $(\mM, \mN)$ measurable. Then 
$$ \mu_\phi : \mN \to [0, \infty], \quad A \mapsto \mu(\phi^{-1}(A))$$
is a measure on $(Y, \mN)$. It is called the image of $\mu$ under $\phi$.
\begin{proof}
We need to show that $\mu_\phi(\emptyset) = 0$ and that $\mu_\phi$ is countably additive. For the former, $\mu_\phi(\emptyset) = \mu(\phi^{-1}(\emptyset)) = \mu(\emptyset) = 0$.  For the latter, suppose $(A_n)$ is a countable collection of pairwise disjoint sets.  Then $\mu(\phi^{-1}(\bigcup_n A_n)) = \mu(\bigcup_n \phi^{-1} (A_n)) = \sum_n \mu(\phi^{-1}(A_n)) = \sum_n \mu_\phi(A_n)$, where the second equality follows from the fact that $\phi^{-1}(A_j) \cap \phi^{-1}(A_k) = \phi^{-1}(A_j \cap A_k) = \emptyset$ for all $j \neq k$.
\end{proof}

\p{3} Let $E \in \mL$ with $m(E) > 0$. Then the set $E - E$ contains an open interval centered at 0.
\begin{proof}
By the inner regularity of $m$, there exists a compact $K \subset E$ with $m(K) > 0$. If $K - K$ contains an open interval, so does $E - E$. Hence WLOG $E$ is compact.

Suppose that $E - E$ does not contain an open interval centered at 0. Then there exists a sequence $x_n \to 0$ such that $x_n \not\in E - E$ for all $n$.  Hence $E + x_n$ is disjoint from $E$ for all $n$.

By the outer regularity of $m$, there exists open $U \supset E$ with $m(U) < 2 \,m(E)$.  

Since $E$ is compact and $U^c$ is closed,  I claim that $d(E, U^c) := \inf\{|x - y| : x \in E, y \in U^c\} > 0$. Suppose $d(E, U^c) = 0$. Then there exists a sequence $(e_n) \subset E$ with $d(e_n, U) \to 0$.  Since $E$ is compact, by passing to a subsequence we may assume $e_n \to e$ for some $e \in E$.  But then $d(e, U) = 0$, so $e$ is a limit point of $U$.  Hence $e \in U$ since $U$ is closed. This contradicts the disjointness of $E$ and $U^c$. Hence $d(E, U^c) > 0$.

Thus there exists $x_n$ such that $E + x_n \subset U$.  But then $m(U) \ge m(E \cup (E + x_n)) = m(E) + m(E + x_n) = 2 \m(E) > m(U)$, a contradiction.
\end{proof}

\p{4} Let $(X , \mM, \mu)$ be a measure space. We call $A \in \mM$ an atom if $\mu(A) > 0$ and if $A = A_1 \cup A_2$ for $A_1, A_2 \in \mM$ disjoint implies that $\mu(A_1) = 0$ or $\mu(A_2) = 0$.

Assume now that $(X, \mM, \mu)$ is an atom-free measure space with $\mu(X) = 1$. Then there is for any $0 \le r \le 1$ an $A \in \mM$ with $\mu(A) = r$. Hint: first show that there is a measurable set whose measure is between 1/3 and 1/2. Secondly show that there is a disjoint sequence of measurable sets $(B_n)$ with $\mu(B_n) = 2^{-n}$. Write $r$ as $r = \sum_{n=1}^\infty r_n2^{-n}$ with $r_n \in \{0, 1\}$.  Therefore $\mu(\bigcup_{r_n=1} B_n = r$.

\begin{proof}
Let $Y \subset X$ with $m(Y) > 0$. Since $X$ is atom-free, there must exist disjoint sets $Y_0$ and $Y_1$ with $Y = Y_0 \cup Y_1$ and $\mu(Y_1) \ge \mu(Y_0) > 0$.  Of all such $Y_0$ and $Y_1$, pick a pair such that $\mu(Y_1) - \mu(Y_0)$ is minimized. 

I claim that $\mu(Y_1) = \mu(Y_0) = m(Y)/2$.  Suppose not.  By the atom-free assumption, there exists a nonempty collection $\mU := \{ U \subset Y_1 : \mu(U) > 0 \}$. 

 I claim that $B := \inf_{U \in \mU} \mu(U) = 0$.  Suppose not.  Pick $U \in \mU$ with $\m(U) < 2 B$.  Then $U = V \cup W$ for disjoint $V,W$ with $\mu(V) \ge \mu(W) > 0$.  But then $\mu(W) \le \mu(U)/2 < B$, a contradiction.  Hence, $B = 0$.  

Thus there exists $U \in \mU$ with $\mu(U) < (\mu(Y_1) - \mu(Y_0))/2$.  Then $Y_1 \setminus U, Y_0 \cup U$ is a partition of $Y$.  Moreover $\m(Y_1 \setminus U) - \m(Y_0 \cup U) = \mu(Y_1) - \mu(Y_0) - 2\mu(U) > 0$, so $\m(Y_1 \setminus U) \ge \m(Y_0 \cup U)$ and $\m(Y_1 \setminus U) - \m(Y_0 \cup U) < \mu(Y_1) - \mu(Y_2)$.  This contradicts the minimality of $\mu(Y_1) - \mu(Y_0)$. 

In summary, we have proved that if $Y \subset X$ with $m(Y) > 0$ there exist disjoint sets $Y_0$ and $Y_1$ with $Y = Y_0 \cup Y_1$ and $\mu(Y_0) = \mu(Y_1)  = \m(Y)/2$.

Applying this lemma to $X$, we get $X = X_1 \cup B_1$ for disjoint $X_1$ and $B_1$ with $\mu(X_1) = \mu(B_1) = 1/2$.    Apply the lemma to $X_1$, to get $X_1 = X_2 \cup B_2$ for disjoint $X_2$ and $B_2$ with $\mu(X_2) = \mu(B_2) = 1/2$. Continuing in this way, we get a sequence $(B_n)$ of pairwise disjoint sets such that $m(B_n) = 2^{-n}$ for each $n$. Following the hint, write $r$ as $r = \sum_{n=1}^\infty r_n2^{-n}$ with $r_n \in \{0, 1\}$.  Then we have $\mu(\bigcup_{r_n=1} B_n) = r$.

\end{proof}

\p{5} Show that there is a measurable set $A \subset [0,1]$ such that $0 < m(A\cap I) < m(I)$ for all nondegenerate intervals $I$.
\begin{proof}
Let $(U_n)$ be a countable base for the topology of $[0,1]$ consisting of bounded open intervals.  For example, take all balls centered at the rationals of rational radius. Let $A_n \subset U_n$ be a fat Cantor set of measure $m(A_n)/2$. Each $A_n$ is closed and nowhere dense, so by the Baire Category Theorem $A := \bigcup_n A_n$ is nowhere dense.  

Let $I$ be a nondegenerate interval.  Since $(U_n)$ is a base, there exists $A_n \subset U_n \subset I$.  Hence $m(A \cap I) > m(A_n) > 0$.   On the other hand, since $A$ is nowhere dense, $(\overline A)^c$ is open and dense.  Hence $I \cap (\overline A)^c$ is a nonempty open set, so contains some nondegenerate open interval $J$.  Thus $J \subset I \cap A^c$, so $\m(A \cap I) =  m(I)  - \m(A^c \cap I) \le m(I) - m(J) < m(I)$.
\end{proof}

\end{document}
