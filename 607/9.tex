\documentclass{article}
\usepackage{../m}

\begin{document}
\noindent Paul Gustafson\\
\noindent Texas A\&M University - Math 607\\ 
\noindent Instructor: Thomas Schlumprecht

\subsection*{HW 9}
\p{1} If $f \in L_1(0,\infty)$, define
$$g(s) = \int_0^\infty e^{-st}f(t) \,dt, \quad 0 < s < \infty.$$
Prove that $g(s)$ is differentiable on $(0,\infty)$ and that
$$g'(s) = - \infty_0^\infty t e^{-st} f(t), \quad 0 < s < \infty.$$
\begin{proof}
\end{proof}

\p{2} Let $(\Omega, \mu, \Sigma)$ be a finite measure space and $(f_n)$ be a sequence of measurable functions on $\Omega$. Suppose that for each $\omega \in \Omega$ there is an $M_\omega \in \R$ so that for all $k \in \N$, $|f_k(\omega) \le M_\omega$. Let $\epsilon > 0$. Show that there is a measurable $A \subset \Omega$ and an $M \in \R$ so that $\mu(\Omega \setminus A) < \epsilon$ and $f_k(\omega) < M$ for all $k \in \N$ and all $\omega \in A$.

\p{3} 57/page 77. Show that $\int_0\infty e^{-sx}x^{-1} \sin x \, dx = \arctan(s^{-1})$ for $s > 0$ by integrating $e^{-sxy}\sin x$ with respect to $x$ and $y$.  (Hints: $\tan(\frac \pi 2) = (\tan \theta)^{-1}$ and Exercise 31d.)

\p{4} 60/page 77. $\Gamma(x) \Gamma(y)/ \Gamma(x + y) = \int_0^1 t^{x-1}(1-t)^{y-1} \, dt$. (Recall that $\Gamma$ was defined in Section 2.3. Write $\Gamma(x) \Gamma(y)$ as a double integral and use the argument of the exponential as a new variable of integration.)

\p{5} Given a bounded function $f : [a,b] \to \R$, define
$$ H(x) = \lim_{\delta \to 0} \sup_{|x - y| \le \delta} f(y), \text{ and } h(x) = \lim_{\delta \to 0} \inf_{|x - y| \le \delta} f(y)$$
\begin{enumerate}[\bfseries a)]
\item For $x \in [a,b]$, $f$ continuous at $x$ $\iff$ $H(x) = h(x)$.
\item Assume now that $(P_k)$ is an increasing sequence of partitions of $[a,b]$ for which the mesh converges to zero. Write $P_k = (t_0^{(k)}, t_1^{(k)}, \ldots, t_{n_k}^{(k)})$. Define for $x \in [a,b]$,
$$G(x) = \lim_{k \to \infty} G_{P_k}(x) \text{ and } g(x) = \lim_{k \to \infty} g_{P_k}(x),$$
where for a partition $P = (t_0, t_1, \ldots, t_n)$
$$G_P = \sum_{i=1}^n \chi_{(t_{i-1}, t_i]} \sup_{t \in (t_{i-1}, t_i]} f(t) \text{ and } g_P = \sum_{i=1}^n \chi_{(t_{i-1}, t_i]} \inf_{t \in (t_{i-1}, t_i]} f(t).$$
Prove that $H = G$ and $h = g$ $m$-a.e.
\item Show that $f$ is Riemann integrable $\iff$ the set of discontinuities of $f$ has Lebesgue measure zero.
\end{enumerate}

\p{6} Problem 30/page 60. Hint: AM-GM. Show that $\lim_{k \to \infty} \int_0^k x^n (1 - k^{-1} x) ^k \, dx = n!$.

\p 7 Problem 1/88. Let $\nu$ be a signed measure on $(X, \mM)$. If $(E_j)$ is an increasing sequence in $\mM$, the $\nu(\bigcup_1^\infty E_j) = \lim_{j \to \infty} \nu(E_j)$. If $(E_j)$ is a decreasing sequence in $\mM$ and $\nu(E_1)$ is finite, then $\nu(\bigcap_1^\infty E_j) = \lim_{j \to \infty} \nu(E_j)$.

\p 8 Problem 4/88. If $\nu$ is a signed measure and $\lambda, \mu$ are positive measures such that $\nu = \lambda - \mu$, then $\lambda \ge \nu^+$ and $\mu \ge \nu^-$.

\p 9 Problem 7/88. Suppose that $\nu$ is a signed measure on $(X, \mM)$ and $E \in \mM$.
\begin{enumerate}[\bfseries a.]

\item $\nu^+(E) = \sup\{\nu(F) : F \in \mM, F \subset E\}$ and $\nu^-(E) = - \inf\{\nu(F) : F \in \mM, F \subset E\}$.

\item $|\nu|(E) = \sup\{\sum_1^n |\nu(E_j)| : n \in \N, E_1, \ldots, E_n \text{ are disjoint, and } \bigcup_1^n E_j = E\}$.
\end{enumerate}


\end{document}
