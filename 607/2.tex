\documentclass{article}
\usepackage{../m}

\begin{document}
\noindent Paul Gustafson\\
\noindent Texas A\&M University - Math 607\\ 

\subsection*{HW 2}
\p{1} Using the fact that $\mathcal B_{\R}$ is generated by the open intervals, show that:
$$ \mathcal B_\R = \mathcal M (\{ [a, \infty ) : a \text { rational } \} ) $$
\begin{proof}
It suffices to show both that $\mathcal B_\R$ contains $[a, \infty )$ for each $a \in \Q$, and that every open interval $(x,y)$ is in $\mathcal M (\{ [a, \infty ) : a \text { rational } \} )$. The former is obvious since $[a, \infty) = (\-infty, a)^c$ for each $a \in \Q$. 

For the latter, suppose $(x,y)$ is an arbitrary open interval. Pick $(x_n), (y_n) \subset \Q$ with $x_n \searrow x$ and $y_n \nearrow y$. Then $(x,y) = \bigcup_n (x_n, y_n)$.
\end{proof}

\p{2} Problem 1/Page 24. A \emph{ring} is a nonempty family of sets closed under finite unions and differences. A ring that is closed under countable unions is called a \emph{$\sigma$-ring}.

a. Rings (resp. $\sigma$-rings)  are closed under finite (resp. countable) intersections.

b. If $\mathcal R$ is a ring (resp. $\sigma$-ring) , then $R$ is an algebra (resp. $\sigma$-algebra) iff $X \in \mathcal R$.

c. If $\mathcal R$ is a $\sigma$-ring, then $\{ E \subset X : E \in \mathcal R \text{ or } E^c \in \mathcal R \}$ is a $\sigma$-algebra.

d. If $\mathcal R$ is a $\sigma$-ring, then $\{ E \subset X : E \cap F \in \mathcal R \text{ for all } F \in \mathcal R \}$ is a $\sigma$-algebra.

\begin{proof}
For (a), let $\mathcal R$ be a ring, and $U, V \in \mathcal R$.  Let $W = U \cup V$.  Then $U \cap V = W \setminus ((W \setminus U) \cup (W \setminus V)) $. This is just one of De Morgan's laws in the restricted universe $W$.  A similar argument works for $\sigma$-rings with $W$ the countable union of the sets involved.

For (b), let $\mathcal R$ be a ring (resp. $\sigma$-ring).  Suppose $X \in \mathcal A$. Since (a) has been verified, we need only check that $R$ contains complements.  This is true since $E^c = X \setminus E$ for any set $E$.  Conversely, suppose $\mathcal R$ is an algebra (resp. $\sigma$-algebra).  Then $\mathcal R$ is nonempty, so there exists $E \in \mathcal R$. Thus, $X = E \cup E^c \in \mathcal R$.

For (c), let $\mathcal M = \{ E \subset X : E \in \mathcal R \text{ or } E^c \in \mathcal R \}$.  Since $\mathcal R$ is nonempty, so is $\mathcal M$.  It is also clear that $\mathcal M$ is closed under complements. For closure under countable unions, let $(E_n) \subset \mathcal M$.  Then $(E_n) = (A_n) \cup (B_n)$ for sequences $(A_n), (B_n)$ such that each $A_n \in \mathcal R$ and each $B_n^c \in \mathcal R$. Let $A = \bigcup A_n \in mathcal R$ and $B = \bigcap B_n^c \in \mathcal R$.  Then $\bigcup_n E_n = \bigcup_n A_n \cup \bigcup B_n = A \cup B^c = (A^c \cap B)^c = (B \setminus A)^c \in \mathcal M$. 

For (d), let $\mathcal M = \{ E \subset X : E \cap F \in \mathcal R \text{ for all } F \in \mathcal R \}$. Since $\mathcal R$ is nonempty, there exists $E \in \mathcal R$.  Hence $\emptyset = E \setminus E \in \mathcal R$. Then it follows from the definition of $\mathcal M$ that $\emptyset \in \mathcal M$. In particular, $\mathcal M$ is nonempty.  To see that $\mathcal M$ is closed under complements, suppose $E \in \mathcal M$.  Let $F \in \mathcal R$.  Then $E^c \cap F = F \setminus E \in \mathcal R$.  Hence, $E^c \in \mathcal M$.  For closure under countable unions, let $(E_n) \subset \mathcal M$. Let $F \in mathcal R$.  Then $\bigcup_n(E_n) \cap F = \bigcup_n (E_n \cap F)  \in \mathcal R$.  Hence, $\bigcup_n E_n \in \mathcal M$.
\end{proof}


\p{3} Problem 5/Page 24. $\mathcal M( \mathcal E)$ is the union of the $\sigma$-algebras generated by $\mathcal F$ as $\mathcal F$ ranges over all countable subsets of $\mathcal E$.
\begin{proof}
Let $$\mathcal H =  \{ \mathcal M ( \mathcal F ) : \mathcal F \subset \mathcal E \text{ and } \mathcal F \text{ is countable} \},$$ and $\mathcal U = \bigcup \mathcal H$.


Let $\mathcal F \subset \mathcal E$ be countable. Then $\mathcal M (\mathcal F) \subset \mathcal M (\mathcal E))$.   Hence, $\mathcal U \subset \mathcal M ( \mathcal E))$. For the reverse inclusion, it suffices to show that $\mathcal U$ is a $\sigma$-algebra, for then $ \mathcal U$ is a $\sigma$-algebra containing $\mathcal(E)$, hence containing $\mathcal M(\mathcal E)$.  

To see that $\mathcal U$ is a $\sigma$-algebra, first note that $\mathcal(\emptyset) \in \mathcal H$, so $\mathcal U$ is nonempty. To see that $\mathcal U$ is closed under taking complements, let $E \in \mathcal U$.  Then $E \in \mathcal M ( \mathcal F )$ for some countable $\mathcal F \subset \mathcal E$, so $E^c \in \mathcal M ( \mathcal F) \subset \mathcal U$.  

For closure under countable union, let $(U_n) \subset \mathcal U$. Then each $U_n \in \mathcal M ( \mathcal F_n )$ for some countable $\mathcal F_n \subset \mathcal E$. Let $(F_{nm})_m$ be an enumeration of $\mathcal F_n$, and Then $U_n = $

\end{proof}


\p{4} Show that every $\sigma$-algebra has either finite or uncountable many elements.
\begin{proof}
Suppose that $\mathcal M \subset \mathcal P (X)$ is an infinite $\sigma$-algebra. 

\emph{Step 1: $\mathcal M$ countains a sequence of disjoint nonempty sets.}

\emph{Case 1: $\mathcal M$ contains an infinite linearly inclusion-ordered subset $\mathcal L$}.  Let $(E_n)_{n=1}^\infty \subset \mathcal L$ be a pairwise distinct sequence of sets. I claim that $(E_n)$ must have a monotone subsequence. Suppose not. Then $(E_n)_{n=1}^\infty$ must be bounded above by some $E_{n_1}$, for otherwise, given any $E_n$, there exists $m > n$ with $E_n \subset E_m$. This would define an ascending sequence.  Similarly, $(E_n)_{n=n_1}^\infty$ must be bounded below by say $E_{m_1}$.  Then $(E_n)_{n=m_1}^\infty$ must be bounded above by some $E_{n_2} \subset E_{n_1}$. Continuing in this way, we get a subsequence $E_{n_1} \supset E_{n_2} \supset \ldots$, a contradiction.


Since the $E_n$ were distinct, this implies that $(E_n)$ contains either an ascending subsequence $(A_n)$ or a descending subsequence $(D_n)$.  In the former case, let $B_n = A_n \setminus ( \bigcup_{k=1}^{n-1} A_k)$. In the latter case, let $B_n = D_n \setminus D_{n+1}$.  In either case, $(B_n)$ is a sequence of disjoint nonempty sets.

\emph{Case 2: every linearly inclusion-ordered subset of $\mathcal M$ is finite.}  Let $\mathcal L_1 \subset \mathcal M \setminus \{\emptyset\}$ be a maximal such chain.  Since $M$ is infinite, $\mathcal L_1$ must be nonempty.

Inductively assume we are given nonempty finite chains $(\mathcal L_i)_{i=1}^n \subset \mathcal P ( \mathcal M \setminus \{\emptyset\} )$ such that the minimal elements of $(\mathcal L_i)$ are pairwise disjoint. Let $E_i$ denote the minimal element of $\mathcal L_i$ for each $i$.  Since the $L_i$ are finite, the set $\mathcal F = \{ \bigcup_{i=1}^n F_i : \forall i,  F_i \in (L_i \cup \{\emptyset\}) \}$ is a finite set. Hence there exists a maximal nonempty chain $\mathcal L_{n+1} \subset \mathcal M \setminus (\mathcal F \cup {\emptyset})$. By the Case 2 assumption, $L_1$ is finite, so it contains a minimal element $E_{n+1}$.  Suppose $F := E_{n+1} \cap E_k \neq \emptyset$ for some $1\le k \le n$.  Then $F \in \mathcal F$, for otherwise $F$ contradicts the maximality of $\mathcal L_{n+1}$.  But then $F \subsetneq E_{n+1}$, and $G := E_{n+1} \setminus F$ must be in $\mathcal F$ since $\mathcal L_{n+1}$ is maximal.  Hence, since $\mathcal F$ is closed under finite unions, $E_{n+1} = F \cup G \in \mathcal F$, a contradiction. Hence $(E_i)_{i=1}^{n+1}$ are disjoint, and $L_{n+1}$ is a nonempty finite chain. Thus, by induction, we have the sequence $(E_n)_{n=1}^\infty$ of disjoint nonempty sets.

\emph{Step 2: \mathcal M is uncountable.} 
From Step 1, there exists a sequence $(M_n) \subset \mathcal M$ of disjoint nonempty sets. Define $f : \mathcal P (\N) \to \mathcal M$ by $f(U) = \bigcup_{u \in U} M_u$.  To see that $f$ is injective, suppose that $U \neq V$.  WLOG there exists $t \in U \setminus V$.  Since $M_t$ is nonempty, there exists $x \in M_t$, so $x \in \bigcup_{u \in U} M_u = f(U) $. Since the $(M_n)$ are disjoint, $x \not\in M_n$ for $n \neq u$.  
Hence, $x \not\in \bigcup_{v\in V} M_v = f(V)$.  Thus, $f(U) \neq f(V)$, so $f$ is injective.  Thus $\card(\mathcal M) \ge \card(\mathcal P (\N) > \card(\N)$.
\end{proof}


\p{5} Let $(\Omega_j , \mathcal M_j)$ be measure spaces for $j \in [n]$. Show that 
$$ \mathcal E = \left\{ \prod_{j=1}^n E_j : E_j \in \mathcal M_j \forall j \right\} $$
is an elementary system.

\begin{proof}
Since $\emptyset \in \mathcal M_j$ for all $j$, we have $\emptyset = \prod_{j=1}^n \emptyset \in \mathcal E$. Now suppose $E, F \in \mathcal E$. Then $E = \prod_j E_j$ and $F = \prod_j F_j$ for $E_j, F_j \in \mathcal M_j$ for all $j$. Hence $E \cap F = \prod_j (E_j \cap F_j) \in \mathcal E$.  Lastly, we need to check that $E^c$ is the finite union of disjoint elements of $\mathcal E$. Let 
$$\mathcal U = \{\prod U_j : U_j \in \{E_j, E_j^c \} \}$$.
Note that $\mathcal U$ is a partition of $\prod_j \Omega_j$, and $\mathcal U \subset \mathcal E$..  Hence $E^c = \bigcup (\mathcal U \setminus {E})$ is a finite union of disjoint elements of $\mathcal E$.
\end{proof}


\end{document}
