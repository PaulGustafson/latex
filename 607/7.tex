\documentclass{article}
\usepackage{../m}


\begin{document}
\noindent Paul Gustafson\\
\noindent Texas A\&M University - Math 607\\ 
\noindent Instructor: Thomas Schlumprecht

\subsection*{HW 7}
\p{1} Assume that $(f_n) \subset L_1(\mu)$ and $f_n \to f$ uniformly.

a) If $\mu(X) < \infty$, then $f \in L_1$ and $\lim_{n\to \infty} \int f_n \, d \mu = \int f \, d\mu$.

b) If $\mu(X) = \infty$, then the conclusion of (a) may fail.

\begin{proof}
For (a), we have
\begin{align*}
\left | \int f \, d \mu - \int f_n \, d \mu \right| & \le \int |f - f_n| \, d \mu
\\ & \le \mu(X) \sup_{x \in X} |f(x) - f_n(x)|
\\ & \to 0.
\end{align*}

For (b), let $(f_n) = (1/n) \chi_{[0,n]}$. Then $f_n \to 0$ uniformly on $\R$, but $\int f_n \, dx = 1$ for all $n$.
\end{proof}

\p{2} Let $f_n, g_n, g \in L_1$, $n \in \N$, and assume that $f_n \to f$, $f$ measurable, and $g_n \to g$ $\mu$-a.e., and that $|f_n| \le g_n$ and $\int g_n \, d \mu \to \int g \, d\mu$. 

Then $\int f_n \, d \mu \to \int f \, d \mu$.

\begin{proof}
Following the proof of the DCT, since $|f_n| \le g_n$ we have $|f| \le g$. Hence $f \in L_1$. We also have $g_n + f_n \ge 0$ a.e. and $g_n - f_n \ge 0$ a.e.  Hence by Fatou's lemma and linearity of the integral on $L_1$,
$$\int g + \int f \le \liminf \int (g_n + f_n)= \liminf \int g_n +  \int f_n = \int g + \liminf \int f_n$$
The last equality follows from the fact that if $(a_n) \to a$ and $(b_n) \subset \R$, then $\liminf a_n + b_n = a + \liminf b_n$.  To see this, pick $\epsilon > 0$ and $N$ such that $|a - a_n| < \epsilon$ for all $n \ge N$.  Hence $\liminf a_n + b_n = \liminf (a_n - a) + a + b_n \le \liminf \epsilon + a +  b_n = \epsilon + a + \liminf b_n$, and similarly $\liminf a_n + b_n \ge -\epsilon + a + \liminf b_n$.   Hence $\liminf a_n + b_n = a + \liminf b_n$.

Similarly,
$$\int g - \int f \le \liminf \int (g_n - f_n) = \liminf \int g_n -  \int f_n = \int g - \limsup \int f_n$$.

Hence $\limsup \int f_n \le \int f \le \liminf \int f_n$, so $\int f = \lim \int f_n$.
\end{proof}

\p{3} Suppose that for $n \in \N$, $f_n = \chi_{E_n}$ for some $E_n \subset \R$, and assume that $f(x) = \lim_{n \to \infty} f_n(x)$ exists a.e.

a) Show that $f = \chi_E$ a.e. for some measurable set $E \subset \R$.

b) Show that for any $g \in L_1$:
$$ \int_E g \, dx = \lim_{n \to \infty} \int_{E_n} g \, dx.$$

c) Establish a necessary and sufficient condition for $f_n \to f$ in $L_1$.

\begin{proof}
For (a), we have $\chi_{E_n} \to f$ on $N^c$ for some null set $N$.  Let $x \in N^c$. Since $(\chi_{E_n}(x))_n$ is a convergent discrete-valued sequence, it must be eventually constant.  Thus, $f(x) \in \{0,1\}$.  Let $E = f^{-1}(1) \cap N^c$.  Hence $f = \chi_E$ on $N^c$, so $f = \chi_E$ a.e. on $\R$. By a previous homework problem, $f$ is measurable since it is the limit of measurable functions.  Hence $E$ is measurable.  

For (b), we have $\chi_{E_n} g \to \chi_E g$ pointwise a.e. by part (a).  Moreover, $\chi_{E_n} g \le |g| \in L_1$. Hence, by the DCT, we have the desired conclusion.

For (c), one such condition is that $m(E_n) \to m(E)$ with $m(E_n), m(E) < \infty$.  Clearly, the latter condition is necessary for $f_n, f$ to be in $L_1$.  For the necessity of the former condition, suppose $f_n \to f$ in $L_1$.  Then $|m(E_n) - m(E)| = |\int f_n - \int f | \le \int |f_n - f| \to 0$.

For sufficiency, suppose $m(E_n) \to m(E)$ with $m(E_n), m(E) < \infty$.    Then $|f_n - f| \le |f_n| + |f| = \chi_{E_n} + \chi_E$ a.e.  Moreover, $\chi_{E_n} + \chi_E \to 2 \chi_E$  a.e., and $\int (\chi_{E_n} + \chi_E) = m(E_n) + m(E) \to 2 m(E) = \int (2 \chi_E)$.  Hence, by the generalized DCT (Exercise 2), we have
$\int |f_n - f| \to \int \lim_n |f_n - f| = 0$.
\end{proof}

\p{4} Let $L_0([0,1])$ be the space of all measurable functions $f : [0,1] \to \R$.

a) for $f,g \in L_0([0,1])$ put
$$d(f, g) = \int_0^1 \min\{1, |f - g|\} \, dx.$$

Show that $(L_0([0,1]), d)$ is a metric space and that for $f, f_n \in L_0([0,1])$:
$$ f_n \to f \text{ in } (L_0([0,1]), d) \iff f_n \to f \text { in measure.}$$

b) Is there a metric $d'$ on $L_0([0,1])$ for which 
$$f_n \to f \text{ in } (L_0([0,1]), d') \iff f_n \to f a.e.  $$

\begin{proof}
For (a), to see that $d$ is a metric we need to show that $d$ is positive definite, symmetric, and satisfies the triangle inequality.  The function $d$ is clearly nonnegative and $0 = d(f,g) = \int_0^1 \min\{1, |f - g|\} \, dx$ implies that $f = g$ a.e.  The function $d$ is obviously symmetric.  For the triangle inequality, I first claim that for $x,y,z \in \R$ we have  $\min\{1, |x - y|\} \le \min\{1, |x - z| \} + \min\{1, |y-z|\}$. 

We have four cases from the RHS of the inequality. 

\emph{Case $|x - z| \le 1$ and $|y - z| \le 1$.}  We have $\min\{1, |x - y|\}  \le |x - y| \le |x - y| + |y - z| = \min\{1, |x - z|\}  + \min\{1, |z - y|\}$.

\emph{Case $|x - z| \le 1$ and $|y - z| > 1$.} We have $\min\{1, |x - y|\} \le \min\{1, |x - z| + |z- y|\} \le \min\{1, 1 + |z- y|\} = 1 + |z- y| = \min\{1, |x - z|\}  + \min\{1, |z - y|\}$.

\emph{Case $|x - z| > 1$ and $|y - z| \le 1$.} Analogous to previous case.

\emph{Case $|x - z| > 1$ and $|y - z| > 1$.} We have $\min\{1, |x - y|\} \le \min\{1, |x - z| + |z- y|\} = 1 \le \min\{1, |x - z|\}  + \min\{1, |z - y|\}$.

Hence $\min\{1, |x - y|\} \le \min\{1, |x - z| \} + \min\{1, |y-z|\}$ for all $x,y,z \in \R$.  Thus, if $f,g,h \in L_0([0,1])$ then 
$d(f,g) = \int_0^1 \min\{1, |f - g|\} \, dx \le \int_0^1 \min\{1, |f - h|\} + \min\{1, |h - g|\}  \, dx =  d(f,h) + d(g,h)$.  Thus, $d$ is a metric.

Suppose $f_n \to f$ in $(L_0([0,1]), d)$.  Let $0 < \epsilon < 1$.  Pick $N$ such that $d(f, f_n) < \epsilon^2$ for all $n \ge N$.  Then for all $n \ge N$,
we have 
\begin{align*}
m(\{ |f - f_n| \ge \epsilon \})  & = \int_{\{ \epsilon \le |f - f_n| < 1 \}} \,dx + \int_{\{ |f - f_n| \ge 1 \}} \, dx
\\ & \le \int_{\{ \epsilon \le |f - f_n| < 1 \}} \frac { |f - f_n|} {\epsilon} \,dx + \int_{\{ |f - f_n| \ge 1 \}} \min\{1, |f - f_n|\} \, dx
\\ & \le \int_{\{ \epsilon \le |f - f_n| < 1 \}} \epsilon^{-1}  \min\{1, |f - f_n|\}  \,dx + \int_{\{ |f - f_n| \ge 1 \}} \min\{1, |f - f_n|\} \, dx
\\ & \le \epsilon^{-1} \int \min\{1, |f - f_n|\} \, dx
\\ & < \epsilon
\end{align*}

Conversely, suppose $f_n \to f$ in measure.  Let $0 < \epsilon < 1$. Pick $N$ such that $m( \{ |f - f_n \ge \epsilon \})  \le \epsilon$ for all $n \ge N$.  Then for all $n \ge N$ we have
\begin{align*}
\int \min \{1, |f_n - f|\} \, dx & \le  \int_{\{ 0 \le |f_n - f| < \epsilon \}} |f_n - f| \, dx + \int_{\{  |f_n - f| \ge \epsilon \}} \, dx
\\ & \le \int_{\{ 0 \le |f_n - f| < \epsilon \}} \epsilon \, dx + m( \{ |f - f_n \ge \epsilon \}) 
\\ & \le 2 \epsilon.
\end{align*}

For (b), I use the following fact about convergence in metric spaces. Let $(M, d)$ be a metric space, $(x_n)_{n \in \N} \subset M$ , and $x \in M$.  I claim that if every subsequence of $(x_n)$ has a further subsequence converging to $x$, then $x_n \to x$.  

Suppose $x_n \not \to x$.  Then there exists $\epsilon > 0$ and a subsequence $(x_n)_{n \in N_1}$ such that $d(x, x_n) \ge \epsilon$ for all $n \in N_1$. This subsequence cannot have a further subsequence converging to $x$, contradicting the hypothesis.

Thus it suffices to find a sequence that does not converge pointwise a.e., but each subsequence has a subsequence that converges a.e. to $0$.  For $n \in \N$, write $n$ as $n = 2^j + k$ for $j \ge 0$ and $0 \le k < 2^j$.  Let $E_n = [k 2^{-j}, (k+1) 2^{-j}]$ and $f_n = \chi_{E_n}$.  Every element of $[0,1]$ is contained in infinitely many $E_n$ and infinitely many $E_n^c$, so $(f_n)$ does not converge pointwise a.e.  

On the other hand, suppose $(f_n)_{n \in N_1}$ is a subsequence of $(f_n)_{n \in \N}$.  For each $n$, pick $x_n \in E_n$.  Then by the sequential compactness of $[0,1]$, there exists an infinite set $N_2 \subset N_1$ and $x_0 \in [0,1]$ such that $x_n \to x_0$ as $n \to \infty, n \in N_2$. Since $\diam(E_n) \to 0$, we have $f_n(x) \to 0$ as $n \to \infty, n \in N_2$ for $x \neq x_0$.  Hence, $f_n \to 0$ as $n \to \infty, n \in N_2$ pointwise a.e.
\end{proof}


\p{5} 
\begin{enumerate}[(a)]
\item $\displaystyle \lim_{n\to \infty} \int_0^\infty \left( 1 + \frac x n \right)^{-n} \sin \left(\frac x n \right) \, dx = 0$
\item $\displaystyle \lim_{n \to \infty} \int_0^1 \frac{1 + n x^2} {(1 + x^2)^n} = 0$
\item $ \displaystyle \lim_{n \to \infty} \int_0^\infty n \sin \left( \frac x n \right) [x(1+x^2)]^{-1} \, dx = \frac \pi 2$
\item $ \displaystyle \lim_{n \to \infty} \int_a^\infty \frac n {1 + n^2 x^2} \, dx= \left\{
\begin{array}{ll}
   0 & \text{ if } a > 0
\\ \pi/2 & \text{ if } a = 0
\\ \pi  & \text{ if } a < 0
\end{array} \right.
$
\end{enumerate}

\begin{proof}
For (a), we have  $\int_0^\infty \left( 1 + \frac x n \right)^{-n} \sin \left(\frac x n \right) \, dx  = \int_0^\infty n^2 ( 1 + n u )^{-n} \sin(nu) \, du$.

For $n \ge 2$, we have $\left| n^2 ( 1 + n u )^{-n} \sin(nu) \right| \le  n^2 (1 + nu)^{-2} \le u^{-2}$, which is integrable on $(1, \infty)$.  Hence, by the DCT, 
\begin{align*}
 \lim_{n\to \infty} \int_1^\infty n^2 ( 1 + n u)^{-n} \sin(nu) \, du  & = \int_1^\infty \lim_{n\to \infty} n^2 ( 1 + n u)^{-n} \sin(nu) \, du
 = 0
\end{align*}

For the rest of the integral, we have
\begin{align*}
\left| \int_0^1 n^2 ( 1 + n u)^{-n} \sin(nu) \, du \right| & \le \int_0^1 n^2 (1 + nu )^{-n} nu \, du 
\\ & = \left[ \frac{n^2}{1 - n} u (1+nu)^{1-n} \right]_{u=0}^1 - \int_0^1 \frac{n^2}{1-n} (1+nu)^{1-n} \, du
\\ & = \frac{n^2}{1-n} (1+n)^{1-n} - \left[ \frac{n}{(1-n)(2-n)} (1+nu)^{2-n} \right]_{u=0}^1
\\ & \to 0
\end{align*}
as $n \to \infty$.  This proves (a).

For (b), for $x \ge 0$ we have $\frac{1 + n x^2} {(1 + x^2)^n} \le 1$. Hence, by the DCT we have
\begin{align*}
\lim_{n \to \infty} \int_0^1 \frac{1 + n x^2} {(1 + x^2)^n} \, dx & = \int_0^1 \lim_{n \to \infty} \frac{1 + n x^2} {(1 + x^2)^n} \, dx 
 = 0.
\end{align*}

For (c), for $x \in [0,\infty)$ we have $ \left| n\sin \left( \frac x n \right) [x(1+x^2)]^{-1} \right| \le n \left(\frac x n \right) [x(1+x^2)]^{-1} = (1+x^2)^{-1}$.  We have $(1+x^2)^{-1} \in L_1([0,\infty))$ by applying the MCT to $\int_0^M (1+x^2)^{-1} \, dx$ as $M \to \infty$.  Hence, by the DCT we have

\begin{align*}
\lim_{n \to \infty} \int_0^\infty n \sin \left( \frac x n \right) [x(1+x^2)]^{-1} \, dx  & = \int_0^\infty \lim_{n \to \infty} n \sin \left( \frac x n \right) [x(1+x^2)]^{-1} \, dx 
\\ & = \int_0^\infty \lim_{m \to 0^+}  \frac {\sin (m)}{m}  (1+x^2)^{-1} \, dx 
\\ & = \int_0^\infty  (1+x^2)^{-1} \, dx 
\\ & = \lim_{M \to \infty} \int_0^M (1+x^2)^{-1} \, dx 
\\ & = \lim_{M \to \infty} \tan^{-1}(M)
\\ & = \pi/2,
\end{align*}
where the fourth equality follows from the MCT.

For (d),
\begin{align*}
\lim_{n \to \infty} \int_a^\infty \frac n {1 + n^2 x^2} \, dx  & = \lim_{n \to \infty} \int_{na}^\infty \frac {du} {1 + u^2} 
\\ & = \lim_{n \to \infty} \pi/2 - \tan^{-1}(na)
\\ & = \left\{ 
\begin{array}{ll}
   0 & \text{ if } a > 0
\\ \pi/2 & \text{ if } a = 0
\\ \pi  & \text{ if } a < 0
\end{array} \right.,
\end{align*}
where the second equality follows from the MCT applied to $\int_{na}^M \frac {du} {1+u^2}$ as $M \to \infty$.



\end{proof}

\end{document}
