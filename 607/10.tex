\documentclass{article}
\usepackage{../m}

\begin{document}
\noindent Paul Gustafson\\
\noindent Texas A\&M University - Math 607\\ 
\noindent Instructor: Thomas Schlumprecht


\subsection*{HW 10}
\p{1}  Problem 11/page 92. Let $\mu$ be a positive measure on $(X \mM)$. A collection of functions $(f_\alpha)_{\alpha \in A}$ is called \emph{uniformly integrable} if for every $\epsilon > 0$ there is a $\delta > 0$ such that $\left| \int_E f_\alpha d\mu \right| < \epsilon$ for all $\alpha \in A$ whenever $\mu(E) < \delta$.
\begin{enumerate}[a)]
\item Any finite subset of $L_1$ is uniformly integrable.
\item A sequence $(f_n)$ which is convergent in $L_1$ is uniformly integrable.
\end{enumerate}
\begin{proof}
For (a), it suffices to show that a single function $f \in L_1$ is uniformly integrable.  Pick an integrable simple function $\phi$ with
 $\|f - \phi\|_1 < \epsilon/2$.  Pick $\delta > 0$ such that $\int_E |\phi| d\mu < \epsilon/2$ for all $\mu(E) < \delta$.  
Then $\left| \int_E f d\mu \right| \le \int_E |f - \phi| d\mu + \int_E |\phi| d\mu < \epsilon$.

For (b), let $\epsilon > 0$ and let $f$ be the $L_1$-limit of the sequence $(f_n)$.  Pick $N$ such that $\| f_n - f \|_1 \le \epsilon/2$ for all $n \ge N$.  By part (a), pick $\delta$ such that $\int_E |g| d\mu  < \epsilon/2$ for all $\mu(E) < \delta$, $g \in \{f\} \cup \{f_n\}_{n < N}$.  
For $\mu(E) < \delta$ and $n \ge N$, we have 
$\left| \int_E f_n d\mu \right| \le \int_E |f_n -f| d\mu + \int_E |f| d\mu \le \epsilon$.
\end{proof}

\p 2 Problem 13/page 92.  Let $X = [0,1], \mM = \mB_\R$ and $\mu$ be the counting measure $[0,1]$.
\begin{enumerate}[a)]
\item $m << \mu$ but there is no $f \in L_0^+$ so that $dm = f d\mu$,
\item $\mu$ has no Lebesgue decomposition with respect to $m$.
\end{enumerate}
\begin{proof}
For (a), the only null sets of $\mu$ are empty, so $m << \mu$.  For the other part, suppose $dm = f d\mu$ for some $f \in L_0^+$.  Then $0 = \int_{\{x\}} dm = \int_{\{x\}} f d\mu = f(x)$ for all $x \in [0,1]$, a contradiction.

For (b), suppose $\mu = \lambda + \rho$ with $\lambda \perp m$ and $\rho << m$. Let $x \in [0,1]$.  Since $\rho << m$, $\rho(\{x\}) = 0$.  Hence $1 = \mu(\{x\}) = \lambda(\{x\})$. Then $\lambda([0,1]) \ge 1$ and $m([0,1]) = 1$, which contradicts the assumption that $\lambda \perp m$.
\end{proof}

\p 3 Assume that $(\Omega, \mM, \PP)$ is a probability space and that $\tilde \mM \subset \mM$ is a sub-$\sigma$-algebra of $\mM$. Let $X$ be an integrable random variable. Then there exists a random variable $\tilde X$ so that:
\begin{enumerate}[a)]
\item $\tilde X$ is $\tilde \mM$-measurable.
\item for all $A \in \tilde \mM$,
$$\EE_\PP(\chi_A X) = \EE_\PP(\chi_A \tilde X).$$
Furthermore $\tilde X$ is unique, i.e. for every random variable $Y$ which has properties (a) and (b) it follows that $Y = \tilde X$ almost surely. (Hint: consider the signed measure $d\nu = X d\PP$ and restrict that measure.  Use the Radon Nikodym theorem.)
\end{enumerate}
\begin{proof}
 Let $\tilde \PP$ be the restriction of $\PP$ to $\tilde \mM$.  Let $\nu$ be the signed measure on $\tilde \mM$ defined by $\nu(A) = \int_A X d\tilde \PP$.  Let $\tilde X$

\end{proof}


\p 4 Assume that $(\Omega, \mM \PP)$ is a probability space and that $\tilde \mM \subset \mM$ is a sub-$\sigma$-algbra of $\mM$. Let $X$ and $Y$ be integrable random variables. Then
\begin{enumerate}[a)]
\item (Linearity)
$$\EE(\alpha X + \beta Y | \tilde \mM) = \alpha \EE(X | \mM) + \beta \EE(Y | \tilde \mM).$$
\item (Positivity)
$$X \le Y \quad \text{$\PP$-almost surely} \implies \EE(X | \tilde \mM) \le \EE(Y | \tilde \mM) \quad \text{$\PP$-a.s.}$$
\item (Tower-Property) Assume $\mN \subset \tilde \mM$ is a sub-$\sigma$-algebra of $\tilde \mM$. Then
$$\EE(\EE(X | \tilde \mM) | \mN) = \E(X | \mN) \quad \text{$\PP$-a.s.}$$
\item (Factorization) If $Y$ is $\tilde \mM$-measurable and $XY$ is integrable, then
$$\EE(Y | \tilde \mM) = Y \text{ and } \EE(YX | \tilde \mM) = Y \EE(X | \tilde \mM) \quad \text{$\PP$-a.s.}
\item (Absolute value)
$$|\EE(X | \tilde \mM)| \le \EE(|X||\mM) \quad \text{$\PP$-a.s.}$$
\end{enumerate}

\p 5 State and prove the Monotone Convergence Theorem and the Dominated Convergence Theorem for conditional expectations.

\p 6 Assume that $(\Omega, \mM, \PP)$ is a probability space and that $\tilde \mM \subset \mM$ is a sub-$\sigma$-algebra of $\mM$ generated by $A_1, A_2, \ldots, A_n \in \mM$, a partition of $\Omega$. Assume that $X$ is an integrable random variable on $(\Omega, \mM, \PP)$. Compute $\EE(X | \tilde \mM)$.

\end{document}
