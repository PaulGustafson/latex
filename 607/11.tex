\documentclass{article}
\usepackage{../m}

\begin{document}
\noindent Paul Gustafson\\
\noindent Texas A\&M University - Math 607\\ 
\noindent Instructor: Thomas Schlumprecht

\subsection*{HW 11} 
\p{1} Let $f$ be increasing on $[0,1]$ and 
$$
g(x) = \limsup_{h\to 0} \frac{f(x+h) - f(x-h)}{2h}, \quad 
\text{for $0 < x < 1$.}
$$
Prove that if $A = \{x \in (0,1) : g(x) > 1\}$ then
$$
f(1) -f(0) \ge m^*(A).
$$
Hint: Vitali's Lemma.

\begin{proof}
To avoid worrying about endpoints, extend $f$ to be constant on $(-\infty, 0]$ and $[1, \infty)$. This does not change $A$.

For each $x \in A$, pick a sequence $(h_{x,n})$ with $\lim_{n \to \infty} h_{x,n} \to 0$ and
$$
\lim_{n \to \infty} \frac { f(x + h_{x,n}) - f(x - h_{x,n})}{2 h_{x,n}} > 1,
$$
for all $n$.

Then $\mB = \{B(h_{x,n}, x) : x \in A\}$ forms a Vitali cover for $A$.  Let $\epsilon > 0$. 
We can pick a finite set $\mF \subset \mB$ of disjoint balls with $m(\bigcup \mF) > m^*(A) - \epsilon$.
Let $(a_i,b_i)_{i=1}^n$ be an enumeration of $\mF$ with $a_1 < b_1 < a_2 < \ldots < b_n$.  Then
$f(1) - f(0) \ge \sum_{i=1}^n f(b_i) - f(a_i) \ge b_i - a_i > m^*(A) - \epsilon$.  Letting $\epsilon \to 0$, we have $f(1) - f(0) \ge m^*(A)$.
\end{proof}


\p{2} Let $f : [a,b] \to \R$ be an increasing function. Using Vitali's lemma, show that 
$$
m(\{D^+f(x) \neq D^-f(x)\}) = 0.
$$
where $D^+(f)$ is the upper derivative from the right, and $D^-(f)$ is the lower derivative from the right.

\begin{proof}
Acknowledgement: I looked at http://www.math.ucla.edu/$\sim$ralston/245a.1.08f/Vitali.pdf for hints.

It suffices to show that $E_{p,q} = \{x \in [a,b] : D^-f(x) < p < q < D^+f(x)\}$ has measure $0$ for every $p,q \in \Q$ with $p < q$. 

Let $\epsilon > 0$. Pick an open set $U \supset E_{p,q}$ with $m(U) < m^*(E_{p,q}) + \epsilon$.

If $x \in E_{p,q}$, then there exist arbitrarily small $h$ for which $\frac{f(x+h) - f(x)}{h} < p$.  
Thus intervals of the form $[x, x+h) \subset U$ with this property form a Vitali cover for $E_{p,q}$. 
By the Vitali lemma, we can pick
a disjoint finite subset of these intervals $([x_k, x_k + h_k))_{k=1}^n$ such that $\sum_k h_k > m^*(E_{p,q}) - \epsilon$.

Similarly for $y \in E_{p,q} \cap \bigcup_k [x_k, x_k + h_k)$ there exist arbitrarily small $l$ for which $\frac{f(y+l) - f(y)}{l} > q$.  Thus,
sets of the form $[y, y+l)$ with this property form a Vitali cover for $E_{p,q} \cap \bigcup_k [x_k, x_k + h_k)$.  
Moreover, by throwing sets out of the cover, we can assume that each interval $[y, y+l)$ lies within an interval $[x_k, x_k + h_k)$.
By the Vitali Lemma, we get a disjoint finite subset of these intervals $([y_j, y_j + l_k))_{j=1}^m$ with 
\begin{align*}
\sum_k l_k & > m^*(E_{p,q} \cap \bigcup_k [x_k, x_k + h_k)) - \epsilon 
\\ & = m(\bigcup_k [x_k, x_k + h_k)) - m^*(E_{p,q}^c \cap \bigcup_k [x_k, x_k + h_k)) - \epsilon 
\\ & > (m^*(E_{p,q}) - \epsilon) - m^*(E_{p,q}^c \cap U) - \epsilon
\\ & > m^*(E_{p,q}) - 3 \epsilon
\end{align*}

Then we have
\begin{align*}
q(m^*(E_{p,q}) - 3 \epsilon)  &   = q \sum_k l_k
\\ & < \sum_j f(y_j + l_j) - f(x_j) 
\\ & \le \sum_k f(x_k + h_k) - f(x_k)
\\ & < p \sum_k h_k 
\\ & < p (m^*(E_{p,q}) - \epsilon).
\end{align*}
Letting $\epsilon \to 0$, we have $0 \le (p-q) m^*(E_{p,q})$, so  $m^*(E_{p,q}) = 0$.
\end{proof}

\p 3 Assume that $f: [a,b] \to \R$ is continuous and that $D^+f(x) > 0$, for all $x \in [a,b]$.
Show that $f$ is nondecreasing on $[a,b]$.

\begin{proof}
Suppose $f$ is not nondecreasing.  Then there exist $a \le c < d \le b$ with $f(c) > f(d)$. By the extreme value theorem, $f$ achieves a maximum $M$ on $[c,d]$. 
Let $u = \sup\{x \in [c,d] : f(x) = M\}$.  Since $f$ is continuous, $f(u) = M$. Since $M \ge f(c) > f(d)$, we have $u < d$. Moreover, $f(x) < M$ for all $x \in [u,d]$.  Thus $D^+f(u) \le 0$, a contradiction.
\end{proof}



\p 4 Determine whether or not the following functions are of bounded variation on $[-1,1]$.
\begin{enumerate}[(a)]
\item $f(x) = x^2 \sin(1/x^2), \quad x \neq 0, f(0) = 0$
\item $f(x) = x^2 \sin(1/x), \quad x \neq 0, f(0) = 0.$
\end{enumerate}

\begin{proof}
For (a), we have 
\begin{align*}
T_{-1}^1(f) & \ge \sum_{n = 1}^N  |f((n \pi)^{-1/2} - f((n \pi + \pi/2)^{-1/2})|
\\ & = \sum_{n = 1}^N  |(n \pi + \pi/2)^{-1})|
\\ \to \infty
\end{align*}
as $N \to \infty$, so $f$ is not of bounded variation.

For (b), if $(x_n)_{n=0}^N$ is a partition of $[-1,1]$
\begin{align*}
\sum_{n = 1}^N  |f(x_n) - f(x_{n-1})| & \le C + 2 \sum_{n = 1}^\infty | f((n\pi - \pi/2)^{-1}) - f((n\pi + \pi/2)^{-1})|
\\ & = C + 2 \sum_{n = 1}^\infty (n\pi - \pi/2)^{-2} + (n\pi + \pi/2)^{-2},
\end{align*}
which converges.  Hence $f$ is of bounded variation.
\end{proof}


\p 5 Let $f$ be of bounded variation on $[a,b]$, then 
$$
\int_a^b |f'(t)| dt \le T_a^b(f).
$$

\begin{proof}
We have
\begin{align*}
\int_a^b |f'(t)| dt  & = \int_a^b |\frac 1 2 (T_a^t(f) + f)' -  \frac 1 2 (T_a^t(f) - f)'| dt 
\\ & \le \frac 1 2 \int_a^b |(T_a^t(f) + f)'| +  |(T_a^t(f) - f)'| dt 
\\ & = \frac 1 2 \int_a^b (T_a^t(f) + f)' +  (T_a^t(f) - f)' dt 
\\ & = \int_a^b (T_a^t(f))' dt 
\\ & \le T_a^b(f),
\end{align*}
where the last inequality follows from decomposing the function $t \mapsto T_a^t(f)$ into its absolutely continuous and singular parts.
\end{proof}

 
\p 6 Construct an increasing function on $\R$ whose discontinuities are $\Q$.

\begin{proof}
Let $\delta_x$ denote the Dirac measure at $x$. Let $(q_n)$ be an enumeration of $\Q$.  Let $\nu = \sum_{n=1}^\infty 2^{-n} \delta_{q_n}$. Let 
$f(x) = \nu((-\infty, x))$. Then $f$ is increasing and has discontinuities at every rational point. 

 If $x$ is irrational and $\epsilon > 0$, pick $N$ such that $2^{-N} < \epsilon$.  
Pick $\delta > 0$ such that $d(x,q_n) > \delta$ for all $n \le N$.  

Suppose $d(x,y) < \delta$.  WLOG suppose $x < y$. We have $|f(x) - f(y)| = \nu((x,y)) \le \sum_{n = N + 1}^\infty 2^{-n} = 2^{-N} < \epsilon$.
\end{proof}


\end{document}
