\documentclass{article}
\usepackage{../m}

\begin{document}
\noindent Paul Gustafson\\
\noindent Texas A\&M University - Math 607\\ 
\noindent Instructor: Thomas Schlumprecht

\subsection*{HW 11} 
\p{1} Let $f$ be increasing on $[0,1]$ and 
$$
g(x) = \limsup_{h\to 0} \frac{f(x+h) - f(x-h)}{2h}, \quad 
\text{for $0 < x < 1$.}
$$
Prove that if $A = \{x \in (0,1) : g(x) > 1\}$ then
$$
f(1) -f(0) \ge m^*(A).
$$
Hint: Vitali's Lemma.

\begin{proof}
For $x \in A$, pick a sequence $(h_{x,n})$ with $\lim_{n \to \infty} h_{x,n} \to 0$ and
$$
\lim_{n \to \infty} \frac { f(x + h_{x,n}) - f(x - h_{x,n})}{2 h_{x,n}} > 1.
$$

Then $\mB = \{B(h_{x,n}, x) : x \in A\}$ forms a Vitali cover for $A$.  Let $\epsilon > 0$. 
We can pick a finite set $\mF \subset \mB$ of disjoint balls with $m(\bigcup \mF) > m^*(A) - \epsilon$.


\end{proof}


\p{2} Let $f : [a,b] \to \R$ be an increasing function. Using Vitali's lemma, show that 
$$
m(\{D^+f(x) \neq D^-f(x)\}) = 0.
$$
where $D^+(f)$ is the upper derivative from the right, and $D^-(f)$ is the lower derivative from the right.



\p 3 Assume that $f: [a,b] \to \R$ is continuous and that $D^+f(x) > 0$, for all $x \in [a,b]$.
Show that $f$ is nondecreasing on $[a,b]$.

\begin{proof}

\end{proof}



\p 4 Determine whether or not the following functions are of bounded variation on $[-1,1]$.
\begin{enumerate}[(a)]
\item $f(x) = x^2 \sin(1/x^2), \quad x \neq 0, f(0) = 0$
\item $f(x) = x^2 \sin(1/x), \quad x \neq 0, f(0) = 0.$
\end{enumerate}




\p 5 Let $f$ be of bounded variation on $[a,b]$, then 
$$
\int_a^b |f'(t)| dt \le T_a^b(f).
$$

\begin{proof}
We have
\begin{align*}
\int_a^b |f'(t)| dt  & = \int_a^b |\frac 1 2 (T_a^t(f) + f)' -  \frac 1 2 (T_a^t(f) - f)'| dt 
& \le \frac 1 2 \int_a^b |(T_a^t(f) + f)'| +  |(T_a^t(f) - f)'| dt 
& = \frac 1 2 \int_a^b (T_a^t(f) + f)' +  (T_a^t(f) - f)' dt 
& = \int_a^b (T_a^t(f))' dt 
& = T_a^b(f) dt 
\end{align*}
\end{proof}

 
\p 6 Construct an increasing function on $\R$ whose discontinuities are $\Q$.

\begin{proof}
Let $\delta_x$ denote the Dirac measure at $x$. Let $(q_n)$ be an enumeration of $\Q$.  Let $\mu = \sum_n 2^{-n} \delta_{q_n}$. Let 
$f(x) = \nu((-\infty, x))$.
\end{proof}


\end{document}
