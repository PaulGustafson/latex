\documentclass{article}
\usepackage{../m}

\begin{document}
\noindent Paul Gustafson\\
\noindent Texas A\&M University - Math 607\\ 
\noindent Instructor: Thomas Schlumprecht

\subsection*{HW 1}
\p{1} Let $f: X \to Y$. Prove that 

 a) if $A,B \subset Y$, then $f^{-1}(A \cap B) = f^{-1}(A) \cap f^{-1}(B)$ and
$f^{-1}(A \cup B) = f^{-1}(A) \cup f^{-1}(B)$

b) For a family $(A_\lambda)_{\lambda \in \Lambda} \subset P(X)$,  show that
$f^{-1}(\bigcup_{\lambda \in \Lambda} A_\lambda) = \bigcup_{\lambda \in \Lambda} f^{-1}(A_\lambda)$ and $f^{-1}(\bigcap_{\lambda \in \Lambda} A_\lambda) = \bigcap_{\lambda \in \Lambda} f^{-1}(A_\lambda)$

 and give examples for the following situations

c) $f^{-1}(f(A)) \neq A$, for some $A \subset X$,

d) $f(f^{-1}(B)) \neq B$ for some $B \subset Y$,

e) $f(\cap_{\lambda \in \Lambda}) \neq \cap_{\lambda \in \Lambda} f(A_\lambda)$, for some family $(A_\lambda)_{\lambda \in \Lambda} \subset P(X)$.

\begin{proof}
(a) is a subcase of (b).  To prove the first part of (b), 
\begin{align*}
x \in f^{-1}(\bigcup_{\lambda \in \Lambda} A_\lambda) & \iff & 
f(x) \in \bigcup_{\lambda \in \Lambda} A_\lambda 
\\ & \iff & f(x) \in A_\lambda \text{ for some } \lambda
\\ & \iff & x \in f^{-1}(A_\lambda) \text{ for some } \lambda
\\ & \iff & x \in \bigcup_\lambda f^{-1}(A_\lambda).
\end{align*}

For the second part,
\begin{align*}
x \in f^{-1}(\bigcap_{\lambda \in \Lambda} A_\lambda) & \iff & 
f(x) \in \bigcap_{\lambda \in \Lambda} A_\lambda 
\\ & \iff & f(x) \in A_\lambda \text{ for all } \lambda
\\ & \iff & x \in f^{-1}(A_\lambda) \text { for all } \lambda
\\ & \iff & x \in \bigcap_\lambda f^{-1}(A_\lambda) 
\end{align*}

For (c), let $X = \{0,1\}$ and $Y = \{0\}$.  Let $A = \{0\} \subset X$.  Let $f:X \to Y$ be the constant function.  Then $f^{-1}(f(A)) = f^{-1}(Y) = X \neq A$.

For (d), let $X = \{0\}$ and $B = Y = \{0, 1\}$.   Let $f:X \to Y$ be the constant function at $1$. Then $f(f^{-1}(B)) = f(X) = \{1\} \neq B$.

For (e), let $X = \{0,1\}$ and $Y = \{0\}$. Let $A_1 = \{0\}$ and $A_2 = \{ 1 \}$.  Let $f:X \to Y$ be the constant function.  Then $f(A_1 \cap A_2) = f(\emptyset) = \emptyset$, but $f(A_1) \cap f(A_2) = \{0\}$.

\end{proof}

\p{2} Show that the following two statements are equivalent for two nonempty sets $A$ and $B$.

a) There is an injection $\phi : A \to B$.

b) There is a surjection $\psi : B \to A$.

\begin{proof}
Suppose (a) holds.  Let $(U_b)_{b \in B}$ be defined by $U_b = \phi^{-1}(\{b\})$ if $b \in \phi(A)$ and $U_b = A$ otherwise.  By the axiom of choice, there exists $f \in \prod_{b \in B} U_b$. Since each $U_b \subset A$, there exist identity injections $i_b:U_b \to A$ for each $b \in B$.  Define $\psi: B \to A$ by $\psi(b) = i_b(f(b))$.

To see that $\psi$ is surjective, let $a \in A$.  Since $\phi$ is injective, $\phi^{-1}(\phi(\{a\}))$ contains only $a$.  Hence, $f(\phi(a)) \in (U_{\phi(a)} = \phi^{-1}(\phi(\{a\})))$ implies that $f(\phi(a)) = a$. Thus, $\psi(\phi(a)) = i_{\phi(a)} f(\phi(a)) = i_{\phi(a)}(a) = a$.

Now suppose (b) holds. Let $(U_a)_{a \in A}$ be defined by $U_a = \psi^{-1}(\{a\})$, which are non-empty since $\psi$ is surjective.   By AC, there exists $f \in \prod_{a \in A} U_a$.  Since each $U_a \subset B$, there exist identity injections $i_a:U_a \to B$.  Define $\phi: A \to B$ by $\phi(a) = i_a(f(a))$.

To see that $\phi$ is injective, let $b \in B$ and suppose $x, y \in \phi^{-1}(\{b\})$. Then $x \in f^{-1}(i_x^{-1}(\{b\}))$, so $f(x) \in i_x^{-1}(\{b\}) = \{b\}$ and similarly for y.  Hence, $f(x) = b = f(y)$.  Hence, $b \in (U_x \cap U_y)$. But $U_x \cap U_y = \psi^{-1}(\{x\}) \cap \psi^{-1}(\{y\}) = \psi^{-1}(\{x\} \cap \{y\})$.  Thus, $\{x\} \cap \{y\}$ is nonempty, so $x = y$.

\end{proof}

\p{3} Find nonhomeomorphic metric spaces $M_1$ and $M_2$ such that there exist injective continuous functions $f:M_1 \to M_2$ and $g: M_2 \to M_2$.
\begin{proof}
Let $M_1 = (0,1)$ and $M_2 = (0,1) \cup (2,3)$ with distances inherited from $\R$.  $M_1$ and $M_2$ are not homeomorphic since $M_1$ is connected but $M_2$ is disconnected. %FIXME

Let $f:M_1 \to M_2$ be defined by $f(x) = x$. $f$ is clearly injective since it is the restriction of an injective function (the identity on $\R$).  For continuity, suppose $U$ is open in $M_2$. Then $U = V \cap M_2$ for some open set $V \in \mathbb{R}$, so $U$ is open in $\R$. Thus, $f^{-1}(U) = U \cap (0,1)$, which is open in $R$ hence $M_1$.

Let $g:M_2 \to M_1$ be defined by $g(x) = x/3$. As with $f$, $g$ is clearly injective. For continuity, suppose $U \subset M_1$ is open.  Then $U$ is open in $\R$. Hence, $3U$ is open in $\R$ since $x \mapsto 3x$ is a homeomorphism from $\R \to \R$.  Thus, $g^{-1}(U) = (3U) \cap M_2$ is open in $M_2$.
\end{proof}

\p{4} Prove that every real vector space has a basis.
\begin{proof}
Let $V$ be a real vector space. Let $\mathcal I \subset P(V)$ be the collection of sets $U$ such that $\span(U) \neq V$. Define an equivalence relation $\~$ on $\mathcal I$ by $S \~ T$ if $\span(S) = \span(T)$.   If $S \~ S'$ and $T \~ T'$, then $\span(S) \subset \span(T)$ implies $\span(S') = \span(S) \subset \span(T) = \span(T')$. Hence, the relation $\le$ on $\mathcal{I}/\~$ defined by $\bar{S} \le \bar{T}$ iff $span(S) \subset span(T)$ is well-defined. I claim that $\le$ is a partial order. Transitivity: suppose $\bar{S} \le \bar{T}$ and $\bar{T} \le \bar{U}$.  Then $\span(S) \subset \span(T) \subset \span(U)$. Hence, $\bar{S} \le \bar{U}$. Similarly, reflexivity follows from the reflexivity of inclusion.  Anti-symmetry:  suppose $\bar{S} \le \bar{T}$ and $\bar{T} \le \bar{S}$.  Then $\span(S) \subset \span(T)$ and vice versa, so $S \~ T$.


Let $(\bar{U}_\alpha)$ be a linearly ordered subset of $(\mathcal{I}/\~, \le)$. Let $U := \bigcup_\alpha (U_\alpha)$. I claim $\bar U$ is a bound for $(\bar{U}_\alpha)$. 


\end{proof}

\end{document}
