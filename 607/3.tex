\documentclass{article}
\usepackage{../m}

\begin{document}
\noindent Paul Gustafson\\
\noindent Texas A\&M University - Math 607\\ 
\noindent Instructor: Thomas Schlumprecht

\subsection*{HW 3}

\p{1} Problem 8/Page 27. If $(X, \mM, \mu)$ is a measure space and $(E_j)_{j=1}^\infty \subset \mM$, then $\mu(\liminf E_j) \le \liminf \mu(E_j)$. Also, $\mu(\limsup E_j) \ge \limsup \mu(E_j)$ provided that $\mu(\bigcup_j E_j) < \infty$.

\begin{proof}
Let $F_k := \bigcap_{j \ge k} E_j$. Then $(F_k)$ is an ascending sequence, so $\mu(\bigcup_k F_k) = \lim_k \mu(F_k)$. Hence, $v b
\end{proof}


\p{2} Assume $\mu$ is finitely additive on a sigma algebra $\mathcal M$

a) $\mu$ is $\sigma$-additive $\leftrightarrow$ $\u$ is continuous from below.

b) Assume $\mu(X) < \infty$. Then $\mu$ is $\sigma$-additive $\leftrightarrow$ $\u$ is continuous from above.

\p{3} Suppose $(X, \mathcal M, \mu)$ is a measure space. We call
$$ \mathcal N = \{ A \subset X : \exists B \in \mathcal M  \: A \subset B \text { and } \mu(B) = 0 \}$$
the \emph{nullsets} of $(X \mathcal M, \mu)$.

a) Show that 
$$ \widebar{\mathcal M} = \{ A \cup N : A \in \mathcal M \text{ and } N \in \mathcal N \}$$
is a $\sigma$-algebra.

b) Show that
$$ \widebar{\mu}: \widebar{\mathcal M} \to [0, \infty], \: A \cup N \mapsto \mu(A), \text{ if } A \in \mathcal M , N \in \mathcal N $$
is well-defined and a measure.

\p{4} Let $(X , \mathcal M, \mu)$ be a finite measure space. 

a) If $E, F \in \mathcal M$ and $\mu(E \Delta F) = 0$ then $\mu(E) = \mu(F)$.

b) We say that $E \tilde F$ if $\mu(E \Delta F) = 0$. Show that $\tilde$ is an equivalence relation.

c) For $E, F \in \mathcal M$ put $\rho(E, F) = \mu(E \Delta F)$, show that $\rho$ induces a metric on $\mathcal M / \tilde$.

\p{5} If $\mu*$ is an outer measure on $X$ and $(A_j)_{j \in \N}$ a sequence of disjoint $\mu^*$-measurable sets, then 
$\mu^*(E \cap (\bigcup_{j=1}^\infty A_j) = \sum_{j=1}^\infty \mu^*(E \cap A_j)$ for any $E \subset X$.

\p{6} Assume that the algebra $\mathcal A$ generates the $\sigma$-algebra $\mathcal M$ and assume that $\mu$ is a finite measure on $\mathcal M$. Show that for any $\epsilon > 0$ and any $A \in \mathcal M$, there is an $~A \in \mathcal A$ so that $\mu(A \delta ~A) < \epsilon$.




\end{document}
