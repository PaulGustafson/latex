\documentclass{article}
\usepackage{amssymb, amsmath, amsthm, verbatim}

\newtheorem{theorem}{Theorem}
\newtheorem{definition}{Definition}
\newtheorem{corollary}{Corollary}
\newtheorem{lemma}{Lemma}
\newtheorem{example}{Example}

\newcommand{\diam}{\mathrm{diam}}
\newcommand{\length}{\mathrm{length}}
\newcommand{\R}{\mathbb{R}}
\newcommand{\N}{\mathbb{N}}
\newcommand{\m}{m^*_\alpha}

\begin{document}
\noindent Paul Gustafson\\
\noindent Texas A\&M University - Math 482 \\
\noindent Instructor: Dr. David Larson

\section*{Cantor Sets in $\R$}
Recall that a \emph{perfect} set is a set for which every point is a limit point. A set $S$ is called \emph{totally disconnected} if for every $x, y \in S$, there exist disjoint open sets $U,V \subset S$ such that $x \in U$, $y \in V$, and $U \cup V = S$.

\begin{definition}
  A \emph{Cantor set} is a non-empty, totally disconnected, perfect, compact metric space.
\end{definition}

\begin{example}
Let $C_0 := [0,1]$, $C_1 := [0,1/3] \cup [2/3, 1]$, and $C_2 = [0,1/9] \cup [2/9, 1/3] \cup [2/3, 7/9] \cup [8/9, 1]$.
Similarly, for $i > 2$, let $C_i$ be the closed set given by removing the open middle third of each interval of $C_{i-1}$.
The \emph{ternary Cantor set} $$\Delta := \bigcap_{i=0}^\infty {C_i}$$ is a Cantor set.
\end{example}
\begin{proof}
Since $0 \in C_i$ for all $i$, $\Delta$ is non-empty. Since each interval in $C_i$ is of length $3^{-i}$, $\Delta$ is totally disconnected. It is closed and bounded,
so compact by the Heine-Borel theorem. 

To see that $\Delta$ is perfect, first note that the endpoints of any interval in any $C_i$ remain endpoints of intervals in $C_{i+1}$, and $C_{i+1} \subset C_i$.
Hence, every point that is an endpoint of an interval in some $C_i$ is in $\Delta$. Now, fix $x \in \Delta$. Given $\epsilon > 0$, there exists a $C_i$ whose intervals are of length less than $\epsilon$. Hence, both endpoints of the interval in $C_i$ containing $x$ are within $\epsilon$ of $x$, and are members of $\Delta$.  Thus, $x$ is a limit point, so $\Delta$ is perfect.
\end{proof}


% ${\{0,1\}}^\mathbb{N}$.

\begin{theorem}\label{subc}
Let $K$ be a Cantor set. If $A \subset K$ is nonempty and clopen, then $A$ is Cantor.
\end{theorem}
\begin{proof}
$A$ is compact since it is closed in $K$, and totally disconnected since it is open. 
To see that $A$ is perfect, let $x \in A$. Since $K$ is perfect, there exists a sequence $\left(x_n\right) \subset K$ such that $x_n \rightarrow x$.
Since $A$ is open, all but a finite number of $x_n$ lie in A.
\end{proof}

\begin{theorem}\label{hom}
If $A \subset \mathbb{R}$ is a Cantor set, then there is a order-preserving homeomorphism $f:A \rightarrow {\{0,1\}}^\mathbb{N}$, where $\{0,1\}^\N$ is ordered lexicographically and equipped with the product metric $d(x,y) = \sum_{i=1}^n |x(i) - y(i)| 2^{-n}$.
\end{theorem}
\begin{proof}
Step 1. Let $a := \inf (A)$, and $d := \sup (A) - a = \mathrm{diam}(A)$. Since $A$ is totally disconnected, there exists $c \in [a + \frac d 4, a + \frac {3d} 4] \setminus A$. Then $M_0 := (-\infty, c) \cap A $ and $M_1 := (c, \infty) \cap A$ are clopen relative to $A$, hence Cantor sets by Theorem~\ref{subc}. Moreover, $\mathrm{diam}(M_i) \leq \frac 3 4 \mathrm{diam}(A)$ for $i = 0,1$.

Step 2. For $n > 1$, apply Step 1 to $M_t$ for each $t \in {\{0,1\}}^{n-1}$ to get clopen Cantor sets $M_{t,0}, M_{t,1} \subset M_t$ with $M_{t,0} < M_{t,1}$ and $\mathrm{diam}(M_{t,i}) \leq \frac 3 4 \mathrm{diam}(M_t)$ for $i=0,1$. By recursion on $n$, for all $r,s \in {\{0,1\}}^n$ we have $\mathrm{diam}(M_s) \leq \left(\frac 3 4\right)^n \mathrm{diam} (A)$, and if $r < s$ in the lexicographical ordering then $M_r < M_s$, i.e. $x \in M_r, y\in M_s$  implies $x < y$. Moreover, for any fixed $n$, $A = \bigcup_{s \in {\{0,1\}}^n} M_s$.

Step 3. Fix $x \in A$. The construction in Step 2 generates a descending sequence of sets ${\left(M_{t_n}\right)}_{t_n \in {\{0,1\}}^n}$, each containing $x$. Since for all $n$ we have $t_{n+1} =t_n,i$ for some $i \in \{0,1\}$, this sequence of sets determines a unique element $f(x) \in {\{0,1\}}^\mathbb{N}$ such that, for any $n$, the first $n$ entries of $f(x)$ are $t_n$. To see that $f$ is bijective, note that if $t \in {\{0,1\}}^\mathbb{N}$ and $t_n = (t(1), t(2), ... t(n))$, then $f^{-1}(t) = \bigcap_{n = 1}^\infty {M_{t_n}}$ contains exactly one point, since $M_{t_n}$ is a descending chain of compact sets with diameters going to $0$.

To see that $f$ is continuous, let $x \in A$. If $x_m \rightarrow x$ then, for every $M_{t_n}$ containing $x$, all but finitely many $x_m$ lie in $M_{t_n}$ since $M_{t_n}$ is open relative to A. Thus, $f(x_m) \rightarrow f(x)$ since $\diam(f(M_{t_n})) = 2^{-n} \rightarrow 0$ as $n \rightarrow \infty$. Since $A$ is compact, the continuity of $f$ implies $f^{-1}$ is also continuous.

To see that $f$ is order-preserving, if $x < y$ there exists $n$ so large that $x \in M_s, y \in M_t$ for $s,t$ of length $n$ with $s \neq t$. By Step 2, this implies $s < t$.
Hence, $f(x) < f(t)$.
\end{proof}


\begin{theorem}\label{cfunction}
If $S \subset R$ is a Cantor set, there exists a nondecreasing, onto, continuous function $g:S \rightarrow [0,1]$.
\end{theorem}
\begin{proof}
Let $h:{\{0,1\}}^\N \rightarrow [0,1]$ be defined by $h(x) = \sum_{i=0}^\infty x(i)2^{-i}$. Defining $f$ as in Theorem~\ref{hom}, let $g = h \circ f$. Thus, it suffices to show that $h$ is nondecreasing, onto, and continuous.

Let $x,y \in {\{0,1\}}^\N$. Then $|h(x) - h(y)| = |\sum_{i=0}^\infty (x(i)-y(i))2^{-i}| \leq \sum_{i=0}^\infty |x(i)-y(i)|2^{-i} = d(x,y)$, so $h$ is continuous. If $x<y$, then there exists a minimal $n$ such that $x(n) \neq y(n)$. By the definition of lexicographical ordering, $x(n) = 0$ and $y(n) = 1$.  Thus, $h(y) - h(x) = \sum_{i=n}^\infty (y(i)-x(i))2^{-i} = 2^{-n} + \sum_{i=n+1}^\infty (y(i)-x(i))2^{-i} \geq 2^{-n} + \sum_{i=n+1}^\infty (-1)2^{-i} = 0$. Hence, $h$ is nondecreasing. To see that $h$ is onto, let $E_n := \{x \in {\{0,1\}}^\N : x(i) = 0 \text{ for all } i>n\}$. Then each $h(E_n)$ is a $2^{-n+1}$-net for $[0,1]$, so the image of $h$ is dense in $[0,1]$.  Since $S$ is compact, $h(S)$ is compact, so $h$ is onto.
\end{proof}



\begin{lemma}\label{continuous}
If $f:[a,b] \rightarrow [0,1]$ is nondecreasing and onto, then $f$ is continuous.
\end{lemma}
\begin{proof}
Let $c \in (a,b]$.  Since $f$ is nondecreasing, $\sup_{x < c} {f(x)} \leq f(c) = \inf_{x \geq c} {f(x)}$. Hence, since $f$ is onto, $\sup_{x < c} {f(x)} = f(c)$.
To see that $f(c-) = f(c)$, set $\epsilon > 0$. By the definition of supremum, there exists $a<c$ such $f(c) - f(a) < \epsilon$. Then if $a<x<c$, since $f$ is nondecreasing, 
$f(c) - f(x) < \epsilon$. Hence, $f(c-) = f(c)$.  The proof for right continuity is analogous.
\end{proof}

\begin{lemma}\label{split}
Every compact metric space K can be written as 
$K = A \cup B$, where $A$ is perfect (hence compact), $B$ is countable, and $A \cap B = \emptyset$.
\end{lemma}
\begin{proof}
Let $U$ be a countable base for $K$. Let $V := \{S \in U : S \text{ is countable}\}$, and $W := U \setminus V$. Then $B:= \bigcup_{S\in V}{S}$ is countable and open.
Let $A:= K\setminus B$. Then A is closed, hence compact. 

I claim that $\widetilde{W} := \{S \cap A : S \in W\}$ is a base for the topology of $A$.  Suppose $C \subset A$ is open in $A$, and $x \in C$. Then $C \cup B$ is open in $K$, so there exists $S \in U$ with $x \in S \subset (C \cup B)$. Since $x \notin B$, $S$ cannot be countable, so $S \in W$. Hence, $x \in S \cap A \subset C$, so $\widetilde{W}$ is a base for $A$.

Note that every element of $W$ is uncountable, so, since $B$ is countable, every element of $\widetilde{W}$ is also uncountable. Thus, $A$ has no isolated points, so $A$ is perfect.
\end{proof}

\begin{definition}
Given an nondecreasing function $\alpha:\mathbb{R} \rightarrow \mathbb{R}$, the \emph{$\alpha$-exterior measure} of a set $E \subset \mathbb{R}$ is defined to be
$$\m(E) := \inf\{\sum_{i=1}^\infty {\alpha(b_i) - \alpha(a_i)}: E \subset \bigcup_{i=1}^\infty {(a_i,b_i)} \}$$
\end{definition}

\begin{theorem}
If $E \subset R$ is closed and $\m(E) = 0$ for all nondecreasing, continous $\alpha:\mathbb{R} \rightarrow \mathbb{R}$, then $E$ is countable.
\end{theorem}
\begin{proof}
Suppose $E$ were uncountable. If $E$ contains a nontrivial interval, then let $\alpha$ be the identity. Since $E$ contains an interval, it contains a compact set of the form $[a,b]$ for $a<b$. Hence, any cover of $E$ by open intervals must contain a finite subcover of $[a,b]$. The sum of the lengths of intervals in this subcover must be at least $b-a$, so $\m(E) \geq b - a > 0$, a contradiction.

Suppose $E$ does not contain any nontrivial intervals. Note that $E \cap [n,n+1]$ must be uncountable for some $n$, so WLOG, $E$ is compact. Then, by Lemma~\ref{split}, $E = A \cup B$ where $A$ is a Cantor set and $B$ is countable. Since $A \subset E$, $\m(A) \leq \m(E)$, so it suffices to show that $\m(A) > 0$.

Let $f:A \rightarrow [0,1]$ be the increasing, onto, continuous function defined in Theorem~\ref{cfunction}. Define
 \begin{displaymath}
   \alpha(x) = \left\{
     \begin{array}{lr}
       0 & : x \leq \inf(A)\\
       \sup\{f(y): y \in A \cap (-\infty, x)\} & : x > \inf(A)
     \end{array}
   \right.
\end{displaymath} 
Since $A$ is closed and $f$ is onto $[0,1]$, $\alpha$ is onto $[0,1]$. Also, $\alpha$ is clearly nondecreasing. Since $\alpha$ is constant outside $(\inf(A), \sup(A))$, Lemma~\ref{continuous} implies $\alpha$ is continuous.

Let $U$ be a cover of $A$ by open intervals. Since $A$ is compact, there exists a finite subcover $F \subset U$. Denote the elements of $F$ by  $((a_i, b_i))_{i=1}^n$, sorted so that $a_i \leq a_{i+1}$ for all $i<n$. If $b_{i+1} < b_i$ for some $i < n$, then $(a_{i+1}, b_{i+1}) \subset (a_i, b_i)$. Since $F$ is finite, we can recursively throw out all such redundant sets. This procedure only reduces the sum of interval lengths of $F$, so we may assume $b_i \leq b_{i+1}$ for all $i<n$. For $i < n$, if $b_i \geq a_{i+1}$, then $\alpha(b_i) - \alpha(a_{i+1}) \geq 0$ since $\alpha$ is nondecreasing. On the other hand, if $b_i < a_{i+1}$, then $\alpha(b_i) - \alpha(a_{i+1}) = 0$ since $A \cap [b_i, a_{i+1}] = \emptyset$. 

Thus, $\sum_{i=1}^n {\alpha(b_i) - \alpha(a_i)} \geq \alpha(b_n) - \alpha(a_1) = 1$. Hence, $\m(A) \geq 1$.
\end{proof}

\begin{comment}
\begin{corollary}
Let $f:\R\rightarrow\R$, and $D(f)$ denote the set of discontinuities of $f$. If $\m(D(f)) = 0$ for all nondecreasing, continous $\alpha:\mathbb{R} \rightarrow \mathbb{R}$, then $D(f)$ is countable.
\end{corollary}
\end{comment}
\end{document}
