\documentclass{article}
\usepackage{../m}

\begin{document}
\noindent Paul Gustafson\\
\noindent Representations of Lie Algebras

\subsection*{HW 1}
\p 1 Prove that over an algebraically closed field of characteristic not equal 2, if $B$ is a nondegenerate symmetric bilinear form over an f.d. vector space $V$, then $\mathfrak{so}(V,B) \simeq \mathfrak{so}(V)$.
\begin{proof}
I claim that there exists a vector $v \in V$ such that $B(v,v) \neq 0$.  Suppose not.  By nondegeneracy, there exist vectors $x,y$ with $B(x,y) \neq 0$.  By assumption, $0 = B(x+y, x+y) = B(x,x) + 2B(x,y) + B(y,y) = 2B(x,y)$, a contradiction.

Thus, we may choose a vector $v$ such that $B(v,v) \neq 0$. By dividing if necessary, WLOG $B(v,v) = 1$.  Let $W = \{w \in V : B(v,w) = 0\}$.  

I claim that $W$ has codimension $1$.  Suppose $x \in V$.  Let $p = B(x,v) v$.   It suffices to show that $x - p \in W$.  We have $B(x-p, v) = B(x,v) - B(p,v) = B(x,v) - B(x,v) B(v,v) = 0$.  Thus $V = W + \lambda v$.  Since $B(v,v) \neq 0$, this is a direct sum.

By induction, we get an orthonormal basis $v_1, \ldots, v_n$ for $V$ with respect to $B$.
\end{proof}

\p 2  Show that $(\RR^3, \times)$ is a real Lie algebra.  Is this related to $\mathfrak{su}(2)$?
\begin{proof}
The cross-product is antisymmetric.  For the Jacobi identity, we only need to check each cyclic ordering of basis vectors $i:= e_1, j:= e_2, k:= e_3$.  For the cyclic ordering $(ijk)$, we have
$i \times (j \times k) + k \times (i \times j) + j \times (k \times i) = i \times i + k \times k + j \times j = 0$.  The other cyclic ordering is similar.

For the comparison to $\mathfrak{su}(2)$, first we have the commutation relations
\begin{align*}
[i\sigma_x, i\sigma_y] & = -2 i\sigma_z \\
[i\sigma_y, i\sigma_z] & = -2 i\sigma_x \\
[i\sigma_z, i\sigma_x] & = -2 i\sigma_y 
\end{align*}
Thus, by swapping the cyclic orders, we get an isomorphism  $(\RR^3, \times) \simeq \mathfrak{su}(2)$.  For example $i \mapsto \frac{i}{\sqrt 2} \sigma_x$, $j \mapsto \frac{i}{\sqrt 2} \sigma_z$, and $k \mapsto \frac{i}{\sqrt 2} \sigma_y$.
\end{proof}

\p 3 Consider $\HH$.
\begin{proof}
The quaternions are not a Lie algebra.  For example, the Jacobi identity fails since
$1ij + j1i + ij1 = k -k + k \neq 0$.  However if we mod out the subspace spanned by $1$,
 we have $\HH/\RR \simeq (\RR^3, \times) \simeq  \mathfrak{su}(2)$ as real Lie algebras.
\end{proof}

\end{document}
