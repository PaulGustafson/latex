\documentclass{article}
\usepackage{../m}
\usepackage{amsmath,amssymb,amsthm,fullpage,mathptmx, hyperref,dsfont,framed, graphicx, subcaption, textcomp} 

\DeclareMathOperator{\hht}{ht}
\DeclareMathOperator{\Id}{Id}
\DeclareMathOperator{\diag}{diag}
\newcommand{\mf}{\mathfrak}

\newtheorem{thm}{Theorem}[section]
\newtheorem*{uthm}{Theorem}
\newtheorem{lem}[thm]{Lemma}
\newtheorem*{ulem}{Lemma}
%\newtheorem{prop}[thm]{Proposition}
%\newtheorem*{uprop}[thm]{Proposition}
\newtheorem{cor}[thm]{Corollary}
\newtheorem{conj}[thm]{Conjecture}
\newtheorem{defn}[thm]{Definition}
\newtheorem{rmk}[thm]{Remark}
\newtheorem{prob}[thm]{Open problem}
\newtheorem{ques}[thm]{Question}
\newtheorem{fact}[thm]{Fact}
\newtheorem{ex}[thm]{Exercise}


\begin{document}
\noindent Paul Gustafson\\
\noindent Representations of Lie Algebras


\subsection*{HW 3}
\p 1 Prove that the roots systems for $\mf{so}(5)$ and $\mf{sp}(4)$ are isomorphic.

\begin{proof}
We already proved in class that the root system for $\mf{so}(5)$ is isomorphic to 
$$( \{ (\pm 1, 0) , (0, \pm 1), (\pm 1, \pm 1)\}, \RR^2 ).$$

Using the representation of $\mf{sp}(4)$ on pages 72-73 of Goodman and Wallach,
the Lie algebra $\mf{sp}(4)$ consists of all matrices
$$ A = 
\begin{pmatrix}
    a_{11} & a_{12} & b_{11} & b_{12}
\\  a_{21} & a_{22} & b_{21} & b_{11}
\\  c_{11} & c_{12} & -a_{22} & -a_{12}
\\  c_{21} & c_{11} & -a_{21} & -a_{11}
\end{pmatrix}
$$
with basis
$$\{e_{11} - e_{44}, e_{22} - e_{33}, e_{12} - e_{34}, e_{21} - e_{43},  
e_{13} + e_{24}, e_{14}, e_{23}, e_{31} + e_{42}, e_{32}, e_{41} \}$$

The choice of Cartan subalgebra $\mf{h}$ is the span of the first two basis elements,
$x_1 := e_{11} - e_{44}$ and $x_2 := e_{22} - e_{33}$.

The adjoint action of the Cartan subalgebra $\mf{h}$ on $\mf{sp}(4)$ is 
$$
\begin{array}{c}
    \ad(x_1) = \diag(0, 0, 1, -1, 1, 2, 0, -1, 0, -2)
 \\ \ad(x_2) = \diag(0, 0, -1, 1, 1, 0, 2, -1, -2, 0)
\end{array}
$$

Up to rearragement of basis vectors, this is the same as
the adjoint action of the Cartan subalgebra of $\mf{so}(5)$ on $\mf{so}(5)$
that we calculated in class.  Since the adjoint action of a choice of Cartan subalgebra
on the full Lie algebra determines the root system, it follows that the corresponding root systems
are isomorphic.
\end{proof}

\p 2 Let $(R,E)$ be a roots system, with Weyl group $W$.  Show that $W$ is a normal subgroup of the group of automorphisms of $(R,E)$ (that is, the group of linear automorphisms of $E$, preserving $R$ as a set, and preserving the Cartan integers). 
\begin{proof}
This follows from the Lemma on page 43 of Humphreys.  
\end{proof}

\p 3 Fill in the details of the proofs of the results in 10.2 and 10.3 of Humphreys book.

\begin{lem} If $\alpha$ is positive but not simple, then $\alpha - \beta$ is a root 
(necessarily positive) for some $\beta \in \Delta$
\end{lem}
\begin{proof}
If $(\alpha, \beta) \le 0$ for all $\beta \in \Delta$, suppose
$0 = \sum_{\beta \in \Delta \cup \{ \alpha \}} r_\beta \beta$. Let $\delta \in \mf{C}(\Delta)$.
 Separating into sets for which $r_\beta \ge 0$ and $r_\beta \le 0$, we can rewrite this as
$\sum s_\beta \beta = \sum t_\gamma \gamma$ where $s_\beta$ and $t_\gamma$
are nonnegative with disjoint support.  Let $\epsilon$ denote the value
of these two sums.  Then
$$(\epsilon, \epsilon) = \sum_{\beta, \gamma \in \beta \in \Delta \cup \{ \alpha \}} s_\beta t_\gamma (\beta, \gamma).$$
If $\beta \neq \gamma \in \Delta$, then $(\beta, \gamma) \le 0$ since $\Delta$ is a root base. By assumption,
$(\alpha, \beta) \le 0$ for all $\beta \in \Delta$.  Hence, 
$(\epsilon, \epsilon) \le 0,$
so $\epsilon = 0$ since the inner product is positive-definite.  
Since $\alpha$ is a positive root, $(\delta, \alpha) > 0$.  Hence,
$$0 = (\delta, \epsilon) = \sum_{\beta \in \Delta \cup \{ \alpha \}} s_\beta (\delta, \beta) = \sum_{\gamma \in \Delta \cup \{ \alpha \}} t_\gamma (\delta, \gamma).$$
Thus, $s_\beta = t_\gamma = 0$ for all $\beta, \gamma \Delta \cup \{ \alpha \}$.  Hence, $\Delta \cup \{ \alpha \}$ is linearly independent, a contradiction.

Thus, there exists $\beta \in \Delta$ such that $(\alpha, \beta) > 0$.  Hence, Lemma 9.4 implies that $\alpha - \beta$ is a root.
\end{proof}

\begin{cor} Each $\beta \in \Phi^+$ can be written in the form $\alpha_1 + \ldots + \alpha_k$ ($\alpha_i \in \Delta$, not
necessarily distinct) in such a way that each partial sum is a root.
\end{cor}
\begin{proof}
Use the lemma and induction on $\hht(\beta)$.  Given $\beta \in \Phi^+$ not simple, write it as a sum of simple roots with $\hht(\beta)$ terms.
Applying the lemma, we get another positive root with lower height.
\end{proof}

\begin{lem}  Let $\alpha$ be simple.  Then $\sigma_\alpha$ permutes the positive roots other than $\alpha$.
\end{lem}
\begin{proof}
The proof in the book shows that $\sigma_\alpha$ maps $\Phi^+ - \{\alpha\}$ to itself.  Since $\sigma_\alpha$ is invertible, its restriction to $\Phi^+ - \{\alpha\}$ must be a permutation.
\end{proof}

\begin{cor} Set $\delta = \frac{1}{2} \sum_{\beta \succ 0} \beta$. Then $\sigma_\alpha(\delta) = \delta - \alpha$ for all $\alpha \in \Delta$.
\end{cor}
\begin{proof}
The map $\sigma_\alpha$ permutes the roots other than $\alpha$, but maps $\alpha$ to $-\alpha$.  
Thus $\sigma_\alpha(\delta) - \delta = \frac{1}{2} (-\alpha - \alpha) = -\alpha$.
\end{proof}

\begin{lem} Let $\alpha_1, \ldots, \alpha_t \in \Delta$ (not necessarily distinct. Write $\sigma_i = \sigma_{\alpha_i}$.
If $\sigma_1 \ldots \sigma_{t-1}(\alpha_t)$ is negative, then for some index 
$1 \le s < t$, $\sigma_1 \ldots \sigma_t = \sigma_1 \ldots \sigma_{s-1} \sigma_{s+1} \ldots
\sigma_{t-1}$.
\end{lem}
\begin{proof}
A slight clarification of the last sentence of the proof:
\begin{align*}
\sigma_s & = \sigma_{\alpha_s}
\\ & = \sigma_{\beta_s}
\\ & = \sigma_{\sigma_{s+1} \ldots \sigma_{t-1}(\alpha_t)}
\\ & = \sigma_{s+1} \ldots \sigma_{t-1} \sigma_{\alpha_t}   (\sigma_{s+1} \ldots \sigma_{t-1})^{-1}
\\ & = \sigma_{s+1} \ldots \sigma_{t-1} \sigma_t   (\sigma_{s+1} \ldots \sigma_{t-1})^{-1}
\\ & = \sigma_{s+1} \ldots \sigma_{t-1}  (\sigma_{s+1} \ldots \sigma_{t-1} \sigma_t)^{-1}
\end{align*}
Substituting this value for $\sigma_s$ into the product $\sigma_1 \ldots \sigma_t$ gives
the desired identity.
\end{proof}

\begin{cor} If $\sigma = \sigma_1 \ldots \sigma_2$ is an experession for $\sigma \in \mathcal W$ in terms of reflections corresponding to simple roots, with $t$ as small as possible, then $\sigma(\alpha_t) \prec 0$.
\end{cor}
\begin{proof}
Suppose not.  Then 
\begin{align*}
0 & \prec \sigma(\alpha_t) 
\\ &  = \sigma_1 \ldots \sigma_t (\alpha_t)
\\ &  = - \sigma_1 \ldots \sigma_{t-1} (\alpha_t)
\end{align*}
This implies that $\sigma_1 \ldots \sigma_{t-1} (\alpha_t)$ is negative, which contradicts the lemma.
\end{proof}

\begin{thm} Let $\Delta$ be a base of $\Phi$.
\begin{enumerate}[(a)]
\item If $\gamma \in E$, $\gamma$ regular, there exists $\sigma \in \mathcal W$ such that $(\sigma(\gamma), \alpha) > 0$ for all $\alpha \in \Delta$ (so $\mathcal W$ acts transitively on Weyl chambers).
\item If $\Delta'$ is another base of $\Phi$, then $\sigma(\Delta') = \Delta$ for some $\sigma \in \mathcal W$ (so $\mathcal W$ acts transitively on bases).
\item If $\alpha$ is any root, there exists $\sigma \in \mathcal W$ such that $\sigma(\alpha) \in \Delta$.
\item $\mathcal W$ is generated by the $\sigma_\alpha$ ($\alpha \in \Delta$).
\item If $\sigma(\Delta = \Delta$, $\sigma \in \mathcal W$, then $\sigma = 1$ (so $\mathcal W$ acts simply transitively on bases.
\end{enumerate}
\end{thm}
\begin{proof}
\begin{enumerate}[(a)]
\item The only thing to add here is that  the fact that $\sigma_\alpha$ is orthogonal follows from considering its action on $\RR \alpha \oplus \alpha^\perp$.
\item Weyl chambers are in 1-1 correspondence with bases.  Since $\mathcal W'$ consists of orthogonal linear maps,  positive roots for a Weyl chamber get mapped to positive roots for the image of that Weyl chamber, and simple roots get mapped to simple roots.
\item Suppose $\alpha$ were decomposable with respect to the base $\Delta(\gamma')$.  Then $\alpha$ is a $\Z_+$-linear combination of elements $\alpha$ of $\Delta(\gamma')$, each of which satisfies $(\alpha, \gamma') > \epsilon$, a contradiction.
\item The proof in Humphreys is clear.
\item Suppose not.  The we can write $\sigma = \sigma_1 \ldots \sigma_t$, with $\alpha_t$ the simple root corresponding to $\sigma_t$, as in Lemma 10.2C and its corollary.  Since $\sigma(\Delta) = \Delta$ and $\alpha_t$ is a positive root,  $\sigma(\alpha_t)$ is a positive root by the orthogonality of $\sigma$.  This contradicts the corrollary to Lemma 10.2C.
\end{enumerate}
\end{proof}

\begin{lem} For all $\sigma \in \mathcal W$, $l(\sigma) = n(\sigma)$, where $l$ is the length and $n(\sigma)$ is the number of positive roots for which $\sigma(\alpha) \prec 0$.
\end{lem}
\begin{proof}
To see that $n(\sigma \sigma_\alpha) = n(\sigma) -1$,  Lemma 10.2B says that $\sigma_\alpha$ permutes the positive roots other than $\alpha$.   Hence the only possible change between $n(\sigma)$ and $n(\sigma \sigma_\alpha)$ is due to $\alpha$. As mentioned earlier in the proof, $\sigma(\alpha) \prec 0$.  Thus $\sigma \sigma_\alpha (\alpha) = - \sigma(\alpha) \succ 0$.  Hence $n(\sigma \sigma_\alpha) = n(\sigma) -1$.
\end{proof}

\begin{lem} Let $\lambda, \mu \in \overline{\mf{C}(\Delta)}$.  If $\sigma \lambda = \mu$ for some $\sigma \in \mathcal W$, then $\sigma$ is a product of simple reflections which fix $\lambda$; in particular, $\lambda = \mu$.
\end{lem}
\begin{proof}
The proof in the book is clear.
\end{proof}
\end{document}
