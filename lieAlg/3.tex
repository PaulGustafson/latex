\documentclass{article}
\usepackage{../m}

\DeclareMathOperator{\Id}{Id}
\newcommand{\mf}{\mathfrak}

\newtheorem*{uthm}{Theorem}
\newtheorem{lem}[thm]{Lemma}
\newtheorem*{ulem}{Lemma}
\newtheorem{prop}[thm]{Proposition}
%\newtheorem*{uprop}[thm]{Proposition}
\newtheorem{cor}[thm]{Corollary}
\newtheorem{conj}[thm]{Conjecture}
\newtheorem{defn}[thm]{Definition}
\newtheorem{rmk}[thm]{Remark}
\newtheorem{prob}[thm]{Open problem}
\newtheorem{ques}[thm]{Question}
\newtheorem{fact}[thm]{Fact}
\newtheorem{ex}[thm]{Exercise}


\begin{document}
\noindent Paul Gustafson\\
\noindent Representations of Lie Algebras


\subsection*{HW 3}
\p Prove that the roots systems for $\mf{so}(5)$ and $\mf{sp}(4)$ are isomorphic.

\begin{proof}
We already proved in class that the root system for $\mf{so}(5)$ is isomorphic to 
$$\{(\pm 1, 0) , (0, \pm 1), (\pm 1, \pm)}, \RR^2\}.$$

Using the representation of $\mf{sp}(4)$ on pages 72-73 of Goodman and Wallach,
the Lie algebra $\mf{sp}(4)$ consists of all matrices
$$ A = 
\begin{pmatrix}
    a_{11} & a_{12} & b_{11} & b_{12}
\\  a_{21} & a_{22} & b_{21} & b_{11}
\\  c_{11} & c_{12} & -a_{22} & -a_{12}
\\  c_{21} & c_{11} & -a_{21} & -a_{11}
\end{pmatrix}
$$
with basis
$$\{e_{11} - e_{44}, e_{22} - e_{33}, e_{12} - e_{34}, e_{21} - e_{43},  
e_{13} + e_{24}, e_{14}, e_{23}, e_{31} + e_{42}, e_{32}, e_{41} \}$$

The choice of Cartan subalgebra $\mf{h}$ is the span of the first two basis elements,
$x_1 := e_{11} - e_{44}$ and $x_2 := e_{22} - e_{33}$.

The adjoint action of the Cartan subalgebra $\mf{h}$ on $\mf{sp}(4)$ is 
$$
\begin{array}{c}
    \ad(x_1) = \diag(0, 0, 1, -1, 1, 2, 0, -1, 0, -2)
 \\ \ad(x_2) = \diag(0, 0, -1, 1, 1, 0, 2, -1, -2, 0)
\end{array}
$$

Up to rearragement of basis vectors, this is the same as
the adjoint action of the Cartan subalgebra of $\mf{so}(5)$ on $\mf{so}(5)$
that we calculated in class.  Since the adjoint action of a choice of Cartan subalgebra
on the full Lie algebra determines the root system, it follows that the corresponding root systems
are isomorphic.
\end{proof}

\p Let $(R,E)$ be a roots system, with Weyl group $W$.  Show that $W$ is a normal subgroup of the group of automorphisms of $(R,E)$ (that is, the group of linear automorphisms of $E$, preserving $R$ as a set, and preserving the Cartan integers). 
\begin{proof}
This follows from the Lemma on page 43 of Humphreys.  In fact, any automorphism of $E$ preserving $R$ automatically preserves the Cartan integers.
\end{proof}

\p Fill in the details of the proofs of the results in 10.2 and 10.3 of Humphreys book.

\begin{lem} If $\alpha$ is positive but not simple, then $\alpha - \beta$ is a root 
(necessarily positive) for some $\beta \in \Delta$
\end{lem}
\begin{proof}
If $(\alpha, \beta) \le 0$ for all $\beta \in \Delta$, suppose
$0 = \sum_{\beta \in \Delta \cup \{ \alpha \}} r_\beta \beta$. Let $\delta \in \mf{C}(\Delta)$.
 Separating into sets for which $r_\beta \ge 0$ and $r_\beta \le 0$, we can rewrite this as
$\sum s_\beta \beta = \sum t_\gamma \gamma$ where $s_\beta$ and $t_\gamma$
are nonnegative with disjoint support.  Let $\epsilon$ denote the value
of these two sums.  Then
$$(\epsilon, \epsilon) = \sum_{\beta, \gamma \in \beta \in \Delta \cup \{ \alpha \}} s_\beta t_\gamma (\beta, \gamma).$$
If $\beta \neq \gamma \in \Delta$, then $(\beta, \gamma) \le 0$ since $\Delta$ is a root base. By assumption,
$(\alpha, \beta) \le 0$ for all $\beta \in \Delta$.  Hence, 
$(\epsilon, \epsilon) \le 0,$
so $\epsilon = 0$ since the inner product is positive-definite.  
Since $\alpha$ is a positive root, $(\delta, \alpha) > 0$.  Hence,
$$0 = (\delta, \epsilon) = \sum_{\beta \in \Delta \cup \{ \alpha \}} s_\beta (\delta, \beta) = \sum_{\gamma \in \Delta \cup \{ \alpha \}} t_\gamma (\delta, \gamma).$$
Thus, $s_\beta = t_\gamma = 0$ for all $\beta, \gamma \Delta \cup \{ \alpha \}$.  Hence, $\Delta \cup \{ \alpha \}$ is linearly independent, a contradiction.

Thus, there exists $\beta \in \Delta$ such that $(\alpha, \beta) > 0$.  Hence, $\alpha - \beta$ is a root by Lemma 9.4.
\end{proof}

\end{document}
