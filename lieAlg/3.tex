\documentclass{article}
\usepackage{../m}

\DeclareMathOperator{\Id}{Id}
\newcommand{\mf}{\mathfrak}


\begin{document}
\noindent Paul Gustafson\\
\noindent Representations of Lie Algebras


\subsection*{HW 3}
\p Prove that the roots systems for $\mf{so}(5)$ and $\mf{sp}(4)$ are isomorphic.

\begin{proof}
We already proved in class that the root system for $\mf{so}(5)$ is isomorphic to 
$$\{(\pm 1, 0) , (0, \pm 1), (\pm 1, \pm)}, \RR^2\}.$$

Using the representation of $\mf{sp}(4)$ on pages 72-73 of Goodman and Wallach,
the Lie algebra $\mf{sp}(4)$ consists of all matrices
$$ A = 
\begin{pmatrix}
    a_{11} & a_{12} & b_{11} & b_{12}
\\  a_{21} & a_{22} & b_{21} & b_{11}
\\  c_{11} & c_{12} & -a_{22} & -a_{12}
\\  c_{21} & c_{11} & -a_{21} & -a_{11}
\end{pmatrix}
$$
with basis
$$\{e_{11} - e_{44}, e_{22} - e_{33}, e_{12} - e_{34}, e_{21} - e_{43},  
e_{13} + e_{24}, e_{14}, e_{23}, e_{31} + e_{42}, e_{32}, e_{41} \}$$

The choice of Cartan subalgebra $\mf{h}$ is the span of the first two basis elements,
$x_1 := e_{11} - e_{44}$ and $x_2 := e_{22} - e_{33}$.

The adjoint action of the Cartan subalgebra $\mf{h}$ on $\mf{sp}(4)$ is 
$$
\begin{array}{c}
    \ad(x_1) = \diag(0, 0, 1, -1, 1, 2, 0, -1, 0, -2)
 \\ \ad(x_2) = \diag(0, 0, -1, 1, 1, 0, 2, -1, -2, 0)
\end{array}
$$

Up to rearragement of basis vectors, this is the same as
the adjoint action of the Cartan subalgebra of $\mf{so}(5)$ on $\mf{so}(5)$
that we calculated in class.  Since the adjoint action of a choice of Cartan subalgebra
on the full Lie algebra determines the root system, it follows that the corresponding root systems
are isomorphic.
\end{proof}

\p Let $(R,E)$ be a roots system, with Weyl group $W$.  Show that $W$ is a normal subgroup of the group of automorphisms of $(R,E)$ (that is, the group of linear automorphisms of $E$, preserving $R$ as a set, and preserving the Cartan integers). 
\begin{proof}
Let $g \in \Aut(R,E)$, $\sigma_\alpha$ be the reflection through the hyperplane normal to a root $\alpha$, and $\beta$ another root.
Then $g \sigma_\alpha g^{-1} (\beta) = $

\end{proof}

\p Fill in the details of the proofs of the results in 10.2 and 10.3 of Humphreys book.

\begin{proof}
\end{proof}

\end{document}
