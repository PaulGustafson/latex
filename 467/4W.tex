\documentclass{article}
\usepackage{m467}

\begin{document}
\noindent Paul Gustafson\\
\noindent Texas A\&M University - Math 467\\ 
\noindent Instructor: Stephen Fulling

\subsection*{HW 4 (W)}
\p{19} Show that in a semi-Euclidean plane, every angle inscribed in a semicircle is a right angle.  Discuss what occurs when the plane is not semi-Euclidean.
\begin{proof}
In a semi-Euclidean plane, suppose $\angle ABC$ is inscribed in a semicircle with $A, C$ on the diameter of the semicircle. Let $O$ denote the center of the semicircle.

Note that $\triangle AOB$ is isosceles with $\angle OAB \cong \angle OBA$.  Similarly, $\angle OAC \cong \angle AOC$. 

Since the plane is semi-Euclidean, the degree sum of the angles in any triangle is $180^\circ$. Hence, for $\triangle ABC$, we have
\begin{align*}
180^\circ & = (\angle ABC)^\circ + (\angle BAO)^\circ + (\angle OCB)^\circ
\\ & = (\angle ABC)^\circ + (\angle ABO)^\circ + (\angle CBO)^\circ
\\ & = 2 (\angle ABC)^\circ,
\end{align*}
so $\angle ABC$ is a right angle.

When the plane is not semi-Euclidean, every triangle has angle sum less than $180^\circ$. Analogously to the reasoning above, then, we have $180^\circ >  2 (\angle ABC)^\circ$, so $\angle ABC$ is acute.
\end{proof}
\end{document}
