\documentclass{article}
\usepackage{m467}

\begin{document}
\noindent Paul Gustafson\\
\noindent Texas A\&M University - Math 467\\ 
\noindent Instructor: Stephen Fulling

\subsection*{HW 4}
\p{4} To Show: If $\angle BAC$ and $\angle B^\prime A^\prime C^\prime$ are right angles and $AB \cong A'B'$ and $BC \cong B'C'$, then $\triangle ABC \cong \triangle A'B'C'$.
\begin{proof}
Following the hint, construct $D$ on the ray opposite to $\ray{AC}$ such that $AD \cong A'C'$.  Then by SAS, $\triangle DAB \cong \triangle C'A'B'$. Thus, $BD \cong BC$, so $\triangle DBC$ is isosceles with $\angle D \cong \angle C$.  Hence, by SAA, $\triangle ABC \cong \triangle ABD \cong A'B'C'$.
\end{proof}

\p{30} To Show: If $\square ABCD$ is a convex quadrilateral and $l$ is a line intersecting $AB$ between $A$ and $B$, then exactly one of the following holds:
\begin{enumerate}
\item There exists a point $O$ such that $B*O*C$ and $O$ is incident to $l$.
\item There exists a point $O$ such that $C*O*D$ and $O$ is incident to $l$.
\item There exists a point $O$ such that $A*O*D$ and $O$ is incident to $l$.
\item $C$ is incident to $l$.
\item $D$ is incident to $l$.
\end{enumerate}
\begin{proof}
Case $C$ is incident to $l$.  
\end{proof}

\p{32} Using Figure 4.33, note that $\angle A'B'B''$ is supplementary to $\angle A'B'B$. Moreover, $\angle ABB''$ is supplementary to $\angle B'BC$.  Thus, since two angles are congruent iff their supplementary angles are congruent, $\angle A'B'B'' \cong \angle ABB''$ iff $\angle A'B'B \cong \angle B'BC$, one of the pairs of alternate interior angles.  We get a similar equivalence between the other pair of corresponding angles and alternate interior angles.


\end{document}
