\documentclass{article}
\usepackage{m467}

\begin{document}
\noindent Paul Gustafson\\
\noindent Texas A\&M University - Math 467\\ 
\noindent Instructor: Stephen Fulling

\subsection*{The Alternate Interior Angle Theorem and its Converse}
The Alternate Interior Angle (AIA) theorem is a theorem of neutral geometry.  It states that in any Hilbert plane, if two lines cut by a transversal have a pair of congruent alternate interior angles with respect to that transversal, then the two lines are parallel.  

The proof is by contradiction. The basic idea of the proof is the following.   Recall that two lines are parallel if they have no points of intersection.   Hence, if the two lines were not parallel, then they would have to have a point of intersection on one side of the transversal.  However, by symmetry (in particular, by the SAS congruence axiom), this implies that that there is another point of intersection on the opposite side, a contradiction.

An important corollary of the AIA theorem is that if $l$ is any line and $P$ is any point not on $l$, there exists at least one line $m$ through $P$ parallel to $l$.  That is, Hilbert planes cannot have the elliptic parallel property.

The converse of AIA does not hold in every Hilbert plane.  In fact, in a Hilbert plane, the converse of AIA is equivalent to Hilbert's parallel postulate which is also equivalent to Euclid's parallel postulate.

\end{document}
