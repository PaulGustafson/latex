\documentclass{article}
\usepackage{m467}

\begin{document}
\noindent Paul Gustafson\\
\noindent Texas A\&M University - Math 467\\ 
\noindent Instructor: Stephen Fulling

\section*{Intersections of Two Circles}

The purpose of this paper is to answer Major Exercise 3.1 (p. 153) of Greenberg \cite{g}:
\begin{quote}
In the real Euclidean plane, let $\gamma$ be a circle with center $A$ and radius of length $r$. Let $\gamma^\prime$ be another circle with center $A'$ and radius of length $r'$, and let $d$ be the distance from $A$ to $A'$. There is a hypothesis about the numbers $r$, $r'$, and $d$ that ensures that the circles $\gamma$ and $\gamma'$ intersect in two distinct points. Figure out what this hypothesis is. \ldots What hypothesis on $r$, $r'$, and $d$ ensures that $\gamma$ and $\gamma'$ intersect in only one point \ldots?
\end{quote}

I will use the following extension of Greenberg's Triangle Inequality theorem (p. 171).
\begin{theorem}
Given distinct points $A,B,C$ in a real Euclidean plane, then $AB + BC \ge AC$ and equality holds iff $A*B*C$.
\end{theorem}
\begin{proof}
If $A,B,C$ are not collinear, then $AB + BC > AC$ by Greenberg's Triangle Inequality. If $A*C*B$, then $AB > AC$, so $AB + BC > AC$.  Similarly, if $B*A*C$, then $BC >AC$, so $AB + BC > AC$.  Lastly, if $A*B*C$, then $AB + BC \cong AC$ by the definition of addition for line segments.

On the other hand, if $AB + BC \cong AC$, then 
\end{proof}

By the order trichotomy for segments, exactly one of the following occurs:
\begin{itemize}
\item $r + r' < d$
\item $r + r' = d$
\item $r + r' > d.$
\end{itemize}

Case $r + r' < d$.  I claim there is no intersection between $\gamma$ and $\gamma'$.  Suppose $I \in \gamma \cap \gamma'$.  Then $|AI| + |IA'| = r + r' < d = |AA'|$, which contradicts the triangle inequality.

Case $r + r' = d$. I claim there is exactly one point of intersection between $\gamma$ and $\gamma'$. By line-circle continuity, the line $\gline{AA'}$ intersects $\gamma$ at two distinct points -- call them $B, C$.  By the triangle inequality, $B*A*C$. %CHECKME
Thus, either $A*B*A'$ or $A*C*A'$, WLOG suppose the former holds.  Then

\bibliographystyle{plain}
\bibliography{paper1}
\end{document}
