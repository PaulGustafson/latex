\documentclass{article}
\usepackage{../m}

\begin{document}
\noindent Paul Gustafson\\
\noindent Texas A\&M University - Math 641\\ 
\noindent Instructor - Fran Narcowich

\subsection*{HW 8}
\p{1}  Let $f(x) = x^2$ for $-1 \le x \le 2$. Find two simple functions $s_1 \le f \le s_2$ and
$$ \int_{-1}^2 s_2(x) dx - \int_{-1}^2 s_1(x) dx < 0.01.$$
How well do these integrals compare with $\int_{-1}^2 f(x)dx$?
\begin{proof}
Let $t_j = -1 + 0.001 j$ for $0 \le j \le 3000$. 
Let $$s_1(x) =  4\chi_{\{2\}} +  \sum_{j=0}^{2999} \left(\inf_{x \in [t_j, t_{j+1})} f(x) \right) \chi_{[t_j, t_{j+1})}$$ and
$$s_2(x) = 4\chi_{\{2\}} + \sum_{j=0}^{2999} \left(\sup_{x \in [t_j, t_{j+1})} f(x) \right) \chi_{[t_j, t_{j+1})}.$$
Then $s_1 \le f \le s_2$. We have
\begin{align*}
\int_{-1}^2 s_1 dx & = \sum_{j=0}^{999} (0.001)f(t_{j+1})  + \sum_{j=1000}^{2999} (0.001)f(t_{j}) 
\\ & = \sum_{j=1}^{1000} (0.001) (-1 + 0.001 j)^2  + \sum_{j=1000}^{2999} (0.001) (-1 + 0.001 j)^2
\\ &  = 2.9975
\end{align*}
and
\begin{align*}
\int_{-1}^2 s_2 dx & = \sum_{j=0}^{999} (0.001)f(t_{j})  + \sum_{j=1000}^{2999} (0.001)f(t_{j+1}) 
\\ & = \sum_{j=0}^{999} (0.001) (-1 + 0.001 j)^2  + \sum_{j=1001}^{3000} (0.001) (-1 + 0.001 j)^2
\\ &  = 3.0025,
\end{align*}
so $\int_{-1}^2 s_2(x) dx - \int_{-1}^2 s_1(x) dx = 0.005$.

We also have $\int_{-1}^2 f(x) dx = \int_{-1}^2 x^2 dx = \left[ \frac {x^3} 3 \right]_{-1}^2 = \frac 8 3 + \frac 1 3 = 3$.
\end{proof}


\p 2 Let $F(s) = \int_0^\infty e^{-st} f(t) dt$ be the Laplace transform of $f \in L^1([0,\infty))$. Use the DCT to show that $F$ is continuous from the right as $s = 0$.
\begin{proof}
Note that for $s,t > 0$ we have $|e^{-st}f(t)| \le |f(t)|$. Hence, by the DCT,
 $\lim_{s \to 0^+} F(s) = \lim_{s \to 0^+}  \int_0^\infty e^{-st} f(t) dt \ =  \int_0^\infty \lim_{s \to 0^+}  e^{-st} f(t) dt =  \int_0^\infty f(t) dt = F(0)$.
\end{proof}


\p 3 Let $f_n = n^{3/2} x e^{-nx}$, where $x \in [0,1]$ and $n = 1,2,3, \ldots$
\begin{enumerate}[a.]
\item Verify that the pointwise limit of $f_n$ is $f = 0$.
\item Show that $\|f_n\|_{C[0,1]} \to \infty $ as $n \to \infty$, so that $f_n$ does not converge uniformly to $0$.
\item Find a constant $C$ such that for all $n$ and $x$ fixed $f_n(x) \le C x^{-1/2}, x \in (0,1]$.
\item Use the DCT to show that 
$$\lim_{n \to \infty} \int_0^1 f_n(x) dx = 0.$$
\end{enumerate}
\begin{proof}
For (a), note that $f_n(0) = 0$ for all $n$.  For fixed $x > 0$, we have $\lim_{n\to\infty} f_n(x) = \lim_{n\to\infty} x e^{-nx + (3/2) \log n} = x \lim_{u \to -\infty} e^u = 0$.

For (b), we have $\sup_{x \in [0,1]}  n^{3/2} x e^{-nx} = n^{1/2} \sup_{u \in [0,n]}  u e^{-u} \le  n^{1/2} \sup_{u \in [0,\infty]}  u e^{-u} \to \infty$  as $n \to \infty$.

For (c), for $x > 0$ we have $\frac {f_n(x)} {x^{-1/2}} = (nx)^{3/2} e^{-nx} \le \sup_{u \in [0,\infty]}  u e^{-u}$.

For (d), since $x^{-1/2} \in L_1(0,1)$, part (c) and the DCT imply
$$\lim_{n \to \infty} \int_0^1 f_n(x) dx =  \int_0^1 \lim_{n \to \infty} f_n(x) dx =  0.$$
\end{proof}

\p 4 Let $L$ be a bounded linear operator on Hilbert space $\mH$. Show that the two formulas for $\|L\|$ are equivalent:
\begin{enumerate}[i.]
\item $\|L\| = \sup\{\|Lu\| : u \in \mH, \|u\| = 1\}$
\item $\|L\| = \sup\{|\langle Lu, v \rangle| : u, v \in \mH, \|u\| = \|v\| = 1 \}$
\end{enumerate}

\begin{proof}
Fix $u \in \mH$ with $\|u\| = 1$.  If $Lu = 0$, then $\|Lu\| = 0 = \langle Lu, v \rangle$ for all $v$. If $Lu \neq 0$, we have $\|Lu\| = \langle Lu, \frac {Lu} {\|Lu\|} \rangle$. Hence, in either case $\|Lu\| \le |\langle Lu, v\rangle |$ for some $\|v\| = 1$.  Hence, $\sup\{\|Lu\| : u \in \mH, \|u\| = 1\} \le \sup\{|\langle Lu, v \rangle| : u, v \in \mH, \|u\| = \|v\| = 1 \}$.

On the other hand, $|\langle Lu, v \rangle| \le \|Lu\|$ for all $\|v\| = 1$ by Cauchy-Schwartz. Thus, $ \sup\{|\langle Lu, v \rangle| : u, v \in \mH, \|u\| = \|v\| = 1 \} = \sup\{\|Lu\| : u \in \mH, \|u\| = 1\}$.
\end{proof}

\p 5 Let $V$ be a Banach space and let $L : V \to V$ be linear. Show that $L$ is bounded iff $L$ is continuous.
\begin{proof}
Suppose $L$ is bounded. Let $\epsilon > 0$. If $\|w - v\| < \epsilon/\|L\|$ then $\|Lw - Lv\| \le \|L\|\|w - v\| < \epsilon$.

Suppose $L$ is continuous.  Pick $\delta > 0$ such that $\|Lv\| < 1$ for all $\|v\| \le \delta$.  Then for all $\|w\| \le 1$, we have $\|Lw\| = \delta^{-1} \|L (\delta w ) \| < \delta^{-1}$.
\end{proof}

\p 6 Consider the BVP $-u''(x) = f(x)$ for $0 \le x \le 1, f \in C[0,1], u(0) = 0$ and $u'(1) = 0$.
\begin{enumerate}[a.]
\item Verify that the solution is given by $u(x) = \int_0^1 k(x,y) f(y) dy$, where
$$ k(x,y) = \left\{ \begin{array}{cc}
   y, & 0 \le y \le x
\\ x, & x \le y \le 1
\end{array}\right. 
$$
\item Let $L$ be the integral operator $Lf = \int_0^1 k(x,y)f(y) dy$. Show that $L: C[0,1] \to C[0,1]$ is bounded and that the norm $\|L\|_{C[0,1] \to C[0,1]} \le 1$. Try to show that $\|L\|_{C[0,1] \to C[0,1]} = 1/2$.
\item Show that $k(x,y)$ is a Hilbert-Schmidt kernel and that $\|L\|_{L^2 \to L^2} \le \sqrt{\frac 3 {20}}$.
\end{enumerate}

\begin{proof}
To see that $u(x) = \int_0^1 k(x,y) f(y) dy$ is a solution of the BVP, first note that 
$\left| \frac{k(x+h,y) - k(x,y)}{h} f(y) \right| \le \frac{k(x+h,y) - k(x,y)}{h} \|f\|_\infty \le \|f\|_\infty$ for all $h$ for which the quotient is defined. The last inequality follows from case analysis on $k$ (one can consider all slopes of secant lines of $k(x,y)$ for any fixed $y$).

Hence, by the DCT we have
$u' = \int_0^1 \pder[k]{x} f(y) dy = \int_x^1 f(y) dy$.
Thus, by the Fundamental Theorem of Calculus, $u'' = - f(y)$.

For uniqueness, suppose $v$ satisfies the BVP. Then $(u-v)'' = 0$ and $(u-v)(0) = 0$ and $(u-v)'(1) = 0$.  Thus, $u = v$.

For (b), for $x \in [0,1]$ we have $|Lf(x)| \le \int_0^1 |k(x,y)| |f(y)| dy \le  \int_0^1 k(x,y) \|f(y)\|_{C[0,1]} dy = \left( \frac {x^2} 2 + x(1 - x) \right) \|f(y)\|_{C[0,1]} = (x - x^2/2)  \|f(y)\|_{C[0,1]}$. Thus $L$ is bounded and of norm no greater than $\sup_{x \in [0,1]} x - x^2/2 = 1/2$.  Moreover, this bound is attained if $f$ is a constant function. Hence $\|L\|_{C[0,1] \to C[0,1]} = 1/2$.

For (c), $k(x,y)$ is bounded, so it must have finite $L^2([0,1]^2)$-norm. Hence $k(x,y)$ is a Hilbert-Schmidt kernel. 

Moreover, 
\begin{align*}
\|Lu\|_2^2 & = \int_0^1 \left|\int_0^1 k(x,y) f(y) dy \right|^2 dx
\\ & \le \int_0^1 \left(  \int_0^1 k(x,y) |f(y)| dy \right)^2 dx
\\ & = \int_0^1 \left( \int_0^x y |f(y)| dy + \int_x^1 x |f(y)| dy \right)^2 dx
\\ & \le \int_0^1 \left( \|f\|_2 \int_0^x y^2 dy  + \|f\|_2 \int_x^1 x^2 dy \right)^2 dx
\\ & = \|f\|_2^2 \int_0^1 \left( \frac 1 3 x^3  + x^2 - x^3  \right)^2 dx
\\ & = \|f\|_2^2 \int_0^1 \left( - \frac 2 3 x^3 + x^2  \right)^2 dx
\\ & = \|f\|_2^2 \int_0^1  \frac 4 9 x^6  -\frac 4 3 x^5 + x^4  \, dx
\\ & = \|f\|_2^2 \left( \frac 4 {63}  -\frac 4 {18}  + \frac 1 5 \right)
\\ & =  \frac {13} {315} \|f\|_2^2 
\end{align*}
Hence $\|L\|_{L^2 \to L^2} \le \sqrt{\frac {13} {315}} \le \sqrt{ \frac 3 {20}}$.
\end{proof}
\end{document}

