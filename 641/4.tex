\documentclass{article}
\usepackage{../m}

\begin{document}
\noindent Paul Gustafson\\
\noindent Texas A\&M University - Math 641\\ 
\noindent Instructor - Fran Narcowich

\subsection*{HW 4}
\p{1}  Let $A$ and $B$ be self-adjoint matrices, which may be real or
  complex. We say that $A\le B$ if and only if $\langle A\mathbf
  x,\mathbf x\rangle \le \langle B\mathbf x,\mathbf x\rangle$ for all
  $\mathbf x$. 

a. If $\lambda_1\ge \lambda_2,\ldots,\lambda_n$ are the eigenvalues
  of $A$ and $\tilde \lambda_1\ge \tilde \lambda_2,\ldots,\tilde
  \lambda_n$ are the eigenvalues of $B$, then show that $\lambda_k \le
  \tilde \lambda_k$.

b. Show that $\text{Trace}(A) \le \text{Trace}(B)$ if $A\le B$.

c. Show that if we increase a diagonal entry of $A$, then the
  resulting matrix $B$ satisfies $A\le B$.

d. (Keener, problem 1.3(b)). Use the previous part to estimate the
lowest eigenvalue of the matrix below. Keener gets $-\frac13$. Using
matlab you get less than about $-2$. Can you beat $-\frac13$?

\[ 
A=\begin{pmatrix}8 & 4 & 4\\ 4 & 8 & -4 \\ 4 &
-4 & 3\end{pmatrix} 
\]

\begin{proof}
For (a), 

Since the trace of a matrix is the sum of its eigenvalues, (b) follows directly from (a).

For (c), suppose $A \le B$. Let $(e_i)$ be the standard basis.  
Then $\tr(A) = \sum_{i=1}^n \langle A e_i, e_i \rangle \le \sum_{i=1}^n \langle B e_i, e_i \rangle = \tr(B)$

\end{proof}



\p{2} Let $A$ be a self-adjoint matrix with eigenvalues $\lambda_1\ge
\lambda_2,\ldots,\ge \lambda_n$. Show that for $ 2\le k < n$ we have 

\[ \max_U \sum_{j=1}^k \langle Au_j,u_j \rangle =\sum_{j=1}^k
\lambda_j, \]

where $U=\{u_1,\ldots,u_k\}$ is any o.n. set.  (Hint: Put $A$ in
diagonal form and use a judicious choice of $B$.)

\begin{proof}
By the spectral theorem for self-adjoint matrices, $A$ is diagonal with respect to an orthonormal basis $V = (v_i)_{i=1}^n$. By reordering the $v_i$, WLOG the eigenvalue of each $v_i$ is $\lambda_i$.  Let $W = \spn (v_i)_{i=1}^k$, which is an invariant subspace of $A$.  Hence we have $\sum_{j=1}^k \lambda_j = \text{Trace}_W(A) = \sum_{i=1}^k \langle A v_i, v_i \rangle$. Hence $\max_U \sum_{j=1}^k \langle Au_j,u_j \rangle \ge \sum_{j=1}^k \lambda_j$.

\end{proof}


\p{3} Show that $\ell^\infty$ is a Banach space under the norm
  $\|\{x_j\}\|= \sup_j |x_j|$

\begin{proof}
To see that $\|\cdot\|$ is a norm, we need to show that it is positive definite, homogenous, and satisfies the triangle inequality. The norm is clearly nonnegative since the absolute value function is nonnegative.  Moreover if $x = (x_j) \in \ell^\infty$ and $\|x\| = 0$, then $\sup_j |x_j| = 0$. Hence $\|x_j| \le 0$ for all $j$, so $x_j = 0$ for all $j$.

For homogeneity, let $c \in \R$.  Then $\|c x\| = \sup_j |c x_j| = |c| \sup_j |x_j| = |c| \|x\|$.  For the triangle inequality, let $y = (y_j)$.  Then $\|x + y \| = \sup_j |x_j + y_j| \le \sup_j |x_j| + |y_j| \le \sup_j |x_j| + \sup_j |y_j| \le \|x\| + \|y\|$.

To see that $\ell^\infty$ is complete, suppose $(x_n) \subset \ell^\infty$ is Cauchy.  Write each $x_n$ as $(x_{nj})_j$.  

Fix $j$. Since $(x_n)$ is Cauchy, given $\epsilon > 0$ there exists $N$ such that $\|x_n - x_m \| < \epsilon$ for all $n,m \ge N$.  Thus $|x_{nj} - x_{mj}| \le \sup_k \|x_{nk} - x_{mk} \| < \epsilon$ for all $n,m \ge N$.  Hence $(x_{nj})_n$ is Cauchy in $\R$, so has a limit $y_j$. 

Let $y = (y_j)_j \in \ell^\infty$.  I need to show that $y \in \ell^\infty$ and $x_n \to y$.  For the former, note that since $(x_n)$ is Cauchy, there exists $M$ such that $\|x_n\| \le M$ for all $n$.  Hence $|x_{nj}| \le M$ for all $n, j$.  Thus for each $j$, we have $|y_j| = |\lim_n x_{nj}| = \lim_n |x_{nj}| \le M$. Thus, $y \in \ell^\infty$.  

To see that $x_n \to y$, pick $\epsilon > 0$.  Since $(x_n)$ is Cauchy, we can pick $N$ such that $\|x_n - x_m\| < \epsilon/2$ for all $n,m \ge N_1$. Since each $x_{nj} \to y_j$ for each $1 \le j \le N$, we can pick $N_j$ such that $\|x_{nj} - y_j \| < \epsilon/2$ for all $n \ge N_j$.  Let $K = \max(N, \max_j N_j))$.  Then for $n \ge K$, we have $\|x_n - y \| \le \|x_n - x_K\| + \|x_K - y\| < \epsilon/2 + \sup_j |x_{Kj} - y_j| < \epsilon/2 + \sup_j (\epsilon/2) = \epsilon$.
\end{proof}

\p{4} Show that $\ell^2$ is a Hilbert space under the inner product 

\[
\langle \{x_j\},\{y_j\} \rangle :=\sum_{j=1}^\infty \bar y_j x_j.
\]

\begin{proof}
To see that $\langle \cdot , \cdot \rangle$ maps into $\R$, let $x = (x_j) \in \ell^2$ and $y = (y_j) \in \ell^2$.  Then for every $N$, we have $\sum_{j=1}^N \bar y_j x_j \le \left( \sum_{j=1}^N |y_j|^2)^{1/2} (\sum_{j=1}^N |x_j|^2 \right)^{1/2} \le \|x\| \|y\|$ by Cauchy-Schwartz on $\C^N$.  Hence, letting $N \to \infty$, we have $\langle x, y \rangle \le \|x\| \|y\| < \infty$.

To see that $\langle \cdot, \cdot \rangle$ defines an inner product, we need to check that it is positive definite, linear in the first component, and conjugate symmetric.  For positive definiteness, note that $\langle x, x \rangle = \sum_j |x_j|^2 \ge 0$, and equality holds iff $x_j = 0$ for all $j$.  Linearity in the first component and conjugate symmetry are both immediate from the definition of $\langle \cdot, \cdot \rangle$.

To see that $\ell^2$ is complete, suppose $(x_n) \subset \ell^2$ is Cauchy.  For each $n$, we can write $x_n = (x_{nj})_j$.  Fix $j$, and let $\epsilon > 0$.  Since $(x_n)$ is Cauchy, we can pick $N$ such that $\|x_n - x_m \| < \epsilon$ for $n,m \ge N$.  Hence $\|x_{nj} - x_{mj}|^2 < \sum_k |x_{nk} - x_{mk}|^2 = \|x_n - x_m\| < \epsilon$.  Hence $(x_{nj})_n$ is Cauchy in $\R$, so converges to some $y_j$.  

To see that $y := (y_j)$ is in $\ell^2$, we use the fact that $(x_n)$ is bounded in $\ell^2$ since it is Cauchy.  That is, there exist an $M$ such that $\sum_j |x_{nj}|^2 = \|x_n\| < M$ for all $n$.  


\end{proof}

\p{5} Let $0\le \delta \le 1$. We define the modulus of continuity for
$f\in C[0,1]$ by

\[
\omega(f;\delta) := \sup_{|\,s-t\,|\,\le\, \delta}|f(s)-f(t)|,
  \ \text{where }\ s,t \in [0,1].
\]

a. Explain why $\omega(f;\delta)$ exists for every $f\in C[0,1]$.

b. Fix $\delta$. Let $S_\delta = \{ \epsilon > 0 : |f(t) - f(s)| < \epsilon \text{ for all } |s - t| \le \delta \}$.  Show that $\omega (f ; \delta) = \inf S_\delta$.

c. Show that $\omega(f ; \delta)$ is nondecreasing as a function of $\delta$.

d. Show that $\lim_{\delta \downarrow 0} \omega(f; \delta) = 0$.

\begin{proof}
For (a), if $f \in C[0,1]$ then there exists $M > 0$ such that $|f(x)| \le M$ for all $x$. This is because the image of a compact set under a continuous function is compact. Hence for all $s, t \in [0,1]$, we have $|f(s) - f(t)| \le 2M$.  Thus $\omega(f;\delta) \le 2M$.

For (b), if $\epsilon \in S_\delta$, then $\omega(f; \delta) = \sup_{|\,s-t\,|\,\le\, \delta} |f(s)-f(t)| \le \epsilon$. Hence $\omega(f; \delta) \le \inf S_\delta$.  On the other hand, if $\eta > 0$, then $|f(s) - f(t)| < \omega(f; \delta) + \eta$ for all $|s - t| \le \delta$.  Hence, $\omega(f; \delta) + \eta \in S_\delta$.  Thus $inf S_\delta \le \omega(f; \delta) + \eta$.  Letting $\eta \to 0$, we have $\inf S_\delta \le \omega(f ; \delta)$.

For (c), suppose $\delta < \gamma$.  If $\epsilon \in S_\gamma$, then $|f(t) - f(s)| < \epsilon$ for all $|s - t| \le \gamma$, hence for all $|s - t| \le \delta$.  Thus, $S_\gamma \subset S_\delta$.  Therefore $\omega(f; \delta) = \inf S_\delta \le \inf S_\gamma = \omega(f ; \gamma)$.

For (d), let $\epsilon > 0$.  Since $f$ is continuous on the compact set $[0,1]$, it is uniformly continuous on $[0,1]$.  Hence we can pick $\delta > 0$ such that $|f(s) - f(t)| < \epsilon$ for all $|s - t| < \delta$.  Thus, $\omega(f; \delta) < \epsilon$.  By (c), if $0 < \gamma \le \delta$, then $\omega(f ; \gamma) \le \omega(f ; \delta) < \epsilon$.  Hence $\lim_{\delta \downarrow 0} \omega(f; \delta) = 0$.
\end{proof}

\end{document}
