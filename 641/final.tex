\documentclass{article}
\usepackage{../m}

\begin{document}
\noindent Paul Gustafson\\
\noindent Texas A\&M University - Math 641\\ 
\noindent Instructor - Fran Narcowich

\subsection*{Final}
\p{1} Let $f \in C[0,2 \pi]$ with $f(0) = f(2\pi)$.  Let 
$Q_{trap}(f) =  \frac {2\pi} n \sum_{k=0}^{n-1} f \left(\frac {2 \pi k} n \right)$.
Let $E_n = \left| \int_0^{2 \pi} f(x) dx - Q_n(f) \right|$.

\begin{enumerate}[(a)]
\item Show $Q_{trap}(e^{ikx}) = \left\{ 
\begin{array}[ll]
0 & k \not\equiv 0 \pmod n
\\ 2\pi & k \equiv 0 \pmod n
\end{array}
\right.$

\begin{proof}
If $k \equiv 0 \pmod n$, we have $Q_{trap} = \frac {2\pi} n \sum_{j=0}^{n-1} e^{\frac {2 \pi i jk} n} = \frac{2 \pi}{n} \sum_{j=0}^{n-1} 1 = 2\pi$.

Otherwise, we have 
\begin{align*}
Q_{trap}(e^{ikx}) & = \frac {2\pi} n \sum_{j=0}^{n-1} e^{\frac {2 \pi i jk} n}
\\ & = \frac {2\pi} n \frac{1 -  e^{\frac {2 \pi i k}}{1 - e^{\frac {2 \pi i k} n}}
\\ & = 0
\end{align*}
\end{proof}

\item Let $f(x)$ be the $2 \pi$-periodic function that equals $x^2(2\pi - x)^2$ when $x \in [0, 2\pi]$. Show that $\int_0^{2\pi} f(x) dx = 16 \pi^5/15$. Prove that $E_n \le C n^{-4}$.  
(Hint: $f(x) = \frac{8\pi^4}{15} - \frac{24}{\pi} \sum_{k \neq 0} e^{ikx} k^{-4}$.)

\begin{proof}
We have 
\begin{align*}
\int_0^{2 \pi} x^2(2\pi - x)^2 dx & = \int_0^{2 \pi} 4\pi^2 x^2 - 4 \pi x^3 + x^4 \, dx
\\ & = \frac 4 3 \pi^2 (2\pi)^3 - \pi(2 \pi)^4 + \frac 1 5 (2 \pi)^5
\\ & = \frac 4 3 \pi^2 (2\pi)^3 - \pi(2 \pi)^4 + \frac 1 5 (2 \pi)^5
\\ & = (32 \cdot 5 - 16 \cdot 15 + \32 \cdot 3) \pi^5/15
\\ & = 16 \pi^5/15
\end{align*}


\end{proof}

\end{enumerate}

\p 2 Let $mH$ be a complex Hilbert space, and let $L \in \mB(\mH)$.
\begin{enumerate}
\item Verify that 
$\langle L(u + e^{i \alpha} v), u + e^{i \alpha} v \rangle - 
\langle L(u - e^{i \alpha} v), u - e^{i \alpha} v \rangle
= 2 e^{-i \alpha} \langle L u, v \rangle + 2 e^{-i \alpha} \langle Lv, u \rangle$

\begin{proof}
We have 
\begin{align*}
\langle L(u + e^{i \alpha} v), u + e^{i \alpha} v \rangle - 
\langle L(u - e^{i \alpha} v), u - e^{i \alpha} v \rangle
& = \lange Lu, u \rangle + \langle Lu, e^{i \alpha} v \rangle +
\langle e^{i \alpha} v), u \rangle + \langle u, v \rangle
& \quad - \lange Lu, u \rangle + \langle Lu, e^{i \alpha} v \rangle +
\langle e^{i \alpha} v), u \rangle - \langle u, v \rangle
& = 2 e^{-i \alpha} \langle L u, v \rangle + 2 e^{-i \alpha} \langle Lv, u \rangle
\end{proof}

\item Show that if $L = L^*$, then $\|L \| = \sup_{\|u\| = 1} | \langle L u, u \rangle |$.

\begin{proof}
By Cauchy-Schwarz, we have $\sup_{\|u\| = 1} | \langle L u, u \rangle | \le 
\sup_{\|u\| = 1} \|L u\| = \|L \|$.

For the reverse inequality, we have 
$\|L\|^2 = \sup_{\|u\| = 1} \|L u\|^2 = \sup_{\|u\| = 1} |\langle L u, L u \rangle |^2$ 
\end{proof}

\item Show that if $M = \sup_{\|u\| = 1} |\langle Lu, u \rangle|$, then $M \le \|L\| \le 2M$, whether or not $L$ is self-adjoint. Give an example that shows that this result is false in a real Hilbert space.

\begin{proof}
By Cauchy-Schwarz, we have $M = \sup_{\|u\| = 1} | \langle L u, u \rangle | \le 
\sup_{\|u\| = 1} \|L u\| = \|L \|$.

For the other inequality, we have 



\end{proof}

\end{enumerate}

\p 3 Let $K \in \mC(\mH)$ be self adjoint. Show that the only possible limit point of the set of eigenvalues of $K$ is 0.

\p 4 Let $K \in \mC(\mH)$ be self adjoint. Suppose the range of $K$ contains a dense subset of $\mH$, and that an o.n. basis has been chosen for the igenspace of each nonzero eigenvalue of $K$. Show that the set of all these eigenvectors form a complete orthonormal set.

\p 5 Let $\mH = L^2[0,1]$. Consider the boundary value problem
$$Lu := \frac d {dx} \left( (1 + x) \frac {du} {dx} \right) = f(x),
u(0) = 0, u'(1) = 0.$$

\begin{enumerate}[(a)]
\item Find $G(x,y)$, the Green's function for this BVP.
\itme Let $Gf(x) = \int_0^1 G(x,y) f(y) dy$. Sho


\p 6 Let $\| \cdot \|_{op}$ be the operator norm for $\mB(\mH)$.
\begin{enumerate}
\item Show that $(\mB(\mH), \|\cdot \|_{op})$ is a Banach space.
\begin{proof}
We need to show that $\| \cdot \|_{op}$ is positive definite, homogeneous, and satisfies the triangle inequality.  

It is clearly positive. Moreover, if $\|T\|_{op} = 0$, then $\|Tu\| = 0$ for all $\|u\| = 1$.  Hence $Tv = \|v\| T \frac v {\|v\|} = 0$ for all $v \in \mH \setminus \{0\}$.  Thus $Tv = 0$.

For the homogeneity, we have $\|cT\|_{op} = \sup_{\|u\| = 1} |c|\|Tu\| = |c| \|T\|_{op}$ for all $c \in \R$ and $T \in \mH$.

For the triangle inequality, we have 
$\|S + T \|_{op} = \sup_{\|u\|=1} \|Su + Tu\| \le \sup_{\|u\|=1} \|Su \| + \|Tu\| 
\le \sup_{\|u\|=1} \|Su \| + \sup_{\|v\| = 1} \|Tv\|  = \|S\| + \|T\|_{op}$.
\end{proof}

\item Consider the operator $L = I - \lambda M$, with $M \in \mB(\mH)$. Show that if $|\lambda| < \|M \|_{op}^{-1}$, then 
$\sum_{k=0}^\infty \lambda^k M^k = (I - \lambda M)^{-1}$.
\end{enumerate}

\p 7 Show that if $B, B^{-1}$ are in $\mB(\mH)$, and $K \in \mC(\mH)$, then the range of $L = B + \lambda K$ is closed.

\begin{proof}
Suppose $Lx_n \to y$. By passing to a subsequence, since $\lambda K$ is compact, we may assume $\lambda K x_n \to z$.  Then $B x_n \to y - z$.   Since $B^{-1}$ is continuous, we have $x_n \to B^{-1}(y-z)$.

We have $L B^{-1}(y-z) = (y - z) + \lambda K B^{-1}(y-z) =  (y - z) + \lambda K \lim_n x_n = y$.
\end{proof}

\end{document}











