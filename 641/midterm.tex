\documentclass{article}
\usepackage{../m}

\DeclareMathOperator{\diag}{diag}

\begin{document}
\noindent Paul Gustafson\\
\noindent Texas A\&M University - Math 641\\ 
\noindent Instructor - Fran Narcowich

\subsection*{Midterm}

\p{1} Use the Courant-Fischer mini-max theorem to show that $\lambda_2 < 0$ for the matrix 
$$A = \begin{pmatrix} 
   0 & 1 & 3
\\ 1 & 0 & 2
\\ 3 & 2 & 0
\end{pmatrix}$$

\begin{proof}
The characteristic polynomial for $A$ is $f(x) := x^3 +6 + 6 - 9x -4x - x = x^3 - 14x + 12$.  We have $\lim_{x\to -\infty} f(x) < 0$, $f(0) > 0$, $f(1) < 0$, and $\lim_{x \to \infty} f(x) > 0$. Thus $\lambda_2 < 0$.
\end{proof}

\p{2} Let $A$ be an $n \times n$ complex matrix that satisfies $A^* A = A A^*$. Show that $A$ is diagonalizable and that there is a unitary matrix $U$ for which $U^*AU = \Lambda = \diag(\lambda_1, \ldots, \lambda_n)$.

\begin{proof}
\emph{Step 1: $A$ and $A^*$ are simultaneously diagonalizable}.  Let $J \in M_n(\C)$ be the Jordan Normal Form for $A$.  I claim that $J$ is diagonal. Suppose not. Then $J$ contains an $m \times \m$ Jordan block $B$ for $ 1< m \le n$.  If $\lambda$ is the generalized eigenvalue corresponding for $B$, then we have $[B, B^*]_{11} = (B B^*)_{11} - (B^*B)_{11} = (|\lambda|^2 + 1) - |\lambda|^2 \neq 0$.  Hence $[J, J^*] \neq 0$, so $[A, A^*] \neq 0$, a contradiction. Thus, $J$ is diagonal. The matrix $J^* = \overline{J}^T$ is clearly diagonal also.

\emph{Step 2: $A$ is unitarily diagonalizable.}  The proof is by induction on $n$. The base case is trivial. For the inductive step, recall that $A$ must have an eigenvector.  Let $v$ be an normalized eigenvector of $A$.  Let $w \in v^\perp$.  Then $\langle v, Aw \rangle = \langle A^*v , w \rangle = 0$ since $v$ is an eigenvector of both $A$ and $A^*$ by Step 1.  Hence $v^\perp$ is an invariant subspace of $A$, and we can apply the inductive hypothesis to $A \vert_{v^\perp}$.  
\end{proof}



\p{3} Let $f$ be continuous on $[0,1]$, with $f(0) = f(1) = 0$ and let $s \in S^{1/n}(1,0)$ be the linear spline interpolant to $f$, with knots at $x_j = \frac j n$.
\begin{enumerate}[(a)]
\item Let $\lambda \in \R$. Show that $\left|\int_0^1 s(x) e^{i \lambda x} dx \right| \le \frac{2 n^2}{\lambda^2} \omega(f, 1/n)$.

\begin{proof}
We have
\begin{align*}
\left|\int_0^1 s(x) e^{i \lambda x} dx \right| & = \left| \sum_{k=0}^{n-1} \int_{k/n}^{(k+1)/n}  s(x) e^{i \lambda x} dx \right|
\\ & = \left| \sum_{k=0}^{n-1} \left[\frac{1}{i  \lambda} s(x) e^{i \lambda x} \right]_{x = k/n}^{(k+1)/n} -
\frac{1}{i  \lambda} \int_{k/n}^{(k+1)/n}  s'(x) e^{i \lambda x} dx \right|
\\ & = \left| \sum_{k=0}^{n-1} \frac{1}{i  \lambda} \int_{k/n}^{(k+1)/n}  s'(x) e^{i \lambda x} dx \right|
\\ & = \left| - \frac{1}{\lambda^2} \sum_{k=0}^{n-1}  \left[s'(x) e^{i \lambda x} \right]_{x = k/n}^{(k+1)/n} \right|
\\ & \le \frac{1}{\lambda^2} \sum_{k=0}^{n-1} \left| s' \left(\frac {k+1} n - \right) \right|  + \left| s'\left(\frac {k} n +\right) \right| 
\\ & \le \frac{1}{\lambda^2} \sum_{k=0}^{n-1} 2 n \omega(f, 1/n)
\\ & = \frac{2 n^2}{\lambda^2} \omega(f, 1/n).
\end{align*}
\end{proof}

\item Use the previous part to show that $\left| \int_0^1 f(x) e^{i\lambda x} dx \right| \le \omega(f, 1/n) + \frac{2n^2}{\lambda^2} \omega(f, 1/n)$.

\begin{proof}
We have
\begin{align*}
\left|\int_0^1 f(x) e^{i \lambda x} dx \right| & \le \left|\int_0^1 f(x) - s(x) e^{i \lambda x} dx \right| + \left|\int_0^1 s(x) e^{i \lambda x} dx \right| 
\\ & \le \int_0^1 |f(x) - s(x)| dx +  \frac{2 n^2}{\lambda^2} \omega(f, 1/n)
\\ & \le \int_0^1 \omega(f, 1/n) dx +  \frac{2 n^2}{\lambda^2} \omega(f, 1/n)
\\ & \le \omega(f, 1/n) +  \frac{2 n^2}{\lambda^2} \omega(f, 1/n)
\end{align*}
\end{proof}
\end{enumerate}

  
\p{4}  Let $\{\phi_n(x)\}_{n=0}^\infty$ be a set of polynomials orthogonal with respect to a weight function $w(x)$ on a domain $[a,b]$. Assume that the degree of $\phi_n$ is $n$, and that the coeffiction of $x^n$ in $\phi_n(x)$ is $k_n > 0$. In addition, suppose that the continuous functions are dense in $L_w^2[a,b] = \{f : [a,b] \to \C : \int_a^b |f(x)|^2 w(x) dx < \infty \}$.

\begin{enumerate}[(a)]
\item Show that $\phi_n$ is orthogonal to all polynomials of degree $n - 1$ or less.
\begin{proof}
The set $\{\phi_k\}_{0 \le k < n}$ spans the polynomials of degree less than $n-1$.
\end{proof}
\item Show that $\{ \phi_n \}_{n=0}^\infty$ is complete in $L_w^2[a,b]$.
\begin{proof}
% Since $\phi_0 \in L_w^2[a,b]$, we have $0 < \int_a^b \phi_0^2 w(x) dx < \infty$.
Let $g \in L_w^2[a,b]$ be continuous.  Let $\epsilon > 0$.  By the Weierstrauss Approximation Theorem, pick a polynomial $p$ such that $\|g - p \|_{C[a,b]} < \epsilon$.  Then
$\|g - p\|_{L_w^2[a,b]} = \int_a^b |g - p|^2 w dx \le \epsilon^2 \int_a^b w dx$.  Since $\phi_0 \in L_w^2[a,b]$, this last integral is finite.  Hence, the polynomials are dense in $L_w^2[a,b]$.

Now suppose $\{ \phi_n \}_{n=0}^\infty$ is not complete. By a previous homework problem, there exists a normalized function $f \in L_w^2[a,b]$ with $\langle f, \phi_n \rangle = 0$ for all $n$.  Thus for any polynomial $p$, we have $\|f - p\|_{L_w^2[a,b]}^2 = \|f\|_{L_w^2[a,b]}^2 + \|p\|_{L_w^2[a,b]}^2 \ge 1$.  This contradicts the fact that the polynomials are dense in ${L_w^2[a,b]}$.

\end{proof}
\item Show that the polynomials satisfy the recurrence relation $\phi_{n+1}(x) = (A_nx + B_n) \phi_n(x) + C_n \phi_{n-1}(x)$. Find $A_n$ in terms of the $k_n$'s.  
\begin{proof}
We have 
$\phi_{n+1} = A_n x \phi_n + \sum_{j=0}^{n} a_j \phi_j$ 
for some unique $A_n$ and $(a_j)_{j=1}^n$.  

For $1 \le l \le n-2$, part (a) implies that 
\begin{align*}
0 & = \langle \phi_{n+1} , \phi_l \rangle 
\\ & = \left\langle A_n x \phi_n + \sum_{j=0}^{n} a_j \phi_j, \phi_l \right\rangle
\\ & = \langle A_n x \phi_n , \phi_l \rangle +  a_l \langle \phi_l, \phi_l \rangle
\\ & =  A_n  \langle \phi_n , x \phi_l \rangle +  a_l \langle \phi_l, \phi_l \rangle
\\ & = a_l \langle \phi_l, \phi_l \rangle
\end{align*}

Hence $a_l = 0$ for $ 1 \le l \le n - 2$, so
$\phi_{n+1} = A_n x \phi_n + B_n \phi_n + C_n \phi_{n-1}.$ 

By comparing leading coefficients, $A_n = \frac {k_{n+1}}{k_n}$.
\end{proof}
\end{enumerate}


\p{5} Suppose that $f(\theta)$ is a $2 \pi$-periodic function in $C^m( \R)$, and that $f^{(m+1)}$ is piecewise continuous and $2 \pi$-periodic. Here $m > 0$ is a fixed integer.  Let $c_k$ denote the $k$-th (complex) Fourier coefficient for $f$ and let $c_k^{(j)}$ denote the $k$-th Fourier coefficient for $f^{(j)}$.

\newcommand{\ckj}{c_k^{(j)}}

\begin{enumerate}[(a)]
\item Show that $\ckj = (ik)^j c_k$ for $1 \le j \le m+1$.

\begin{proof}
Integrating by parts $j$ times, we have
\begin{align*}
\ckj  & =  \frac 1 {2 \pi} \int_0^{2 \pi} f^{(j)}(x) e^{- i k x} dx 
\\ & = \frac 1 {2 \pi} \left( \left[ f^{(j-1)}(x) e^{2 \pi k x}  \right]_0^{2 \pi} + (ik)  \int_0^{2 \pi} f(x) e^{2\pi k x} dx \right)
\\ & = \frac  {ik} {2 \pi} \left( \int_0^{2 \pi} f^{(j-1)}(x) e^{2\pi k x} dx \right)
\\ & \vdots
\\ & = \frac  {(ik)^j} {2 \pi} \left( \int_0^{2 \pi} f(x) e^{2\pi k x} dx \right)
\\ & = (ik)^j c_k
\end{align*}
\end{proof}

\item For $k \neq 0$, show that the Fourier coefficient $c_k$ satisifies the bound
$$ | c_k | \le \frac 1 {2 \pi |k|^{m+1}} \|f^{(m+1)}\|_{L_1[0,2\pi]}$$

\begin{proof}
Integrating by parts $m+1$ times, we have
\begin{align*}
|c_k| & = \left| \frac 1 {2 \pi} \int_0^{2 \pi} f(x) e^{- i k x} dx \right|
\\ & = \left| \frac 1 {2 \pi (ik)^{m+1}} \int_0^{2 \pi} f^{(m+1)}(x) e^{- i k x} dx \right|
\\ & \le \frac 1 {2 \pi |k|^{m+1}} \|f^{(m+1)}\|_{L_1[0,2\pi]}
\end{align*}
\end{proof}

\item Let $S_n(\theta) = \sum_{k=-n}^n c_k e^{ik\theta}$ be the $n$-th partial sum of the Fourier series for $f$, $n \ge 1$. Show that both of the following hold for $f$:
$$\|f - S_n \|_{L_2} \le C \frac{\|f^{(m+1)}\|_{L_1}} {n^{m + \frac 1 2}} \text{ and }  \|f - S_n\|_{C[0,2\pi]} \le C' \frac{\|f^{(m+1)}\|_{L_1}}{n^m}.$$
\begin{proof}
By Parseval's theorem, we have 
\begin{align*}
\|f - S_n \|_{L_2} & = \left(\sum_{k > n} |c_k|^2\right)^{-1/2}
\\ & \le \left(\sum_{k > n} \frac C {|k|^{2m+2}} \|f^{(m+1)}\|_{L_1[0,2\pi]}^2 \right)^{-1/2}
\\ & = \left(\sum_{k > n} \frac C {|k|^{2m+2}} \right)^{-1/2} \|f^{(m+1)}\|_{L_1[0,2\pi]}
\\ & \le \left( \int_{k > n} \frac {C_1} {|k|^{2m+2}} dk \right)^{-1/2} \|f^{(m+1)}\|_{L_1[0,2\pi]}
\\ & = \left(  \frac {C_2} {n^{2m+1}} \right)^{-1/2} \|f^{(m+1)}\|_{L_1[0,2\pi]}
\\ & = \frac {C_3} {n^{m+1/2}} \|f^{(m+1)}\|_{L_1[0,2\pi]}
\end{align*}
Using part (b), we have
\begin{align*}
\|f - S_n \|_{C[0, 2\pi]} & = \sup_{x \in [0,2\pi]} \left| \sum_{k > n} c_k(x) e^{ikx} \right|
\\ & \le \sup_{x \in [0,2\pi]}  \sum_{k > n} |c_k(x)|
\\ & \le \sum_{k > n} \frac 1 {2 \pi |k|^{m+1}} \|f^{(m+1)}\|_{L_1[0,2\pi]}
\\ & \le \left(\int_{k > n} \frac {C'} {|k|^{m+1}} dk\right) \|f^{(m+1)}\|_{L_1[0,2\pi]}
\\ & = \frac {C''} {n^{m}} \|f^{(m+1)}\|_{L_1[0,2\pi]}.
\end{align*}
\end{proof}


\item Let $f(x)$ be the $2\pi$-periodic function that equals $x^2(2 \pi - x)^2$ when $x \in [0,2 \pi]$. Verify that $f$ satisfies the conditions above with $m = 1$. With the help of (a), calculate the Fourier coefficients for $f$. (Hint: look at $f''$.)

\begin{proof}
To see that $f$ satisfies the conditions with $m=1$, we need to check that  $f'(0+) = f'(2\pi -)$ and $f''$ is piecewise continuous ($f''$ is $2 \pi$-periodic since $f$ is). The former follows from the fact that $f$ has double roots at $0$ and $2 \pi$.  The latter is obvious.

For $x \in (0, 2\pi)$, we have 
\begin{align*}
f(x) & = x^4 - 4 \pi x^3 + 4 \pi^2 x^2
\\ f'(x) & = 4x^3 - 12 \pi x^2 + 8 \pi^2 x
\\ f''(x) & = 12x^2 - 24 \pi x + 8 \pi^2 
\end{align*}

From (a), the Fourier coefficient $c_k$ for $f$ is
\begin{align*}
c_k & = (ik)^{-2} \ckj
\\  & = -\frac 1 {2 \pi k^2} \int_0^{2\pi} f''(x) e^{-ikx} dx
\\  & = -\frac 1 {2 \pi k^2} \int_0^{2\pi} (12x^2 - 24 \pi x) e^{-ikx} dx
\\  & = -\frac {1} {2 \pi k^2}  \left( \left[ \frac{12x^2 - 24 \pi x}{-ik} e^{-ikx}\right]_0^{2\pi}  + \frac 1 {ik} \int_0^{2 \pi} (24x - 24 \pi)e^{-ikx} dx \right)
\\  & = -\frac {1} {2 \pi k^2}  \left( \frac {24} {ik} \right) \int_0^{2 \pi} xe^{-ikx} dx 
\\  & = \frac {24i} {2 \pi k^3} \int_0^{2\pi} x e^{-ikx} dx
\\  & = \frac {24i} {k^3} \left( \frac i k \right)
\\ & = - \frac{24} {k^4},
\end{align*}
where the penultimate equality uses the homework problem calculating the Fourier series of $g(x) = x, 0 \le x < 2 \pi$.
\end{proof}


\end{enumerate}

\end{document}
