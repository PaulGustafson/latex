\documentclass{article}
\usepackage{../m}

\begin{document}
\noindent Paul Gustafson\\
\noindent Texas A\&M University - Math 641\\ 
\noindent Instructor - Fran Narcowich

\subsection*{HW 6}
\p{1} Let $(y_j)_{j=0}^n \subset \R$. Consider the cubic-Hermite spline $s(x) \in S^h(3,1)$, with $h = 1/n$, that satisfies $s(j/n) = y_j$ and minimizes $\int_0^1 (s'')^2 \, dx$. Show that $s(x) \in S^h(3,2)$.
\begin{proof}
For any sequence $(q_j)_{j=0}^n$, let $U((q_j)_j)$ be the space of all $u \in S^h(3,1)$ such that $u(j/n) = q_j$ for all $j$.  Let $f(x)$ be the line interpolating $(0, y_0)$ and $(1, y_n)$.  Then $u \in U((y_j - f(j/n))_j)$ iff  $u + f \in U((y_j)_j$.  Moreover, $u'' = (u + f)''$ since $f$ is a line.  Hence since $s$ minimizes $\int (s'')^2 \, dx$ over $s \in U((y_j)_j$, we have $u = s - f$ minimizes $\int (u'')^2 \,dx$ for $u \in U((y_j - f(j/n))_j)$.  Moreover if $s -f \in C^2$ so is $s$.  Thus, it suffices to consider the case where $y_j$ is replaced by $y_j - f(j/n)$; that is, the case where $y_0 = y_n = 0$.

Let $V = \{f \in S^h(3,1) : f(0) = f(1) = 0 \text{ for all } j\}$.  Define $\langle \cdot, \cdot \rangle_V : V \times V \to \R_{\ge 0}$ by $\langle f, g \rangle_V = \int_0^1 f''(x) g''(x) \, dx$.  This function is bilinear, and $\langle f, f \rangle_V = 0$ implies $f'' = 0$ which implies that $f = 0$ since $f(0) = f(1) = 0$. Hence, $\langle \cdot, \cdot \rangle_V$ is an inner product.

Let $W = \{f \in S^h(3,1) : f(j/n) = 0 \text{ for all } j\}$.  Pick any $l \in S^h(3,1)$ interpolating the points $((j/n, y_j))_j$, then $l + W = U((y_j)_j)$.

We have $s = w + v$ for some $w \in W$ and $v \in W^\perp$.   Hence $v = s - w \in l + W$, and $\|s\|_V^2 = \|w\|_V^2 + \|v\|_V^2$.  Hence, $\|v\|_V \le \|s\|_V$ with equality iff $w = 0$.  Thus, since $s$ minimizes $\|s\|_V$ over $l + W$, we must have $s = v$.  Thus $s \in W^\perp$.

Now suppose that $s \not \in S^h(3,2)$. Let $j_0$ be a knot for which $s''$ is discontinuous.  I claim there exists $u \in W$ such that $u'(j_0) = 1$ and $u'(j) = 0$ for all $j \neq j_0$.  To construct such a $u$, for $x \in [(j_0-1)/n, j_0/n]$ let 
 $u(x) = \int_{(j_0-1)/n}^{x} A(t-(j_0-1)/n)^2 + B(t -(j_0 -1)/n) \, dt$ 
where real numbers $A, B$ are chosen such that $u'(j_0/n) = A(1/n)^2 + B(1/n) = 0$, and $u(j_0) = A/(3n^3) + B/(2n^2) = 0$.  We can do similarly for $x \in [j_0/n, (j_0+1)/n]$.  Lastly, set $u(x) = 0$ for all other $x$. 

Then we have 
\begin{align*}
\langle s, u \rangle & = \int_0^1 s'' u'' \, dx
\\ & = \sum_j \int_{j/n}^{(j+1)/n} s'' u'' \, dx
\\ & = \sum_j [s'' u']_{x=j/n}^{(j+1)/n} - \int_{j/n}^{(j+1)/n} s''' u' \, dx
\\ & = \sum_j [s'' u']_{x=j/n}^{(j+1)/n} -  s''' \int_{j/n}^{(j+1)/n}  u' \, dx
\\ & = \sum_j [s'' u']_{x=j/n}^{(j+1)/n} 
\\ & = \lim_{x \to j_0^+} s'' - \lim_{x \to j_0^-} s''
\\ & \neq 0,
\end{align*}
which contradicts the fact that $s \in W^\perp$. 
\end{proof}

\p{2} We want to solve the boundary value problem (BVP): 
$$-u'' = f(x), u(0) = u(1) = 0, f \in C[0,1]$$.

\p{a.} Let $H$ be the set of all continuous functions vanishing at $x = 0$ and $x = 1$, and having $L^2$ derivatives. Also let $H$ have the inner product
$$\langle f, g \rangle_H = \int_0^1 f'(x) g'(x) \, dx.$$
Use integration by parts to convert the BVP into its ``weak'' form:
$$\langle u, v \rangle_H = \int_0^1 f(x) v(x) \, dx \text{ for all } v \in H.$$
\begin{proof}
We have $\langle u, v \rangle_H = \int_0^1 u' v' \, dx =  - \int_0^1 u'' v \, dx = \int_0^1 f v \, dx$ for all $v \in \mH$.
\end{proof}

\p{b.} Conversely, suppose that $u \in H$ is also in $C^2[0,1]$ and 
$$\langle u, v \rangle_H  = \int_0^1 f(x) v(x) \, dx \text{ for all } v \in H.$$
Show that $u$ satisfies the BVP.
\begin{proof}
Suppose not. Then $-u''(x_0) \neq f(x_0)$ for some $x_0 \in (0,1)$. WLOG assume $u''(x_0) + f(x_0) > 0$.  By continuity, this inequality holds on some interval $(x_0 - \delta, x_0 + \delta)$. Let $v \in H$ be a nonnegative with support in $(x_0 - \delta, x_0 + \delta)$ and $v(x_0) > 0$.  Then $\int f v \, dx - \langle u, v \rangle_H = \int f v \, dx + \int u'' v \, dx  > 0$, a contradiction.
\end{proof}


\p{c.} Let $V = S^h(1,0)$, with $h = 1/n$. Thus, $V$ is spanned by $\phi_j(x) := N_2(nx - j+1), 1 \le j \le n-1 $ Show that the least squares approximation to $u$ from $V$ is $y = \sum_j \alpha_j \phi_j(x) \in V$, where $G \alpha = \beta$ for $\beta_j = \langle y , \phi_j \rangle_H$ and $G_{kj} = \langle \phi_j, \phi_k \rangle_H$.

\begin{proof}
This is just the definition of the normal equations, which was proved in HW 1.
\end{proof}

\p{d.} Show that $G_{kj} = \langle \phi_j, \phi_k \rangle_H$ is given by
$G_{j,j} = 2n$, $G{j,j-1} = -n$, and $G_{j, j+1} = n$ and $G_{j,k} = 0$ for all other $k$.
\begin{proof}
Recall that $N_2(x) = x_+ - 2(x-1)_+ + (x-2)_+$. Thus, for all $j$, we have $\phi_j' = n \chi_{[(j-1)h, jh]} -n \chi_{[jh, (j+1)h]}$.  
Hence,
\begin{align*}
\langle \phi_j, \phi_k \rangle  & = \int (n \chi_{[(j-1)h, jh]} -n \chi_{[jh, (j+1)h]})(n \chi_{[(k-1)h, kh]} -n \chi_{[kh, (k+1)h]}) \, dx
\\ & = \int n^2 \chi_{[(j-1)h, jh] \cap [(k-1)h, kh]} - n^2 \chi_{[(j-1)h, jh] \cap [(kh, (k+1)h]}
\\ & \quad \quad - n^2 \chi_{[jh, (j+1)h] \cap [(k-1)h, kh]} + n^2 \chi_{[jh, (j+1)h] \cap [kh, (k+1)h]} \, dx
\\ & = 2n \delta_{j,k} - n \delta_{j-1, k} - n \delta_{j, k-1}
\end{align*}
\end{proof}


\p{3} Let $S = \{ s \in C^2(\R) : \forall j \in \Z \quad s \text{ is a cubic on } [j, j+1] \}$.  Suppose that $s \in S$ has compact support in $[0,M]$. Determine the smallest value of $M$ such that $s \not \equiv 0$.

\begin{proof}
Suppose $s \in S$ with nonnegative compact support. Since $s'''$ is piecewise constant with knots at the integers, the support of $s$ must be an interval of the form $[a,b]$ for $a,b \in \N$. Further suppose that $s \not \equiv 0$ and $M = b$ is minimized by $s$. Then $a = 0$ since otherwise $s(x+a) \in S$ with support $[0, M-a]$.  Hence the support of $s$ is precisely $[0,M]$ for some $M \in \N$.

For $j \in \Z$, let $s_j$ be the cubic defined by $s$ on $[j, j+1]$. Since $s \in C^2$, we have $s_{j}''(j) - s_{j-1}''(j) = s_{j}'(j) - s_{j-1}'(j) = s_{j}(j) - s_{j-1}(j) = 0$ for all $j$.  Thus, by comparing Taylor expansions at $x = j$, we have $s_{j} = s_{j-1} + a_j (x-j)^3$ for some $a_{j} \in \R$. Since $s_{-1} = 0$, we have $s_j = \sum_{k=0}^{j} a_k (x - k)^3$ for $j \ge 0$.

Let $P^3$ be the vector space of polynomials of degree $3$ or less. Let $T: P^3 \to P^3$ be defined by the matrix $T_{j,k} = (-1)^{k-j} {k \choose j}$ for $0 \le j,k \le 3$ with respect to the basis $(1, x, x^2, x^3)$. Then $T x^k = \sum_{j=0}^k (-1)^{k-j} {k \choose j} x^j = (x-1)^k$ for all $0 \le k \le 3$. Thus $s_j = \sum_{k=0}^{j} a_k T^k x^3$.

Thus, $M$ is equal to the degree of the minimal nonzero polynomial $f$ such that $f(T^k)x^3  = 0$. Note that $(T - I) x^k = (x - 1)^k - x^k$, so if $\deg(g) = k > 0$ then $\deg((T-I)g) = k - 1$.  Thus, $((T-I)^n x^3)_{n=0}^3$ forms a basis for $P^3$. Hence $(T^n x^3)_{n=0}^3$ forms a basis for $P^3$. Hence $M = \deg(f) = 4$.
\end{proof}

\p{4} Let $U := \{u_j\}_{j=1}^\infty$ be an orthonormal set in a Hilbert space $\mH$. Show that the following are equivalent:

i. $U$ is maximal in the sense that there is no non-zero vector in $\mH$ that is orthogonal to $U$.

ii. Every vector in $\mH$ may be uniquely represented as the series $f = \sum_{j=1}^\infty \langle f , u_j \rangle u_j$.

\begin{proof}
Suppose (i) holds but (ii) does not.  Then there exists $f \in \mH$ with $f \neq \sum_{j=1}^\infty \langle f , u_j \rangle u_j =: v$.  Let $g = f - v$.  Then if $u_j \in U$, then $\langle g, u_j \rangle = \langle f, u_j \rangle - \langle v, u_j \rangle = \langle f, u_j \rangle - \langle f, u_j \rangle = 0$. Thus $g$ is a non-zero vector orthogonal to $U$, a contradiction.

For the converse, suppose (ii) holds but (i) does not. Let $w$ be a nonzero vector orthogonal to $U$.  By (ii), $w = \sum_{j=1}^\infty \langle w, u_j \rangle u_j = 0$, a contradiction.
\end{proof}

\p{5} Show that every separable Hilbert space $\mH$ has an o.n. basis.
\begin{proof}
Let $\mU$ be the poset of all o.n. subsets of $mH$.  If $\mL$ is a chain in $\mU$, then $\bigcup \mL$ is orthonormal.  Hence, by Zorn's Lemma, $\mU$ contains a maximal element $U$. 

I claim that $U$ is countable.  Suppose not. If $u,v \in U$ with $u \neq v$, then $\|u - v\|^2 = \|u\|^2 - \langle u, v \rangle - \langle v, u \rangle + \|v\|^2 = 2$.  This implies that $U$ is not separable, a contradiction.

Hence $U$ satisfies condition (i) of Exercise 4, so $U$ is a basis.
\end{proof}

\end{document}
