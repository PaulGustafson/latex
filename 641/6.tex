\documentclass{article}
\usepackage{../m}

\begin{document}
\noindent Paul Gustafson\\
\noindent Texas A\&M University - Math 641\\ 
\noindent Instructor - Fran Narcowich

\subsection*{HW 6}
\p{1} Let $(y_j)_{j=1}^n \subset \R$. Consider the cubic-Hermite spline $s(x) \in S^h(3,1)$, with $h = 1/n$, that satisfies $s(j/n) = y_j$ and minimizes $\int_0^1 (s'')^2 \, dx$. Show that $s(x) \in S^h(3,2)$.


\p{2} We want to solve the boundary value problem (BVP): 
$$-u'' = f(x), u(0) = u(1) = 0, f \in C[0,1]$$.

\p{a.} Let $H$ be the set of all continuous functions vanishing at $x = 0$ and $x = 1$, and having $L^2$ derivatives. Also let $H$ have the inner product
$$\langle f, g \rangle_H = \int_0^1 f'(x) g'(x) \, dx.$$
Use integration by parts to convert the BVP into its ``weak'' form:
$$\langle u, v \rangle_H = \int_0^1 f(x) v(x) \, dx \text{ for all } v \in H.$$
\begin{proof}
We have $\langle u, v \rangle_H = \int_0^1 u' v' \, dx =  - \int_0^1 u'' v \, dx = \int_0^1 f v \, dx$ for all $v \in \mH$.
\end{proof}

\p{b.} Conversely, suppose that $u \in H$ is also in $C^2[0,1]$ and 
$$\langle u, v \rangle_H  = \int_0^1 f(x) v(x) \, dx \text{ for all } v \in H.$$
Show that $u$ satisfies the BVP.
\begin{proof}
Suppose not. Then $-u''(x_0) \neq f(x_0)$ for some $x_0 \in (0,1)$. WLOG assume $u''(x_0) + f(x_0) > 0$.  By continuity, this inequality holds on some interval $(x_0 - \delta, x_0 + \delta)$. Let $v \in H$ be a nonnegative with support in $(x_0 - \delta, x_0 + \delta)$ and $v(x_0) > 0$.  Then $\int f v \, dx - \langle u, v \rangle_H = \int f v \, dx + \int u'' v \, dx  > 0$, a contradiction.
\end{proof}


\p{c.} Let $V = S^h(1,0)$, with $h = 1/n$. Thus, $V$ is spanned by $\phi_j(x) := N_2(nx - j+1), j \in [n-1]$ Show that the least squares approximation to $u$ from $V$ is $y = \sum_j \alpha_j \phi_j(x) \in V$, where $G \alpha = \beta$ for $\beta_j = \langle y , \phi_j \rangle_H$ and $G_{kj} = \langle \phi_j, \phi_k \rangle_H$.

\begin{proof}
This is just the definition of the normal equations, which was proved in HW 1.
\end{proof}

\p{d.} Show that $G_{kj} = \langle \phi_j, \phi_k \rangle_H$ is given by
$G_{j,j} = 2n$, $G{j,j-1} = -n$, and $G_{j, j+1} = n$ and $G_{j,k} = 0$ for all other $k$.



\p{3} Let $S = \{ s \in C^2(\R) : \forall j \in \Z \quad s \text{ is a cubic on } [j, j+1] \}$.  Suppose that $s \in S$ has compact support in $[0,M]$. Determine the smallest value of $M$ such that $s \not \equiv 0$.

\begin{proof}
Define $s_0 \in S$ by ... %FIXME

Since $s'''$ is piecewise constant with knots at the integers, the support of $s$ must be an interval of the form $[a,b]$ for $a,b \in \N$. Pick an $s \in S$ such that $s \not \equiv 0$ and $b$ is minimized.  Clearly $a = 0$ since otherwise $s(x+a) \in S$ with support $[0, b-a]$.  

We have $x^n = ((x - 1) + 1)^n = \sum_j {n \choose j} (x - 1)^j$.

Since $s$ is $C^2$, on $[0,1]$ we have $s = a_0 x^3 = a_0 (x-1)^3 + 3a_0 (x - 1)^2 + 3a_0 (x - 1) + a_0$. 

 On $[1,2]$, we have
$s = a_1 (x-1)^3 + 3a_0 (x - 1)^2 + 3a_0 (x - 1) + a_0 
= a_1 (x - 2)^3 + (3a_1 + 3a_0)(x - 2)^2 + (3a_1 + 6a_0 + 3a_0)(x - 2) + (a_1 + 3a_0 + 3a_0 + a_0) (x - 2)
= a_1 (x - 2)^3 + (3a_1 + 3a_0)(x - 2)^2 + (3a_1 + 9 a_0)(x - 2) + (a_1 + 7 a_0) (x - 2)$.

On $[2,3]$, we have

\end{proof}

\p{4} Let $U := \{u_j\}_{j=1}^\infty$ be an orthonormal set in a Hilbert space $\mH$. Show that the following are equivalent:

i. $U$ is maximal in the sense that there is no non-zero vector in $\mH$ that is orthogonal to $U$.

ii. Every vector in $\mH$ may be uniquely represented as the series $f = \sum_{j=1}^\infty \langle f , u_j \rangle u_j$.

\begin{proof}
Suppose (i) holds but (ii) does not.  Then there exists $f \in \mH$ with $f \neq \sum_{j=1}^\infty \langle f , u_j \rangle u_j =: v$.  Let $g = f - v$.  Then if $u_j \in U$, then $\langle g, u_j \rangle = \langle f, u_j \rangle - \langle v, u_j \rangle = \langle f, u_j \rangle - \langle f, u_j \rangle = 0$. Thus $g$ is a non-zero vector orthogonal to $U$, a contradiction.

For the converse, suppose (ii) holds but (i) does not. Let $w$ be a nonzero vector orthogonal to $U$.  By (ii), $w = \sum_{j=1}^\infty \langle w, u_j \rangle u_j = 0$, a contradiction.
\end{proof}

\p{5} Show that every separable Hilbert space $\mH$ has an o.n. basis.
\begin{proof}
Let $\mU$ be the poset of all o.n. subsets of $mH$.  If $\mL$ is a chain in $\mU$, then $\bigcup \mL$ is orthonormal.  Hence, by Zorn's Lemma, $\mU$ contains a maximal element $U$. 

I claim that $U$ is countable.  Suppose not. If $u,v \in U$ with $u \neq v$, then $\|u - v\|^2 = \|u\|^2 - \langle u, v \rangle - \langle v, u \rangle + \|v\|^2 = 2$.  This implies that $U$ is not separable, a contradiction.

Hence $U$ satisfies condition (i) of Exercise 4, so $U$ is a basis.
\end{proof}

\end{document}
