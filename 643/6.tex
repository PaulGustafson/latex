\documentclass{article}
\usepackage{../m}

\begin{document}
\noindent Paul Gustafson\\
\noindent Math 643 - Algebraic Topology I

%Problems: Problems: 4.9*, 4.10*, 4.11*, 4.13*, 4.16*, 
\subsection*{HW 6}
\p{4.9} \begin{enumerate}[(i)]
\item Using the explicit formula for $\beta_{n+1}$, show that
$$\partial_{n+1} \beta_{n+1} = ( {\lambda_1^\Delta}_\# - {\lambda_0^\Delta}_\# - P^\Delta_{n-1} \partial_n)(\delta)$$
for $n = 0$ and $n=1$.

\begin{proof}

For $n=0$, we have
\begin{align*}
\partial_1 \beta_1 & = \partial ([a_0, b_0]) \\
& = [b_0] - [a_0] \\
& = ( {\lambda_1^\Delta}_\# - {\lambda_0^\Delta}_\# - P^\Delta_{-1} \partial_1)(\delta) 
\end{align*}


For $n=1$, we have 
\begin{align*}
\partial_2 \beta_2 & = \partial ([a_0, b_0, b_1] - [a_0, a_1, b_1]) \\
& = [b_0,b_1] - [a_0,b_1] + [a_0, b_0] - [a_1, b_1] + [a_0, b_1] - [a_0, a_1] \\
& = [b_0,b_1]  - [a_0, a_1] + [a_0, b_0] - [a_1, b_1] \\
& = [b_0,b_1]  - [a_0, a_1] - (\partial_2 \delta \times 1)_\# \beta_1 \\
& = ( {\lambda_1^\Delta}_\# - {\lambda_0^\Delta}_\# - P^\Delta_{0} \partial_2)(\delta) 
\end{align*}
\end{proof}

\item Give an explicit formula for $P_1^X(\sigma)$, where $\sigma: \Delta^1 \to X$ is a 1-simplex.
\end{enumerate}
\begin{proof}
We have 
\begin{align*}
P_1^X(\sigma) & = (\sigma \times 1)_\#(\beta_2)
 = (\sigma \times 1)_\#([a_0,b_0, b_1] - [a_0, a_1, b_1])
\end{align*}
\end{proof}

\p{4.10} Prove that $P_n$ is natural.
\begin{proof}
The diagram holds when $f$ is a simplex.  Extend by linearity.
\end{proof}

\p{4.11} If $X$ is a deformation retract of $Y$, then $H_n(X) \simeq H_n(Y)$ for all $n \ge 0$. In fact, if
$i: X \to Y$ is the inclusion, then $H_n(i)$ is an isomorphism.
\begin{proof}
Since $X$ is a deformation retract of $Y$, $i \simeq 1_Y$. Hence $H_n(i) = H_n(1_Y) = 1$ is an isomorphism. In particular $H_n(X) \simeq H_n(Y)$ for all $n \ge 0$.
\end{proof}

\p{4.13} Prove that the Hurewicz map $\phi$ is natural.
\begin{proof}
We have 
\begin{align*}
\phi h_* [f] &  = \phi [h f] \\
& = \cls (h f \eta) \\
& = h_* \cls (f \eta) \\
& = h_* \phi([f])
\end{align*}
\end{proof}

\p{4.16} If $f:S^1 \to S^1$ is continuous, define the degree of $f$ to be $m$ if the induced map $f_*:H_1(S^1) \to H_1(S^1)$ is 
multiplication by $m$. Show that this definition of degree coincides with the degree of a pointed map $(S^1, 1) 
\to (S^1, 1)$ defined in terms of $\pi_1(S_1,1)$.
\begin{proof}
Let $f:(S^1, 1) \to (S^1, 1)$ be a pointed map with degree $m$ with respect to the new definition.
Let $i:(S^1, 1) \to (S^1, 1)$ be the identity map.

By Theorem 4.29, the Hurewicz map is an isomorphism $\phi:\pi_1(S^1,1) \simeq H_1(S^1)$.
Hence $\pi_1(f) = \phi^{-1} H_1(f) = \phi^{-1} H_1(f \circ i) = \phi^{-1}(m H_1(i))
= (\pi_1(i))^m$.
Thus, the degree of $f$ is $m$ with respect to the old definition also.
\end{proof}

\end{document}
