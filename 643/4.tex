\documentclass{article}
\usepackage{../m}

\begin{document}
\noindent Paul Gustafson\\
\noindent Math 643 - Algebraic Topology I

%Problems

\subsection*{Problem Set 1}
\p{1.12} Let $X,Z$ be compact Hausdorff spaces, and let $h:X \to Z$ be a continuous
surjection. Prove that $\phi: X /\ker h \to Z$, defined by $[x] \to h(x)$, is a homeomorphism.
\begin{proof}
Since $X/\ker h$ is a continuous image of the compact set $X$, it is compact. Since $Z$
is Hausdorff and $\phi$ is a continuous bijection, this implies $\phi$ is a homeomorphism
($\phi$ maps compact sets to compact sets, hence closed sets to closed sets).
\end{proof}

\p{1.13} For fixed $t$ with $0 \le t < 1$, prove that $f: x \mapsto [x,t]$ defines a homeomorphism
from a space $X$ to a subspace of $CX$.
\begin{proof}
This map is continuous since the map $x \mapsto (x,t) \subset X \times I$ is continuous and
respects the equivalence relation $\sim$.  The map $f$ is also bijective since $\sim$ only identifies
points of the form $[x,1]$.  

To see that $f^{-1}$ is continuous, let $U \subset X$ be open. Then $U \times [0, (t+1)/2) \subset X \times I$
is open.  Let $\pi: X \times I \to CX$ be the canonical quotient map.
Then $f(U) = \pi(U \times [0, (t+1)/2)) \cap f(X)$ is open in $f(X)$.
\end{proof}

\p{2.9}

\p{2.10}

\p{5}  Show that if $x$ deformation retracts to $A$ in the weak sense, then the
inclusion map $A \to X$ is a homotopy equivalence.

\end{document}
