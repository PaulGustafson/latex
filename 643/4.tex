\documentclass{article}
\usepackage{../m}

\begin{document}
\noindent Paul Gustafson\\
\noindent Math 643 - Algebraic Topology I

%Problems

\subsection*{Problem Set 1}
\p{1.12} Let $X,Z$ be compact Hausdorff spaces, and let $h:X \to Z$ be a continuous
surjection. Prove that $\phi: X /\ker h \to Z$, defined by $[x] \to h(x)$, is a homeomorphism.
\begin{proof}
Since $X/\ker h$ is a continuous image of the compact set $X$, it is compact. Since $Z$
is Hausdorff and $\phi$ is a continuous bijection, this implies $\phi$ is a homeomorphism
($\phi$ maps compact sets to compact sets, hence closed sets to closed sets).
\end{proof}

\p{1.13} For fixed $t$ with $0 \le t < 1$, prove that $f: x \mapsto [x,t]$ defines a homeomorphism
from a space $X$ to a subspace of $CX$.
\begin{proof}
This map is continuous since the map $x \mapsto (x,t) \subset X \times I$ is continuous and
respects the equivalence relation $\sim$.  The map $f$ is also bijective since $\sim$ only identifies
points of the form $[x,1]$.  

To see that $f^{-1}$ is continuous, let $U \subset X$ be open. Then $U \times [0, (t+1)/2) \subset X \times I$
is open.  Let $\pi: X \times I \to CX$ be the canonical quotient map.
Then $f(U) = \pi(U \times [0, (t+1)/2)) \cap f(X)$ is open in $f(X)$.
\end{proof}

\p{2.9} If $\{p_0, p_1, \ldots, p_m\}$ is affine independent with barycenter $b$, then $\{b, p_0, \cdots, 
\hat p_i, \ldots, p_m \}$ is affine independent for each $i$.
\begin{proof}
Fix $0 \le i \le m$.  Suppose $sb + \sum_{j \neq i} s_j p_j = 0$ with $s + \sum_{j \neq i} s_j = 0$ for some $s,s_j \in \R$.
Then we have
\begin{align*}
0 & = sb + \sum_{j \neq i} s_j p_j \\
& = \frac{s}{m+1} \sum_j p_j + \sum_{j \neq i} s_j p_j \\
& = \sum_j t_j p_j,
\end{align*}
where $t_j = \frac{s}{m+1} + s_j$ for $j \neq i$, and $t_i = \frac{s}{m+1}$.
Hence $\sum_j t_j = s + \sum_{j \neq i} s_j = 0$.  Since $\{p_i\}_i$ is affine independent,
this implies $t_j = 0$ for all $j$.  Thus $s = 0$ and $s_j = 0$ for all $j \neq i$.
Thus $\{b, p_0, \cdots, \hat p_i, \ldots, p_m \}$ is affine independent.

It is false that the diameter of a simplex with the barycenter replacing a vertex is
always strictly smaller than the diameter of the original simplex. 
Take an equilateral triangle as an example.
\end{proof}

\p{2.10} Show that for $0 \le i \le m$, $[p_0, \ldots, p_m]$ is homeomorphic to the cone
$C[p_0, \ldots, \hat p_i, \ldots, p_m]$ with vertex $p_i$.
\begin{proof}
Since $C[p_0, \ldots, \hat p_i, \ldots, p_m]$ is the continuous image of the compact set
$[p_0, \ldots, \hat p_i, \ldots, p_m]$, it is compact.  Also $[p_0, \ldots, p_m]$ is Hausdorff.
Hence it suffices to find a continuous bijection from $C[p_0, \ldots, \hat p_i, \ldots, p_m]$
to $[p_0, \ldots,  p_m]$.

Define $\phi: [p_0, \ldots, \hat p_i, \ldots, p_m] \times I \to [p_0, \ldots,  p_m]$ by
$\phi( \sum_{j \neq i} s_j p_j, t) = t p_i + \sum_{j \neq i} s_j (1-t) p_j$.
%Let $\pi: [p_0, \ldots, \hat p_i, \ldots, p_m] \times I \to C[p_0, \ldots, \hat p_i, \ldots, p_m]$ be the canonical quotient map.  
Since $\phi(x, 1) = p_i$ for all $x$, $\phi$ induces a continuous map 
$\bar \phi : C[p_0, \ldots, \hat p_i, \ldots, p_m] \to [p_0, \ldots,  p_m]$. Since $\phi$ is surjective, so is $\bar \phi$.

To see that $\bar \phi$ is injective, first note that $\phi$ maps only the vertex to the point $p_i$. Next if
$\bar \phi [\sum_{j \neq i} s_j p_j, t]  = \bar \phi [\sum_{j \neq i} r_j p_j, u] \neq p_i$, then
$(u-t) p_i + \sum_{j \neq i} (r_j - s_j) p_j$.  Hence, $u = t$ and $r_j = s_j$ for all $j \neq i$ since
the $p_j$ are affine independent.  Thus $\bar \phi$ is injective.
\end{proof}

\p{5}  Show that if $x$ deformation retracts to $A$ in the weak sense, then the
inclusion map $A \to X$ is a homotopy equivalence.
\begin{proof}

\end{proof}

\end{document}
