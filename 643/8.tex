\documentclass{article}
\usepackage{../m}

\begin{document}
\noindent Paul Gustafson\\
\noindent Math 643 - Algebraic Topology I

%Problems: 3.20, 3.21, 3.24, fix proof, 5.14
\subsection*{Problem Set 2} 
\p{3.20} Let $X$ be a space with basepoint $x_0$, and let $\{U_j : j \in J\}$ be an open cover of $X$
by path connected subspaces such that:
\begin{enumerate}[(i)]
\item $x_0 \in U_j$ for all $j$;
\item $U_j \cap U_k$ is path connected for all $j,k$.
\end{enumerate}
Prove that $\pi_1(X,x_0)$ is generated by the subgroups $\im i_{j*}$ where $i_j:(U_j, x_0) \to (X, x_0)$ is the inclusion.
\begin{proof}
The set $\{f^{-1}(U_j)\}_j$ is an open cover of $I$. Let $\delta$ be its Lebesgue number. Pick $N$ such that $1/N < \delta$.

For each $1 \le i \le N$, there exists $U_{j_i}$ such that $f([(i-1)/N), i/N]) \subset U_{j_i}$ by the definition of Lebesgue number.
For $1 \le i < N$, the point $f(iN)$ lies in $U_i \cap U_{i+1}$, a path connected space.  Let $p_i$ be a path from $x_0$ to $f(iN)$
in $U_i \cap U_{i+1}$. If $i = 0$ or $i=N$, let $p_i$ be the constant path at $x_0$.
Then for $1 \le i \le N$ Then $g_i := p_{i-1} * f_{|[(i-1)/N), i/N]} * p_{i}^{-1}$ defines a path in $U_i$ with endpoints at $x_0$.
The fact that $[f] = [g_1]*[g_2]* \ldots * [g_N]$ proves the desired result.
\end{proof}

\p{3.21} If $n \ge 2$, prove that $S^n$ is simply connected.
\begin{proof}
Let $U_1, U_2 \subset S^n$ be the complements of the north and south poles, respectively.  By Exercise 3.20,
it suffices to show that both $U_1$ and $U_2$ are simply connected.  However, the stereographic projection
from the north pole defines a homeomorphism from $U_1$ to $\R^n$ which is simply connected. Thus $U_1$ is
simply connected.  A similar argument works for $U_2$.
\end{proof}

\p{3.24} Let $G$ be a simply connnected topological group and let $H$ be a discrete closed
normal subgroup. Prove that $\pi_1(G/H, 1) \simeq H$.
\begin{proof}
Let $\nu: G \to G/H$ be the canonical quotient map. Let $f$ and $(X,x_0)$ be defined as in Lemma 3.14 with
codomain $(G/H,1)$ instead of $(S^1, 1)$. I will prove the analog of Lemma 3.14 with
$\nu$ replacing $\exp$ and $G$ replacing $\RR$.

Since $H$ is discrete, there exists an open $U \subset G$ with $U \cap H = 1$. 

I claim that there exists $V \subset U$, a symmetric open neighborhood of $1$ with $V*V \subset U$.
Indeed, there exists open neighborhoods $A,B$ of $1$ with $A*B \subset U$ by the continuity of 
group multiplication.  Therefore $W := A \cap B$ is an open neighborhood of $1$ with $W*W \subset U$.
Moreover $W^{-1}$ is an open neighborhood of $1$.  Hence $V:= W \cap W^{-1}$ works.

Note that $Vg$ is the inverse image of $V$ under multiplication by $g^{-1}$ on the right, so is open for all $g \in G$. 
Therefore $\{f^{-1}(Vg)\}_g$ is an open cover of $X$. Let $\epsilon$ be the Lebesgue number of this cover. It follows that if $\|x - x'\| < \epsilon$
then $f(x) f(x')^{-1} \in Vg g^{-1}V = V$.

The rest of the proof of Lemma 3.14 follows with $V$ replacing $(-\frac 1 2, \frac 1 2)$ and $\nu(V)$ replacing $S^1 - \{-1\}$. The only
sticky part is showing that $\nu_{|V}$ is a homeomorphism onto its image.

First $\nu_{|V}$ is injective since $H \cap V = \{1\}$. Thus $\nu_{|V}$ is a continuous bijection onto its image, so it suffices to show that
$\nu$ is an open map. Suppose $U \subset G$ is open. Then $\nu^{-1}(\nu(U)) = \bigcup_{h \in H} Vh$ is the union of open sets, hence open.
Hence $\nu(U)$ is open since $\nu$ is a quotient map.  Thus $\nu_{|V}$ is a homeomorphism onto its image.

The rest of the proof follows with the obvious modifications (replacing $+$ with the group multiplication, and $\exp$ with $\nu$).
\end{proof}

\p{4}
%1) reformulate the naturally condition
%2) check it still holds for the start of the induction 
%3) fix the part of the proof of the key formula in the text (using the new naturally condition)
%4) verify the new naturally condition holds for n.

\p{5.14}\begin{enumerate}[(i)]
\item \begin{proof}
Since $i_{n-1}$ is injective,  we have $0 = \ker A_{n-1} = \im (C_n \to A_{n-1})$ by the exactness of the 
long sequence at $A_{n-1}$. Hence, by the exactness at $C_n$ in the long sequence, we have $\im p_n = C_n$.

The exactness at $B_n$ in the long sequence implies exactness at $B_n$ in the short sequence.  Since
$i_n$ is injective, we get exactness at $A_n$ in the short sequence.
\end{proof}

\item If $A$ is a retract of $X$, prove that for all $n \ge 0$,
$$ H_n(X) \simeq H_n(A) \oplus H_n(X,A).$$

\begin{proof}
Since $A$ is a retract of $X$, the s.e.s $0 \to A \to X \to X/A to 0$ splits.

Therefore $0 \to S_*(A) \overset{i} \to S_*(X) \overset{p} \to S_*(X/A) to 0$ also splits.

Applying Theorem 5.6, we get an exact sequence
$$ \ldots \to H_n(A) \overset{i_*}\to H_n(X) \overset{p_*}\to H_n(X,A) \to H_{n+1}(A) \to \ldots. $$
Since $i$ has a left inverse, so does $i_*$ for all $n$ since $H_n$ is a functor.
In particular, $i_*$ is invertible for all $n$.

Thus by (i), we have the split exact sequence
$$0 \to H_n(A) \to H_n(X) \to H_n(X,A) \to 0.$$
\end{proof}

\item if $A$ is a deformation retract of $X$, then $H_n(X,A) = 0$ for all $n \ge 0$.
\begin{proof}
Since $A$ is a deformation retract of $X$, the inclusion $i:A \to X$ is a homotopy equivalence.
Since $H_n$ factors through $\mathbf{hTop}$, the map $i_*:= H_n(i)$ is an isomorphism,
and in particular surjective.
The map $p_*$ in the s.e.s.
$$0 \to H_n(A) \overset{i_*} \to H_n(X)  \overset{p_*} \to H_n(X,A) \to 0,$$
is must be the $0$-map and also be surjective.  Hence $H_n(X,A) = 0$.
\end{proof}

\end{enumerate}  

\end{document}
