\documentclass{article}
\usepackage{../m}

\begin{document}
\noindent Paul Gustafson\\
\noindent Math 643 - Algebraic Topology I

%1.14* or Top* instead of Top, 1.15, 1.19*, 1.22, 1.23*, 1.24(read), 1.29, 1.30, 1.31(read) 1.32*, 1.33, 1.34*, 2.8*
%Problems with an asterisk (*) are due 9/23. 

\subsection*{HW 3}
\p{1.14} Prove that $X \mapsto CX$ defines a functor $\mathbf{Top}  \to \mathbf{Top}$. (Hint: use exercise 1.11)
\begin{proof}
If $f:X \to Y$ is continuous, then $f \times 1: X \times I \to Y \times I$ is continuous. Since $f \times 1$ maps
$(x,1)$ to $(f(x), 1)$ for all $x$, it preserves the relation $\sim$ from the definition of the cone over a space.
Hence, by Exercise 1.11, we get a map $\overline f : CX \to CY$. 

To see that the association $f \mapsto \overline f$ is functorial, we need to check that it preserves identities
and composition. It is clear from the definition that $\overline 1_x = 1_{CX}$.  For composition, for $t \neq 1$,
we have $\overline{fg}(x,t) = ((fg)(x),t) = (\overline f \circ \overline g) (x,t)$.  For the other case, we have
$\overline{fg}(x,1) = (*,1) = (\overline f \circ \overline g) (*,1)$ where
$(*,1)$ denotes the vertex of the cone.
\end{proof}

\p{1.19} \begin{enumerate}[(i)] 
\item A space $X$ is path connected iff every two constant maps $X \to X$ are homotopic.
\item If $X$ is contractible and $Y$ is path connected, then any two continuous maps
$X \to Y$ are homotopic (and each is nullhomotopic).
\end{enumerate}

\begin{proof}
For (i), first suppose $X$ is path connected.  Let $c_x, c_y: X \to X$ be the constant maps at $x$ and $y$.
Since $X$ is path connected, there exists a path $p:I \to X$ from $x$ to $y$.  
The map $H:X \times I \to X$ by $H(x,t) = p(t)$ is a homotopy from $c_x$ to $c(y)$.

Conversely, every two constant maps $X \to X$ are homotopic. Let $x,y \in X$ and $H: c_x \simeq c_y$. 
Define $p:I \to X$ by $p(t) = H(x_0,t)$ for any fixed $x_0 \in X$. Then $p$ is a path from $x$ to $y$.

For (ii), let $f, g: X \to Y$. Since $X$ is contractible there exists $x_0 \in X$ with a homotopy $H: 1_X \simeq c_{x_0}$.
Let $p$ be a path from $f(x_0)$ to $g(x_0)$. Let $G:X \times I \to Y$ be defined by the concatenation 
$(f \circ H) * p(t) * (g \circ H^{-1})$. Then $G: f \simeq g$.
\end{proof}

\p{1.23} 
\begin{enumerate}[(i)] 
\item The $\sin(1/x)$ space $X$ has exactly two path components: the vertical line $A$ 
and the graph $G$.
\item Show that the graph $G$ is not closed. Conclude that, in contrast to components
(which are always closed), path components may not be closed.
\item Show that the natural map $\nu: X \to X/A$ is not an open map. (Hint: Let $U$
be the open disk with center $(0/\frac 1 2)$ and radius $\frac 1 4$; show that $\nu(X \cap U)$
is not open in $X/A (\approx [0, \frac 1 {2 \pi}])$.)
\end{enumerate}

\begin{proof}
\begin{enumerate}[(i)] 
\item It is clear that $A$ and $G$ are both path connected. Hence it suffices to show that
there is no path in $X$ from $(0,1)$ to $(1/\pi, 0)$. Suppose such a path $p$
exists. Then $\lim_{t \to 0} p(t) = (0,1)$. Hence, there exists $t_0$ such that the $y(p(t)) > 1/2$
for all $t < t_0$, where $y(\cdot)$ denotes the projection on the $y$-coordinate. 
Pick $n$ so large that $(2\pi n)^{-1} < t_0$.  Then since $\sin(1/x) < 0$ for 
$(2\pi n + 2\pi)^{-1} < x < (2\pi n + \pi)^{-1}$, the graph of $p$ cannot meet this strip.
Thus, the graph of $p$ is disconnected, a contradiction.

\item The sequence $(1/(n\pi), 0)$ lies in $G$, but its limit is the origin.

\item Following the hint, let $U$ be the open disk with center $(0, \frac 1 2)$ and radius $\frac 1 4$. 
I claim that $\nu(X \cap U)$ is not open in $X / A$.  If it were, then $\nu^{-1} \nu(X \cap U) = A \cup (X \cap U)$ is open in $X$.
This is not true since any neighborhood of, say, $(0, - \frac 1 2)$ must intersect $G$ below the $x$-axis.
\end{enumerate}

\end{proof}



\p{1.32} Assume that $X$, $Y$, and $Z$ are spaces with $X \subset Y$. If $X$ is a retract, then 
every continuous map $f: X \to Z$ can be extended to a continuous map $\tilde f : Y \to Z$, 
namely, $\tilde f = fr$, where $r: Y \to X$ is a retraction. Prove that if $X$ is a retract of
$Y$ and if $f_0$ and $f_1$ are homotopic continuous maps $X \to Z$, then $\tilde f_0 \simeq \tilde f_1$.

\begin{proof}
Let $F: f_0 \simeq f_1$. Let $G: Y \times I \to Z$ be defined by $G(y,t) = F(r(y), t)$. Then 
$G(y,0) = F(r(y),0) = f_0r(y) = \tilde f_0(y)$, and similarly for $G(y,1)$. Hence $G$ is the desired homotopy.
\end{proof}


\p{1.34} \begin{enumerate}[(i)]
\item Define $i:X \to M_f$ by $i(x) = [x,0]$ and $j: Y \to M_f$ by $j(y) = [y]$. Show that
$i$ and $j$ are homeomorphisms to subspaces of $M_f$.
\begin{proof}
It is easy to see that $i,j$ are continuous. The map $i$ is injective since the relation $\sim$ does
not affect its image.  The map $j$ is also injective since the fibers of $f$ are disjoint (hence only
one $y \in Y$ is in each equivalence class of $\sim$). Thus it suffices to show that $i,j$ are open maps.

Let $U \subset X$ be open.  Then $i(U) = \pi_X^{-1}(U) \cap i(X)$ is open, where $\pi_X: X \times I \to X$ is the canonical projection.
Hence $i$ is an open map.

Let $U \subset Y$ be open.  Let $\nu: (X \times I ) \sqcup Y \to M_f$ be the quotient map, and
$i_1: X \times I \to  (X \times I ) \sqcup Y$ and $i_2: Y \to  (X \times I ) \sqcup Y$ be the 
canonical injections.   Then
$\nu^{-1}j(U) = i_1(f^{-1}(U) \times \{1\}) \cup i_2(U) = (i_1(f^{-1}(U)) \cap j(Y)) \cup i_2(U)$
is open, hence $j(U)$ is open.
\end{proof}

\item Define $r:M_f \to Y$ by $r[x,t] = f(x)$ for all $(x,t) \in X \times I$ and $r[y] = y$. Prove
that $r$ is a retraction $rj = 1_Y$.

\begin{proof}
If $y \in Y$ and $[x,1] \sim y$, then $r[x,1] = f(x) = y = r[y]$.  Hence $r$ is well-defined. 

It is also clear that $rj = 1_Y$.  To see that $r$ is continuous, note that $r = \nu \circ ((f\pi_X) \sqcup id_Y)$,
where $\nu$ is the quotient map defined above and $\pi_X:X \times I \to X$ is the canonical projection.
\end{proof}


\item Prove that $Y$ is a deformation retract of $M_f$. (Hint: Define $F:M_f \times I \to M_f$ by
\begin{align*}
F([x,t], s) & = & [x, (1-s)t+ s] & \text{ if } x \in X, t \in I; \\
F([y], s) & = & [y] & \text{ if } y \in Y, s \in I
\end{align*}
.)

\begin{proof}
Let $F$ be defined as in the hint. To see that $F$ is a well-defined continous map, let 
$G: ((X \times I) \sqcup Y) \times I \to ((X \times I) \sqcup Y) \times I$ be defined by
$G(x,t,s) = (x, (1-s)t+ s)$ and $G(y,s) = y$ for $(x,t) \in X \times I$ and $y \in Y$.
Then $G$ is continuous, and $G(x,1,s) = (x,1)$ and $G(y,s) = y$ for all $x \in X$ and $y \in Y$. Thus,
$G$ respects the identification map defining $M_f$.  Hence $F$, the induced map, is a continuous map.

Moreover, $F([x,t], 0) = [x,t]$ and $F([y],0) = [y]$, so $F( \cdot, 0) = 1_{M_f}$. Lastly,
$F([x,t], 1) = [x, 1] \in Y$ and $F([y],1) = [y]$.  Thus, $Y$ is a deformation retract of $M_f$.
\end{proof}



\item Show that every continuous map $f: X \to Y$ is homotopic to $r \circ i$, where $i$ is
 an injection and $r$ is a homotopy equivalence.

\begin{proof}
Let $i$ be the injection $i$ from part (i). Let $r$ be defined as in (ii).  The proof
of (iii) shows that $j \circ r \simeq 1_{M_f}$, so $r$ is a homotopy equivalence.
\end{proof}


\end{enumerate}

\p{2.8} Let $A \subset \R^n$ be an affine set and let $T: A \to \R^k$ be an affine map. If
$X \subset A$ is affine (or convex), then $T(X) \subset \R^k$ is affine (or convex). In
particular, if $a,b$ are distinct points in $A$ and if $l$ is the line segment with endpoints
$a,b$, then $T(l)$ is the line segment with points $T(a), T(b)$ if $T(a) \neq T(b)$, and 
$T(l)$ collapses to the point $T(a)$ if $T(a) = T(b)$.

\begin{proof}
To see that $T(X)$ is affine (convex), let $T(a), T(b) \in T(X)$.  Then
for any $\alpha + \beta = 1$ (with $\alpha,\beta$ nonnegative in the convex case),
 we have $\alpha T(a) + \beta T(b) = T(\alpha a + \beta b) \in T(X)$ since 
$\alpha a + \beta b \in X$ since $X$ is affine (convex).  
\end{proof}

\end{document}
