\documentclass{article}
\usepackage{../m}

\begin{document}
\noindent Paul Gustafson\\
\noindent Math 643 - Algebraic Topology I

%Problems: Problems: 1.1, 1.3, 1.4, 1.6*, 1.8*, 1.9*, 1.10*, 1.11* 
%Problems with and asterisk (*) are due 9/16. 
\subsection*{HW 2}
\p{1.6} Contractible sets and hence convex sets are connected. 
\begin{proof}
Suppose not.  Then there exists a contractible space $X$ and a disconnection $X = U \cup V$. Let $x$ be a point for which $1_X \simeq c_x$. WLOG $x \in U$. Let $F:X \times I \to X$ be a homotopy from $1_x$ to $c_x$. 

Pick $y \in V$. Let $f(t) = F(y,t)$. Then by considering $f(I) \cap U$ and $f(I) \cap V$, we see that $f(I)$ is disconnected.  Hence $I$ is disconnected, a contradiction.
\end{proof}

\p{1.8} \begin{enumerate}[(i)]
\item Give an example of a continuous image of a contractible space that is not contractible.
\item Show that a retract of a contractible space is contractible.
\end{enumerate}
\begin{proof}
For (i), the circle is the image of the line under the winding map.

For (ii), let $X$ be contractible and $r: X \to Y$ be a retraction. Pick any $y \in Y$. Since $X$ is contractible, there exists $x \in X$ and  a homotopy  $F:1_X \simeq c_x$. 

Define $G: Y \times I \to Y$ by $G = r \circ F$. Then $G(y, 0) = r(F(y, 0)) = r(y) = y$, and $G(y, 1) = r(F(y,1)) = r(x)$.  Hence $G$ is a homotopy from $Y$ to $1_{r(x)}$.
\end{proof}



\p{1.9} If $f: X \to Y$ is nullhomotopic and if $g: Y \to Z$ is continuous, then $g \circ f$ is 
null-homotopic.

\begin{proof}
Since $f$ is nullhomotopic there exists a homotopy $F:f \simeq c_x$ for some constant map $c_x:X \to Y$.  Define $G:X \times I \to Z$ by $G = g \circ F$. Then $G(x,0) = g(F(x,0)) = g(f(x))$ and $G(x,1) = g(F(x,1)) = g(c_x)$ is a constant map. Thus $g \circ f$ is nullhomotopic.
\end{proof}

\p{1.10} Let $f:X \to Y$ be an identification, and let $g: Y \to Z$ be a continuous surjection.
Then $g$ is an identification iff $gf$ is an identification.

\begin{proof}
Suppose $g$ is an identification. Then $gf$ is a continuous surjection.
Suppose $U \subset Z$ with $(gf)^{-1}(U) = f^{-1} g^{-1}(U)$ open. Since $f,g$ are identifications,
we have $g^{-1}(U)$ is open and then $U$ is open. Hence $gf$ is an identification.

For the converse, suppose $gf$ is an identification.  Suppose $U \subset Z$ with $g^{-1}(U)$ open.
Then $(gf)^{-1}(U) = f^{-1}g^{-1}(U)$ is open since $f$ is continuous.  Hence $U$ is open since $gf$
is an identifiction. Since $g$ is already a continuous surjection, $g$ is an identification.
\end{proof}


\p{1.11} Let $X$ and $Y$ be spaces with equivalence relations $\sim$ and $\square$, respectively, and
let $f: X \to Y$ be a continuous map preserving the relations (if $x \sim x'$, then $f(x) \square f(x')$).
Prove that the induced map $\overline f: X/\sim \to Y / \square$ is continuous; moreover, if
$f$ is an identification, then so is $\overline f$.

\begin{proof}
Let $\psi$ and $\pi$ be the identification maps for $\sim$ and $\square$, respectively. Suppose
$U \subset Y / \square$ is open.  Then $(\overline f)^{-1}(U) = \psi f^{-1} \pi^{-1}(U)$ is open
since $\psi, \pi$ are identifications and $f$ is continuous. Hence $\overline f$ is continuous.

If $f$ is an identification, then $\overline f$ is surjection since $f$ is a surjection.
Suppose $(\overline f)^{-1}(U) = \psi f^{-1} \pi^{-1}(U)$ is open.  Then since $\psi, f, \pi$ are
identifications, we have that $U$ is open.  Hence, $\overline f$ is an identification.
\end{proof}


\end{document}