\documentclass{article}
\usepackage{../m}

\begin{document}
\noindent Paul Gustafson\\
\noindent Math 663 

%Problems: 

\subsection*{Problems}
\p{1} Give a proof of the mean ergodic theorem using the spectral theorem for unitary operators.
\begin{proof}
Let $U$ be a unitary operator on a Hilbert space $H$. By the spectral theorem, there exists a unitary map $T: H \to L^2(X, \mu)$ for some finite measure space $(X, \mu)$ with $U = T^{-1}ST$ where $S$ is multiplication by a function $f$ taking values on the unit circle.

Note that we have $\lim_{N \to \infty} \frac 1 N \sum_{n=0}^N f^n = \chi_{f^{-1}(1)}$. Thus, $P := \lim_{N \to \infty} \frac 1 N \sum_{n=0}^N U^n$ exists and is a orthogonal projection. 

I claim that $P$ is the orthogonal projection onto $\ker(I - U)$. If $v \in \ker(I - U)$, then $T^{-1}STv = Uv = v$. Hence $S(Tv) = Tv$. Therefore $f(x) = 1$ for all $x$ where $Tv(x) \neq 0$. This implies that $Pv = T^{-1} \chi_{f^{-1}(1)} Tv = v$. All these steps are reversible, so the range of $P$ is precisely $\ker(I-U)$.
\end{proof}

\p{2} Prove Khintchine's recurrence theorem: If $G$ is a countable amenable group and $G$ acts on $(X,\mu)$ via a p.m.p. action then for every measurable
set $A \subset X$ and $\epsilon > 0$ the set
$S:= \{ s \in G: \mu(sA \cap A) \ge \mu(A)^2 - \epsilon\}$
is syndetic.
\begin{proof}
Let $t \in G$, $\{F_n\}_{n=1}^\infty$ be a tempered Folner sequence in $G$, and $P$ be the orthogonal projection onto the subspace of $G$-invariant functions in $L^2(X)$. From the mean ergodic theorem, we have
\begin{align*}
 \lim_{n \to \infty} \frac 1 {|F_n|} \sum_{s \in tF_n} \mu(sA \cap A) 
& = \lim_{n \to \infty} \frac 1 {|F_n|} \sum_{s \in tF_n} \langle  \chi_{sA}, \chi_{A} \rangle \\
& =  \langle \lim_{n \to \infty} \frac 1 {|F_n|} \sum_{s \in F_n}  s \chi_{A}, t^{-1} \chi_{A} \rangle \\
& = \langle P \chi_{A} , t^{-1} \chi_{A} \rangle \\
& = \langle t P \chi_{A} , \chi_{A} \rangle \\
& = \langle P \chi_{A} , \chi_{A} \rangle \\
& = \|P\chi_A\|_2^2 \\
& \ge \langle P\chi_A, 1 \rangle ^2 \\
& = \langle \chi_A, 1 \rangle ^2 \\
& = \mu(A)^2.
\end{align*}

Note that in the limiting step the error is 
$$
|\langle P \chi_{A} - \frac 1 {|F_n|} \sum_{s \in F_n}  s \chi_{A}, t^{-1} \chi_{A} \rangle| \le \|P \chi_{A} - \frac 1 {|F_n|} \sum_{s \in F_n}  s \chi_{A}\|_2 \|\chi_{A}\|_2,
$$
and the last bound is independent of $t$.

It follows that by choosing $n$ sufficiently large we can ensure that 
$$\frac 1 {|F_n|} \sum_{s \in tF_n} \mu(sA \cap A) \ge \mu(A)^2 - \epsilon$$
for all $t \in G$. This implies that for any $t \in G$ there exists $s \in F_n$ such that $ts \in S$. Thus $S$ is syndetic.
\end{proof}

%\p{3} Let $G$ be a countable group acting on $(Y^G, \nu^G)$ by a Bernoulli action. Determin the associated Koopman representation of $G$ on 
%$L^2(Y^G, \nu^G)$.
%\begin{proof}
%The Bernoulli action of $s \in G$ on $x \in Y_G$ is defined by $(sx)_t = x_{s^{-1}t}$.
%\end{proof}


\p{5} Give examples of unitary representations $\pi$ and $\rho$ of $\ZZ$ such that $\pi \otimes \rho$ is ergodic but neither $\pi$ nor $\rho$ is weakly mixing.
\begin{proof}
Let $\pi = \rho$ be the one-dimensional representation taking $1$ to multiplication by $i$.  Since this representation is finite-dimensional, it is not weakly mixing. Moreover, $(\pi \otimes \rho)(1)$ acts by multiplication by $-1$, so $\pi \otimes \rho$ is ergodic.
\end{proof}

%\p{6} Give an example of a unitary representation of $\ZZ$ on an infinite-dimensional Hilbert space which is weakly mixing but not mixing.
%\begin{proof}
%Let $\{e_n\}_{n = 1}^\infty$ be the standard basis for $l^2$.
%
%Note that $\{e_1 - e_2 - e_3 - \ldots -e_{n} + n e_{n+1}\}_{n=1}^\infty$ forms an orthogonal set. Let us call the normalized version $\{u_n\}_{n=1}^\infty$.
%
%Now note that $e_1 -\sum_{j=2}^k e_j$ is orthogonal to $\{u_n\}_{n=1}^{k-1}$. Thus, the distance from $e_1$ to $\spn\{u_n\}_{n=1}^{k-1}$ is equal to $1/\sqrt{k}$. It follows that $e_1$ is in the closed linear span of $\{u_n\}_{n=1}^\infty$.
%
%Define $\pi(1)$ by mapping $e_n$ to $u_n$ for all $n \ge 1$.  This defines an isometry with dense range, hence a unitary operator.
%
%This representation is not mixing because $\langle \pi(s) e_1, e_1 \rangle \ge 2^{-s/2} + 3^{-s/2} + \ldots $ does not go to $0$ as $s \to \infty$.
%
%This representation is weakly mixing because it has no nonzero finite dimensional subrepresentations.
%\end{proof}

\p{7} Show that a countable discrete group $G$ is amenable iff every continuous action of $G$ on a compact Hausdorff space has an invariant Borel probability measure.
\begin{proof}
Suppose $G$ is amenable.  The canonical map $\beta:l^\infty(G) \to C(\beta G)$ (extending a bounded function to the Stone-Cech compactification) is a $G$-equivariant $C^*$-algebra isomorphism. Thus, the left-invariant mean on $l^\infty(G)$ induces a $G$-invariant state on $\beta G$.  By the Riesz Representation theorem, this gives us an invariant Borel probability measure on $\beta G$.

Now  suppose $G$ acts continuously on a compact Hausdorff space $K$. Fix any point $x_0 \in K$. Define $f:G \to K$ by $f(s) = sx_0$. Clearly $f$ is $G$-equivariant, so $\beta f: \beta G \to K$ is equivariant also. The pushforward of the measure on $\beta G$ by the function $\beta f$ is the desired measure.

Now suppose the converse holds. Then action of $G$ on $\beta G$ gives us an invariant Borel probability measure on $\beta G$. Integrating against this measure and making use of the properties of the map $\beta$ mentioned above, we get an invariant mean on $l^\infty(G)$.
\end{proof}

\p{8} Show that a subgroup $H$ of an amenable countable discrete group $G$ is amenable.
\begin{proof}
Suppose $H$ were not amenable.  Then it admits a paradoxical decomposition $C \sim D \sim H$. Let $R$ be a complete set of representatives for the right cosets of $H$ in $G$. Then $\{C_iR\}_{C_i \in C}$, $\{D_iR\}_{D_i \in D}$ forms a paradoxical decomposition for $G$, a contradiction.
\end{proof}
\end{document}
