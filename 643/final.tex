\documentclass{article}
\usepackage{../m}

\begin{document}
\noindent Paul Gustafson\\
\noindent Math 663 

%Problems: 

\subsection*{Problems}
\p{1} Give a proof of the mean ergodic theorem using the spectral theorem for unitary operators.
\begin{proof}
Let $U$ be a unitary operator on a Hilbert space $H$. By the spectral theorem, there exists a unitary map $T: H \to L^2(X, \mu)$ for some finite measure space $(X, \mu)$ with $U = T^{-1}ST$ where $S$ is multiplication by a function $f$ taking values on the unit circle.

Note that we have $\lim_{N \to \infty} \frac 1 N \sum_{n=0}^N f^n & = \chi_{f^{-1}(1)}$. Thus, $P := \lim_{N \to \infty} \frac 1 N \sum_{n=0}^N U^n$ exists and is a orthogonal projection. 

I claim that $P$ is the orthogonal projection onto $\ker(I - U)$. If $v \in \ker(I - U)$, then $T^{-1}STv = Uv = v$. Hence $S(Tv) = Tv$. Therefore $f(x) = 1$ for all $x$ where $Tv(x) \neq 0$. This implies that $Pv = T^{-1} \chi_{f^{-1}(1)} Tv = v$. All these steps are reversible, so the range of $P$ is precisely $\ker(I-U)$.
\end{proof}

\p{2} Prove Khintchine's recurrence theorem: If $G$ is a countable amenable group and $G$ acts on $(X,\mu)$ via a p.m.p. action then for every measurable
set $A \subset X$ and $\epsilon > 0$ the set
$\{ s \in G: \mu(sA \cap A) \ge \mu(A)^2 - \epsilon\}$
is syndetic.
\begin{proof}
Let $\{F_n\}_{n=1}^\infty$ be a tempered Folner sequence in $G$. From the pointwise ergodic theorem, we have
\begin{align*}
 \lim_{n \to \infty} \frac 1 {|F_n|} \sum_{s \in F_n} \mu(sA \cap A) & =  
 \lim_{n \to \infty} \frac 1 {|F_n|} \sum_{s \in F_n} \langle  \chi_{sA}, \chi_{A} \rangle_L^2(X) \\
& = \langle P \chi_{A} , \chi_{A} \rangle,
\end{align*}
where  $P$ is the orthogonal projection onto the subspace of $G$-invariant functions in $L^2(X)$. 

On the other hand, we have
\begin{align*}
\mu(A)^2 & = \langle \chi_A, 1 \rangle ^2 \\
& = \langle P\chi_A, 1 \rangle ^2 \\
& \le \|P\chi_A\|_2^2 \\
& = \langle P \chi_{A} , \chi_{A} \rangle.
\end{align*}

Hence,  $\mu(A)^2 \le  \lim_{n \to \infty} \frac 1 {|F_n|} \sum_{s \in F_n} \mu(sA \cap A)$.

Pick $n$ so large that $$

 It follows that $\mu(sA \cap A) \ge \mu(A)^2 - \epsilon$ for some $s \in F_n$.


\end{proof}

\p{5} Give examples of unitary representations $\pi$ and $\rho$ of $\ZZ$ such that $\pi \otimes \rho$ is ergodic but neither $\pi$ nor $\rho$ is weakly mixing.
\begin{proof}
Let $\pi = \rho$ be the one-dimensional representation taking $1$ to multiplication by $i$.  Since this representation is finite-dimensional, it is not weakly mixing. Moreover, $(\pi \otimes \rho)(1)$ acts by multiplication by $-1$, so $\pi \otimes \rho$ is not ergodic.
\end{proof}

\p{8} Show that a subgroup $H$ of an amenable countable discrete group $G$ is amenable.
\begin{proof}
Suppose $H$ were not amenable.  Then it admits a paradoxical decomposition $C \sim D \sim H$. Let $R$ be a complete set of representatives for the right cosets of $H$ in $G$. Then $\{C_iR\}_{C_i \in C}$, $\{D_iR\}_{D_i \in D}$ forms a paradoxical decomposition for $G$, a contradiction.
\end{proof}
\end{document}
