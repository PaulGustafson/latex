\documentclass{article}
\usepackage{../m}

\begin{document}
\noindent Paul Gustafson\\
\noindent Math 663 

%Problems: 

\subsection*{Problems}
\p{1} Give a proof of the mean ergodic theorem using the spectral theorem for unitary operators.
\begin{proof}
Let $U$ be a unitary operator on a Hilbert space $H$. By the spectral theorem, there exists a unitary map $T: H \to L^2(X, \mu)$ for some finite measure space $(X, \mu)$ with $U = T^{-1}ST$ where $S$ is multiplication by a function $f$ taking values on the unit circle.

Note that we have $\lim_{N \to \infty} \frac 1 N \sum_{n=0}^N f^n & = 1_{f = 1}$. Thus, $P := \lim_{N \to \infty} \frac 1 N \sum_{n=0}^N U^n$ exists and is a orthogonal projection. 

I claim that $P$ is the orthogonal projection onto $\ker(I - U)$. If $v \in \ker(I - U)$, then $T^{-1}STv = Uv = v$. Hence $S(Tv) = Tv$. Therefore $f(x) = 1$ for all $x$ where $Tv(x) \neq 0$. This implies that $Pv = T^{-1} 1_{f=1} Tv = v$. All these steps are reversible, so the range of $P$ is precisely $\ker(I-U)$.
\end{proof}

\p{8} Show that a subgroup $H$ of an amenable countable discrete group $G$ is amenable.
\begin{proof}
Suppose $H$ were not amenable.  Then it admits a paradoxical decomposition $C \sim D \sim H$. By appending $G \setminus H$ to both $C$ and $D$ we get a paradoxical decomposition of $G$, a contradiction.
\end{proof}
\end{document}
