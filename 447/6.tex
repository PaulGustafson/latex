\documentclass{article}
\usepackage{../m}

\begin{document}
\noindent Paul Gustafson\\
\noindent Texas A\&M University - Math 447\\ 
\noindent Instructor: Dr. Johnson

\subsection*{HW 7}
\p{19.43} No to both; let $f(x) = x \chi_{[0,1)}$. Then $||f||_\infty = 1$, but $\{|f| = 1\} = \emptyset$.

\p{50} Note that if $f \in L_\infty$ then $f$ is a.e. equal to a bounded function, so the simple functions of the Basic Construction converge uniformly to $f$ a.e. Hence, the simple functions are dense in $L_\infty$.  If $E$ is of finite measure, all simple functions defined on $E$ are integrable.  If $m(E) = \infty$, then the integrable simple functions are not dense: take $f = \chi_E$. If $||\phi - f||_\infty < 1/2$, then $||\phi||_\infty > 1/2$, so $\int |\phi| = \infty$.

\p{51} There's a typo in the problem statement: the exponent of $m(E)$ should be $1/p$, not $(1-1/p)$.  To see why the latter can't be right, let $f = \chi_E$, then $||f||_p = m(E)^{1/p} = m(E)^{1/p} ||f||_\infty$. Take $m(E) = 2$ and $p = 1$ to see that the problem can't be correct as stated.

If $f \in L_\infty(E)$ with $m(E) < \infty$ and $1 \le p < \infty$, we have
$||f||_p \le ||( ||f||_\infty) ||_p = (m(E))^{1/p} ||f||_\infty$. This implies $L_\infty(E) \subset L_p(\R)$ with the convention that a function $f$ defined on $E$ is set to $0$ outside of $E$.  

If $E = [0,1]$, this inequality reduces to $||f||_p \le ||f||_\infty$. To see $||f||_1 \le ||f||_p$, use H\"{o}lder's inequality: $||f||_1 = ||(1)f||_1 \le ||1||_q ||f||_p = ||f||_p$

\p{52} Let $f \in L_\infty[0,1]$. To see that $||f||_p$ is increasing as a function of $p$,  let $1\le r < s \le \infty$. By H\"{o}lder,
$$||f||_r = (\int |f|^r (1))^{1/r} \le ((\int |f|^{r(s/r)})^{r/s})^{1/r} = ||f||_s.$$
Since by (51) every $||f||_p$ is bounded above by $||f||_\infty$, we have $\lim_{p\to\infty} ||f||_p$ exists.

To show that $||f||_\infty \le \lim_{p\to\infty} ||f||_p$, let $\epsilon > 0$. We have
\begin{align*}
||f||_p & \ge \left(\int_{\{|f| > ||f||_\infty - \epsilon\}} |f|^p \right)^{1/p}
\\ & \ge \left((||f||_\infty - \epsilon)^p \,m\{|f| > ||f||_\infty - \epsilon\}\right)^{1/p}
\\ & = (||f||_\infty - \epsilon) (m\{|f| > ||f||_\infty - \epsilon\})^{1/p}
\\ & \to ||f||_\infty - \epsilon,
\end{align*}
as $p \to \infty$, since $m \{ |f| > ||f||_\infty - \epsilon\} > 0$. Thus,
$||f||_\infty \le \lim_{p\to\infty} ||f||_p \le ||f||_\infty$, so $\lim_{p\to\infty} = ||f||_\infty$.

\p{62}
Pick a step function $h$ such that $||f - h||_p < \epsilon/2$. If $m(A) < \delta := (\frac \epsilon {2 ||h||_\infty})^p$, then
\begin{align*}
||f \chi_A||_p & \le ||h \chi_A||_p + ||(f-h) \chi_A||_p
\\ & \le || ||h||_\infty \chi_A||_p + (\epsilon/2)
\\ & \le ||h||_\infty m(A)^{1/p} + (\epsilon/2)
\\ & < \epsilon.
\end{align*}

If $p = \infty$, this will not work.  Take $f(x) := 1$. If $m(A) >0$, then $||f\chi_A||_\infty = ||\chi_A||_\infty = 1$.

\p{64(a)} \emph{Case $p>1$.} For the boundedness, since $1<p<\infty$, we can use H\"{o}lder:
$$|h(x)| = |\int f T_x(g)| \le ||f||_p||T_x(g)||_q = ||f||_p||g||_q,$$
where the last equality follows from (63), which was proved in class.

For continuity,
$$|h(x) - h(y)| = |\int f \cdot (T_x - T_y)g)| \le ||f||_p||(T_x - T_y)g||_q \to 0$$
as $y\to x$ by (63)(c).

\emph{Case $p=1$}. For the boundedness, note 
$$|h(x)| \le \int |f T_x(g)| \le \int |f| \cdot ||g||_\infty = ||f||_1||g||_\infty.$$

For continuity, first note the previous estimate shows that $f T_x(g) \in L_1$, so by (63) we have
\begin{align*}
h(x) & = \int f T_x (g) 
\\ & = \int (f T_x (g))^+ - \int (f T_x (g))^-
\\ & = \int T_{-x} ((f T_x (g))^+) - \int T_{-x} ((f T_x (g))^-)
\\ & = \int (T_{-x}(f) g)^+ - \int (T_{-x} (f) g)^-
\\ & = \int T_{-x}(f) g,
\end{align*}
where the penultimate equality is justified by the fact that for any function $F$, we have $T_x(F^+) = T_x(1/2(|F| + F)) = 1/2(|T_xF| + T_xF) = (T_xF)^+$ and similarly for $F^-$.

Thus, $|h(x) - h(y)| = |\int (T_{-x}-T_{-y})(f) g| \le ||g||_\infty \int |(T_{-x} - T_{-y})(f)|  \to 0$
as $y\to x$ by (63)(c).

\p{64(b)}  
Suppose $g$ is continuously differentiable.

\emph{Case $f = \chi_{[a,b]}$.}  We have $h(x) = \int_a^b g(x+t) \,dt$, so
 $(h(x + k) - h(x))/k = \int_a^b (g(x +k + t) - g(x + t))/k \,dt
  \to \int_a^b g'(x + t) \,dt  \,(k \to 0)$
 by the DCT, since $|(g(x +k + t) - g(x + t))/k| \le sup_{t \in [a-1, b+1]} |g'(x+t)|$ for $t \in [a,b]$ and $k < 1/2$ by the MVT.

The step function case follows by linearity of the integral.

Pick a step functions $s_n$ such that $||f-s_n||_p \to 0$ and $s_n \to f$ a.e.
Then 
\begin{align*}
(h(x + k) - h(x))/k & = \int f(t)(g(x +k + t) - g(x + t))/k dt
\\ &
\end{align*}


\end{document}
