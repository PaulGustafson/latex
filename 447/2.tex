\documentclass{article}
\usepackage{../m}

\begin{document}
\noindent Paul Gustafson\\
\noindent Texas A\&M University - Math 447\\ 
\noindent Instructor: Dr. Johnson

\subsection*{HW 2, due February 7}
\p{16.58} Suppose that $m^*(E) < \infty$. Prove that $E$ is measurable if and only if, for every $\epsilon > 0$, there is a finite union of bounded intervals $A$ such that $m^*(E\triangle A) < \epsilon$ (where $E\triangle A$ is the symmetric difference of $E$ and $A$).
\begin{proof}
\begin{lemma}\label{lem1}
If $S,T,U$ are sets, then $S\triangle U \subset (S\triangle T) \cup (T\triangle U)$.
\end{lemma}
\begin{proof}
\begin{align*} 
S \triangle U & = (S\setminus U) \cup (U \setminus S) \\
& \subset  (((S \setminus T) \cup T) \setminus U) \cup (((U \setminus T) \cup T) \setminus S) \\
&\subset  (S \setminus T) \cup (T \setminus U) \cup (U \setminus T) \cup (T \setminus S) \\
& =  (S\triangle T) \cup (T\triangle U)
\end{align*}
\end{proof}

Suppose $E$ is measurable. Let $\epsilon > 0$. Pick an open set $U \supset E$ with $m(U \setminus E) < \epsilon/2$. Since $m(U) < \infty$, $U = \bigcup_{n=1}^\infty I_n$ where the $I_n$ are disjoint bounded open intervals.  Pick $N$ such that $\sum_{n=N+1}^\infty I_n < \epsilon/2$. Let $A := \bigcup_{n=1}^N I_n$. Then, by the lemma, $m(A \triangle E) \le m(A \triangle U) + m(U \triangle E)  = m(U \setminus A) + m(U\setminus E) < \epsilon$.

Conversely, let $\epsilon > 0$ and suppose such an $A$ exists. Let $U \supset E\triangle A$ be an open set such that $m(U) < \epsilon$. There exists an open set $J\supset A$ such that $m(J\setminus A) < \epsilon$. Then $G := U \cup J$ is open, and $G \supset (E\triangle A) \cup A \supset E$. Moreover, 
\begin{align*}
m^*(G \setminus E) & \leq  m(U) + m^*(J \setminus E) 
\\ & \le \epsilon + m^*(J \triangle E)
\\ & \le \epsilon + m(J \triangle A) + m^*(A \triangle E)
\\ & = \epsilon + m(J \setminus A) + m^*(A \triangle E)
\\ & < 3\epsilon
\end{align*}
\end{proof}
\p{16.60} If $E$ is a measurable set, show that $E+x$ and $rE$ are measurable for any $x, r \in \R$. [Hint: Use Theorem 16.21].
\begin{proof}
If $r = 0$, $rE = \{0\}$ is measurable, so we may assume $r \ne 0$.
Let $\epsilon > 0$, and let $U \supset E$ be open with $m(U \setminus E) < \epsilon$. 

I claim that $I$ is in open interval iff $rI +x$ is an open interval. Let $f(y) := ry + x$. Since $r \ne 0$, $f$ is a homeomorphism. It also preserves betweenness since $a < b < c$ implies $f(a) < f(b) < f(c)$ if $r > 0$, and $f(a) > f(b) > f(c)$ if $r > 0$. If $f(I)$ is an interval, let $a<b<c$ with $a,c$ in $I$. Then $f(b)$ is between $f(a)$ and $f(c)$, so $f(b) \in f(I)$ which implies $b \in I$. The converse follows from the fact that $f^{-1}$ has the same form as $f$.

 Hence, since $U$ is a countable union of open intervals, $rU +x$ is open. Moreover, $(I_n)$ is a cover of $U \setminus E$ by open intervals iff $(rI_n + x)$ covers $(rU + x) \setminus (rE +x)$. Therefore, $m^*((rU+x) \setminus (rE+x)) = r m(U \setminus E)$.
\end{proof}
\p{16.70} Prove that an arbitrary union of positive-length intervals is measurable. [Hint: Let $\mathcal{C}$ be the collection of all closed intervals $J$ such that $J \subset I_\alpha$ for some $\alpha$.]
\begin{proof} Let $(I_\alpha)$ be an arbitrary collection of positive length intervals, and $U = \bigcup_\alpha I_\alpha$. Let $\mathcal{C}$ be the collection of all closed intervals $J$ such that $J \subset I_\alpha$ for some $\alpha$.

Let $(q_n)$ be a countable dense-in-$\R$ set. Let $E_n = \bigcup \{J \in \mathcal{C} : q_n \in J\}$. I claim $E_n$ is an interval. Suppose $a<b<c$ with $a,c \in E_n$. Then there exists $J_a \in \mathcal C$ with $q_n, a \in J_a$ and $J_c$ with $q_n, c \in J_c$. Hence, if $q_n \le b$, then $b \in J_c$; and if $q_n > b$, then $b \in J_a$. Hence, $E_n$ is an interval.

To see that $E := \bigcup_n E_n \supset U$, suppose $x \in U$. Then $x \in I_\alpha$ for some $\alpha$. There exists a closed interval $I$ such that $x \in I \subset I_\alpha$. There exists some $q_n$ such that $q_n \in I$. Hence, $x \in I \subset E_n$. 

The reverse inclusion, $E \subset U$, follows from the fact that each $E_n \subset U$. Hence, $U = E$ is the countable union of intervals.

\end{proof}
\p{16.78} If $E$ is a measurable subset of $A$, show that $m^*(A) = m(E) + m^*(A \setminus E)$. Thus $m^*(A \setminus E) = m^*(A) - m(E)$ provided that $m(E) < \infty$.
\begin{proof}
Let $\epsilon > 0$. There exists an open set $U \supset A$ such that $m(U) \le m^*(A) + \epsilon$. Thus, $m^*(A) \ge m(U) + \epsilon = m(E) + m(U\setminus E) + \epsilon \ge m(E) + m^*(A\setminus E) + \epsilon$.
\end{proof}
\p{J16.4} Prove that every set of positive outer measure contains a nonmeasurable subset.
\begin{proof}
Following Carothers' hint, let $A \subset \R$ with $m^*(A) > 0$. Then since $m^*(A) \ge \sum_n A \cap [n, n+1)$, some $A \cap [n, n+1)$ has positive outer measure. Let $N_r \subset [0,1)$ be defined as in Carothers' construction of an unmeasurable set. Since $\bigcup_r N_r + n = [n, n+1)$, some $A \cap (N_s +n) =: E$ must have positive outer measure. 

Suppose $E$ were measurable. Then $E - n \subset N_s$ has positive measure. Let $F_r := E - n + r \text{(mod 1)}$. Then $m(F_r) = m(E)$, and $F_r \subset N_r$. Hence, the $F_r$ are disjoint, so $m([0,1))  \geq m(\bigcup_r F_r) = \sum_r m(F_r) = \sum_r m(E) = \infty$, a contradiction.
\end{proof}
\p{J16.5} Prove that $m$ is Lipschitz with constant 1 on $(\mathcal{M}_1, d)$, where $\mathcal{M}_1$ denotes the measurable subsets of $[0,1]$, and $d(E,F) = m(E\triangle F)$. Prove that $(\mathcal M_1, d)$ is complete. [Hint: If $(E_n)$ is $d$-Cauchy, then, by passing to a subsequence, you may assume that $d(E_n, E_{n+1}) < 2^{-n}$. Now argue that $(E_n)$ converges to, say, $\limsup_{n \rightarrow \infty}E_n$.]
\begin{proof}
Let $E,F \in \mathcal M_1$. WLOG, $m(E) \le m(F)$. Then $|m(F) - m(E)| = m(F) - m(E) \le m(F\setminus E) \le m(E \triangle F) = d(E,F)$.

For the completeness, suppose $(E_n)$ is $d$-Cauchy. By passing to a subsequence, assume $d(E_n, E_{n+1}) < 2^{-n}$. Let $E = \limsup_{n\rightarrow \infty} E_n$.  For any $k$, we have 
\begin{align*}
d(E_k, E) & = m(E_k \triangle \bigcap_{m=1}^\infty\bigcup_{n=m}^\infty E_n)
\\ & = m((\bigcup_{m=1}^\infty\bigcap_{n \ge m} E_n \setminus E_k) \cup ((\bigcup_{m=1}^\infty\bigcap_{n \ge m} E_n )^c \setminus E_k^c) )
\\ & \le m(\bigcup_{m=1}^\infty\bigcap_{n \ge m} E_n \setminus E_k) + m(\bigcap_{m=1}^\infty\bigcup_{n \ge m} E_n^c  \setminus E_k^c )
\\ & \le m(\bigcup_{n > k} E_n \setminus E_k) + m((\bigcup_{n > k} E_n^c) \setminus E_k^c )
\\ & \le m(\bigcup_{n > k} (E_n \setminus E_{n-1}) \setminus E_k) + m((\bigcup_{n > k} E_n^c \setminus E_{n-1}^c) \setminus E_k^c )
\\ & \le m(\bigcup_{n > k} E_n \setminus E_{n-1}) + m(\bigcup_{n > k} E_n^c \setminus E_{n-1}^c)
\\ & \le m(\bigcup_{n > k} E_n \triangle E_{n-1}) + m(\bigcup_{n > k} E_n \triangle E_{n-1})
\\ & \le 2^{-k} + 2^{-k}
\end{align*}
so $E_k \rightarrow E$.

\end{proof}
\p{J16.6} Let $X$ be a metric space, $E\subset X$, and $\mathcal B := \{B_{r(x)}(x) : x \in S \}$ be a cover of $E$ such that $\sup_{x\in S} r(x) < \infty$. Prove that there is a (finite or infinite) sequence $\{B_{r(x_i)}(x_i)\}_{i=1}^N$ of disjoint balls in $\mathcal B$ so that either
\begin{enumerate}
\item $N = \infty$ and $\inf_i r(x_i) > 0$, or
\item $E \subset \bigcup_{n=1}^N B_{5r(x_i)}(x_i)$. ($N$ can be either finite or infinite in this case.)
\end{enumerate}
Hint: Greed is good.
\begin{proof}
Let $S_1 := S$. Recursively, for $i = 1, 2, \ldots$,
\begin{enumerate}
\item Choose $x_i \in S_i$ such that $r(x_i) > 1/2 \sup_{x\in S_i} r(x)$.
\item Let $S_{i+1} =  \{x\in S : B_{r(x)}(x) \cap \bigcup_{j=1}^i B_{r(x_j)} (x_j) = \emptyset\}$.
\item If $S_{i+1} = \emptyset$, stop.
\end{enumerate}
By (2), $\{B_{5r(x_i)}(x_i)\}_{n=1}^N$ are disjoint.

Let $y \in E$. There exists $x \in S$ such that $y\in B_{r(x)}(x)$.

\emph{Case $N < \infty$}:  Since $S_{N+1} = \emptyset$, we have $B_{r(x)}(x) \cap \bigcup_{j=1}^N B_{r(x_j)} \ne  \emptyset$. Thus, for some minimal $j$, $B_{r(x_j)}(x_J) \cap B_{r(x)}(x) \ne \emptyset$. By construction step (1), since $j$ is minimal, $r(x) < 2r(x_j)$. Thus, by the triangle inequality, $B_{r(x)}(x) \subset B_{5r(x_j)}(x_j)$.

\emph{Case $\inf_i r(x_i) = 0$}: Suppose $B_{r(x)}(x) \cap \bigcup_{j=1}^N B_{r(x_j)} =  \emptyset$. Since $\inf_i r(x_i) = 0$, there exists $j$ such that $r(x_j) < 1/2 r(x_i)$.  But this contradicts the choice of $x_j$ in construction step (1).

Hence, as in the previous case, $B_{r(x)}(x) \subset B_{5r(x_j)}(x_j)$ for some $j$.

\end{proof}
\end{document}
