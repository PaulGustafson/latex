\documentclass{article}
\usepackage{../m}

\newcommand{\A}{\mathcal{A}}
\newcommand{\dm}{\,dm}

\begin{document}
\noindent Paul Gustafson\\
\noindent Texas A\&M University - Math 447\\ 
\noindent Instructor: Dr. Johnson

\subsection*{HW 4, due 2/21}
\p{18.3} Prove that $\int_1^\infty (1/x)\, dx = \infty$.
\begin{proof}
Let $f(x)$ = $(\chi_{(1,\infty)}(x)) (1/x)$. Define $\phi_m = \sum_{n=1}^m \frac 1 {n+1} \chi_{(n, n+1)}$. Then for all $m \ge 1$, we have $\phi_m  \le f$. Hence, by the monotonicity of the integral, $\int f\, dm \ge \int \phi_m\, dm = \sum_{n=1}^m \frac 1 {n+1} \to \infty$ as $m \to \infty$. 
\end{proof}
\p{4} Find $(f_n)$ nonnegative measurable functions that converge uniformly to $0$, but $\lim_{n\to\infty} \int f_n = 1$.
\begin{proof}
Let $f_n = (1/n) \chi_{(0,n)}$.
\end{proof}
\p{6} 
Suppose $(f_n)$ nonnegative, measurable decrease pointwise to $f$, and that $\int f_k < \infty$ for some $k$. Prove that $\int f = \lim_{n\to\infty} f_n$.  Also, give an example showing that the condition $\int f_k < \infty$ is necessary.
\begin{proof}
For the counterexample, let $f_n = \chi_{(n,\infty)}$ for $n \ge 1$.

%% Would be better to throw out all n<k
For the other part of the problem, for all $n \ge k$, let $g_n = f_k - f_n$.  Since $(f_n)$ is nonnegative and decreasing, $(g_n)_{n\ge k}$ is increasing and nonnegative. Since $g_n \le f_k$, we have $\int g_n < \infty$ for all $n \ge k$. Hence, using the linearity of the integral on integrable functions and the MCT,
\begin{align*}
\int f\,dm  & =  -\int(f - f_k)\dm  +  \int f_k\, dm 
\\ & = -\int\lim_{n\to\infty} f_k - f_n \,dm + \int f_k\,dm
\\ & = -\lim_{n\to\infty} \left(\int  f_k - f_n \,dm \right) + \int f_k\,dm 
\\ & = -\lim_{n\to\infty} \int -f_n \,dm 
\\ & = \lim_{n\to\infty} \int f_n \,dm
\end{align*}
\end{proof}
\p{7} Let $\mu : \A \to [0,\infty]$ be a nonnegative, finitely additive, set function defined on a $\sigma$-algebra $\A$. Prove that:
\begin{enumerate}
\item $\mu(E) \le \mu(F)$ whenever $E,F \in\A$ satisfy $E \subset F$.
\item if $\mu(\emptyset) \ne 0$, then $\mu(E) = \infty$ for all $E \in\A$.
\end{enumerate}
\begin{proof}
For (1), we have $\mu(F) = \mu(E) + \mu(E\setminus F) \ge \mu(E)$.
For (2), if $\mu(\emptyset) \ne 0$, we have $\mu(E) = \mu(E \cup \bigcup_{i=1}^n \emptyset) = \mu(E) + n\mu(\emptyset) \to \infty$ as $n \to \infty$.
\end{proof}
\p{8} Define $\mu$ and $\A$ as in (7). Prove that TFAE:
\begin{enumerate}
\item $\mu(\bigcup_{n=1}^\infty E_n) = \sum_{n=1}^\infty \mu(E_n)$ for every pairwise disjoint $(E_n) \subset \A$.
\item $\mu(\bigcup_{n=1}^\infty E_n) = \lim_{n\to\infty} \mu(E_n)$ for every increasing $(E_n) \subset \A$.
\end{enumerate}
\begin{proof}
\begin{comment}
%By  (7)(2), if $\mu(\emptyset) \ne 0$, everything has infinite measure, so all the conclusions follow trivially. Hence, we may assume $\mu(\emptyset) = 0$.

For (2) implies (1), in the case that $\sum_{n=1}^\infty \mu(E_n) = \infty$,  let $M > 0$.  Pick $N$ such that $\sum_{n=1}^N \mu(E_n) > M$. Then $\mu(\bigcup_{n=1}^\infty E_n) \ge \sum_{n=1}^M \mu(E_n) > M$.  Letting $M \to \infty$, we get the desired equality.

Now consider the case that $\sum_{n=1}^\infty \mu(E_n) < \infty$. We have $\mu(\bigcup_{n=1}^\infty E_n) \ge \mu(\bigcup_{n=1}^N E_n) =  \sum_{n=1}^N \mu(E_n) $. Letting $N \to\infty$, we get $\mu(\bigcup_{n=1}^\infty E_n) \ge \sum_{n=1}^\infty \mu(E_n)$.
\end{comment}
To prove (2) implies (1), let $F_k = \bigcup_{n=1}^k E_n$. Then $(F_k)$ is an increasing sequences of sets in $\A$, so, by (2), $\mu(\bigcup_{n=1}^\infty E_n) = \mu(\bigcup_{k=1}^\infty F_k)  = \lim_{k\to\infty} \mu(F_k) = \lim_{k\to\infty} \sum_{n=1}^k \mu(E_n) = \sum_{n=1}^\infty \mu(E_n)$.

For (1) implies (2), let $(F_n)$ be the disjointification of $(E_n)$.  That is, $F_n := E_n \setminus (\bigcup_{k<n} E_k)$, so for all $N$, we have $\bigcup_{n=1}^N F_n = \bigcup_{n=1}^N E_n$. Then, applying (1) to $F_n$, we have $\mu(\bigcup_{n=1}^\infty E_n) 
= \mu(\bigcup_{n=1}^\infty F_n) = \sum_{n=1}^\infty \mu(F_n) = \lim_{N\to\infty} \sum_{n=1}^N \mu(F_n) = \lim_{N\to\infty} \mu(E_N)$.
\end{proof}

\p{15} Let $f$ be nonnegative and measurable. Prove that $\int f < \infty$ if and only if $\sum_{i=-\infty}^\infty 2^k m\{ f > 2^k \} < \infty$.
\begin{proof}
Suppose $\int f < \infty$. Then $\sum_{i=-N}^N 2^k m\{ f > 2^k \} = \int \sum_{i=-N}^N 2^k \chi_{\{ f > 2^k \}} \le \int f$.  Letting $N \to \infty$, we see that $\sum_{i=-\infty}^\infty 2^k m\{ f > 2^k \} \le \int f < \infty$.

Conversely, suppose $\sum_{i=-\infty}^\infty 2^k m\{ f > 2^k \} < \infty$. Let $\phi \le f$ be an integrable simple function with standard representation 
$\phi(x) = \sum_{i=0}^n a_i \chi_{A_i}$.  %%TODO
\end{proof}


\p{16} Let $f \ge 0$ be integrable. Given $\epsilon >0$, show that there is a measurable set $E$ with $m(E) < \infty$ such that $\int_E f > \int f - \epsilon$. Moreover, show that $E$ can be chosen so that $f$ is bounded on $E$.
\begin{proof}
\begin{comment}
Pick an integrable, nonnegative, simple function $\phi \le f$ such that $\int f - \int \phi \le \epsilon/2$. 
 Write $\phi$ in standard form as $\phi = \sum_{i=0}^n a_i \chi_{A_i}$ where $a_0 = 0$.  Note that since $\phi$ is integrable, we have $m(A_0^c) = \sum_{i=1}^n m(A_i) \le (\min_{i\ge 1} a_i)^{-1} \sum_{i=1}^n a_i m(A_i) = (\min_{i\ge 1} a_i)^{-1} \int \phi \le \infty$. Hence, if $F := A_0^c$, we have $\int_{F} f \ge \int_{F} \phi  = \int_\R \phi \ge \int f - \epsilon/2$.

%To get the bounded part, let $E_k := \{f > k\} \cap A_0^c$. Since $f$ is integrable, it is finite a.e., so $m(\bigcap_k E_k) = 0$.  Hence, since $m(E_1) \le m(A_0^c) < \infty$, we have $lim_{k\to\infty} E_k = 0$.  Pick $K$ such that $m(E_k) < \epsilon/2$.  Then, if $E = A_0^c $

To get the bounded part, pick simple functions $(\phi_n)$ defined on $F$ such that $\phi_n \to f$ on $F$.  By Egorov, there exists a set $G$ with $m(G) < \epsilon$
\end{\comment}
\end{proof}



\p{17} If $f$ is nonnegative and integrable, prove that the function $F(x) = \int_{-\infty}^x f$ is continuous. In fact, even more is true: Given $\epsilon > 0$, show that there is a $\delta > 0$ such that $\int_E f < \epsilon$ whenever $m(E) < \delta$. [Hint: This is easy when $f$ is bounded; see (16)]
\begin{proof}
\end{proof}
\p{14} Define $f: [0,1] \to [0, \infty)$ by $f(x) = 0$ if $x$ is rational and $f(x) = 2^n$ if $x$ is irrational with exactly $n = 0,1,2,\ldots$ leading zeros in its decimal expansion. Show that $f$ is measurable, and find $\int_0^1 f$.
\begin{proof}
\end{proof}
\p{J18.1} Suppose that $f$ is a nonnegative integrable function and $A$ is a measurable set. Define $F$ on $\R$ by $F(t) = m_f(A+t)$. Show that $F$ is a continuous function. Recall that $m_f(E) := \int \chi_E f \dm$. (Hint: First treat the case where $A$ is a bounded interval.)
\begin{proof}
\end{proof}
\end{document}
