\documentclass{article}
\usepackage{../m}

\begin{document}
\noindent Paul Gustafson\\
\noindent Texas A\&M University - Math 655\\ 
\noindent Instructor: Bill Johnson

\subsection*{HW 1}
\p{J1.1.1} Show that no Banach space has a countable Hamel basis.
\begin{proof}
Let $X$ be a Banach space.  Suppose $(x_i)_{i=1}^\infty \subset X$ is a countable Hamel basis.  Let $X_n = \spn((x_i)_{i=1}^n$ for each $n$.  I claim each $X_n$ is nowhere dense. Suppose $X_n$ contained some ball $B_\epsilon(x)$. Then it contains $B_\epsilon(0)$, which contains a multiple of $x_{n+1}$, a contradiction.  Hence, each $X_n$ is nowhere dense.  But $U = \bigcup_n X_n$, which contradicts the Baire Category Theorem.
\end{proof}

% \p{J1.1.2} Show that any two algebraic bases $X$ and $Y$ for a vector space $V$ have the same cardinality.
% \begin{proof}
% Acknowledgment: I looked in Hungerford's Algebra for a hint on this problem. WLOG suppose $|X| \le |Y|$.  If $|X|$ is finite, use standard f.d. linear algebra. If $|X|$ is infinite, use trickery.
% \end{proof}

\p{J1.2} Show that $l_\infty$ is not separable.
\begin{proof}
Let $U \subset l_\infty$ be the set of all sequences consisting of only $0$ and $1$.  Note that $|U| = |P(\N)| > |\N| $.  Let $x,y \in U$ be distinct. Then $\|x - y \| = 1$, so $B_{1/2}(x) \cap B_{1/2}(y) = \emptyset$.  Thus, $(B_{1/2}(x))_{x\in U}$ is an uncountable set of disjoint balls, so $l_\infty$ cannot be separable.
\end{proof}


\p{J1.3} Show that every Banach space with a basis is separable.
\begin{proof}
Let $(x_n)$ be a basis for the Banach space $X$. WLOG $\|x_n\| = 1$ for all $n$. Let $Q$ be a countable dense set in the scalar field.  Let $U = \bigcup_n \bigoplus_{k=1}^n Q x_k$. Then $U$ is countable since it is the countable union of countable sets.  To see that $U$ is dense, let $x \in B$ and $\epsilon > 0$. Then $x = \sum_n a_n x_n$ for some scalars $a_n$. Pick $N$ such that $\|x - \sum_{n=1}^N a_n x_n\| < \epsilon/2$.  Pick $q_n$ within $\frac \epsilon {2N}$  of $a_n$.  Then $ |x - \sum_{n=1}^N q_n x_n| \le |x - \sum_{n=1}^N a_n x_n| + |\sum_{n=1}^N (a_n - q_n)x_n| < \epsilon$.
\end{proof}

\p{J1.4} Find a basis $(x_i)$ for a normed space $X$ for which some $x_i$ is not continuous. (Hint: consider the algebraic span of $(e_i) \subset l_2$).

%Sun 7:40 pm

Need to find $(x_i)$ and $(q_i)$ with $\|q_i\| \le 1$ such that $x_1^#(q_i) \to \infty$. Will x_i = e_i work?  No, e_i* are continuous on l_2 hence on X. Need something that would not be a basis on l_2. Because if it were a basis on l_2, the duals would be continuous on l2 hence on X.  

Basis of X but not l_2.       (\sum^k e_i)_k ?  I'm not very familiar with duals/ operator norm.


\p{J1.5.1} Construct an M-basis $(e_n)$ for $l_2$ that is not a Schauder basis.

\p{J1.5.2} Same except also have $\sup_n \|e_n^*\| \|e_n\| < \infty$

\p{J1.5.4} Show that if $(e_n, e_n^*)$ is an M-basis for $l_2$ and $\forall n, \|e_n^*\| = 1 = \| e_n \|$, then $(e_n)$ is orthonormal.

\p{J1.6} Show that every separable Banach space has an M-basis.

a) $X$ separable implies that there exists a total sequence $(x_n^*) \subset X$.

b) Show that if $(x_n)$ is fundamental in $x$ and $(x_n^*)$ is total over $X$, then there is an M-basis $(e_n, e_n^*)$ for $X$ such that 
$\spn(x_1, \ldots, x_n) = \spn(e_1, \ldots, e_n) $ and
$\spn(x_1^*, \ldots, x_n^*) = \spn(e_1^*, \ldots, e_n^*)$ for all $n$.

\end{document}
