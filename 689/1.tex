\documentclass{article}
\usepackage{../m}

\begin{document}
\noindent Paul Gustafson\\
\noindent Math 689 Commutative and Homological Algebra

\subsection*{HW 1}
All rings are commutative with $1 \neq 0$.
\p 1 Let $A$ be a ring and let $\mathfrak S$ be the set of all multiplicative subsets of $A$ that do not contain $0$, ordered by inclusion.  Prove that an element $S$ of $\mathfrak S$ is maximal in $\mathfrak S$ if and only if $R - S$ is a minimal prime ideal.
\begin{proof}
Suppose $S$ is a maximal element of $\mathfrak S$. By a proposition from class, there exists a prime ideal $I$ disjoint from $S$.  Suppose $J$ is a prime ideal contained in $I$.  Then $R - J$ is a multiplicative subset of $A$ not containing $0$, i.e, an element of $\mathfrak S$.  But we also have $S \subset R - I \subset R -J$. Thus, by the maximality of $S$, we have $S = R - I = R - J$.  Hence $I = J$. Since this for all prime ideals $J \subset I$, $I$ is a minimal prime ideal.

For the opposite implication, suppose $S$ is not a maximal element of $\mathfrak S$. Pick $T \in \mathfrak S$ with $S$ properly contained in $T$.  Pick a prime ideal $I$ disjoint from $T$. Then
$I \subset R - T \subset R - S$, where the last inclusion is proper.  Therefore, $R - S$ is not a minimal prime ideal.
\end{proof}

\p 2 Let $I$ and $J$ be ideals of a ring $A$. Show that 
$$\sqrt{IJ} = \sqrt{ I \cap J }   = \sqrt I \cap \sqrt J .$$

\begin{proof}
First we have $IJ \subset I \cap J$, so $\sqrt{IJ} \subset \sqrt{ I \cap J}$.

Similarly, $I \cap J \subset I$ and $I \cap J \subset J$ implies that $\sqrt{ I \cap J }   = \sqrt I \cap \sqrt J$.


Lastly, let $P$ be a prime ideal containing $IJ$.  I claim that $P$ contains $I$ or $J$. Suppose not.  Then there exists $x \in I - P$ and $y \in J - P$.
But then $xy \in IJ \subset P$, which contradicts the fact that $P$ is a prime ideal. Thus, $P$ contains $I$ or $J$.

Since $\sqrt I \cap \sqrt J$ intersection of all prime ideals containing $I$ or $J$, this implies that $\sqrt I \cap \sqrt J \subset \sqrt{IJ}$.
\end{proof}

\p 3 Let $\phi: A \to B$ be a ring homomorphism, and $I$ an ideal of $B$. Prove that $\phi^{-1}(\sqrt I) = \sqrt{\phi^{-1}(I)}$.

\begin{proof}

Since inverse functions preserve intersections and containment, it is good enough to show that inverse maps preserve prime ideals. 

Let $J$ be a prime ideal.  Let $a \in A$ and $x,y \in \phi^{-1}(J)$.  Then we have $\phi(x + y) = \phi(x) + \phi(y) \in J$, and $\phi(ax) = \phi(a)\phi(x) \in J$.  Thus $\phi^{-1}(J)$ is an ideal.   Now suppose $a,b \in A$ with $ab \in \phi^{-1}(J)$.  Then $\phi(a)\phi(b) = \phi(ab) \in J$,  hence $\phi(a)$ or $\phi(b)$ is in $J$ since $J$ is prime.  Thus, $\phi^{-1}(J)$ is also a prime ideal.
\end{proof}

\p 4 Let $n$ be a positive integer, $n \ge 2$. Find $\nil(\Z/(n))$.
\begin{proof}
The maximal ideals of $\Z/(n)$ are the ideals generated by the prime factors of $n$.  The intersection of these ideals, $\nil(\Z/(n))$,  is the ideal generated by the product of the distinct prime factors of $n$ taken without multiplicity.
\end{proof}

\p 5 Let $\phi: A \to B$ be a surjective ring homomorphism.
\begin{enumerate}[(a)]
\item Prove that $\phi(\rad(A)) \subset \rad(B)$.
\begin{proof}
Let $a \in \rad(A)$ and $b \in B$.  Since $\phi$ is surjective, there exists $c \in A$ with $\phi(c) = b$.  Since $a \in \rad(A)$, the element $1 - ca$ is a unit. Thus, $\phi(1 - ca) = 1 - b\phi(a)$ is a unit.  Since this holds for all $b \in B$, it follows that $\phi(a) \in \rad(B)$.  Since this holds for all $a \in \rad(A)$, we have $\phi(\rad(A)) \subset \rad(B)$.
\end{proof}
\item Give an example to show that the inclusion need not be an equality.
\begin{proof}

\end{proof}
\end{enumerate}

\p 6 Let $A$ be a local ring.  Prove that $A$ contains no idempotent elements other than $0$ and $1$.
\begin{proof}
Suppose not.  Let $a \in A - \{0,1\}$ be idempotent.  Then $a(1-a) = 0$.  Thus $a$ and $1-a$ are both non-units, a contradiction.
\end{proof}

\p 7 Let $A$ be a local ring and $M,N$ finitely generated $A$-modules. Prove that if $M \otimes_A  N = 0$, then $M = 0$ or $N = 0$.
\begin{proof}
Let $k$ be the residue field of $A$, and $\mathfrak{m}$ the maximal ideal.  We have $0 = (M \otimes_A N) \otimes_A (k \otimes_A k) = (M \otimes_A k) \otimes_A (N \otimes_A k) = (M \otimes_A k) \otimes_k (N \otimes_A k)$.  The last expression is just a tensor product of finite dimensional vector spaces over a field, so either $(M \otimes_A k)$ or $(N \otimes_A k)$ is 0.  WLOG suppose the former is 0. Then $0 = M \otimes_A k = M/\mathfrak{m}M$, so by Nakayama's lemma $M = 0$.
\end{proof}

\p 8 Let $A$ be a ring. Suppose that $A^m \cong A^n$ as $A$-modules.  Show that $m = n$. (Hint: reduce to the case of a field.)
\begin{proof}
Let $M$ be a maximal ideal of $A$.  Then $A^m/MA^m \cong A^n/MA^n$ as $A/MA$-vector spaces. Since $A^m/MA^m \cong (A/MA)^m$ and the same for $n$, the invariance of dimension for vector spaces implies that $m = n$.
\end{proof}


\p 9 Let $I_1, \ldots, I_n$ be ideals of a ring for which $I_1 \cap \ldots \cap I_n = (0)$. Prove that if $A/I_j$ is Noetherian for each $j$, then $A$ is Noetherian.
\begin{proof}
To prove this, I'll use induction on $n$.  When $n = 1$, it is trivially true.

Suppose the theorem holds for any collection of $n-1$ or fewer ideals.  Let $\phi:A \to A/I_n$ be the canonical surjection.  Then we have $\phi(I_1) \cap \ldots \cap \phi(I_{n-1})$
\end{proof}

\p 10 Let $A, B, C$ be rings and suppose that $\phi: A \to C, \psi: B \to C$ are surjective ring homomorphisms. Prove that if $A$ and $B$ are Noetherian, then $A \times_C B$ is Noetherian. (Hint: use 9.).

\end{document}
