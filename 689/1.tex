\documentclass{article}
\usepackage{../m}

\begin{document}
\noindent Paul Gustafson\\
\noindent Math 689 Commutative and Homological Algebra

\subsection*{HW 1}
All rings are commutative with $1 \neq 0$.
\p 1 Let $A$ be a ring and let $\mathfrak S$ be the set of all multiplicative subsets of $A$ that do not contain $0$, ordered by inclusion.  Prove that an element $S$ of $\mathfrak S$ is maximal in $\mathfrak S$ if and only if $R - S$ is a minimal prime ideal.
\begin{proof}
Suppose $S$ is a maximal element of $\mathfrak S$. By a proposition from class, there exists a prime ideal $I$ disjoint from $S$.  Suppose $J$ is a prime ideal contained in $I$.  Then $R - J$ is a multiplicative subset of $A$ not containing $0$, i.e, an element of $\mathfrak S$.  But we also have $S \subset R - I \subset R -J$. Thus, by the maximality of $S$, we have $S = R - I = R - J$.  Hence $I = J$. Since this for all prime ideals $J \subset I$, $I$ is a minimal prime ideal.

For the opposite implication, suppose $S$ is not a maximal element of $\mathfrak S$. Pick $T \in \mathfrak S$ with $S$ properly contained in $T$.  Pick a prime ideal $I$ disjoint from $T$. Then
$I \subset R - T \subset R - S$, where the last inclusion is proper.  Therefore, $R - S$ is not a minimal prime ideal.
\end{proof}

\p 2 Let $I$ and $J$ be ideals of a ring $A$. Show that 
$$\sqrt{IJ} = \sqrt{ I \cap J }   = \sqrt I \cap \sqrt J .$$

\begin{proof}

\end{proof}

\p 3 Let $\phi: A \to B$ be a ring homomorphism, and $I$ an ideal of $B$. Prove that $\phi^{-1}(\sqrt I) = \sqrt{\phi^{-1}(I)}$.

\p 4 Let $n$ be a positive integer, $n \ge 2$. Find $\nil(\Z/(n))$.

\p 5 Let $\phi: A \to B$ be a surjective ring homomorphism.
\begin{enumerate}[(a)]
\item Prove that $\phi(\rad(A)) \subset \rad(B)$.
\item Give an example to show that the inclusion need not be an equality.
\end{enumerate}

\p 6 Let $A$ be a local ring.  Prove that $A$ contains no idempotent elements other than $0$ and $1$.

\p 7 Let $A$ be a local ring and $M,N$ finitely generated $A$-modules. Prove that if $M \otimes_A  N = 0$, then $M = 0$ or $N = 0$.

\p 8 Let $A$ be a ring. Suppose that $A^m \cong A^n$ as $A$-modules.  Show that $m = n$. (Hint: reduce to the case of a field.)

\p 9 Let $I_1, \ldots, I_n$ be ideals of a ring for which $I_1 \cap \ldots \cap I_n = (0)$. Prove that if $A/I_j$ is Noetherian for each $j$, then $A$ is Noetherian.

\p 10 Let $A, B, C$ be rings and suppose that $\phi: A \to C, \psi: B \to C$ are surjective ring homomorphisms. Prove that if $A$ and $B$ are Noetherian, then $A \times_C B$ is Noetherian. (Hint: use 9.).

\end{document}
