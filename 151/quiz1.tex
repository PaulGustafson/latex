\documentclass[12pt]{amsart}
\usepackage{amssymb, latexsym, amsmath, amsthm, amsfonts, amsbsy, enumerate, tabularx}
\usepackage{enumerate}
\usepackage{fancyhdr}
\usepackage{graphics}
\usepackage{ltxtable,tabularx,tabulary}

\usepackage[latin1]{inputenc}
\usepackage{wrapfig}
\usepackage{tikz}
\usepackage{pgfplots}
\usepackage{multicol}

%\usepackage[pdftex]{graphicx}

%%%%%%%%%%%%% Margins  %%%%%%%%%%%%%%%%%%%%

\setlength{\oddsidemargin}{-0.25in}
\setlength{\evensidemargin}{-0.25in}

% Set width of the text - What is left will be the right margin.
% In this case, right margin is 8.5in - 1.25in - 6in = 1.25in.
\setlength{\textwidth}{7in}
\setlength{\headwidth}{7in}
%\setlength{\headheight}{16pt}
% Set top margin - The default is 1 inch, so the following
% command sets a 0.75-inch top margin.
\setlength{\topmargin}{-0.25in}

% Set height of the text - What is left will be the bottom margin.
% In this case, bottom margin is 11in - 0.75in - 9.5in = 0.75in
\setlength{\textheight}{10in}

\setlength{\parskip}{1ex plus 0.5ex minus 0.2ex}
\setlength{\baselineskip}{-5pt}
\setlength{\parindent}{0mm}
%%%%%%%%%%%%%%%%%%%%%%%%%%%%%%%%%%%%%%%


\pagestyle{fancy}
\lhead{NAME:}
\chead{MATH 151 Section:}
\rhead{September 6, 2017}
%\cfoot{}

\title{Quiz 1}
%\author{Good Luck}
%\date{}

\begin{document}

\maketitle
\thispagestyle{fancy}

%\begin{itemize}
%\item Show all your work and indicate your final
%answer clearly. You will be graded not merely on the final answer, but also on the work leading up to it.
%\end{itemize}
\vskip0.25in
\begin{enumerate}


\item[\textbf{1. }] (3 points) Find the sum of the vectors $\langle 1, 2 \rangle$ and $\langle -3, 0 \rangle$.  Illustrate the sum geometrically.

\vskip 2in

\item[\textbf{2. }](3 points)  Find the vector projection of the vector $\langle 3, 0 \rangle$ onto the vector $\langle 1, 1 \rangle$.  Illustrate the projection geometrically.
 
\vskip 2.5in

\item[\textbf{3. }] (4 points) Find a vector equation, a parametric equation, and a Cartesian equation for the line that passes through
the point $(1,2)$ and is parallel to the vector $\langle -1, 1 \rangle$.




\end{enumerate}
\end{document} 
