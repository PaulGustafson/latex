\documentclass[12pt]{amsart}
\usepackage{amssymb, latexsym, amsmath, amsthm, amsfonts, amsbsy, enumerate, tabularx}
\usepackage{enumerate}
\usepackage{fancyhdr}
\usepackage{graphics}
\usepackage{ltxtable,tabularx,tabulary}

\usepackage[latin1]{inputenc}
\usepackage{wrapfig}
\usepackage{tikz}
\usepackage{pgfplots}
\usepackage{multicol}

%\usepackage[pdftex]{graphicx}

%%%%%%%%%%%%% Margins  %%%%%%%%%%%%%%%%%%%%

\setlength{\oddsidemargin}{-0.25in}
\setlength{\evensidemargin}{-0.25in}

% Set width of the text - What is left will be the right margin.
% In this case, right margin is 8.5in - 1.25in - 6in = 1.25in.
\setlength{\textwidth}{7in}
\setlength{\headwidth}{7in}
%\setlength{\headheight}{16pt}
% Set top margin - The default is 1 inch, so the following
% command sets a 0.75-inch top margin.
\setlength{\topmargin}{-0.25in}

% Set height of the text - What is left will be the bottom margin.
% In this case, bottom margin is 11in - 0.75in - 9.5in = 0.75in
\setlength{\textheight}{10in}

\setlength{\parskip}{1ex plus 0.5ex minus 0.2ex}
\setlength{\baselineskip}{-5pt}
\setlength{\parindent}{0mm}
%%%%%%%%%%%%%%%%%%%%%%%%%%%%%%%%%%%%%%%


\pagestyle{fancy}
\lhead{NAME:}
\chead{MATH 151 Section:}
\rhead{October 17, 2013} %date
%\cfoot{}

\title{Quiz 6}%number
%\author{Good Luck}
%\date{}

\begin{document}

\maketitle
\thispagestyle{fancy}

\begin{itemize}
\item Show all your work and indicate your final
answer clearly. You will be graded not merely on the final answer, but also on the work leading up to it.
\end{itemize}
\vskip0.25in
\begin{enumerate}


\item[\textbf{1. }] (3 points) A 20 m long ladder leans against a vertical wall. If the bottom of the ladder slides away from the wall at a rate of 2 m/s, how fast is the top of the ladder sliding down the wall when the bottom of the ladder is 6m away from the wall?


\vskip 2in

\item[\textbf{2. }](4 points) Find the equation of the tangent line to the curve 
$$x(t) = t^2 + 1 \quad \quad y(t) = t^3 -2$$
at the point $(2, -1)$.
 
\vskip 2in

\item[\textbf{3. }] (3 points) Use a linear approximation to estimate $\sqrt {26}$.


\end{enumerate}
\end{document} 
