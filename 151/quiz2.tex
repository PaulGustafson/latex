\documentclass[12pt]{amsart}
\usepackage{amssymb, latexsym, amsmath, amsthm, amsfonts, amsbsy, enumerate, tabularx}
\usepackage{enumerate}
\usepackage{fancyhdr}
\usepackage{graphics}
\usepackage{ltxtable,tabularx,tabulary}

\usepackage[latin1]{inputenc}
\usepackage{wrapfig}
\usepackage{tikz}
\usepackage{pgfplots}
\usepackage{multicol}

%\usepackage[pdftex]{graphicx}

%%%%%%%%%%%%% Margins  %%%%%%%%%%%%%%%%%%%%

\setlength{\oddsidemargin}{-0.25in}
\setlength{\evensidemargin}{-0.25in}

% Set width of the text - What is left will be the right margin.
% In this case, right margin is 8.5in - 1.25in - 6in = 1.25in.
\setlength{\textwidth}{7in}
\setlength{\headwidth}{7in}
%\setlength{\headheight}{16pt}
% Set top margin - The default is 1 inch, so the following
% command sets a 0.75-inch top margin.
\setlength{\topmargin}{-0.25in}

% Set height of the text - What is left will be the bottom margin.
% In this case, bottom margin is 11in - 0.75in - 9.5in = 0.75in
\setlength{\textheight}{10in}

\setlength{\parskip}{1ex plus 0.5ex minus 0.2ex}
\setlength{\baselineskip}{-5pt}
\setlength{\parindent}{0mm}
%%%%%%%%%%%%%%%%%%%%%%%%%%%%%%%%%%%%%%%


\pagestyle{fancy}
\lhead{NAME:}
\chead{MATH 151 Section:}
\rhead{September 12, 2013}
%\cfoot{}

\title{Quiz 2}
%\author{Good Luck}
%\date{}

\begin{document}

\maketitle
\thispagestyle{fancy}

%\begin{itemize}
%\item Show all your work and indicate your final
%answer clearly. You will be graded not merely on the final answer, but also on the work leading up to it.
%\end{itemize}
\vskip0.25in
\begin{enumerate}


\item[\textbf{1. }]\textbf{a)} (3 points) Find the equation of the line passing through $(1,1)$ and perpendicular to $- \mathbf i +  2 \mathbf j$.

\vskip 2in

\item[\textbf{2. }](3 points) Find the vertical and horizontal asymptotes of $\displaystyle y = \frac {x^2 - 1} {x^2 + 2x + 1}$.
 
\vskip 2in

\item[\textbf{3. }] (4 points) Let 
\[
f(x) = \begin{cases} 
0 & \text{ if } x < 0 
\\ x^2 & \text{ if } 0 \le x < 1
\\ 3 & \text{ if } x \ge 1
\end{cases} 
\]

Calculate $\displaystyle \lim_{x \to 0} f(x)$ if it exists, or show that it does not exist.  Do the same for $\displaystyle \lim_{x \to 1} f(x)$.




\end{enumerate}
\end{document} 
