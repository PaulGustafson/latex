\documentclass[12pt]{amsart}
\usepackage{amssymb, latexsym, amsmath, amsthm, amsfonts, amsbsy, enumerate, tabularx}
\usepackage{enumerate}
\usepackage{fancyhdr}
\usepackage{graphics}
\usepackage{ltxtable,tabularx,tabulary}

\usepackage[latin1]{inputenc}
\usepackage{wrapfig}
\usepackage{tikz}
\usepackage{pgfplots}
\usepackage{multicol}

%\usepackage[pdftex]{graphicx}

%%%%%%%%%%%%% Margins  %%%%%%%%%%%%%%%%%%%%

\setlength{\oddsidemargin}{-0.25in}
\setlength{\evensidemargin}{-0.25in}

% Set width of the text - What is left will be the right margin.
% In this case, right margin is 8.5in - 1.25in - 6in = 1.25in.
\setlength{\textwidth}{7in}
\setlength{\headwidth}{7in}
%\setlength{\headheight}{16pt}
% Set top margin - The default is 1 inch, so the following
% command sets a 0.75-inch top margin.
\setlength{\topmargin}{-0.25in}

% Set height of the text - What is left will be the bottom margin.
% In this case, bottom margin is 11in - 0.75in - 9.5in = 0.75in
\setlength{\textheight}{10in}

\setlength{\parskip}{1ex plus 0.5ex minus 0.2ex}
\setlength{\baselineskip}{-5pt}
\setlength{\parindent}{0mm}
%%%%%%%%%%%%%%%%%%%%%%%%%%%%%%%%%%%%%%%


\pagestyle{fancy}
\lhead{NAME:}
\chead{MATH 151 Section:}
\rhead{November 1, 2017}
%\cfoot{}

\title{Quiz 7}
%\author{Good Luck}
%\date{}

\begin{document}

\maketitle
\thispagestyle{fancy}

\begin{itemize}
\item Show all your work and indicate your final
answer clearly. You will be graded not merely on the final answer, but also on the work leading up to it.
\end{itemize}
\vskip0.25in
\begin{enumerate}


\item[\textbf{1. }] (3 points) Find the absolute minimum and maximum values of $f(x) = 2x^3 + 3x^2 -12x - 5$ on $[-3,3]$.

\vskip 2in

\item[\textbf{2. }] (3 points) Find the critical points of $f(x) = x^{2/3}(x-2)^2$.

\vskip 2in

\item[\textbf{3. }] (4 points) Let $f(x) = x^2 - 1$.
  \begin{enumerate}[(a)]
    \item Does $f(x)$ satisfy the
      hypotheses of the Mean Value Theorem on the interval $[-2,2]$?
    \item Find all numbers $c \in [-2,2]$ such that $f'(c) = \frac{f(2) - f(-2)}{4}$.
  \end{enumerate}

\end{enumerate}
\end{document} 
