\documentclass{article}
\usepackage{../m}

\begin{document}
\noindent Paul Gustafson\\   
\noindent Texas A\&M University - Math 637\\ 
\noindent Instructor: Zoran Sunik

\subsection*{HW 4}
\p{7} The closure of the open disk contains all the (infinitely many) 1-cells, so the (C) condition fails. 

For the (W) condition, suppose $A \cap \overline{e}$ is closed in $X$ for every cell $e$. Taking $e$ to be the 2-cell, we have that $A$ is closed in $X$.
The converse (that $A \subset X$ closed implies $A \cap \overline{e}$ is closed for all cells $e$) is obvious.

\p{8} The set $S := \{(1,1/n) \vert n \text{ a positive integer } \}$ intersects every 1-cell's closure in at most a single point; however, this set has a limit point at $(1,0)$.  Hence the set $S$ is not closed, contradicting the (W) condition.

For the (C) condition, note that the closure of any $1$-cell is simply two of the $0$-cells.

\p{9}

\end{document}
