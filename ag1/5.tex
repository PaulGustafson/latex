\documentclass{article}
\usepackage{../m}

\begin{document}
\noindent Paul Gustafson\\
\noindent Texas A\&M University - Math 620\\ 
\noindent Instructor: Frank Sottile

\subsection*{HW 4} %p83-5
\p{1} Let $X$ and $Y$ be two irreducible algebraic subsets of $k^n$ of respective dimensions $r$ and $s$. 
Our aim is to prove that any irreducible component of $X \cap Y$ is of dimension $\ge r + s -n $.
\begin{enumerate}[a)]
\item Prove that this result is true if $X$ is a hypersurface in $k^n$.
\item Let $\Delta$ be the diagonal in $k^n \times k^n$ (cf. Problem I,4). Prove that
the variety $X \cap Y$ is isomorphic to $(X \times Y) \cap \Delta$. Show that $X \times Y$
is of dimension $r + s$ (use the theorem on dimensions of fibres).
\item Using b) and explicit equations for $\Delta$, finish the problem by reducing to
the case of a hypersurface.
\end{enumerate}

\p 3 Let $p,q$ be integers $ > 0$ and $r$ an integer such that $0 \le r \le \inf(p,q)$. We
denote by $\MM_{p,q}$ the set of matrices $p \times q$ with coefficients in $k$. We endow
this set with its natural affine space structure of dimension $pq$. We set
$$C_r = \{A \in \MM_{p,q} \vert \rank(A) \le r \} \quad \text{ and } \quad 
 C_r' = \{A \in \MM_{p,q} \vert \rank(A) = r \}.$$
\begin{enumerate}[a)]
\item Prove that $C_r$ is a closed set in $\MM_{p,q}$ and $C_r'$ is open in $C_r$.
\item We set
$$J = \begin{pmatrix}
   I_r & 0
\\ 0 & 0
\end{pmatrix}
\in \MM_{p,q},
$$
where $I_r$ is the $r \times r$ identity matrix. Show that the map $\phi: \MM_{p,p} \times \MM_{q,q} \to C_r$
given by $\phi(P,Q) = PJQ$ is a surjective morphism of varieties. Deduce that $C_r$ is irreducible.
\item We consider $\phi'$, the restriction of $\phi$ to $GL(p,k) \times GL(q,k)$. Prove that the image
of $\phi'$ is equal to $C_r'$. Prove that all the fibres of $\phi'$ are isomorphic. Calculate the
dimesion of $\phi'^{-1}(J)$.
\item Deduce from b) and c) the dimension of $C_r$ and the codimension of $C_r$ in $\MM_{p,q}$.
\end{enumerate}

\p 6 Let $E$ be a $k$-vectore space of dimension $n$ and let $G_{n,p}$ be the set of subspaces of $E$ of
dimension $p$. We will admit the result that $G_{n,p}$ is a projective algebraic variety (called a Grassmanian).

Prove that the map $\phi$, associating to a $p$-tuplet of independent vectors of $E$ the subspace they generate, 
induces a surjections from an open set of $E^p$ to $G_{n,p}$. Determine the fibres of $\phi$ and calculate
$\dim G_{n,p}$. (You may use the fact that $\phi$ is a morphism. See also the June 1993 exam.)

\end{document}
