\documentclass{article}
\usepackage{../m}

\begin{document}
\noindent Paul Gustafson\\
\noindent Texas A\&M University - Math 620\\ 
\noindent Instructor: Frank Sottile

\subsection*{HW 4}
\p{B-3 p67} Let  $n, d > 0$ and 
$$ A = \{\alpha = (\alpha_0, \ldots, \alpha_n) \in \N^{n+1} | \alpha_i \ge 0 \text{ and } \sum_{i=0}^n \alpha_i = d \}.$$
We note that $|A| = {{n+d} \choose d }$. We set $|A| = N + 1$.

The cardinality of the support of $\alpha \in A$ is called the breadth of $\alpha$.

Consider distinct integers $i,j \in [0,n]$. We denote by $(i,j)$ (resp. $(i)$) the element $\alpha \in A$ defined by $\alpha_k = 0$ for $k \neq i,j$; $\alpha_i = d- 1$; $\alpha_j = 1$ (resp. $\alpha_k = 0$ for $k \neq i$ and $\alpha_i = d$).

If $X_0, \ldots, X_n$ are variables, then for any $\alpha \in A$ we set $X^\alpha = X_0^{\alpha_0} \cdots X_n^{\alpha_n}$.

We consider the map $\phi : \PP^n \to \PP^N$ defined by the formula
$$\phi(x_0, \ldots, x_n) = (x^\alpha)_{\alpha \in A}.$$
\begin{enumerate}[1)]
\item Prove that $\phi$ is an injective map.
\begin{proof}
Suppose $\phi(x) = \phi(y)$.
\end{proof}

\item Consider the ring homomorphism
$$\theta : k[(Y_\alpha)]_{\alpha \in A} \to k[X_0, \ldots X_n]$$
defined by $\theta(Y_\alpha) = X^\alpha$. Set $I = \ker \theta$ and consider $V = V_P(I)$ (the Veronese variety).
Prove that $I$ is a homogeneous ideal and $\phi(\PP^n) \subset V$.
\item Consider $\alpha, \beta, \gamma, \delta \in A$. We assume $\alpha + \beta = \gamma + \delta$. Prove that $Y_\alpha Y_\beta - Y_\gamma Y_\delta$ is in the ideal $I$.
\item Prove that the open sets $D^+(Y_{(i)})$ cover $V$. (Hint in book).
\item We define $\psi : D^+(Y_{(i)}) \cap V \to D^+(X_i)$ by the formula
$$\psi((y_\alpha)) = (y_{(i,0)}, y_{(i,1)}, \ldots, y_{(i)}, \ldots, y_{(i,n)}).$$
Prove that $\phi$ and $\psi$ are mutually inverse morphisms on the open sets in question.
\item Prove that $\phi$ gives an isomorphism from $\PP^n$ to the Veronese variety $V$.
\end{enumerate}

\end{document}
