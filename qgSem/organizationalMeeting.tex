\documentclass{article}
\usepackage{amsmath}

\DeclareMathOperator{\Irr}{Irr}
\DeclareMathOperator{\spn}{span}

\begin{document}
\section{Resources}
Book: Timmerman . ``Invitation to QG's and Duality.'' Section on $SU_q(2)$.

Papers. Functional analysis.
\begin{itemize}
\item Woronowicz.  ``Twisted SU(2) group.''
\item Woronowicz. ``Compact matrix pseudogroups.''
\item Woronowicz. ``Compact Quantum Groups.''
\end{itemize}

Peter-Weyl theory.
\begin{itemize}
\item Koornwinder. ``Orthogonal polynomials in connection to QGs''
\item Masuda, Mimachi, Makagami, et al. ``Representations of QGs and a q-analog of orthogonal polynomials.''
\end{itemize}

\section{Compact (Matrix) Quantum Groups}
Vague idea: $G \subset M_N(\mathbf C)$, a compact group.  
Things you'd like: representations, harmonic analysis.  
$$L^2(G) = \bigoplus_{\pi \in \Irr(G)} L^2_\pi(G)$$
where
$$L^2_\pi(G) = \spn\{\pi_{ij}\}_{i,j \le \dim(\pi)}$$

So, try to reformulate ``without points.'' I.e. think in terms of algebras in stead of matrices, etc. From Gelfand: $G \leftrightarrow C(G)$. Can recover topology. To encode group structure, take duals of multiplication maps.

\end{document}
