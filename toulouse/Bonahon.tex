Francis Bonahon - Miracoulous cancellation in $\Gamma_?(sl_2)$

When $YX = qXY$ with $q^n = 1$ primitive, then 
$$(X + Y)^n = X^ + Y^n.$$

When $YX = qXY$ with $q^n = 1$ primitive and $n odd$,
$$T_n(X + XY + X^{-1}) = X^n + X^nY^n + X^{-n}$$
where $T_n \in \ZZ[t]$ is the normalized $n$-the Chebyshev
polynomial of the first type (jw Helen Wong).  $T_n$ has the property
that 
$$\Tr A^n = T_n (\Tr A)$$
for all $A \in \SL_w(\CC)$.



Found by studying a surface's Kauffman bracket skein algebra $\mathcal{S}^q(S)$.

Heuristic: When $q^n = 1$, the operation ``take $n$-th powers'' translates
properties of the properties of the quantum group $U_q(\mathfrak g)$ to
the associated Lie group $G$, and back

Evidence: such a phenomenon exists for the quantum cluster $\mathcal A^q$ 
associated to a cluster algebra $\mathcal A$. For most $n$,

$$\{\text{irreducible representations of $\mathcal A^q when $q^n = 1$} \}$$
$$ \updownarrow $$
$$\{\text{points of $\mathcal A$} \}$$

Correspondence uses $(X + Y)^n = X^n + Y^n$.

-----

Instead of $U_q(sl_2)$, consider the dual $SL_q(2)$.

When $q$ is a root of unity (and $q^2$ is a primitive $n$-th root of unity), 
the Chebyshev polynomial $T_n(t)$ enables us to go from $SL_q^2(\mathcal A)$
to $SL_2(\text{commutative } \mathcal A \subset \mathcal A)$.

Analogs for $sl_d$ -> Speculation: Adams operations 


