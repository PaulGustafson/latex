% An introduction to Beamer created 6/8/2016
\documentclass{beamer}
\usetheme{Boadilla}

\usepackage{amsmath}
\usepackage{amsfonts}
\usepackage{amssymb}

\usepackage{tikz}
\usepackage{hyperref}

\title{A Brief Introduction to LaTeX}
\author{Paul Gustafson (adapted from Robert Williams)}
\institute{Texas A\&M University}
\date{June 2, 2017}

\begin{document}
	
	\begin{frame}
		\maketitle
	\end{frame}
	
	\begin{frame}{What is \LaTeX?}
		LaTeX (pronounced ``lay-tech'' or ``la-tech'') is type-setting software designed to satisfy technical and scientific needs.
		
		Why use it?
		\pause
		\begin{itemize}
			\item Your math looks nice \pause
			\item Customizable \pause
			\item Outputs pdf's \pause
			\item \alert<5>{The standard in scientific publication} \pause
		\end{itemize}
		
		\begin{exampleblock}{Downloading and Installing LaTeX}
			\url{http://latex-project.org/ftp.html}
		\end{exampleblock}
		\hyperlink{basic tex}{\beamerbutton{I'm new to LaTeX}}
		\quad
		\hyperlink{outline}{\beamerbutton{I'm familiar with LaTeX}}
		\quad
		\hyperlink{resources}{\beamerbutton{Take me to the resources
slide}}
	\end{frame}
	

	\begin{frame}[fragile]{Outline of a beamer file} \label{outline}
		\begin{verbatim}
		\documentclass[handout]{beamer} %handout collapses frames into
1 slide
		\usetheme{theme}
		\usecolortheme{color}
		
		\usepackage{amsmath, amsfonts, amssymb}
		\usepackage{tikz}
		
		\title{My Informative Title}
		\author{My Name}
		\institute{My Institution}
		\date{Date of presentation} or \date{\today}
		\end{verbatim}
	\end{frame}
	
	\begin{verbatim}
	
	
	
	
		\begin{document}
		
			\begin{frame}
				\maketitle
			\end{frame}
			
			\begin{frame}{Frame Title}
				Contents of paper or presentation
			\end{frame}
		
		\end{document}
	\end{verbatim}
	
	\begin{frame}{Beamer vs Article}
		Inside of each frame, you can write as you would in a normal LaTeX file. There is also a little bit of extra functionality present in beamer:
		\begin{itemize}
			\item You can give individual frames titles \pause
			\item You can segment frames into multiple slides using $\backslash$pause \pause
			\item You can create blocks to sort information \pause
			\item You can draw attention to points during some of these slides by using $\backslash$alert$\langle\#\rangle$\{text\}\pause
			\item You can separate a slide into multiple columns using the columns environment \pause
			\item You can create \hyperlink{outline}{\beamerbutton{hyperlinks}} to other parts of your presentation
		\end{itemize}
	\end{frame}
	
	\begin{frame}{Beamer example}
		The best way to learn beamer is to use it. Let's recreate the example beamer file.
        \end{frame}
                
	
	\begin{frame}{Basic Tex Reference} \label{basic tex}
		Symbols for type-setting commands:
		\begin{itemize}
			\item \% is used for comments
			\item $\backslash$ tells LaTeX you are writing a command
			\begin{itemize}
				\item some commands have \{inputs contained in curly brackets\} or [options contained in square braces]
			\end{itemize}
			\item \$ tells LaTeX to enter or exit math mode
		\end{itemize}
			
		\pause
			
		Common unintuitive commands:
		\begin{itemize}
			\item $\backslash\backslash$ is a line return
			\item \$\$ enters or exits display math mode
			\begin{itemize}
				\item Alternatively, $\backslash[ \text{ math goes here } \backslash]$ also uses display math mode
			\end{itemize}
		\end{itemize}		
	\end{frame}
	
	\begin{frame}[fragile]
		When in math mode, use \^~to write superscripts and \_ to write subscripts. Enclose the (super-)subscript in \{\} if it is more than 1 character long. \pause \\
		\begin{verbatim} $\sum_{n=1}^\infty \frac{1}{2^n} = 1$
\end{verbatim} $\sum_{n=1}^\infty \frac{1}{2^n} = 1$ \pause \\
		\begin{verbatim} $$\sum_{n=1}^\infty \frac{1}{2^n} = 1$$
\end{verbatim}
		$$\sum_{n=1}^\infty \frac{1}{2^n} = 1$$
	\end{frame}
	
	\begin{frame}[fragile]{Creating Lists}
		\begin{columns}
			\begin{column}{0.4\textwidth}
				\begin{verbatim}
					\begin{itemize}
						\item Item 1
						\item Item 2
					\end{itemize}
				\end{verbatim}
				\begin{itemize}
					\item Item 1
					\item Item 2
				\end{itemize}
			\end{column}
			\begin{column}{0.4\textwidth}
				\begin{verbatim}
					\begin{enumerate}
						\item Item 1
						\item Item 2
					\end{enumerate}
				\end{verbatim}
				\begin{enumerate}
					\item Item 1
					\item Item 2
				\end{enumerate}
			\end{column}
		\end{columns}
	\end{frame}
	
	\begin{frame}[fragile]{Tables and Matrices}
		\begin{columns}
			\begin{column}{0.4\textwidth}
				\begin{verbatim}
					\begin{tabular}{c|cc}
						a & b & c \\
						\hline
						d & e & f \\
						g & h & i
					\end{tabular}
				\end{verbatim}
				\begin{tabular}{c|cc}
					a & b & c \\
					\hline
					d & e & f \\
					g & h & i
				\end{tabular}
			\end{column}
			\begin{column}{0.4\textwidth}
				\begin{verbatim}
					$
					\begin{bmatrix}
					a & b & c \\
					d & e & f \\
					g & h & i 
					\end{bmatrix}
					$
				\end{verbatim}
				$
				\begin{bmatrix}
				a & b & c \\
				d & e & f \\
				g & h & i 
				\end{bmatrix}
				$
			\end{column}
		\end{columns}
	\end{frame}
	
	\begin{frame}[fragile]		
		\begin{verbatim}
		\begin{theorem}[Name/Attribution of Theorem]
		X iff Y
		\end{theorem}
		\end{verbatim}
		
		\begin{theorem}[Name/Attribution of Theorem]
			X iff Y
		\end{theorem}
		
		\begin{verbatim}
		\begin{proof}
		This is trivial.
		\end{proof}
		\end{verbatim}
		
		\begin{proof}
			This is trivial.
		\end{proof}
		
		\hyperlink{outline}{\beamerbutton{back to beamer}}
	\end{frame}
	
	% Resources Slide
	\begin{frame}{Additional Resources} \label{resources}
		LaTeX has an active community with many resources to help you along the way.
		\begin{itemize}
			\item LaTeX Wikibooks: \url{https://en.wikibooks.org/wiki/LaTeX}
			\item TeX Stack Exchange: \url{http://tex.stackexchange.com}
			\item The Not So Short Introduction to \LaTeXe:
\url{http://tobi.oetiker.ch/lshort/lshort.pdf}
			\item Detexify: \url{http://detexify.kirelabs.org/classify.html}
			\item A collection of beamer themes and colors:
\url{http://hartwork.org/beamer-theme-matrix/}
		\end{itemize}
	\end{frame}
	
\end{document}
