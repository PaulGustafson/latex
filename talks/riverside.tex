\documentclass{beamer}

\usepackage{graphicx, amssymb, listings}
%\usepackage[active]{srcltx}
%\usepackage[all,xdvi]{xy}
%\usepackage{showlabels}
\usepackage[alphabetic,y2k,lite]{amsrefs}

\newcommand{\ZZ}{\mathbf{Z}}
\newcommand{\mB}{\mathcal{B}}

\newtheorem{prop}[theorem]{Proposition}
\newtheorem{thm}[theorem]{Theorem}
\newtheorem{conj}[theorem]{Conjecture}
\DeclareMathOperator{\Obj}{Obj}
\DeclareMathOperator{\FPdim}{FPdim}
\DeclareMathOperator{\Hom}{Hom}
\DeclareMathOperator{\Aut}{Aut}
\DeclareMathOperator{\ev}{ev}
\DeclareMathOperator{\coev}{coev}
\DeclareMathOperator{\id}{id}
\DeclareMathOperator{\End}{End}
\DeclareMathOperator{\MCG}{MCG}
\DeclareMathOperator{\Homeo}{Homeo}
\DeclareMathOperator{\Vect}{Vect}
\DeclareMathOperator{\Mod}{Mod}
\DeclareMathOperator{\PGL}{PGL}



\newcommand{\img}[1]{
\vfill
\centering
\includegraphics[width=\textwidth,height=0.8\textheight,keepaspectratio]{#1}
\vfill
} 

% There are many different themes available for Beamer. A comprehensive
% list with examples is given here:
% http://deic.uab.es/~iblanes/beamer_gallery/index_by_theme.html
% You can uncomment the themes below if you would like to use a different
% one:
%\usetheme{AnnArbor}
%\usetheme{Antibes}
%\usetheme{Bergen}
%\usetheme{Berkeley}
%\usetheme{Berlin}
%\usetheme{Boadilla}
%\usetheme{boxes}
%\usetheme{CambridgeUS}
%\usetheme{Copenhagen}
%\usetheme{Darmstadt}
%\usetheme{default}
%\usetheme{Frankfurt}
%\usetheme{Goettingen}
%\usetheme{Hannover}
%\usetheme{Ilmenau}
%\usetheme{JuanLesPins}
%\usetheme{Luebeck}
\usetheme{Madrid}
%\usetheme{Malmoe}
%\usetheme{Marburg}
%\usetheme{Montpellier}
%\usetheme{PaloAlto}
%\usetheme{Pittsburgh}
%\usetheme{Rochester}
%\usetheme{Singapore}
%\usetheme{Szeged}
%\usetheme{Warsaw}

\title{Towards Property F for metaplectic modular categories}
\date{Paul Gustafson \\ Texas A\&M University}

\begin{document}
\frame{\titlepage}


\begin{frame}{The Property F conjecture}
\begin{conj}[Rowell]
Let $\mathcal C$ be a braided fusion category and let $X$ be a simple object in $\mathcal C$.  The braid group representations $\mathcal B_n$ on $\End(X^{\otimes n})$ have finite image for all $n>0$ if and only if  $X$ is weakly integral (i.e. $\FPdim(X)^2 \in \mathbf Z$).
\end{conj}

\begin{itemize}
\item Verified for modular categories from quantum groups (Rowell, Naidu, Freedman, Larsen, Wang, Wenzl, Jones, Goldschmidt) 
\end{itemize}
\end{frame}


\begin{frame}{A potential approach to property F for modular categories}
  Prove two conjectures:
  \begin{enumerate}[(1)]
  \item Gauging preserves property F
  \item Every weakly integral modular category is a gauging of a pointed or pointed$\boxtimes$Ising MTC
  \end{enumerate}

  \pause

  \begin{thm}[Natale, 2017]
    Every weakly group-theoretical modular category is a gauging of a pointed or pointed$\boxtimes$Ising MTC
  \end{thm}
\end{frame}


\begin{frame}{Gauging a modular category}
  \begin{itemize}
  \item Starting point: a group homomorphism $\rho: G \to \Aut_\otimes^{br}(\mB)$
  \item Extend $\mB$ by $\rho$ to get a $G$-crossed braided category $\mB_G^\times$
    \begin{itemize}
    \item Cohomological obstructions must vanish for the extension to exist
    \item Choices for fusion rules, associators 
    \end{itemize}
  \item Equivariantize 
  \end{itemize}
\end{frame}


\newcommand{\one}{1}{

\begin{frame}{Metaplectic categories}
  A metaplectic modular category is a unitary modular category with the same fusion rules as $SO(N)_2$ for some odd $N > 1$. It has $2$ simple objects $X_1, X_2$ of dimension $\sqrt{N}$, two simple objects $\one, Z$ of dimension $1$, and $\frac{N-1}{2}$ objects $Y_i$, $i=1,\ldots,\frac{N-1}{2}$ of dimension $2$.

  The fusion rules are:
\begin{enumerate}
 \item $Z\otimes Y_i\cong Y_i$, $Z\otimes X_i\cong X_{i+1}$ (modulo $2$), $Z^{\otimes 2}\cong\one$,
 \item $X_i^{\otimes 2}\cong \one\oplus \bigoplus_{i} Y_i$,
 \item $X_1\otimes X_2\cong Z\oplus\bigoplus_{i} Y_i$,
 \item $Y_i\otimes Y_j\cong Y_{\min\{i+j,N-i-j\}}\oplus Y_{|i-j|}$, for $i\neq j$ and $Y_i^{\otimes 2}=\one\oplus Z\oplus Y_{\min\{2i,N-2i\}}$.
\end{enumerate}

\end{frame}


\begin{frame}{Related Work}
% ng schaumburg - genus 1, any modular category
\begin{theorem}[Rowell--Wenzl]
The images of the braid group representations on $\End_{SO(N)_2}(S^{\otimes n})$ for $N$ odd are isomorphic to images of braid groups in Gaussian representations; in particular, they are finite groups.
\end{theorem}

\begin{theorem}[Ardonne--Cheng--Rowell--Wang]
\begin{enumerate}
\item Suppose $\mathcal{C}$ is a metaplectic modular category with fusion rules $SO(N)_2$, then $\mathcal{C}$ is a gauging of the particle-hole symmetry of a $\mathbb{Z}_N$-cyclic modular category.
\item For $N=p_1^{\alpha_1}\cdots p_s^{\alpha_s}$ with distinct odd primes $p_i$, there are exactly $2^{s+1}$ many inequivalent metaplectic modular categories.
\end{enumerate}
\end{theorem}
\end{frame}


\begin{frame}{Distinguishing metaplectics}
  Ardonne--Finch--Titsworth classify metaplectic fusion categories up to monoidal equivalence and give modular data for low-rank cases.

  Key invariants for distinguishing different metaplectics of the
  same $SO(N)_2$ are the Frobenius-Schur indicators $\nu_2(X_i)$
  of the spin objects.
\end{frame}

\begin{frame}{Strategies}
  In apparent order of increasing difficulty:
  \begin{itemize}
    \item Modify the quantum group construction to get all metaplectic modular categories
    \item Relate the $R$-matrix of a metaplectic category to that of the corresponding $SO(N)_2$
    \item Relate $R$-matrices to $\ZZ_N$ 
  \end{itemize}
\end{frame}


\begin{frame}{Modify the quantum group construction}
  \begin{itemize}
  \item Naive approach: take Galois conjugates of $q^{1/2}$
  \item Not general enough -- doesn't modify $\nu_2(X_i)$
  \item Zesting? (Twist tensor product, can get new central charges)
  \end{itemize}
\end{frame}


\begin{frame}{Compare $R$-matrices with $SO(N)_2$}
  \begin{itemize}
    
    \item Ardonne--Finch--Titsworth compute the $R$-matrices for metaplectic modular categories

    \item Nontrivial ratios of R-symbols ($\zeta^{\nu_2(X_i)}$ for some $\zeta^8 = 1$)
 \end{itemize}
\end{frame}

\begin{frame}{Compare $R$-matrices with $\ZZ_N$}
  \begin{itemize}
    \item $R$-symbols for $\ZZ_N^{(\omega)}$ MTC:
  $$R^{ab}_{a+b} = e^{\frac{2\pi i \omega}{N} ab}$$

    \item Look at braiding $c_{A,A}$ for $A := \bigoplus_{a \in \ZZ_N} [a]_N$.
    \item Can we take a ``square root'' (even just for the twists)? 
  \end{itemize}
\end{frame}



\begin{frame}{Thanks}

Thanks for listening!

\end{frame}


\end{document}
