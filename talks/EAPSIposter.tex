\documentclass[final,t, mathserif]{beamer}
\mode<presentation>
{
\usetheme{I6dv}
%  \usetheme{I6pd}
%  \usetheme{I6pd2}
}
% additional settings
\setbeamerfont{itemize}{size=\normalsize}
\setbeamerfont{itemize/enumerate body}{size=\normalsize}
\setbeamerfont{itemize/enumerate subbody}{size=\normalsize}

% additional packages
\usepackage{times}
\usepackage{amsmath,amsthm, amssymb, latexsym}
\usepackage{graphicx, amssymb, listings}
%\usepackage[active]{srcltx}
%\usepackage[all,xdvi]{xy}
%\usepackage{showlabels}
\usepackage[alphabetic,y2k,lite]{amsrefs}

%%%%%%%%%%%%%%%%%% Tikz %%%%%%%%%
\usepackage{tikz}
\usetikzlibrary{arrows,backgrounds}
\usetikzlibrary{shapes.geometric}
\usetikzlibrary{calc}
\usetikzlibrary{scopes}
\usetikzlibrary{decorations.markings}
%\usepackage[labelformat=empty]{caption}

\tikzset{
every picture/.style={line width=0.8pt, >=stealth,
                       baseline=-3pt,label distance=-3pt},
%%%%%%%%%%  Node styles
dotnode/.style={fill=black,circle,minimum size=2.5pt, inner sep=1pt, outer
sep=0},
morphism/.style={circle,draw,thin, inner sep=1pt, minimum size=15pt,
                 scale=0.8},
small_morphism/.style={circle,draw,thin,inner sep=1pt,
                       minimum size=10pt, scale=0.8},
coupon/.style={draw,thin, inner sep=1pt, minimum size=18pt,scale=0.8},
%%%% different line styles:
regular/.style={densely dashed},
edge/.style={thick, dashed, draw=blue, text=black},
boundary/.style={thick,  draw=blue, text=black},
overline/.style={preaction={draw,line width=2mm,white,-}},
drinfeld center/.style={>=stealth,green!60!black, double
distance=1pt,text=black},
%%%%%%% Fill styles %%%%%%%%%%%%%%%
cell/.style={fill=black!10},
subgraph/.style={fill=black!30},
%%%%%%% Mid-path arrows
midarrow/.style={postaction={decorate},
                 decoration={
                    markings,% switch on markings
                    mark=at position #1 with {\arrow{>}},
                 }},
midarrow/.default=0.5
}
%%%%%%%%%%%%%%%%%%%%%%%%%%%%%%%%%%%%%%%%

\newcommand{\ph}{\varphi}
\renewcommand{\Im}{\mathrm{Im}}

\newcommand{\ee}{\mathbf{e}}
\DeclareMathOperator{\Obj}{Obj}
\DeclareMathOperator{\FPdim}{FPdim}
\DeclareMathOperator{\Hom}{Hom}
\DeclareMathOperator{\ev}{ev}
\DeclareMathOperator{\coev}{coev}
\DeclareMathOperator{\id}{id}
\DeclareMathOperator{\End}{End}
\DeclareMathOperator{\MCG}{MCG}
\DeclareMathOperator{\Homeo}{Homeo}
\DeclareMathOperator{\Vect}{Vect}
\DeclareMathOperator{\Mod}{Mod}
\DeclareMathOperator{\PGL}{PGL}

%\newtheorem{prop}[theorem]{Proposition}
%\newtheorem{conj}[theorem]{Conjecture}


\newcommand{\img}[1]{
\vfill
\centering
\includegraphics[width=\textwidth,height=0.8\textheight,keepaspectratio]{#1}
\vfill} 

%\usepackage{xy}

\usepackage{exscale}
%\boldmath
\usepackage{booktabs, array}
%\usepackage{rotating} %sideways environment
\usepackage[english]{babel}
\usepackage[latin1]{inputenc}
\usepackage[orientation=landscape,size=custom,width=120,height=90,scale=1.65]{beamerposter}
\listfiles
%\graphicspath{{figures/}}
% Display a grid to help align images
%\beamertemplategridbackground[1cm]



\title{\huge Finiteness of mapping class group representations from twisted Dijkgraaf-Witten theory }
\author{Paul Gustafson}
\institute{Texas A\&M University}

% abbreviations
\usepackage{xspace}
\makeatletter
\DeclareRobustCommand\onedot{\futurelet\@let@token\@onedot}
\def\@onedot{\ifx\@let@token.\else.\null\fi\xspace}
\def\eg{{e.g}\onedot} \def\Eg{{E.g}\onedot}
\def\ie{{i.e}\onedot} \def\Ie{{I.e}\onedot}
\def\cf{{c.f}\onedot} \def\Cf{{C.f}\onedot}
\def\etc{{etc}\onedot}
\def\vs{{vs}\onedot}
\def\wrt{w.r.t\onedot}
\def\dof{d.o.f\onedot}
\def\etal{{et al}\onedot}
\makeatother

\newcommand{\arXiv}[1]{\url{http://arxiv.org/abs/#1}}
\newcommand{\doi}[1]{\href{http://dx.doi.org/#1}{{\tt DOI:#1}}}
\newcommand{\euclid}[1]{\href{http://projecteuclid.org/getRecord?id=#1}{{\tt #1}}}
\newcommand{\mathscinet}[1]{\href{http://www.ams.org/mathscinet-getitem?mr=#1}{\tt #1}}
\newcommand{\googlebooks}[1]{(preview at \href{http://books.google.com/books?id=#1}{google books})}

%%%%%%%%%%%%%%%%%%%%%%%%%%%%%%%%%%%%%%%%%%%%%%%%%%%%%%%%%%%%%%%%%%%%%%%%%%%%%%%%%%%%%%%%%%%%%%%%%%%%%%%%%%%%
\theoremstyle{plain}
\newtheorem{thm}{Theorem}[section]
\newtheorem{cor}[thm]{Corollary}
\newtheorem{conjec}[thm]{Conjecture}
\newtheorem{lem}[thm]{Lemma}
\newtheorem{prop}[thm]{Proposition}
\newtheorem{quest}[thm]{Question}
\theoremstyle{definition}
\newtheorem{defn}[thm]{Definition}
\newtheorem{prob}[thm]{Problem}
\newtheorem{nota}[thm]{Notation}
\newtheorem{notes}[thm]{Notes}
\newtheorem{exs}[thm]{Examples}
\newtheorem{ex}[thm]{Example}
\newtheorem{rem}[thm]{Remark}
\newtheorem{rems}[thm]{Remarks} 
\newtheorem{alg}[thm]{Algorithm} 


% Operators %%%%%%%%%%%%%%%%%%%%%%%%%%%%%%%%%
\DeclareMathOperator{\coker}{coker}
\DeclareMathOperator{\capind}{capind}
\DeclareMathOperator{\caps}{caps}
\DeclareMathOperator{\cupind}{cupind}
\DeclareMathOperator{\cups}{cups}
%\DeclareMathOperator{\id}{id}
\DeclareMathOperator{\im}{im}
\DeclareMathOperator{\ind}{ind}
%\DeclareMathOperator{\Hom}{Hom}
\DeclareMathOperator{\Ob}{Ob}
\DeclareMathOperator{\rel}{rel}
\DeclareMathOperator{\tr}{tr}
\DeclareMathOperator{\sh}{sh}
\DeclareMathOperator{\spann}{span}
% Math %%%%%%%%%%%%%%%%%%%%%%%%%%%%%%%%%
\newcommand{\D}{\displaystyle}
\newcommand{\comment}[1]{}
\newcommand{\hs}{\hspace{.07in}}
\newcommand{\hsp}[1]{\hs\text{#1}\hs}
\newcommand{\be}{\begin{enumerate}}
%\newcommand{\ee}{\end{enumerate}}
\newcommand{\itm}[1]{\item[\underline{\ensuremath{#1}:}]} 
\newcommand{\itt}[1]{\item[\underline{\text{#1}:}]} 
\newcommand{\N}{\mathbb{N}}
\newcommand{\R}{\mathbb{R}}
\newcommand{\PP}{\mathbb{P}}
\newcommand{\C}{\mathbb{C}}
\newcommand{\F}{\mathbb{F}}
\newcommand{\Z}{\mathbb{Z}} 
\newcommand{\W}{\mathcal{W}} 
\newcommand{\T}{\mathcal{T}} 
\newcommand{\I}{\infty} 
\newcommand{\set}[2]{\left\{#1 \big| #2\right\}}
\newcommand{\SET}[2]{\left\{#1 \bigg| #2\right\}}
\newcommand{\thh}{^{\text{th}}}
\newcommand{\stt}{^{\text{st}}}
\newcommand{\A}{\mathcal{A}}
\newcommand{\B}{\mathcal{B}}
\newcommand{\E}{\mathcal{E}}
\newcommand{\V}{\mathcal{V}}
% Categories %%%%%%%%%%%%%%%%%%%%%%%%%%%%%%%%%
\newcommand{\Atl}{{\sf{Atl}}}
\newcommand{\AAA}{{\sf{A}}}
\newcommand{\BB}{{\sf{B}}}
\newcommand{\DD}{{\sf{D}}}
\newcommand{\TT}{{\sf{T}}}
\newcommand{\PO}{{\sf{PO}}}
\newcommand{\Q}{{\mathfrak{Q}}}
\newcommand{\el}{{\mathfrak{L}}}
\newcommand{\CC}{{\mathfrak{C}}}
\newcommand{\QQ}{{\sf{Q}}}
\newcommand{\cAtl}{{\sf{cAtl}}} 
\newcommand{\sAtl}{{\sf{sAtl}}}
\newcommand{\ssAtl}{{\sf{ssAtl}}} 
\newcommand{\aD}{{\sf{a}\Delta}}
\newcommand{\tlD}{{\sf{tl}\Delta}}
\newcommand{\ccD}{{\sf{c}\Delta}}
\newcommand{\sD}{{\sf{s}\Delta}}
\newcommand{\ssD}{{\sf{ss}\Delta}}
\newcommand{\TL}{{\sf{TL}}}
\newcommand{\sTL}{{\sf{sTL}}}
\newcommand{\ssTL}{{\sf{ssTL}}}
\newcommand{\aTL}{{\sf{aTL}}}
\newcommand{\Fun}{\sf{Fun}}
%\newcommand{\Vect}{\sf{Vect}}
\newcommand{\Cat}{\sf{Cat}}
\newcommand{\Set}{\sf{Set}}
\newcommand{\Hilb}{{\sf{Hilb}}}
\newcommand{\RMod}[1]{{\sb{#1}\sf{Mod}}}
\newcommand{\ModR}[1]{{\sf{Mod}_#1}} 
\newcommand{\op}{^{\sf{op}}}   


%%%%%%%%%%%%%%%%%%%%%%%%%%%%%%%%%%%%%%%%%%%%%%%%%%%%%%%%%%%%%%%%%%%%%%%%%%%%%%%%%%%%%%%%%%%%%%%%%%
\begin{document}
\tikzstyle{shaded}=[fill=red!10!blue!20!gray!30!white]
\tikzstyle{shaded line}=[double=red!10!blue!20!gray!30!white, double distance=1.5mm, draw=black]
\tikzstyle{unshaded}=[fill=white]
\tikzstyle{unshaded line}=[double=white, double distance=1.5mm, draw=black]
\tikzstyle{Tbox}=[circle, draw, thick, fill=white, opaque,]
\tikzstyle{empty box}=[circle, draw, thick, fill=white, opaque, inner sep=2mm]
\tikzstyle{background rectangle}= [fill=red!10!blue!20!gray!40!white,rounded corners=2mm] 
\tikzstyle{on}=[very thick, red!50!blue!50!black]
\tikzstyle{off}=[gray]

%These are for resizing a family of drawings at the same time
\tikzstyle{traces}=[scale=.2, inner sep=1mm]
\tikzstyle{quadratic}=[scale=.4, inner sep=1mm, baseline]
\tikzstyle{annular}=[scale=.7, inner sep=1mm, baseline]
\tikzstyle{rectangular}=[scale=.75, inner sep=1mm, baseline]
\tikzstyle{make triple edge size}= [scale=.4, inner sep=1mm,baseline] 
\tikzstyle{icosahedron network}=[scale=.3, inner sep=1mm, baseline]
\tikzstyle{ATLsix}=[scale=.25, baseline]
\tikzstyle{TL12}=[scale=.15,baseline]
\tikzstyle{PAdefn}=[scale=.7,baseline]
\tikzstyle{TLEG}=[scale=.5,baseline]
\begin{frame}{} 
  \begin{columns}[t]
    
\begin{column}{.3\linewidth}

%%%%%%%%%%%%%%%%%%%%%%%%%%%%%%%%%%%%%%%%%%%%%%%%%

  \begin{block}{Mapping class groups}

    \begin{itemize}
    
\item
   The mapping class group of a compact surface $\Sigma$, 
   $\MCG(\Sigma),$
   is the group of isotopy classes of orientation-preserving self-homeomorphisms of $\Sigma$ 
  \begin{itemize}
    \item $\MCG(\mathbf{D} \text{ with } n \text{ marked points}) = B_m$
    \item $\MCG(\mathbf{T}^2) = SL(2,\mathbb Z)$
  \end{itemize}
\end{itemize}
\end{block}

%%%%%%%%%%%%%%%%%%%%%%%%%%%%%%%%%%%%%%%%%%%%%%%%%


\begin{block}{Property F conjecture for mapping class groups (Rowell)}
The Turaev-Viro-Barrett-Westbury (TVBW) mapping class group representation associated to a compact surface $\Sigma$ and spherical fusion category $\mathcal A$ has finite image iff $\mathcal A$ is weakly integral.
\end{block}


\begin{block}{The spherical fusion category $\Vect^\omega_G$}
   $\Vect^\omega_G$, the category $G$-graded vector spaces twisted by a 3-cocycle $\omega$ has the following structural morphisms:
\begin{itemize} 
\item The associator $\alpha_{g,h,k}:(V_g \otimes V_h) \otimes V_k \to V_g \otimes (V_h \otimes V_k)$
            $$ \alpha_{g,h,k} = \omega(g,h,k)$$
\item The evaluator $\ev_g:V_g^* \otimes V_g \to 1$
  $$ ev_g = \omega(g^{-1},g,g^{-1})$$
\item The coevaluator $\coev_g:V_g \otimes V_g^* \to 1$
    $$\coev_g = 1$$
\item The pivotal structure $j_g:V_g^{**} \to V_g$
            $$ j_g = \omega(g^{-1},g,g^{-1})$$
\end{itemize}

\end{block}


\begin{block}{Related Work}
  \begin{itemize}
  \item All $\Vect_G^\omega$ braid group representations have finite images (Etingof--Rowell--Witherspoon)

  \item  If $\omega = 1$, every mapping class group representation of a closed surface with $\le 1$ marked point has finite image
(Fjelstad--Fuchs)

  \item Every $SL(2,\mathbf{Z})$ representation from any modular category has finite image (Ng--Schauenberg)
  \end{itemize}
\end{block}

\end{column}
\begin{column}{.3\linewidth}




\begin{block}{Main result}
The image of any $\Vect^\omega_G$ TVBW representation of a mapping class group of an orientable, compact surface with boundary is finite.
\\ \vspace{1cm}
Proof outline:
\begin{itemize}
  \item Describe a tractable presentation of  the representation space
  \item Find a good finite spanning set $S$ for the representation space
  \item Calculate the action of each Birman generator on $S$
  \item Show that the representation of each Birman generator lies in a quotient of a finite group of monomial matrices.  
\end{itemize}
\end{block}

\begin{block}{The TVBW space associated to a surface}
  \begin{itemize}
    \item 
        The TVBW representation space is canonically isomorphic to
        a vector space of formal linear combinations of $\mathcal A$-colored graphs in $\Sigma$  modulo certain local relations (Kirillov).
      
  \end{itemize}

\end{block}

%% \begin{block}{Colored graphs in $\Sigma$}
%% \begin{itemize}
%% \item Let $\Gamma \subset \Sigma$ be an undirected finite graph embedded in $\Sigma$.

%%  \item Define $E^{or}$ to be the set of orientation edges of $\Gamma$, i.e. pairs $\ee=(e,\text{orientation of } e)$; for such an oriented edge $\ee$, we denote by $\bar{\ee}$ the edge with opposite orientation. 

%%  \item A {\em coloring} of $\Gamma$ is the
%% following data: 
%% \begin{itemize}
%%     \item Choice of an object $V(\ee)\in \Obj \mathcal A$ for every oriented edge  $\ee \in E^{or}$ so that $V(\bar{\ee})=V(\ee)^*$.
%%      \item Choice of a vector $\ph(v)\in \Hom_{\mathcal A}(1, V_1 \otimes \cdots \otimes V_n)$  for  every interior vertex $v$, where 
%%       $\ee_1, \dots, \ee_n$ are edges incident to $v$, taken in counterclockwise
%%       order and with outward orientation.
%% \end{itemize}
%% \end{itemize}
%% \end{block}


\begin{block}{Local relations}
\begin{itemize}
\item Isotopy of the graph embedding
\item Linearity in the vertex colorings
\item And the following:
\end{itemize}
\begin{figure}
  \img{local}
%  \caption{The remaining local relations.}\label{f:local_rels1}
\end{figure}

\end{block}

%% \begin{block}{Consequences of the local relations}

%% \begin{figure}[ht]
%% %%%%%%%%%%%%
%% \begin{tikzpicture}
%% \node[morphism] (ph) at (0,0) {$\ph$};
%% \node[morphism] (psi) at (1.5,0) {$\psi$};
%% \node at (-0.7,0.1) {$\vdots$};
%% \node at (2.2,0.1) {$\vdots$};
%% \draw[->] (ph)-- +(220:1cm) node[pos=1.0,below,scale=0.8] {$V_n$};
%% \draw[->] (ph)-- +(140:1cm) node[pos=1.0,above,scale=0.8] {$V_1$};
%% \draw[->] (psi)-- +(40:1cm) node[pos=1.0,above,scale=0.8] {$W_m$};
%% \draw[->] (psi)-- +(-40:1cm) node[pos=1.0,below,scale=0.8] {$W_1$};
%% \draw[->] (ph) -- (psi) node[pos=0.5,above,scale=0.8] {$X_1\oplus X_2$};
%% \end{tikzpicture}
%% %%%%%%%%%%%%
%% =
%% %%%%%%%%%%%%
%% \begin{tikzpicture}
%% \node[morphism] (ph) at (0,0) {$\ph_1$};
%% \node[morphism] (psi) at (1.5,0) {$\psi_1$};
%% \node at (-0.7,0.1) {$\vdots$};
%% \node at (2.2,0.1) {$\vdots$};
%% \draw[->] (ph)-- +(220:1cm) node[pos=1.0,below,scale=0.8]{$V_n$};
%% \draw[->] (ph)-- +(140:1cm) node[pos=1.0,above,scale=0.8]{$V_1$};
%% \draw[->] (psi)-- +(40:1cm) node[pos=1.0,above,scale=0.8]{$W_m$};
%% \draw[->] (psi)-- +(-40:1cm) node[pos=1.0,below,scale=0.8]{$W_1$};
%% \draw[->] (ph) -- (psi) node[pos=0.5,above,scale=0.8] {$X_1$};
%% \end{tikzpicture}
%% %%%%%%%%%%%%
%% +
%% %%%%%%%%%%%%
%% \begin{tikzpicture}
%% \node[morphism] (ph) at (0,0) {$\ph_2$};
%% \node[morphism] (psi) at (1.5,0) {$\psi_2$};
%% \node at (-0.7,0.1) {$\vdots$};
%% \node at (2.2,0.1) {$\vdots$};
%% \draw[->] (ph)-- +(220:1cm) node[pos=1.0,below,scale=0.8]{$V_n$};
%% \draw[->] (ph)-- +(140:1cm) node[pos=1.0,above,scale=0.8]{$V_1$};
%% \draw[->] (psi)-- +(40:1cm) node[pos=1.0,above,scale=0.8]{$W_m$};
%% \draw[->] (psi)-- +(-40:1cm) node[pos=1.0,below,scale=0.8]{$W_1$};
%% \draw[->] (ph) -- (psi) node[pos=0.5,above,scale=0.8] {$X_2$};
%% \end{tikzpicture}
%% %%%%%%%%%%%%


%% \caption{Additivity in edge colorings. Here $\ph_1,\ph_2$ are compositions
%% of $\ph$ with projector $X_1\oplus X_2\to X_1$ (respectively, 
%% $X_1\oplus X_2\to X_2$), and similarly for $\psi_1,\psi_2$.
%% }
%% \end{figure}

%% \begin{itemize}
%% \item Additivity in edge colorings
%% \end{itemize}

%%   \begin{theorem}[Kirillov, Reshitikhin--Turaev]
%%   A colored graph $\Gamma$ may be evaluated on any disk $D\subset \Sigma$, giving
%%   an equivalent colored graph $\Gamma'$ such that $\Gamma'$ is identical
%%   to $\Gamma$ outside of $D$, has the same colored edges crossing $\partial D$,
%%   and contains at most one colored vertex within $D$.
%%   \end{theorem}

%% \end{block}


\end{column}
\begin{column}{.3\linewidth}

\begin{block}{Spanning set for genus 2 closed surface}
  \begin{figure}
    \vfill
    \centering
    \includegraphics[width=0.5\textwidth,height=0.8\textheight,keepaspectratio]{g2span}
    \vfill
%\includegraphics[width=0.5\textwidth]{basis.jpg}
\caption{Element of the spanning set for a genus 2 surface.  Here $[g,h][k,l] = 1$, and the vertex is labeled by a ``simple'' morphism (a $|G|$-th root of unity times a canonical morphism)}
\label{fig:span}
\end{figure}
\end{block}


%% \begin{block}{Applying the Birman generators to the spanning set}
%% \begin{itemize}
%% \item  The next step of the proof is to apply each Birman generator to each element of the spanning set.
%% \item  In each case, we relate the resulting colored graph to another element of the spanning set by means of local moves
%% \item  The local moves map simple colored graphs to simple colored graphs
%% \item Hence, the Birman generators preserve the finite spanning set.
%% \end{itemize}
%% \end{block}

%% \begin{block}{Applying the Birman generators to the spanning set}
%% \begin{prop}[G.]
%% Let $\Gamma$ be a simple colored graph embedded in a surface $\Sigma$.  Let $\Delta$ be the colored graph given by applying one of the three local moves in Figure \ref{f:local_rels1} to $\Gamma$.  Then
%% \begin{enumerate}
%% \item  each edge of $\Delta$ is labeled by $\delta_g$ for some $g \in G$, and
%% \item  there exists $\alpha \in \mu_{|G|}$ such that 
%% $$\Delta = \alpha \Delta' $$
%% in $H$, where $\Delta'$ is a simple colored graph given by replacing each vertex label in $\Delta$ with a simple morphism.
%% \end{enumerate}
%% \end{prop}
%% \end{block}
  
\begin{block}{One of the generators: a Dehn twist}
\img{t1}
\end{block}

%% \begin{block}{Braid generator}
%% \img{t3}
%% \end{block}

%% \begin{block}{Dragging a point}
%% \img{t4}
%% \end{block}



%%%%%%%%%%%%%%%%%%%%%%%%%%%%%%%%%%%%%%%%%%%%%%%%%

\begin{block}{References}

\bibliographystyle{amsalpha}
\scriptsize{
\begin{thebibliography}{0}

\bibitem{hep-th/9311155} J.\ Barrett and B.\ Westbury. {\em Invariants
of Piecewise-Linear 3-Manifolds}, Trans.\ Amer.\ Math.\ Soc.\ \textbf{348} (1996), 3997--4022.

\bibitem{birman} J.\ Birman. \emph{Mapping class groups and their relationship to braid groups}, Comm.\ Pure Appl.\ Math.\ \textbf{22} (1969) 213--242.


\bibitem{erw} P.\ Etingof, E.\ C.\ Rowell, and S.\ Witherspoon, \emph{Braid group representations from twisted quantum doubles of finite groups}, Pacific J.\ Math.\ \textbf{234} (2008), no. 1, 33--42.


\bibitem{kirillovStringNets} A.\ Kirillov, \emph{String-net model of {Turaev-Viro} invariants}, Preprint (2011), arXiv:1106.6033.


\bibitem{nr} D.\ Naidu and E.\ C.\ Rowell. \emph{A finiteness property for braided fusion categories}, Algebr.\ and Represent.\ Theor.\ \textbf{14} (2011), no. 5, 837--855.

\end{thebibliography}
}

\end{block}

%%%%%%%%%%%%%%%%%%%%%%%%%%%%%%%%%%%%%%%%%%%%%%%%%

\end{column}
\end{columns}
\end{frame}
\end{document}
