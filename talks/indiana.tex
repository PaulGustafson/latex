\documentclass{beamer}

\usepackage{graphicx, amssymb, listings}
%\usepackage[active]{srcltx}
%\usepackage[all,xdvi]{xy}
%\usepackage{showlabels}
\usepackage[alphabetic,y2k,lite]{amsrefs}

%%%%%%%%%%%%%%%%%% Tikz %%%%%%%%%
\usepackage{tikz}
\usetikzlibrary{shapes.geometric}
\usetikzlibrary{calc}
\usetikzlibrary{scopes}
\usetikzlibrary{decorations.markings}
%\usepackage[labelformat=empty]{caption}

\tikzset{
every picture/.style={line width=0.8pt, >=stealth,
                       baseline=-3pt,label distance=-3pt},
%%%%%%%%%%  Node styles
dotnode/.style={fill=black,circle,minimum size=2.5pt, inner sep=1pt, outer
sep=0},
morphism/.style={circle,draw,thin, inner sep=1pt, minimum size=15pt,
                 scale=0.8},
small_morphism/.style={circle,draw,thin,inner sep=1pt,
                       minimum size=10pt, scale=0.8},
coupon/.style={draw,thin, inner sep=1pt, minimum size=18pt,scale=0.8},
%%%% different line styles:
regular/.style={densely dashed},
edge/.style={thick, dashed, draw=blue, text=black},
boundary/.style={thick,  draw=blue, text=black},
overline/.style={preaction={draw,line width=2mm,white,-}},
drinfeld center/.style={>=stealth,green!60!black, double
distance=1pt,text=black},
%%%%%%% Fill styles %%%%%%%%%%%%%%%
cell/.style={fill=black!10},
subgraph/.style={fill=black!30},
%%%%%%% Mid-path arrows
midarrow/.style={postaction={decorate},
                 decoration={
                    markings,% switch on markings
                    mark=at position #1 with {\arrow{>}},
                 }},
midarrow/.default=0.5
}
%%%%%%%%%%%%%%%%%%%%%%%%%%%%%%%%%%%%%%%%

\newcommand{\ph}{\varphi}
\renewcommand{\Im}{\mathrm{Im}}

\newcommand{\ee}{\mathbf{e}}
\DeclareMathOperator{\Obj}{Obj}
\DeclareMathOperator{\FPdim}{FPdim}
\DeclareMathOperator{\Hom}{Hom}
\DeclareMathOperator{\ev}{ev}
\DeclareMathOperator{\coev}{coev}
\DeclareMathOperator{\id}{id}
\DeclareMathOperator{\End}{End}
\DeclareMathOperator{\MCG}{MCG}
\DeclareMathOperator{\Homeo}{Homeo}
\DeclareMathOperator{\Vect}{Vect}
\DeclareMathOperator{\Mod}{Mod}
\DeclareMathOperator{\PGL}{PGL}

\newtheorem{proposition}[theorem]{Proposition}
\newtheorem{conj}[theorem]{Conjecture}


\newcommand{\img}[1]{
\vfill
\centering
\includegraphics[width=\textwidth,height=0.8\textheight,keepaspectratio]{#1}
\vfill
} 

% There are many different themes available for Beamer. A comprehensive
% list with examples is given here:
% http://deic.uab.es/~iblanes/beamer_gallery/index_by_theme.html
% You can uncomment the themes below if you would like to use a different
% one:
%\usetheme{AnnArbor}
%\usetheme{Antibes}
%\usetheme{Bergen}
%\usetheme{Berkeley}
%\usetheme{Berlin}
%\usetheme{Boadilla}
%\usetheme{boxes}
%\usetheme{CambridgeUS}
%\usetheme{Copenhagen}
%\usetheme{Darmstadt}
%\usetheme{default}
%\usetheme{Frankfurt}
%\usetheme{Goettingen}
%\usetheme{Hannover}
%\usetheme{Ilmenau}
%\usetheme{JuanLesPins}
%\usetheme{Luebeck}
\usetheme{Madrid}
%\usetheme{Malmoe}
%\usetheme{Marburg}
%\usetheme{Montpellier}
%\usetheme{PaloAlto}
%\usetheme{Pittsburgh}
%\usetheme{Rochester}
%\usetheme{Singapore}
%\usetheme{Szeged}
%\usetheme{Warsaw}

\title{Computing Quantum Mapping Class Group Representations with Haskell}
\date{Paul Gustafson \\ Texas A\&M University}

\begin{document}
\frame{\titlepage}


\begin{frame}{The Property F conjecture}
\begin{conj}[Rowell]
Let $\mathcal C$ be a braided fusion category and let $X$ be a simple object in $\mathcal C$.  The braid group representations $\mathcal B_n$ on $\End(X^{\otimes n})$
have finite image for all $n>0$ if and only if $\FPdim(X)^2 \in \mathbf Z$.
\end{conj}
\end{frame}

\begin{frame}{Modified Property F conjecture}
\begin{conj}
The Turaev-Viro-Barrett-Westbury (TVBW) mapping class group representation associated to a compact surface $\Sigma$ and spherical fusion category $\mathcal A$ has finite image iff $\mathcal A$ is weakly integral.
\end{conj}

Test cases:

\begin{itemize}
\item $\mathcal A = \Vect_G^\omega$
\item $\mathcal A = \mathcal{TY}(\mathbf{Z}_N, \cdot, \cdot)$
\end{itemize}
\end{frame}


\begin{frame}{The TVBW space associated to a 2-manifold}
  \begin{itemize}
    \item 
        Using Kirillov's definitions, the representation space we consider is
        \[
        H := \frac{\text{$\mathcal A$-colored graphs in $\Sigma$}  }
        {\text{local relations}}
       \]
    \pause
    \item The vector space $H$ is canonically isomorphic to the usual (triangulation-based) TVBW state sum vector space associated to $\Sigma$. 
   \end{itemize}
\end{frame}

\begin{frame}{Colored graphs in $\Sigma$}
\begin{itemize}
\item Let $\Gamma \subset \Sigma$ be an undirected finite graph embedded in $\Sigma$.

\pause \item Define $E^{or}$ to be the set of orientation edges of $\Gamma$, i.e. pairs $\ee=(e,
\text{orientation of } e)$; for such an oriented edge $\ee$, we denote by $\bar{\ee}$ the edge with opposite orientation. 

\pause \item A {\em coloring} of $\Gamma$ is the
following data: 
\begin{itemize}
    \item Choice of an object $V(\ee)\in \Obj \mathcal A$ for every oriented edge  $\ee \in E^{or}$ so that $V(\bar{\ee})=V(\ee)^*$.
    \pause \item Choice of a vector $\ph(v)\in \Hom_{\mathcal A}(1, V_1 \otimes \cdots \otimes V_n)$  for  every interior vertex $v$, where 
      $\ee_1, \dots, \ee_n$ are edges incident to $v$, taken in counterclockwise 
      order and with outward orientation.
\end{itemize}
\end{itemize}
\end{frame}


\begin{frame}{Local relations}
\begin{itemize}
\item Isotopy of the graph embedding
\pause \item Linearity in the vertex colorings
\end{itemize}

\pause
\begin{figure}[ht]
%%%%%%%%%%%%%%%%%%%%%%%%%%%%%%%%%%%%%%%%%%%%%%%%%%%%%%%
%%%%%%%%%%

\begin{tikzpicture}
\node[morphism] (ph) at (0,0) {$\ph$};
\node[morphism] (psi) at (1,0) {$\psi$};
\node at (-0.7,0.1) {$\vdots$};
\node at (1.7,0.1) {$\vdots$};
\draw[->] (ph)-- +(220:1cm) node[pos=1.0,below,scale=0.8]
{$V_n$};
\draw[->] (ph)-- +(140:1cm) node[pos=1.0,above,scale=0.8]
{$V_1$};
\draw[->] (psi)-- +(40:1cm) node[pos=1.0,above,scale=0.8]
{$W_m$};
\draw[->] (psi)-- +(-40:1cm) node[pos=1.0,below,scale=0.8]
{$W_1$};
\draw[->] (ph) -- (psi) node[pos=0.5,above,scale=0.8] {$X$};
\end{tikzpicture}
%%%%%%%%
=
%%%%%%%%
\begin{tikzpicture}
\node[ellipse, thin, scale=0.8, inner sep=1pt, draw] (ph) at (0,0)
             {$\ph\circ_{X}\psi$};
\node at (-0.8,0.1) {$\vdots$};
\node at (0.8,0.1) {$\vdots$};
\draw[->] (ph)-- +(220:1cm) node[pos=1.0,below,scale=0.8] {$V_n$};
\draw[->] (ph)-- +(140:1cm) node[pos=1.0,above,scale=0.8] {$V_1$};
\draw[->] (ph)-- +(40:1cm) node[pos=1.0,above,scale=0.8]  {$W_m$};
\draw[->] (ph)-- +(-40:1cm) node[pos=1.0,below,scale=0.8] {$W_1$};
\end{tikzpicture}
%%%%%%%%%
\\
%%%%%%%%%
\begin{tikzpicture}
\node[dotnode] (ph) at (0,0) {};
\node[dotnode] (psi) at (1.5,0) {};
\node at (-0.7,0.1) {$\vdots$};
\node at (2.2,0.1) {$\vdots$};
\draw[->] (ph)-- +(220:1cm) node[pos=1.0,below,scale=0.8] {$A_n$};
\draw[->] (ph)-- +(140:1cm) node[pos=1.0,above,scale=0.8] {$A_1$};
\draw[->] (psi)-- +(40:1cm) node[pos=1.0,above,scale=0.8] {$B_m$};
\draw[->] (psi)-- +(-40:1cm) node[pos=1.0,below,scale=0.8] {$B_1$};
\draw[out=45,in=135, midarrow] (ph) to (psi)
                node[above,xshift=-0.6cm, yshift=0.25cm, scale=0.8] {$V_k$};
\draw[ out=15,in=165, midarrow] (ph) to (psi);
\draw[ out=-15,in=195, midarrow] (ph) to (psi);
\draw[ out=-45,in=225, midarrow] (ph) to (psi) node[below, xshift=-0.6cm, yshift=-0.3cm, scale=0.8] {$V_1$};
\end{tikzpicture}
%%%%%%%%%%%
=
%%%%%%%%%%%
\begin{tikzpicture}
\node[dotnode] (ph) at (0,0) {};
\node[dotnode] (psi) at (1.5,0) {};
\node at (-0.7,0.1) {$\vdots$};
\node at (2.2,0.1) {$\vdots$};
\draw[->] (ph)-- +(220:1cm) node[pos=1.0,below,scale=0.8] {$A_n$};
\draw[->] (ph)-- +(140:1cm) node[pos=1.0,above,scale=0.8] {$A_1$};
\draw[->] (psi)-- +(40:1cm) node[pos=1.0,above,scale=0.8] {$B_m$};
\draw[->] (psi)-- +(-40:1cm) node[pos=1.0,below,scale=0.8] {$B_1$};
\draw[ ->] (ph) to (psi)
            node[above,xshift=-0.8cm,scale=0.8] {$V_1\otimes \dots\otimes V_k$};
\end{tikzpicture}
%%%%%%%
\qquad $k\ge 0$
\\
%%%%%%%
\begin{tikzpicture}
\node[ellipse, scale=0.8, inner sep=1pt, draw,thin] (ph) at (0,0)
{$\mathrm{coev}$};
\draw[->] (ph)-- +(180:1cm) node[pos=1.0,above,scale=0.8] {$V$};
\draw[->] (ph)-- +(0:1cm) node[pos=1.0,above,scale=0.8] {$V^*$};
\end{tikzpicture}
%%%%%%%%
=
%%%%%%%%
\begin{tikzpicture}
\draw[->] (2,0)-- (0,0) node[pos=0.5,above,scale=0.8] {$V$};
\end{tikzpicture}
%%%%%%%%%%%%%%%%%%%%%%%%%%
\caption{The remaining local relations.
        }\label{f:local_rels1}
\end{figure}

\end{frame}

\begin{frame}{Consequences of the local relations}

\begin{figure}[ht]
%%%%%%%%%%%%
\begin{tikzpicture}
\node[morphism] (ph) at (0,0) {$\ph$};
\node[morphism] (psi) at (1.5,0) {$\psi$};
\node at (-0.7,0.1) {$\vdots$};
\node at (2.2,0.1) {$\vdots$};
\draw[->] (ph)-- +(220:1cm) node[pos=1.0,below,scale=0.8] {$V_n$};
\draw[->] (ph)-- +(140:1cm) node[pos=1.0,above,scale=0.8] {$V_1$};
\draw[->] (psi)-- +(40:1cm) node[pos=1.0,above,scale=0.8] {$W_m$};
\draw[->] (psi)-- +(-40:1cm) node[pos=1.0,below,scale=0.8] {$W_1$};
\draw[->] (ph) -- (psi) node[pos=0.5,above,scale=0.8] {$X_1\oplus X_2$};
\end{tikzpicture}
%%%%%%%%%%%%
=
%%%%%%%%%%%%
\begin{tikzpicture}
\node[morphism] (ph) at (0,0) {$\ph_1$};
\node[morphism] (psi) at (1.5,0) {$\psi_1$};
\node at (-0.7,0.1) {$\vdots$};
\node at (2.2,0.1) {$\vdots$};
\draw[->] (ph)-- +(220:1cm) node[pos=1.0,below,scale=0.8]{$V_n$};
\draw[->] (ph)-- +(140:1cm) node[pos=1.0,above,scale=0.8]{$V_1$};
\draw[->] (psi)-- +(40:1cm) node[pos=1.0,above,scale=0.8]{$W_m$};
\draw[->] (psi)-- +(-40:1cm) node[pos=1.0,below,scale=0.8]{$W_1$};
\draw[->] (ph) -- (psi) node[pos=0.5,above,scale=0.8] {$X_1$};
\end{tikzpicture}
%%%%%%%%%%%%
+
%%%%%%%%%%%%
\begin{tikzpicture}
\node[morphism] (ph) at (0,0) {$\ph_2$};
\node[morphism] (psi) at (1.5,0) {$\psi_2$};
\node at (-0.7,0.1) {$\vdots$};
\node at (2.2,0.1) {$\vdots$};
\draw[->] (ph)-- +(220:1cm) node[pos=1.0,below,scale=0.8]{$V_n$};
\draw[->] (ph)-- +(140:1cm) node[pos=1.0,above,scale=0.8]{$V_1$};
\draw[->] (psi)-- +(40:1cm) node[pos=1.0,above,scale=0.8]{$W_m$};
\draw[->] (psi)-- +(-40:1cm) node[pos=1.0,below,scale=0.8]{$W_1$};
\draw[->] (ph) -- (psi) node[pos=0.5,above,scale=0.8] {$X_2$};
\end{tikzpicture}
%%%%%%%%%%%%


\caption{Additivity in edge colorings. Here $\ph_1,\ph_2$ are compositions
of $\ph$ with projector $X_1\oplus X_2\to X_1$ (respectively, 
$X_1\oplus X_2\to X_2$), and similarly for $\psi_1,\psi_2$.
}
\end{figure}

\begin{itemize}
\item Additivity in edge colorings
\pause \item A colored graph may be evaluated on any disk $D\subset S$, giving
  an equivalent colored graph $\Gamma'$ such that $\Gamma'$ is identical
  to $\Gamma$ outside of $D$, has the same colored edges crossing $\partial D$,
  and contains at most one colored vertex within $D$.
\end{itemize}
\end{frame}


\begin{frame}{Related Work ($\Vect_G^\omega$ case)}
% ng schaumburg - genus 1, any modular category
\begin{theorem}[Ng--Schauenberg]
Every modular representation associated to a modular category has finite image.
\end{theorem}

\begin{theorem}[Etingof--Rowell--Witherspoon]
The braid group representation associated to 
the modular category $\Mod(D^\omega(G))$ has finite image.
\end{theorem}

\begin{theorem}[Fjelstad--Fuchs]
Every mapping class group representation of a closed surface with at most one marked point associated to $\Mod(D(G))$ has finite image.
\end{theorem}
\end{frame}

\begin{frame}{First result}
\begin{theorem}[G.]
The image of any $\Vect^\omega_G$ TVBW representation $\rho$ of a mapping class group of an orientable, compact surface $\Sigma$ with boundary is finite.
\end{theorem}
\end{frame}

\begin{frame}{First Dehn twist}
\img{t1}
\end{frame}

\begin{frame}{Second Dehn twist}
\img{t2}
\end{frame}

\begin{frame}{Braid generator}
\img{t3}
\end{frame}

\begin{frame}{Dragging a point}
\img{t4}
\end{frame}



\begin{frame}{Next step: Tambara-Yamagami categories}
Let $A$ be a finite abelian group, $\chi$ a bicharacter on $A$,
and $\nu \in \{\pm 1\}$. The \textbf{Tamabara-Yamagami category} $\mathcal{TY}(A, \chi, \nu)$ is the skeletal spherical category with
simple objects $\{a : a \in A\} \cup \{m\}$, fusion rules given by
\[ a \otimes b = ab \text{ for $a,b \in A$ } \qquad 
a \otimes m = m \qquad
m \otimes m = \bigoplus_{a \in A} a, 
\]
and the following nontrivial structural morphisms
\[
\alpha_{a,m,b} = \chi(a,b) \id_m \qquad  
\alpha_{m,a,m}  = \bigoplus_{b \in A} \chi(a,b) \id_b 
\]
\[
\alpha_{m,m,m} = (\nu |A|^{-1/2} \chi^{-1}(a,b) \id_m)_{a,b \in A},
\]
\[
j_m = \nu \id_m  \qquad  ev_m = \nu |A|^{1/2} \pi_1
\]
\end{frame}


\begin{frame}{Related Work ($\mathcal{TY}$ case)}
% ng schaumburg - genus 1, any modular category
\begin{theorem}[Rowell--Wenzl]
The images of the braid group representations on $\End_{SO(N)_2}(S^{\otimes n})$ for $N$ odd are isomorphic to images of braid groups in Gaussian representations; in particular, they are finite groups.
\end{theorem}

\end{frame}

\begin{frame}{Motivation}
\begin{itemize}
\item Why are Tambara-Yamagami categories interesting?
\begin{itemize}
\item Multi-fusion channels ($m \otimes m = \bigoplus_{a \in A} a$)
\item Gauging (with respect to group inversion action on $\mathbf{Z}_N$)
\end{itemize}
\item Problem: Want to calculate actual matrices 
\end{itemize}
\end{frame}


\begin{frame}{Calculations = Hard}
Easiest example (first Dehn twist, $\Vect^\omega_G$):
\img{hard}
\end{frame}

\begin{frame}{Why is it hard?}
\begin{itemize}
\item Computationally intensive  
\item High level of abstraction
\end{itemize}
\end{frame}

\begin{frame}[fragile]{Solution: Haskell}
\begin{verbatim}
data ColoredGraph = ColoredGraph
                  { vertices      :: [InteriorVertex]
                  , edges         :: [Edge]
                  , disks         :: [Disk]
                  , perimeter     :: Disk -> [Edge]

                  -- image under contractions
                  , imageVertex    :: Vertex -> Vertex     

                  , edgeTree      :: Vertex -> Tree Edge
                  , morphismLabel :: InteriorVertex 
                                      -> Morphism
                  , objectLabel   :: Edge -> Object
                  }
\end{verbatim}
\end{frame}

\begin{frame}[fragile]{Two-Complex Datatypes}
\begin{verbatim}
data Puncture = LeftPuncture | RightPuncture
data InteriorVertex = Main | Midpoint Edge | Contraction Edge
data Vertex = Punc Puncture | IV InteriorVertex
data InitialEdge = LeftLoop | RightLoop | LeftLeg | RightLeg
data Edge
  = IE InitialEdge
  | FirstHalf Edge
  | SecondHalf Edge
  | Connector Edge Edge Disk
  | TensorE Edge Edge
  | Reverse Edge
data Disk = Outside | LeftDisk | RightDisk | Cut Edge
\end{verbatim}
\end{frame}


\begin{frame}[fragile]{Objects}
\begin{verbatim}
data Object
  = OVar InitialEdge 
  | One 
  | Star Object 
  | TensorO Object Object
\end{verbatim}
\end{frame}

\begin{frame}[fragile]{Morphisms}
\begin{verbatim}
data Morphism
  = Phi
  | Id Object
  | Lambda Object
  | LambdaI Object
  | Rho Object
  | RhoI Object
  | Alpha Object Object Object
  | AlphaI Object Object Object
  | Coev Object
  | Ev Object
  | TensorM Morphism Morphism
  | PivotalJ Object
  | PivotalJI Object
  | Compose Morphism Morphism
\end{verbatim}
\end{frame}

\begin{frame}[fragile]{Local Moves}
\begin{verbatim}
tensor :: Disk -> State Stringnet ()
contract :: Edge -> State Stringnet InteriorVertex
connect :: Edge -> Edge -> Disk -> State Stringnet Edge
addCoev :: Edge 
  -> State Stringnet (InteriorVertex, Edge, Edge)
\end{verbatim}
\end{frame}

\begin{frame}[fragile]{Vertex Hom-Space Moves}
\begin{verbatim}
associateL ::
  InteriorVertex -> Tree Edge -> State Stringnet (Tree Edge)
associateR ::
  InteriorVertex -> Tree Edge -> State Stringnet (Tree Edge)
isolateR :: InteriorVertex -> State Stringnet ()
isolateL :: InteriorVertex -> State Stringnet ()
zMorphism :: Object -> Object -> Morphism -> Morphism
zRotate :: InteriorVertex -> State Stringnet ()
isolate2 :: Edge -> Edge -> InteriorVertex 
  -> State Stringnet ()
\end{verbatim}
\end{frame}

\begin{frame}[fragile]{Braid move}
\begin{verbatim}
  (_,l1,r1) <- addCoev $ IE LeftLoop
  (_,l2,r2) <- addCoev $ IE LeftLeg
  (_,r13,l3) <- addCoev r1
  (_,_,r4) <- addCoev $ IE RightLoop
  e1 <- connect (rev l1) r2 LeftDisk
  e2 <- connect (rev l2) (rev r13) (Cut $ e1)
  e3 <- connect l3 r4 Outside
  contract e1                   
  contract e2
  contract e3
  tensor (Cut $ rev e1)
  tensor (Cut $ rev e2)
  tensor (Cut $ rev e3)
  v <- contract r4
\end{verbatim}
+ some reassociating 
\end{frame}


\begin{frame}[fragile]{TambaraYamagami types}
\begin{verbatim}
newtype AElement = AElement Int 

newtype RootOfUnity = RootOfUnity AElement

data Scalar =  Scalar 
  { coeff :: [Int]
  , tauExp :: Sum Int
  } 
\end{verbatim}

A scalar is represented as $\tau^k \sum_{i=0}^{n-1} a_i \zeta_n^i$
\end{frame}

\begin{frame}[fragile]{TambaraYamagami types}
\begin{verbatim}
data SimpleObject =
  -- Group-element-indexed simple objects
  AE !AElement

  -- non-group simple object
  | M

newtype Object = Object
  { multiplicity_ :: [Int]
  }
\end{verbatim}
\end{frame}

\begin{frame}[fragile]{TambaraYamagami types}
\begin{verbatim}
data Morphism = Morphism 
  { domain   :: Object
  , codomain :: Object
  , subMatrix_ :: [M.Matrix Scalar]
  }
\end{verbatim}
\end{frame}

\begin{frame}[fragile]{TambaraYamagami types}
\begin{verbatim}
data BasisElement = BasisElement
  { initialLabel :: S.InitialEdge -> SimpleObject
  , oneIndex :: Int
  }
\end{verbatim}
\end{frame}


%% \begin{frame}{Output Example}
%% \img{final}
%% \end{frame}

%% \begin{frame}{Output example: R matrix for $\mathcal{TY}(\mathbf{Z}_2, (-1)^{xy}, +)$}

%% \end{frame}

\begin{frame}{Next steps}
\begin{itemize}
%% \item Double check $\coev$ and $\ev$ definitions
%% \item Verify snake equations
%% \item verify the pentagon equations
\item Verify braid relations
\item Compare with Ising $R$-matrices
\item Optimize composition to be local wrt tensor products
\item Other mapping class group generators
\item Other categories
\end{itemize}
\end{frame}

\begin{frame}{Thanks}

Thanks for listening!

\end{frame}


\end{document}
