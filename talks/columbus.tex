\documentclass{beamer}

\usepackage{graphicx, amssymb, listings}
%\usepackage[active]{srcltx}
%\usepackage[all,xdvi]{xy}
%\usepackage{showlabels}
\usepackage[alphabetic,y2k,lite]{amsrefs}

\newcommand{\ZZ}{\mathbf{Z}}
\newcommand{\mB}{\mathcal{B}}

\newtheorem{prop}[theorem]{Proposition}
\newtheorem{thm}[theorem]{Theorem}
\newtheorem{conj}[theorem]{Conjecture}

\DeclareMathOperator{\Rep}{Rep}
\DeclareMathOperator{\Obj}{Obj}
\DeclareMathOperator{\FPdim}{FPdim}
\DeclareMathOperator{\Hom}{Hom}
\DeclareMathOperator{\Aut}{Aut}
\DeclareMathOperator{\ev}{ev}
\DeclareMathOperator{\coev}{coev}
\DeclareMathOperator{\id}{id}
\DeclareMathOperator{\End}{End}
\DeclareMathOperator{\MCG}{MCG}
\DeclareMathOperator{\Homeo}{Homeo}
\DeclareMathOperator{\Vect}{Vect}
\DeclareMathOperator{\Mod}{Mod}
\DeclareMathOperator{\PGL}{PGL}



\newcommand{\img}[1]{
\vfill
\centering
\includegraphics[width=\textwidth,height=0.8\textheight,keepaspectratio]{#1}
\vfill
} 

% There are many different themes available for Beamer. A comprehensive
% list with examples is given here:
% http://deic.uab.es/~iblanes/beamer_gallery/index_by_theme.html
% You can uncomment the themes below if you would like to use a different
% one:
%\usetheme{AnnArbor}
%\usetheme{Antibes}
%\usetheme{Bergen}
%\usetheme{Berkeley}
%\usetheme{Berlin}
%\usetheme{Boadilla}
%\usetheme{boxes}
%\usetheme{CambridgeUS}
%\usetheme{Copenhagen}
%\usetheme{Darmstadt}
%\usetheme{default}
%\usetheme{Frankfurt}
%\usetheme{Goettingen}
%\usetheme{Hannover}
%\usetheme{Ilmenau}
%\usetheme{JuanLesPins}
%\usetheme{Luebeck}
\usetheme{Madrid}
%\usetheme{Malmoe}
%\usetheme{Marburg}
%\usetheme{Montpellier}
%\usetheme{PaloAlto}
%\usetheme{Pittsburgh}
%\usetheme{Rochester}
%\usetheme{Singapore}
%\usetheme{Szeged}
%\usetheme{Warsaw}

\title{On metaplectic modular categories}
\date{Paul Gustafson  \\ Texas A\&M University \\   jww Yuze Ruan and Eric Rowell}

\begin{document}
\frame{\titlepage}


\begin{frame}{Motivation: The Property F conjecture}
\begin{conj}[Rowell]
Let $\mathcal C$ be a braided fusion category and let $X$ be a simple object in $\mathcal C$.  The braid group representations $\mathcal B_n$ on $\End(X^{\otimes n})$ have finite image for all $n>0$ if and only if  $X$ is weakly integral (i.e. $\FPdim(X)^2 \in \mathbf Z$).
\end{conj}

\begin{itemize}
\item Verified for modular categories from quantum groups (Rowell, Naidu, Freedman, Larsen, Wang, Wenzl, Jones, Goldschmidt) 
\end{itemize}
\end{frame}


\begin{frame}{A potential approach to property F for modular categories}
  Prove two conjectures:
  \begin{enumerate}[(1)]
  \item Gauging Conjecture (Ardonne--Cheng--Rowell--Wang): Gauging preserves property F
  \item Weakly Group-theoretical Conjecture (Etingof--Nikshych--Ostrik): Every weakly integral modular category is weakly-group theoretical.
  \end{enumerate}

  \begin{thm}[Natale, 2017]
    Every weakly group-theoretical modular category is a gauging of a pointed or pointed$\boxtimes$Ising MTC
  \end{thm}
\end{frame}


\begin{frame}{Gauging a modular category}
  \begin{itemize}
  \item Starting point: a group homomorphism $\rho: G \to \Aut_\otimes^{br}(\mB)$
  \item Extend $\mB$ by $\rho$ to get a $G$-crossed braided category $\mB_G^\times$
    \begin{itemize}
    \item Cohomological obstructions must vanish for the extension to exist
    \item Choices for fusion rules, associators 
    \end{itemize}
  \item Equivariantize 
  \end{itemize}
\end{frame}


\newcommand{\one}{1}{

\begin{frame}{Odd metaplectic modular categories}
  An odd metaplectic modular category is a unitary modular category with the same fusion rules as $SO(N)_2$ for some odd $N > 1$. It has $2$ simple objects $X_1, X_2$ of dimension $\sqrt{N}$, two simple objects $\one, Z$ of dimension $1$, and $\frac{N-1}{2}$ objects $Y_i$, $i=1,\ldots,\frac{N-1}{2}$ of dimension $2$. All simple objects are self-dual.

  The fusion rules are:
\begin{enumerate}
 \item $Z\otimes Y_i\cong Y_i$, $Z\otimes X_i\cong X_{i+1}$ (modulo $2$), $Z^{\otimes 2}\cong\one$,
 \item $X_i^{\otimes 2}\cong \one\oplus \bigoplus_{i} Y_i$,
 \item $X_1\otimes X_2\cong Z\oplus\bigoplus_{i} Y_i$,
 \item $Y_i\otimes Y_j\cong Y_{\min\{i+j,N-i-j\}}\oplus Y_{|i-j|}$, for $i\neq j$ and $Y_i^{\otimes 2}=\one\oplus Z\oplus Y_{\min\{2i,N-2i\}}$.
\end{enumerate}

\end{frame}

\newcommand{\ot}{\otimes}

\begin{frame}{Previous Work}
% ng schaumburg - genus 1, any modular category
\begin{theorem}[Rowell--Wenzl]
The images of the braid group representations on $\End_{SO(N)_2}(S^{\otimes n})$ for $N$ odd are isomorphic to images of braid groups in Gaussian representations; in particular, they are finite groups.
\end{theorem}

\begin{theorem}[Ardonne--Cheng--Rowell--Wang, Bruillard--Plavnik--Rowell]
\begin{enumerate}
\item Suppose $\mathcal{C}$ is a (not necessarily odd) metaplectic modular category with fusion rules $SO(N)_2$ with $4 \nmid N$, then $\mathcal{C}$ is a gauging of the particle-hole symmetry of a $\mathbb{Z}_N$-cyclic modular category.
\item For $N=p_1^{\alpha_1}\cdots p_s^{\alpha_s}$ with distinct odd primes $p_i$, there are exactly $2^{s+1}$ many inequivalent metaplectic modular categories.
\end{enumerate}
\end{theorem}
\end{frame}

\begin{frame}{``Even-even'' metaplectic modular categories}
  An ``even-even'' metaplectic modular category is a unitary modular category with the same fusion rules as $SO(2k)_2$ for even $k \ge 2$. It has $4$ simple objects $V_1, V_2, W_1, W_2$ of dimension $\sqrt{k}$ and four simple objects $\one, f, g, fg$ of dimension $1$. Setting $r := \frac{k}{2} -1$, there are $k-1$ objects $X_i$, $i=0,\ldots,X_{r-1}$ and $Y_0, \ldots, Y_r$ of dimension $2$.  All simple objects are self-dual.
\end{frame}

\begin{frame}{Even-even metaplectic modular categories}

  The key fusion rules are:
  \begin{itemize}
  \item $f^{\ot 2}=g^{\ot 2}=\one$, $f\ot X_i=g\ot X_i=X_{r-i-1}$ and $f\ot Y_i=g\ot Y_i=Y_{r-i}$
  \item $g\ot V_1=V_2, f\ot V_1=V_1$ and $f\ot W_1=W_2, g\ot W_1=W_1$
  \item $V_1^{\ot 2}=\one\oplus f\oplus\bigoplus_{i=0}^{r-1} X_i$
  \item $W_1^{\ot 2}=\one\oplus g\oplus\bigoplus_{i=0}^{r-1} X_i$
  \item $W_1\ot V_1=\bigoplus_{i=0}^r Y_i$
  \item $X_i\ot X_j=\begin{cases} X_{i+j+1}\oplus X_{j-i-1} & i<j\leq\frac{r-1}{2}\\ \one\oplus fg\oplus X_{2i+1} & i=j<\frac{r-1}{2}\\ \one \oplus f\oplus g\oplus fg & i=j=\frac{r-1}{2}<r-1\end{cases}$
  \item $Y_i\ot Y_j=\begin{cases} X_{i+j}\oplus X_{j-i-1} & i<j\leq\frac{r}{2}\\ \one\oplus fg\oplus X_{2i} & i=j\leq\frac{r-1}{2}\\ \one\oplus f\oplus g\oplus fg &i=j=\frac{r}{2}.\end{cases}$
 
  \end{itemize}

\end{frame}

\newcommand{\mcC}{\mathcal{C}}
\newcommand{\mcD}{\mathcal{D}}
\newcommand{\mbbZ}{\mathbb{Z}}


\begin{frame}{Analogous theorem for even-even metaplectic modular categories}
  \begin{theorem}[Bruillard--G--Plavnik--Rowell]
     If $\mcC$ is a metaplectic modular category with the fusion rules of $SO(N)_2$ with $4\mid N$ then the de-equivariantization $\mcD:=\mcC_{\mbbZ_2}$ by $\langle fg\rangle=\Rep(\mbbZ_2)$ is a  generalized
    Tambara-Yamagami category of dimension $4N$, and, the trivial
    component $\mcD_0:=[\mcC_{\mbbZ_2}]_0\cong\mcC(\mbbZ_N,q)$ is a pointed cyclic modular category.  Moreover, $\mcC$ is obtained from $\mcC(\mbbZ_N,q)$ via a $\mbbZ_2$-gauging of the particle-hole symmetry.
  \end{theorem}

  Degenerate case: we have $SO(4)_2 = Ising \boxtimes Ising$.
\end{frame}


  \begin{frame} {Count for even-even metaplectic modular categories (BGPR)}
    \begin{itemize}
    \item Suppose we have the prime factorization $N = 2^a p_1^{a_1} \cdots p_s^{a_s}$ for $a > 2$.
    \item Let $\rho: \mbbZ_2 \to \Aut(\mbbZ_N)$ denote the map determined by $\rho(1)(n) = -n$, i.e. the particle-hole symmetry.
    \pause
    \item There are two potential obstructions to extending $\mcC(\mbbZ_N,q)$ by $\rho$.
      \begin{enumerate}
      \item The first obstruction $H_\rho^3(\mbbZ_2,\mbbZ_N)$ vanishes since we know that a gauging exists.
      \item The second obstruction vanishes since $H^4(\mbbZ_2,U(1)) \cong 0$.
      \end{enumerate}
    \pause
    \item The action of $H_\rho^2(\mbbZ_2,\mbbZ_N)$ on the fusion rules is trivial.
    \item There is a choice of associativity constraints on the $\mbbZ_2$-extension, so that \emph{a priori} we have $4$ distinct theories.
    \item Labelling ambiguity reduces the factor of $4$ to $3$
    \item  This gives $3(2^{s+2})$ metaplectic modular categories of dimension $4N>16$ with $4\mid N$.  
    \end{itemize}
  \end{frame}

  \begin{frame}{Current work (jww Eric Rowell and Yuze Ruan)}
    \begin{itemize}
    \item Sequential gauging for even metaplectics
    \item Property F for metaplectic modular categories
    \end{itemize}
  \end{frame}

\begin{frame}{Thanks}

Thanks for listening!

\end{frame}


\end{document}
