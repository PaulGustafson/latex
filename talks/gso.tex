\documentclass{beamer}

\usepackage{graphicx, amssymb, listings}
%\usepackage[active]{srcltx}
%\usepackage[all,xdvi]{xy}
%\usepackage{showlabels}
\usepackage[alphabetic,y2k,lite]{amsrefs}

%%%%%%%%%%%%%%%%%% Tikz %%%%%%%%%
\usepackage{tikz}
\usetikzlibrary{shapes.geometric}
\usetikzlibrary{calc}
\usetikzlibrary{scopes}
\usetikzlibrary{decorations.markings}
%\usepackage[labelformat=empty]{caption}

\tikzset{
every picture/.style={line width=0.8pt, >=stealth,
                       baseline=-3pt,label distance=-3pt},
%%%%%%%%%%  Node styles
dotnode/.style={fill=black,circle,minimum size=2.5pt, inner sep=1pt, outer
sep=0},
morphism/.style={circle,draw,thin, inner sep=1pt, minimum size=15pt,
                 scale=0.8},
small_morphism/.style={circle,draw,thin,inner sep=1pt,
                       minimum size=10pt, scale=0.8},
coupon/.style={draw,thin, inner sep=1pt, minimum size=18pt,scale=0.8},
%%%% different line styles:
regular/.style={densely dashed},
edge/.style={thick, dashed, draw=blue, text=black},
boundary/.style={thick,  draw=blue, text=black},
overline/.style={preaction={draw,line width=2mm,white,-}},
drinfeld center/.style={>=stealth,green!60!black, double
distance=1pt,text=black},
%%%%%%% Fill styles %%%%%%%%%%%%%%%
cell/.style={fill=black!10},
subgraph/.style={fill=black!30},
%%%%%%% Mid-path arrows
midarrow/.style={postaction={decorate},
                 decoration={
                    markings,% switch on markings
                    mark=at position #1 with {\arrow{>}},
                 }},
midarrow/.default=0.5
}
%%%%%%%%%%%%%%%%%%%%%%%%%%%%%%%%%%%%%%%%

\newcommand{\ph}{\varphi}
\renewcommand{\Im}{\mathrm{Im}}

\newcommand{\ee}{\mathbf{e}}
\DeclareMathOperator{\Obj}{Obj}
\DeclareMathOperator{\Hom}{Hom}
\DeclareMathOperator{\ev}{ev}
\DeclareMathOperator{\coev}{coev}
\DeclareMathOperator{\id}{id}
\DeclareMathOperator{\End}{End}
\DeclareMathOperator{\MCG}{MCG}
\DeclareMathOperator{\Homeo}{Homeo}
\DeclareMathOperator{\Vect}{Vect}
\DeclareMathOperator{\Mod}{Mod}
\DeclareMathOperator{\PGL}{PGL}

\newtheorem{proposition}[theorem]{Proposition}
\newtheorem{conj}[theorem]{Conjecture}


\newcommand{\img}[1]{
\vfill
\includegraphics[width=\textwidth,height=0.8\textheight,keepaspectratio]{#1}
\vfill
} 

% There are many different themes available for Beamer. A comprehensive
% list with examples is given here:
% http://deic.uab.es/~iblanes/beamer_gallery/index_by_theme.html
% You can uncomment the themes below if you would like to use a different
% one:
%\usetheme{AnnArbor}
%\usetheme{Antibes}
%\usetheme{Bergen}
%\usetheme{Berkeley}
%\usetheme{Berlin}
%\usetheme{Boadilla}
%\usetheme{boxes}
%\usetheme{CambridgeUS}
%\usetheme{Copenhagen}
%\usetheme{Darmstadt}
%\usetheme{default}
%\usetheme{Frankfurt}
%\usetheme{Goettingen}
%\usetheme{Hannover}
%\usetheme{Ilmenau}
%\usetheme{JuanLesPins}
%\usetheme{Luebeck}
\usetheme{Madrid}
%\usetheme{Malmoe}
%\usetheme{Marburg}
%\usetheme{Montpellier}
%\usetheme{PaloAlto}
%\usetheme{Pittsburgh}
%\usetheme{Rochester}
%\usetheme{Singapore}
%\usetheme{Szeged}
%\usetheme{Warsaw}

\title{Computing Quantum Mapping Class Group Representations with Haskell}
\date{Paul Gustafson}

\begin{document}
\frame{\titlepage}

\begin{frame}{Motivation}
Why should we care about quantum mapping class group representations?
\begin{itemize}
\item Topological quantum computation
\vfill
\item Understand mapping class groups  (algebraic geometry)
\vfill
\item Understand tensor categories (representation theory, operator algebras)
\vfill
\item Intrinsic beauty

\end{itemize}
\end{frame}

\begin{frame}{Mapping class group definition}
\begin{itemize}
\item
    Let $\Sigma = \Sigma_{g,b}^m$ be the oriented compact surface of genus $g$ with $b$ boundary components and $m$ marked points in its interior.

    \pause
\item
   The mapping class group of $\Sigma$, 
   $MCG(\Sigma),$
   is the group of isotopy classes of orientation-preserving homeomorphisms of $\Sigma$ that preserve the boundary \emph{pointwise} and preserve the marked points \emph{setwise}.
  
  \pause
  \item Examples
  \begin{itemize}
    \item $\MCG(\Sigma_{0,1}^m) = B_m$
    \item $\MCG(\Sigma_{1,0}^0) = SL(2,\mathbb Z)$
  \end{itemize}

\pause

  \item Birman (1969) found ``nice'' finite generating set for the mapping class group of any compact surface.
\end{itemize}
\end{frame}

\begin{frame}{(2+1)-Topological Quantum Field Theories}
\begin{itemize}
\item Quantum mapping class group representations come from (2+1)-TQFTs
\pause
\item A (2+1)-TQFT assigns a vector space to oriented 2-manifolds
\item It also assigns a vector to oriented 3-manifolds. This vector inhabits the vector space corresponding to the 3-manifold's boundary.
\item To get a mapping class group representation, apply the TQFT to mapping cylinders.
\pause
\item History: Witten and Atiyah (1980s)
\pause
\item Examples of mathematical (2+1)-TQFTs: 
\begin{itemize}
\item Reshitikhin-Turaev TQFT (input: modular category)
\item Turaev-Viro-Barret-Westbury TQFT (input: spherical fusion category)
\end{itemize}
\end{itemize}
\end{frame}

\begin{frame}{Monoidal categories}
A \textbf{monoidal category} is a category $\mathcal C$ equipped with
\begin{itemize}
\item a tensor product -- a  bifunctor $\otimes: \mathcal C \times \mathcal C \to \mathcal C$
\item an associativity isomorphism -- a natural isomorphism $\alpha: (\cdot \otimes \cdot) \otimes \cdot \to \cdot \otimes (\cdot \otimes \cdot)$ 
\item a unit object $1 \in \mathcal C$
\item a left unitor -- a natural isomorphism $\lambda_X: 1 \otimes X \to X$
\item a right unitor -- a natural isomorphism $\rho_X: X \otimes 1 \to X$,
\end{itemize}
satisfying certain coherence conditions (the triangle and pentagon axioms).
\end{frame}


\begin{frame}{Rigid monoidal categories}
\begin{itemize}
\item Let $X$ be an object of a monoidal category $\mathcal C$. A \textbf{left dual} to $X$ is an object $X^*$ equipped with 
\begin{itemize}
\item an evaluation morphism, $\ev_X : X^* \otimes X \to 1$
\item a coevaluation morphism, $\coev_X: 1 \to X \otimes X^*$,
\end{itemize}
satisfying certain coherence conditions (the snake equations).

\pause 
\item Right duals are defined similarly. 
\item An object $X$ is \textbf{rigid} if it
has both a left and right dual.  
\item A monoidal category is rigid if all of
its objects are rigid.
\end{itemize}
\end{frame}

\begin{frame}{Fusion category}
\begin{itemize}
\item A \textbf{fusion category} is a rigid semisimple linear monoidal category (“tensor category”), with only finitely many isomorphism classes of simple objects, such that the endomorphisms of the unit object form just the ground field $k$.

\pause
\item Example: Representation category of a finite group
\end{itemize}
\end{frame}

\begin{frame}{Spherical categories}
\begin{itemize}
\item A \textbf{pivotal category} is a rigid monoidal category equipped with
a pivotal structure, i.e. a tensor natural automorphism $j_X : X \to (X^*)^*$.  

\item A pivotal structure defines left and right traces $\End(X) \to \End(1)$ for every object $X$. 

\pause

\item A \textbf{spherical category} is a monoidal category such that all left and right traces coincide.
\end{itemize}
\end{frame}

\begin{frame}{Example: $\Vect^\omega_G$}
\begin{itemize}
\item  Let $G$ be a finite group, and $\omega: G \times G \times G \to \mathbb{C}$ be a 3-cocycle. The spherical fusion category $\Vect^\omega_G$ is the skeletal category of $G$-graded finite-dimensional vector spaces with the following modified structural morphisms, where $V_g$ is the simple object:   
\begin{itemize} 
\item The associator $a_{g,h,k}:(V_g \otimes V_h) \otimes V_k \to V_g \otimes (V_h \otimes V_k)$
            $$ a_{g,h,k} = \omega(g,h,k)$$
\item The evaluator $ev_g:V_g^* \otimes V_g \to 1$
            $$ ev_g = \omega(g^{-1},g,g^{-1})$$
\item The pivotal structure $j_g:V_g^{**} \to V_g$
            $$ j_g = \omega(g^{-1},g,g^{-1})$$
\end{itemize}
\end{itemize}
\end{frame}


\begin{frame}{The TVBW space associated to a 2-manifold}
  \begin{itemize}
    \item Let $\mathcal A$ be a spherical fusion category, and $\Sigma$ an oriented compact surface with boundary.
    \item 
        Using Kirillov's definitions, the representation space we consider is
        \[
        H := \frac{\text{$\mathcal A$-colored graphs in $\Sigma$}  }
        {\text{local relations}}
       \]
    \pause
    \item The vector space $H$ is canonically isomorphic to the Turaev-Viro state sum vector space associated to $\Sigma$.  This isomorphism commutes with the mapping class group action.
   \end{itemize}
\end{frame}

\begin{frame}{Colored graphs in $\Sigma$}
\begin{itemize}
\item Let $\Gamma \subset \Sigma$ be an undirected finite graph embedded in $\Sigma$.

\pause \item Define $E^{or}$ to be the set of orientation edges of $\Gamma$, i.e. pairs $\ee=(e,
\text{orientation of } e)$; for such an oriented edge $\ee$, we denote by $\bar{\ee}$ the edge with opposite orientation. 

\pause \item A {\em coloring} of $\Gamma$ is the
following data: 
\begin{itemize}
    \item Choice of an object $V(\ee)\in \Obj \mathcal A$ for every oriented edge  $\ee \in E^{or}$ so that $V(\bar{\ee})=V(\ee)^*$.
    \pause \item Choice of a vector $\ph(v)\in \Hom_{\mathcal A}(1, V_1 \otimes \cdots \otimes V_n)$  for  every interior vertex $v$, where 
      $\ee_1, \dots, \ee_n$ are edges incident to $v$, taken in counterclockwise 
      order and with outward orientation.
\end{itemize}
\end{itemize}
\end{frame}


\begin{frame}{Local relations}
\begin{itemize}
\item Isotopy of the graph embedding
\pause \item Linearity in the vertex colorings
\end{itemize}

\pause
\begin{figure}[ht]
%%%%%%%%%%%%%%%%%%%%%%%%%%%%%%%%%%%%%%%%%%%%%%%%%%%%%%%
%%%%%%%%%%

\begin{tikzpicture}
\node[morphism] (ph) at (0,0) {$\ph$};
\node[morphism] (psi) at (1,0) {$\psi$};
\node at (-0.7,0.1) {$\vdots$};
\node at (1.7,0.1) {$\vdots$};
\draw[->] (ph)-- +(220:1cm) node[pos=1.0,below,scale=0.8]
{$V_n$};
\draw[->] (ph)-- +(140:1cm) node[pos=1.0,above,scale=0.8]
{$V_1$};
\draw[->] (psi)-- +(40:1cm) node[pos=1.0,above,scale=0.8]
{$W_m$};
\draw[->] (psi)-- +(-40:1cm) node[pos=1.0,below,scale=0.8]
{$W_1$};
\draw[->] (ph) -- (psi) node[pos=0.5,above,scale=0.8] {$X$};
\end{tikzpicture}
%%%%%%%%
=
%%%%%%%%
\begin{tikzpicture}
\node[ellipse, thin, scale=0.8, inner sep=1pt, draw] (ph) at (0,0)
             {$\ph\circ_{X}\psi$};
\node at (-0.8,0.1) {$\vdots$};
\node at (0.8,0.1) {$\vdots$};
\draw[->] (ph)-- +(220:1cm) node[pos=1.0,below,scale=0.8] {$V_n$};
\draw[->] (ph)-- +(140:1cm) node[pos=1.0,above,scale=0.8] {$V_1$};
\draw[->] (ph)-- +(40:1cm) node[pos=1.0,above,scale=0.8]  {$W_m$};
\draw[->] (ph)-- +(-40:1cm) node[pos=1.0,below,scale=0.8] {$W_1$};
\end{tikzpicture}
%%%%%%%%%
\\
%%%%%%%%%
\begin{tikzpicture}
\node[dotnode] (ph) at (0,0) {};
\node[dotnode] (psi) at (1.5,0) {};
\node at (-0.7,0.1) {$\vdots$};
\node at (2.2,0.1) {$\vdots$};
\draw[->] (ph)-- +(220:1cm) node[pos=1.0,below,scale=0.8] {$A_n$};
\draw[->] (ph)-- +(140:1cm) node[pos=1.0,above,scale=0.8] {$A_1$};
\draw[->] (psi)-- +(40:1cm) node[pos=1.0,above,scale=0.8] {$B_m$};
\draw[->] (psi)-- +(-40:1cm) node[pos=1.0,below,scale=0.8] {$B_1$};
\draw[out=45,in=135, midarrow] (ph) to (psi)
                node[above,xshift=-0.6cm, yshift=0.25cm, scale=0.8] {$V_k$};
\draw[ out=15,in=165, midarrow] (ph) to (psi);
\draw[ out=-15,in=195, midarrow] (ph) to (psi);
\draw[ out=-45,in=225, midarrow] (ph) to (psi) node[below, xshift=-0.6cm, yshift=-0.3cm, scale=0.8] {$V_1$};
\end{tikzpicture}
%%%%%%%%%%%
=
%%%%%%%%%%%
\begin{tikzpicture}
\node[dotnode] (ph) at (0,0) {};
\node[dotnode] (psi) at (1.5,0) {};
\node at (-0.7,0.1) {$\vdots$};
\node at (2.2,0.1) {$\vdots$};
\draw[->] (ph)-- +(220:1cm) node[pos=1.0,below,scale=0.8] {$A_n$};
\draw[->] (ph)-- +(140:1cm) node[pos=1.0,above,scale=0.8] {$A_1$};
\draw[->] (psi)-- +(40:1cm) node[pos=1.0,above,scale=0.8] {$B_m$};
\draw[->] (psi)-- +(-40:1cm) node[pos=1.0,below,scale=0.8] {$B_1$};
\draw[ ->] (ph) to (psi)
            node[above,xshift=-0.8cm,scale=0.8] {$V_1\otimes \dots\otimes V_k$};
\end{tikzpicture}
%%%%%%%
\qquad $k\ge 0$
\\
%%%%%%%
\begin{tikzpicture}
\node[ellipse, scale=0.8, inner sep=1pt, draw,thin] (ph) at (0,0)
{$\mathrm{coev}$};
\draw[->] (ph)-- +(180:1cm) node[pos=1.0,above,scale=0.8] {$V$};
\draw[->] (ph)-- +(0:1cm) node[pos=1.0,above,scale=0.8] {$V^*$};
\end{tikzpicture}
%%%%%%%%
=
%%%%%%%%
\begin{tikzpicture}
\draw[->] (2,0)-- (0,0) node[pos=0.5,above,scale=0.8] {$V$};
\end{tikzpicture}
%%%%%%%%%%%%%%%%%%%%%%%%%%
\caption{The remaining local relations.
        }\label{f:local_rels1}
\end{figure}

\end{frame}

\begin{frame}{Consequences of the local relations}

\begin{figure}[ht]
%%%%%%%%%%%%
\begin{tikzpicture}
\node[morphism] (ph) at (0,0) {$\ph$};
\node[morphism] (psi) at (1.5,0) {$\psi$};
\node at (-0.7,0.1) {$\vdots$};
\node at (2.2,0.1) {$\vdots$};
\draw[->] (ph)-- +(220:1cm) node[pos=1.0,below,scale=0.8] {$V_n$};
\draw[->] (ph)-- +(140:1cm) node[pos=1.0,above,scale=0.8] {$V_1$};
\draw[->] (psi)-- +(40:1cm) node[pos=1.0,above,scale=0.8] {$W_m$};
\draw[->] (psi)-- +(-40:1cm) node[pos=1.0,below,scale=0.8] {$W_1$};
\draw[->] (ph) -- (psi) node[pos=0.5,above,scale=0.8] {$X_1\oplus X_2$};
\end{tikzpicture}
%%%%%%%%%%%%
=
%%%%%%%%%%%%
\begin{tikzpicture}
\node[morphism] (ph) at (0,0) {$\ph_1$};
\node[morphism] (psi) at (1.5,0) {$\psi_1$};
\node at (-0.7,0.1) {$\vdots$};
\node at (2.2,0.1) {$\vdots$};
\draw[->] (ph)-- +(220:1cm) node[pos=1.0,below,scale=0.8]{$V_n$};
\draw[->] (ph)-- +(140:1cm) node[pos=1.0,above,scale=0.8]{$V_1$};
\draw[->] (psi)-- +(40:1cm) node[pos=1.0,above,scale=0.8]{$W_m$};
\draw[->] (psi)-- +(-40:1cm) node[pos=1.0,below,scale=0.8]{$W_1$};
\draw[->] (ph) -- (psi) node[pos=0.5,above,scale=0.8] {$X_1$};
\end{tikzpicture}
%%%%%%%%%%%%
+
%%%%%%%%%%%%
\begin{tikzpicture}
\node[morphism] (ph) at (0,0) {$\ph_2$};
\node[morphism] (psi) at (1.5,0) {$\psi_2$};
\node at (-0.7,0.1) {$\vdots$};
\node at (2.2,0.1) {$\vdots$};
\draw[->] (ph)-- +(220:1cm) node[pos=1.0,below,scale=0.8]{$V_n$};
\draw[->] (ph)-- +(140:1cm) node[pos=1.0,above,scale=0.8]{$V_1$};
\draw[->] (psi)-- +(40:1cm) node[pos=1.0,above,scale=0.8]{$W_m$};
\draw[->] (psi)-- +(-40:1cm) node[pos=1.0,below,scale=0.8]{$W_1$};
\draw[->] (ph) -- (psi) node[pos=0.5,above,scale=0.8] {$X_2$};
\end{tikzpicture}
%%%%%%%%%%%%


\caption{Additivity in edge colorings. Here $\ph_1,\ph_2$ are compositions
of $\ph$ with projector $X_1\oplus X_2\to X_1$ (respectively, 
$X_1\oplus X_2\to X_2$), and similarly for $\psi_1,\psi_2$.
}
\end{figure}

\begin{itemize}
\item Additivity in edge colorings
\pause \item A colored graph may be evaluated on any disk $D\subset S$, giving
  an equivalent colored graph $\Gamma'$ such that $\Gamma'$ is identical
  to $\Gamma$ outside of $D$, has the same colored edges crossing $\partial D$,
  and contains at most one colored vertex within $D$.
\end{itemize}
\end{frame}


\begin{frame}{Overall Strategy}
\begin{itemize}
\item Find a basis of colored graphs for the representation space for a surface
\item ``Calculate'' the representation of each mapping class group generator with respect to this basis
\item Analyze the image of the representation (Is it finite? Can we do universal quantum computation with it (possibly adding extra measurements)?)
\end{itemize}
\end{frame}

\begin{frame}{Modified Property F conjecture}
\begin{conj}[Rowell]
A TVBW mapping class group representation associated to a spherical fusion category $\mathcal A$ has finite image iff $\mathcal A$ is weakly integral, i.e. the squared dimension of every simple object is an integer.
\end{conj}

\end{frame}

\begin{frame}{Related Work}


%ng schaumburg - genus 1, any modular category
\begin{theorem}[Ng--Schauenberg]
Every modular representation associated to a modular category has finite image.
\end{theorem}

\pause


\begin{theorem}[Etingof--Rowell--Witherspoon]
The braid group representation associated to 
the modular category $\Mod(D^\omega(G))$ has finite image.
\end{theorem}

\pause


 ERW also asked if, more generally, all mapping class group representations associated to $\Mod(D^\omega(G))$ have finite image.  

\pause

\begin{theorem}[Fjelstad--Fuchs]
Every mapping class group representation of a closed surface with at most one marked point associated to $\Mod(D(G))$ has finite image.
\end{theorem}

\end{frame}


\begin{frame}{Answering ERW's question: First Dehn twist}
\img{t1}
\end{frame}

\begin{frame}{Second Dehn twist}
\img{t2}
\end{frame}

\begin{frame}{Braid generator}
\img{t3}
\end{frame}

\begin{frame}{Dragging a point}
\img{t4}
\end{frame}

\begin{frame}{First result}

\begin{theorem}
The image of any $\Vect^\omega_G$ TVBW representation $\rho$ of a mapping class group of an orientable, compact surface $\Sigma$ with boundary is finite.
\end{theorem}

\emph{Sketch of proof.}  

\begin{itemize}
\item For any $k$, let $\mu_{|G|}$ denote the set of $|G|$-th roots of unity. Then $\omega$ is cohomologous to a cocycle taking values in $\mu_{|G|}$.  Since cohomologous cocycles give rise to equivalent spherical categories $\Vect^\omega_G$, WLOG $\omega$ takes values in $\mu_{|G|}$.

\pause \item  Let $L \subset \MCG(\Sigma)$ be the Birman generating set. From the figures and the definitions of structural morphisms of $\Vect_G^\omega$,  $\rho(L)$ acts by permutations on $\mu_{|G|} S$ for a spanning set $S$.

\pause \item Let $B \subset S$ be a basis for $H$. Then $\rho(\MCG(\Sigma)) B \subset \rho(\MCG(\Sigma)) S \subset \mu_{|G|}  S$.
 
 \pause \item   Thus, $|\Im(\rho)| < \infty$.
\end{itemize}

\end{frame}

\begin{frame}{Next steps}
\begin{itemize}
\item Tambara-Yamagami categories
\begin{itemize}
\item Simple class of categories with ``multi-fusion channels''
\item Simple examples of gauging (with respect to group inversion action)
\end{itemize}
\item Problem: Want to calculate actual matrices 
\end{itemize}
\end{frame}

\begin{frame}{Calculations = Hard}
Easiest example (first Dehn twist, $\Vect^\omega_G$):
\img{hard}
\end{frame}

\begin{frame}{Why is it hard?}
\begin{itemize}
\item Computationally intensive  
\item High level of abstraction
\end{itemize}
\end{frame}

\begin{frame}[fragile]{Solution: Haskell}
\begin{verbatim}
data Stringnet = Stringnet
                  { vertices      :: [InteriorVertex]
                  , edges         :: [Edge]
                  , disks         :: [Disk]
                  , perimeter     :: Disk -> [Edge]

                  -- image under contractions
                  , imageVertex    :: Vertex -> Vertex     

                  , edgeTree      :: Vertex -> Tree Edge
                  , morphismLabel :: InteriorVertex 
                                      -> Morphism
                  , objectLabel   :: Edge -> Object
                  }
\end{verbatim}
\end{frame}

\begin{frame}[fragile]{Two-Complex Datatypes}
\begin{verbatim}
data Puncture = LeftPuncture | RightPuncture
data InteriorVertex = Main | Midpoint Edge | Contraction Edge
data Vertex = Punc Puncture | IV InteriorVertex
data InitialEdge = LeftLoop | RightLoop | LeftLeg | RightLeg
data Edge
  = IE InitialEdge
  | FirstHalf Edge
  | SecondHalf Edge
  | Connector Edge Edge Disk
  | TensorE Edge Edge
  | Reverse Edge
data Disk = Outside | LeftDisk | RightDisk | Cut Edge
\end{verbatim}
\end{frame}


\begin{frame}[fragile]{Objects}
\begin{verbatim}
data Object
  = OVar InitialEdge 
  | One 
  | Star Object 
  | TensorO Object Object
\end{verbatim}
\end{frame}

\begin{frame}[fragile]{Morphisms}
\begin{verbatim}
data Morphism
  = Phi
  | Id Object
  | Lambda Object
  | LambdaI Object
  | Rho Object
  | RhoI Object
  | Alpha Object Object Object
  | AlphaI Object Object Object
  | Coev Object
  | Ev Object
  | TensorM Morphism Morphism
  | PivotalJ Object
  | PivotalJI Object
  | Compose Morphism Morphism
\end{verbatim}
\end{frame}

\begin{frame}[fragile]{Local Moves}
\begin{verbatim}
tensor :: Disk -> State Stringnet ()
contract :: Edge -> State Stringnet InteriorVertex
connect :: Edge -> Edge -> Disk -> State Stringnet Edge
addCoev :: Edge 
  -> State Stringnet (InteriorVertex, Edge, Edge)
\end{verbatim}
\end{frame}

\begin{frame}[fragile]{Vertex Hom-Space Moves}
\begin{verbatim}
associateL ::
  InteriorVertex -> Tree Edge -> State Stringnet (Tree Edge)
associateR ::
  InteriorVertex -> Tree Edge -> State Stringnet (Tree Edge)
isolateR :: InteriorVertex -> State Stringnet ()
isolateL :: InteriorVertex -> State Stringnet ()
zMorphism :: Object -> Object -> Morphism -> Morphism
zRotate :: InteriorVertex -> State Stringnet ()
isolate2 :: Edge -> Edge -> InteriorVertex 
  -> State Stringnet ()
\end{verbatim}
\end{frame}

\begin{frame}[fragile]{Braid move}
\begin{verbatim}
  (_,l1,r1) <- addCoev $ IE LeftLoop
  (_,l2,r2) <- addCoev $ IE LeftLeg
  (_,r13,l3) <- addCoev r1
  (_,_,r4) <- addCoev $ IE RightLoop
  e1 <- connect (rev l1) r2 LeftDisk
  e2 <- connect (rev l2) (rev r13) (Cut $ e1)
  e3 <- connect l3 r4 Outside
  contract e1                   
  contract e2
  contract e3
  tensor (Cut $ rev e1)
  tensor (Cut $ rev e2)
  tensor (Cut $ rev e3)
  v <- contract r4
\end{verbatim}
+ some reassociating 
\end{frame}

\begin{frame}{Tambara-Yamagami categories}
Let $A$ be a finite abelian group, $\chi$ a bicharacter on $A$,
and $\nu \in \{\pm 1\}$. The \textbf{Tamabara-Yamagami category} $\mathcal{TY}(A, \chi, \nu)$ is the skeletal spherical category with
simple objects $\{a : a \in A\} \cup \{m\}$, fusion rules given by
\[ a \otimes b = ab \text{ for $a,b \in A$ } \qquad 
a \otimes m = m \qquad
m \otimes m = \bigoplus_{a \in A} a, 
\]
and the following nontrivial structural morphisms
\[
\alpha_{a,m,b} = \chi(a,b) \id_m \qquad  
\alpha_{m,a,m}  = \bigoplus_{b \in A} \chi(a,b) \id_b 
\]
\[
\alpha_{m,m,m} = (\nu |A|^{-1/2} \chi^{-1}(a,b) \id_m)_{a,b \in A},
\]
\[
j_m = \nu \id_m  \qquad  ev_m = \nu |A|^{1/2} \pi_1
\]
\end{frame}

\begin{frame}[fragile]{TambaraYamagami types}
\begin{verbatim}
newtype AElement = AElement Int 

newtype RootOfUnity = RootOfUnity AElement

data Scalar =  Scalar 
  { coeff :: [Int]
  , tauExp :: Sum Int
  } 
\end{verbatim}

A scalar is represented as $\tau^k \sum_{i=0}^{n-1} a_i \zeta_n^i$
\end{frame}

\begin{frame}[fragile]{TambaraYamagami types}
\begin{verbatim}
data SimpleObject =
  -- Group-element-indexed simple objects
  AE !AElement

  -- non-group simple object
  | M

newtype Object = Object
  { multiplicity_ :: [Int]
  }
\end{verbatim}
\end{frame}

\begin{frame}[fragile]{TambaraYamagami types}
\begin{verbatim}
data Morphism = Morphism 
  { domain   :: Object
  , codomain :: Object
  , subMatrix_ :: [M.Matrix Scalar]
  }
\end{verbatim}
\end{frame}

\begin{frame}[fragile]{TambaraYamagami types}
\begin{verbatim}
data BasisElement = BasisElement
  { initialLabel :: S.InitialEdge -> SimpleObject
  , oneIndex :: Int
  }
\end{verbatim}
\end{frame}


\begin{frame}{Output Example}
\img{final}
\end{frame}

\begin{frame}{Next steps}
\begin{itemize}
\item Distribute tensor products using dictionary ordering 
\item Double check $\coev$ and $\ev$ definitions
\item Verify snake equations
\item Verify braid relations
\item Compare with Ising $R$-matrices
\item Optimize composition to be local wrt tensor products
\end{itemize}
\end{frame}

\begin{frame}{Thanks}

Thanks for listening!

Any questions?
\end{frame}

\end{document}
