\documentclass{article}
\usepackage{../m}

\begin{document}
\noindent Paul Gustafson\\
\noindent Texas A\&M University - Math 608 \\ 
\noindent Instructor: Grigoris Paouris

\subsection*{HW 7}
% (Due April 23)
% Folland (Chapter 4: Exercise 43, 52, 60, 69, 74)

\p{4.43} For $x \in [0,1)$, let $\sum_1^\infty a_n(x) 2^{-n}$ be the binary expansion of $x$. 
(If $x$ is a dyadic rational, choose the expansion such that $a_n(x) = 0$ for $n$ large.) 
Then the sequence $(a_n) \in \{0, 1\}^{[0,1)}$ has no pointwise convergent sequence.
\begin{proof}
Suppose not. Then there exists a convergent subsequence $(a_{n_k})$. Let $x \in [0,1)$ be defined by letting $x_{n_k} = -1 + (-1)^k$ and $x_n = 0$ otherwise.  Then $a_{n_k}(x)$ does not converge.  
\end{proof}


\p{52} The one-point compactification of $\R^n$ is homeomorphic to the $n$-sphere 
$\{ x \in \R^{n+1} : |x| = 1\}$.
\begin{proof}
The homeomorphism is the usual stereographic projection sending the north pole to $\infty$.  It is easy to see that this map is a bijection and a local homeomorphism away from the north pole. 

Note that if any closed subset of $S^n$ does not contain the north pole then it has a positive distance to the north pole.  Hence the subset maps to a bounded and closed, hence compact, set. Thus, by taking complements, the inverse map is continuous at the north pole. Moreover, the preimage of a compact set in $\R^n$ under the map is compact and does not contain the north pole, so by taking complements the map is continuous at $\infty$.  Hence the map is a homeomorphism.
\end{proof}

\p{60} The product of countably many sequentially compact spaces is sequentially compact.
\begin{proof}
Let $(X_n)_{n=1}^\infty$ be sequentially compact.  Let $(x_n) \subset \prod_n X_n$.  Pick $N_1 \subset \N$ such that $(\pi_1(x_n))_{n \in N_1}$ converges. Pick $N_2 \subset N_1$ such that $(\pi_2(x_n))_{n \in N_2}$ converges, and so on.
Let $(n_k)$ be defined by picking $n_1 \in N_1$, $n_2 > n_1$ with $n_2 \in N_2$, and so on.  Then $\pi_i(x_{n_k})$ converges for all $i$.  Hence $(x_{n_k})$ converges.
\end{proof}

\p{69} Let $A$ be a nonempty set, and let $X = [0,1]^A$. The algebra generated by the coordinate maps $\pi_\alpha: X \to [0,1]$ for $\alpha \in A$ and the constant function $1$ is dense in $C(X)$.
\begin{proof}
By Stone-Weierstrauss it suffices to show that this algebra separates points. This is obvious.
\end{proof}

\p{74} Consider $\N$ as a subset of its Stone-Cech compactification $\beta\N$.
\begin{enumerate}[a.]
\item If $A$ and $B$ are disjoint subsets of $\N$, their closures in $\beta\N$ are disjoint. (Hint: $\chi_A \in C(\N, I)$.)
\item No sequence in $\N$ converges in $\beta\N$ unless it is eventually constant.
\end{enumerate}
\begin{proof}
For part (a), since $\N$ is discrete, $\chi_A: \N \to I$ is continuous, and so is $\chi_B$.  Therefore they have unique continuous extensions to $\beta\N$, $f$ and $g$ respectively. Suppose $x \in \overline{A} \cap \overline{B} \subset \beta\N$. Then $f(x) = 1 = g(x)$.

\end{proof}

\end{document}
