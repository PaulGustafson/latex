\documentclass{article}
\usepackage{../m}

\begin{document}
\noindent Paul Gustafson\\
\noindent Texas A\&M University - Math 608 \\ 
\noindent Instructor: Grigoris Paouris

\subsection*{HW 7}
% (Due April 23)
% Folland (Chapter 4: Exercise 43, 52, 60, 69, 74)

\p{4.43} For $x \in [0,1)$, let $\sum_1^\infty a_n(x) 2^{-n}$ be the binary expansion of $x$. 
(If $x$ is a dyadic rational, choose the expansion such that $a_n(x) = 0$ for $n$ large.) 
Then the sequence $(a_n) \in \{0, 1\}^{[0,1)}$ has no pointwise convergent sequence.
\begin{proof}
Suppose not. Then there exists a subsequence $(a_{n_k})$ with $a_{n_k} \to a$ for some
$a \in \{0,1\}^{[0,1)}$. Pick

% Let U be open containing a. Then eventually a_n_k \in U:

\end{proof}


\p{52} The one-point compactification of $\R^n$ is homeomorphic to the $n$-sphere 
$\{ x \in \R^{n+1} : |x| = 1\}$.

\p{60} The product of countably many sequentially compact spaces is sequentially compact.

\p{69} 

\p{74} Consider $\N$ as a subset of its Stone-Cech compactification $\beta\N$.
\begin{enumerate}[a.]
\item If $A$ and $B$ are disjoint subsets of $\N$, their closures in $\beta\N$ are disjoint. (Hint: $\chi_A \in C(\N, I)$.)
\item No sequence in $\N$ converges in $\beta\N$ unless it is eventuall constant.
\end{enumerate}


\end{document}
