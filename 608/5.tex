\documentclass{article}
\usepackage{../m}

\begin{document}
\noindent Paul Gustafson\\
\noindent Texas A\&M University - Math 608 \\ 
\noindent Instructor: Grigoris Paouris

\subsection*{HW 5}
%Section 4.2: 23, Section 6.4: 38,39,40, Section 6.5 43
\p{4.2.23} Give an elementary proof of the Tietze extension theorem in the case $X = \R$.
\begin{proof}
Let $E \subset \R$ be closed and $f : E \to \R$. Then $E^c = \bigcup_{n=1}^N I_n$ for disjoint open intervals $I_n$ and $N \le \infty$.

We define the function $F:\R \to \R$ to be the extension of $f$ that linearly interpolates the endpoint values on 
each bounded interval $I_n$ and is constant on the closure of each unbounded interval $I_n$.

By symmetry it suffices to show that $F$ is right continuous. Let $x \in \R$. If $x \in E^c$, then $F$ is continous at $x$.  

Now suppose $x \in E$. If $x$ is not an accumulation point of $E \cap [x, \infty)$, then $F$ is right continuous at $x$.  

If $x$ is an accumulation point of $E \cap [x, \infty)$, then let $\epsilon > 0$.  Pick $\delta > 0$ such that $|f(x) - f(y)| < \epsilon$ for all $y \in E$ with $|y - x| < \delta$.
Since $x$ is an accumulation point of $E \cap [x, \infty)$, there exists $y_0 \in E \cap [x, x + \delta)$.  By the construction of $F$ it follows
 that $|F(x) - F(y)| < \epsilon$ for all $y >x$ with $|y - x| < \delta$.
\end{proof}

\p{6.4.38} Show that $f \in L^p$ iff $\sum_{k= -\infty}^\infty 2^{kp} \lambda_f(2^k) < \infty$.
\begin{proof}

Note that $|f|^p \le \sum_{k=-\infty}^\infty 2^{p(k+1)} \chi_{\{2^k < |f| \le 2^{k+1} \}} \le 2^p \sum_{k=-\infty}^\infty 2^{pk} \chi_{\{|f| > 2^k \}}$. Hence, $\int |f|^p \le \int 2^p \sum_{k=-\infty}^\infty 2^{kp} \chi_{\{f >2^k \}} = 2^p \sum_{k=-\infty}^\infty 2^{pk} \lambda_f(2^k)$.

For the opposite inequality, we have 
\begin{align*}
\sum_{k=\infty}^\infty 2^{pk} \chi_{\{f >2^k \}} & = \sum_{k=-\infty}^\infty \sum_{j=k}^{\infty} 2^{pk} \chi_{\{ 2^j < |f| \le 2^{j+1}\}}
\\ & = \sum_{j=-\infty}^\infty \sum_{k=-\infty}^{j} 2^{pk} \chi_{\{ 2^j < |f| \le 2^{j+1}\}} 
\\ & = \sum_{j=-\infty}^\infty \frac{2^{pj}} { 1 - 2^{-p}} \chi_{\{ 2^j < |f| \le 2^{j+1}\}} 
\\ & \le \frac{1}{ 1 - 2^{-p}} |f|^p.
\end{align*}
\end{proof}


\p{39} If $f \in L^p$, then $\lim_{\alpha \to 0} \alpha^p \lambda_f(\alpha) = \lim_{\alpha \to \infty} \alpha^p \lambda_f(\alpha) = 0$. (First suppose $f$ is simple.)
\begin{proof}
Case $f$ is simple.  Then the support of $f$ has finite measure, so $\lim_{\alpha \to 0} \alpha^p \lambda_f(\alpha) = 0$.  Moreover, $f$ is bounded so $\lim_{\alpha \to \infty} \alpha^p \lambda_f(\alpha) = 0$.

General case. Let $\epsilon > 0$. Pick a simple function $\phi \in L^p$ with  $\phi \le f$ and $\|\phi - f\|_p^p < \epsilon$.  Then
\begin{align*}
\limsup_{\alpha \to 0} \alpha^p \lambda_f(\alpha) & \le \limsup_{\alpha \to 0} \alpha^p \lambda_\phi(\frac \alpha 2) + \alpha^p \lambda_{f - \phi})(\frac \alpha 2)
\\ & = \limsup_{\alpha \to 0} \alpha^p \lambda_{f - \phi})(\frac \alpha 2)
\\ & \le \limsup_{\alpha \to 0} \alpha^p \frac 2^p {\alpha^p} \|\phi - f\|_p^p
\\ & = 2^p \epsilon.
\end{align*}
The case where $\alpha \to \infty$ is similar.
\end{proof}

\p{40} If $f$ is a measurable function on $X$, its decreasing rearrangement is the function $f^*: (0,\infty) \to [0,\infty]$ defined by 
$$f^*(t) = \inf\{\alpha : \lambda_f(\alpha) \le t\} \quad (\text{where } \inf \emptyset = \infty).$$
\begin{enumerate}[a.]
\item $f^*$ is decreasing. If $f^* < \infty$ then $\lambda_f(f^*(t)) \le t$, and if $\lambda_f(\alpha) < \infty)$ then $f^*(\lambda_f(\alpha)) \le \alpha$.
\item $\lambda_f = \lambda_{f^*}$, where $\lambda_{f^*}$ is defined with respect to Lebesgue measure on $(0,\infty)$.
\item If $\lambda_f(\alpha) < \infty$ for all $\alpha > 0$ and $\lim_{\alpha \to \infty} \lambda_f(\alpha) = 0$ (so that $f^*(t) < \infty$ for all $t > 0$), and 
$\phi \ge 0$ is a Borel measurable function on $(0,\infty)$, then $\int_X \phi \circ |f| \, d\mu = \int_0^\infty \phi \circ f^*(t) \, dt$. In particulare,
$\|f\|_p = \|f^*\|_p$ for $0 < p < \infty$.
\item If $0<p < \infty$, $[f]_p = \sup_{t > 0} t^{1/p} f^*(t)$.
\item The name \emph{rearrangement} for $f^*$ comes from the case where $f$ is a nonnegative function on $(0,\infty)$. To see why it is appropriate, pick
a step function on $(0, \infty)$ assuming four or five different values and draw the graphs of $f$ and $f^*$.
\end{enumerate}

\p{43} Let $H$ be the Hardy-Littlewood maximal operator on $\R$. Compute $H_{\chi_{(0,1)}}$ explicitly. 
Show that it is in $L^p$ for all $p > 1$ and in weak $L^1$ but not in $L^1$., and that its $L^p$ norm tends to $\infty$ 
like $(p-1)^{-1}$ as $p \to 1$, although $\|\chi_{(0,1)}\|_p = 1$ for all $p$.
\begin{proof}
We have $H_{\chi_{(0,1)}}(x) = \sup_{r>0} \frac 1 {2 r} \int_{x-r}^{x+r} \chi_{(0,1)}(t) \, dt$. If $x \in (0,1)$, then $H_{\chi_{(0,1)}}(x) = 1$.  If $x=0$ or $x =1$, then $H_{\chi_{(0,1)}}(x) = 1/2$.  If $x > 1$, then $H_{\chi_{(0,1)}}(x) = \frac 1 {2x}$.  If $x < 0$, then $H_{\chi_{(0,1)}}(x) = \frac 1 {2-2x}$.

We saw in class that $H$ is weak (1,1) and strong (p,p) for all $p > 1$. Hence, $H_{\chi_{(0,1)}}$ is in weak $L^1$ and in $L^p$ for all $p > 1$. It is clear that $H_{\chi_{(0,1)}}(x)$ is not in $L^1$ since $\int_1^\infty \frac {dx}{2x}$ diverges.






\end{proof}


\end{document}
