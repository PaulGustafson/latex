\documentclass{article}
\usepackage{../m}

\begin{document}
\noindent Paul Gustafson\\
\noindent Texas A\&M University - Math 608 \\ 
\noindent Instructor: Grigoris Paouris

\subsection*{HW 5}
%Section 4.2: 23, Section 6.4: 38,39,40, Section 6.5 43
\p{4.2.23} Give an elementary proof of the Tietze extension theorem in the case $X = \R$.


\p{6.4.38} Show that $f \in L^p$ iff $\sum_{k= -\infty}^\infty 2^{kp} \lambda_f(2^k) < \infty$.

\p{39} If $f \in L^p$, then $\lim_{\alpha \to 0} \alpha^p \lambda_f(\alpha) = \lim_{\alpha \to \infty} \alpha^p \lambda_f(\alpha) = 0$. (First suppose $f$ is simple.)

\p{40} If $f$ is a measurable function on $X$, its decreasing rearrangement is the function $f^*: (0,\infty) \to [0,\infty]$ defined by 
$$f^*(t) = \inf\{\alpha : \lambda_f(\alpha) \le t\} \quad (\text{where } \inf \emptyset = \infty).$$
\begin{enumerate}[a.]
\item $f^*$ is decreasing. If $f^* < \infty$ then $\lambda_f(f^*(t)) \le t$, and if $\lambda_f(\alpha) < \infty)$ then 
\end{enumerate}

\p{43}
\end{document}
