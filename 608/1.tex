
\documentclass{article}
\usepackage{../m}

\begin{document}
\noindent Paul Gustafson\\
\noindent Texas A\&M University - Math 608 \\ 
\noindent Instructor: Grigoris Paouris

\subsection*{HW 1}
\p{1} Show that for every symmetric convex body $K \subset \R^n$, one can define a norm $\| \cdot \|$ whose unit ball is $K$.
\begin{proof}
Define $$\|x\| = \inf \{ c: c > 0, x \in cK \}.$$

Since $K$ contains some small ball $B$ and $-B$, by convexity it must contain a small ball around $0$.  Thus, $0 \le \|x\| \le \infty$ for all $x$.

We have $\|0\| =  0$ since $0 \in cK$ for all $c$.  If $x \neq 0$, we have $\|x\| \neq 0$ since $K$ is bounded.

Let $B$ denote the closed unit ball with respect to $\| \cdot \|$. Clearly $K \subset B$. Now suppose $x \in B \setminus K$.  Note that $cK \subset K$ for $0 \le c \le K$ by convexity since $0 \in K$. Hence, $\|x\| \ge 1$, so $\|x\| = 1$ since $x \in B$.  By the definition of $\|\cdot\|$, there exists a sequence $(c_n) \to 1$ with $x \in c_nK$. Then $c_n^{-1} x \to x$ with $c_n^{-1} x \in K$.  But $K$ is closed, so $x \in K$, a contradiction.

% \|\lambda x \| = | \lambda | \|x\|
To see that $\| \cdot \|$ is homogeneous, first note that if $\lambda = 0$, then $\|\lambda x\| = 0 = |\lamdba| \|x\|$.  If $\lambda \neq 0$, we have $\lambda x \in cK$ iff $|\lambda|x \in cK$ iff $x \in \frac {c}{|\lambda|} K$. Hence $\|\lambda x\| = \inf \{ c: c > 0, x \in \frac{c}{|\lambda|}K \} = 
\inf \{ |\lambda| c: c > 0, x \in cK \} = |\lambda| \|x\|$.

% \|x + y \| \le \|x\| + \|y\|
For the triangle inequality, it suffices to consider the case where $\|x\| + \|y\| = 1$ since the inequality is homogeneous. Then we have
$x + y = \|x\| (x / \|x\|)  + \|y \| (y / \|y\|) \in K$ since the RHS is a convex combination of elements of $K$.  Thus $\|x + y\| \le 1 = \|x\| + \|y\|$.
\end{proof}

\p{2} Let $X$ be a normed space and let $f: X \to \R$ be a nonzero linear functional. Show that the following are equivalent:
\begin{enumerate}
\item $f$ is not bounded
\item For every $x \in X$ and for every $r > 0$, $f(B(x,r)) = \R$.
\item $\ker(f)$ is a dense subspace of $X$.
\end{enumerate}
Conclude the following: For every linear functional $f$ either $\ker(f)$ is closed or $ker(f)$ is dense. 

%TODO

\begin{proof}
(i) $\implies$ (ii): Suppose $f$ is not bounded. Then there exists a sequence $(x_n) \subset B_X$ with $|f(x_n)| \to \infty$.  Let $u \in \R$.  Pick $x_n$ such that $|f(x_n)| > |u - f(x)|/r$.  Then $y := (\frac{u - f(x)}{f(x_n)} x_n + x) \in B(x,r)$, and $f(y) = u$.

(ii) $\implies$ (iii):  $0$ is in the preimage of every ball.

(iii) $\implies$ (i): Suppose $f$ were bounded with $|f(x)| \le M \|x\|$ for all $x \in X$.  Pick $x_0 \in X$ such that $f(x) \neq 0$. By multiplying $x_0$ by a scalar, WLOG $f(x_0) > M$.  By (iii), pick $y \in B(x_0, 1)$ such that $f(y) = 0$.  Then $f(x_0 - y) > M \ge M \|x_0 - y\|$, a contradiction.
\end{proof}


\p{3} Let $X$ be a normed space. Show that the following are equivalent:
\begin{enumerate}
\item Every linear functional $f$ is bounded.
\item Every subspace of $X$ is closed.
\item The unit ball of $X$, $B_X$ is compact
\item $X$ has finite dimension
\end{enumerate}

\begin{proof}
(i) $\implies$ (ii):  Suppose $Y \le X$ is not closed. Pick $(x_n) \subset Y$ with $x_n \to x$ and $x \not\in Y$. Let $V = \spn{x}$. Then $X = V \oplus Z$ for some $Z \le V$.  Define a linear functional $\phi: X \to k$ by $\phi(\lambda x_n + z) = \lambda$ for $\lambda \in k$, $z \in Z$.   Then $\phi(x_n) = 0 \neq \phi(x_n)$, so $\phi$ is not continuous.

(ii) $\implies$ (i): Suppose there exists an unbounded linear functional $f: X \to k$. Then $\ker(f)$ is not closed by Exercise 2, a contradiction.

(ii) $\implies$ (iii): Let $(x_n) \subset X$ be Cauchy.  %TODO

(iii) $\implies$ (iv):  Suppose $X$ is infinite dimensional. By repeatedly applying Riesz's lemma, we can pick a sequence $x_n \subset S_X$ with $\|x_n - x_m\| > 1/2$ for all $n \neq m$  (the space $\spn(x_1, \ldots, x_k)$ is closed by $(iv) \implies (i) \implies (ii)$). Thus, $B_X$ is not totally bounded, hence not compact.

(iv) $\implies$ (i): Let $\phi: X \to k$ be a linear functional, and $(e_i)_{i=1}^n$ be a basis for $X$.  If $x = \sum_i a_i e_i$, then $\phi(x) \le \max_i |\phi(e_i)| \sum_i |a_i| \le C \max_i |\phi(e_i)|  \|x\|$, where the last inequality is from the equivalence of the $l_1$ norm to $\| \cdot \|$ since $X$ is finite dimensional.
\end{proof}

\p 4 Let $(X, \| \cdot \|)$ be a normed space with $\dim(X) = \infty$. Show that
\begin{enumerate}
\item There exists an unbounded injective linear operator from $X$ onto $X$.
\item There exists a norm $\| \cdot \|_1$ in $X$ such that $\|\cdot \|_1$ is not equivalent to $\| \cdot \|$ but the spaces $(X, \| \cdot \|_1)$ and $(X, \| \cdot \|)$ are isometric.
\end{enumerate}
\begin{proof}
(i) Let $(x_n)_{n=1}^\infty$ be a Hamel basis for $X$. Let $T: X \to X$ be the linear transformation induced by $x_n \maps n x_n$ for all $n$.

(ii) %TODO 
\end{proof}



\p 5 Let $X, Y$ be normed spaces and $T; X \to Y$ be a linear operator. Show that 
\begin{enumerate}
\item If for every sequence $(x_n) \subset X$ with $x_n \to 0$ the sequence $(Tx_n) \subset Y$ is bounded, then $T$ is a bounded operator.
\item If for every absolutely convergent series $x_n$ we have $\sum_n Tx_n$ converges, then $T$ is bounded.
\end{enumerate}

\begin{proof}
(i) Suppose $T$ is not bounded.  Then there exists a sequence $(x_n) \subset S_X$ with $\|Tx_n\| \to \infty$.  Then $y_n := x_n \|Tx_n\|^{-1/2} \to 0$ but
$Ty_n = Tx_n^{1/2} \to \infty$, a contradiction.

(ii) Suppose $T$ is not bounded.  Then there exists a sequence $(x_n) \subset S_X$ with $\|Tx_n\| \to \infty$.  


\end{proof}
\end{document}
