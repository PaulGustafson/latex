\documentclass{article}
\usepackage{../m}

\begin{document}
\noindent Paul Gustafson\\
\noindent Texas A\&M University - Math 608 \\ 
\noindent Instructor: Grigoris Paouris

%Folland, Chapter 6, Exercises: 3,4,5, 10, 12,13
\subsection*{HW 4}
\p{6.3} If $1 \le p < r \le \infty$, $L^p \cap L^r$ is a Banach space with norm $\|f\| = \|f\|_p + \|f\|_r$, and if 
$p < q< r$, the inclusion map $L^p \cap L^r \to L^q$ is continuous.
\begin{proof}
The restrictions of $\|\cdot\|_p$ and $\|\cdot\|_r$ to $L^p \cap L^r$ are norms, so their sum is a norm.  To see that $L^p \cap L^r$ is complete, suppose $\sum_n f_n$ converges absolutely with respect to $\| \cdot \|$ for $f_n \in L^p \cap L^r$.  Then the same series converges absolutely in $L^p$ and $L^q$.  Thus the pointwise limit of the series exists a.e. and lies in $L^p \cap L^q$.

To see that the inclusion map $L^p \cap L^r \to L^q$ is continuous, let $f \in L^p \cap L^r$ and pick $\lambda$ as in Prop. 6.10.  Then $\|f\|_q \le \|f\|_p^\lambda \|f\|_r^{1 - \lambda} \le \|f\|^\lambda \|f\|^{1 - \lambda} = \|f\|$.
\end{proof}

\p{4} If $1 \le p < r \le \infty$, $L^p + L^r$ is a Banach space with norm $\|f\| = \inf\{\|g\|_p + \|h\|_r : f = g + h\}$, and if
$p < q < r$, the inclusion map $L^q \to L^p + L^r$ is continuous.

\begin{proof}
To see that $\|\cdot\|$ is positive definite, we must show that $\|f\| = 0$ implies $f = 0$ a.e.  Suppose that $\mu(\{f > 0\}) > 0$.  Then there exist a measurable set $E$ and $\delta > 0$ such that $\mu(E) > 0$ and $f_{|E} \ge \delta$. Suppose $f = g + h$ for $g \in L^p$ and $h \in L^q$.  Then 
\begin{align*}
\|g\|_p + \|h\|_r &  \ge \|g_{|E}\|_p + \|h_{|E}\|_q 
\\ &  \ge \|g_{|E}\|_p + \mu(E)^{1/p - 1/q}\|h_{|E}\|_p
\\ & \ge \min(\mu(E)^{1/p - 1/q},1) (\|g_{|E}\|_p + \|h_{|E}\|_p)
\\ & \ge \min(\mu(E)^{1/p - 1/q},1) \|f_{|E}\|_p
\\  & \ge \min(\mu(E)^{1/p - 1/q},1) \delta^{1/p}
\end{align*}
This implies that $\|f\| \ge \min(\mu(E)^{1/p - 1/q},1) \delta^{1/p} > 0$.

The function $\|\cdot\|$ satisfies the homogeneity condition of a norm.  For the triangle inequality, suppose $f_1, f_2 \in L^p + L^r$.  Suppose $f_1 = g_1 + h_1$ and $f_2 = g_1 + h_2$ for some $g_1,g_2 \in L^p$ and $h_1, h_2 \in L^r$. Then $\|g_1\|_p + \|h_1\|_q +  \|g_2\|_p + \|h_2\|_q \ge \|g_1 + g_2\|_p + \|h_1 + h_2\|_q \ge \|f_1 +f_2\|$.  Thus, $\|f_1\| + \|f_2\| \ge \|f_1 + f_2\|$.

To see that $L^p + L^r$ is complete, suppose $f_n \in L^p + L^r$ and $\sum_n f_n$ converges absolutely. Pick $g_n \in L^p$ and $h_n \in L^r$ such that $\|g_n\|_p + \|h_n\|_r \le \|f\| + 2^{-n}$.  Then $\sum_n g_n$ and $\sum_n h_n$ converge absolutely in $L^p$ and $L^r$ respectively.  Let $g = \sum_n g_n$ and $h = \sum_n h_n$. Then $\sum_n f_n = g + h$ pointwise a.e.  Moreover, 
\begin{align*}
\| \sum_{n \ge N} f_n \|  & \le \sum_{n \ge N} \|f_n\|
\\ & \le \sum_{n \ge N} \|g_n\|_p + \|h_n\|_r
\\ & \underset{N \to \infty}{\to} 0,
\end{align*}
so $\sum_n f_n = g+ h$ in $L^p + L^r$.  Hence $L^p + L^r$ is complete.

To see that the inclusion $L^q \to L^p + L^r$ is continuous, let $f \in L^q$ with $\|f\|_q = 1$. 

Case $f$ is simple. Let $E = \{x : f(x) \le 1\}$. Then 
\begin{align*}
\|f\| & \le \|f \chi_{E^c}\|_p + \|f \chi_E\|_r 
\\ & =  \left( \int_{E^c} |f|^p d\mu \right)^{1/p} + \left( \int_{E} |f|^r d\mu \right)^{1/r}
\\ & \le \left( \int_{E^c} |f|^q d\mu \right)^{1/p} + \left( \int_{E} |f|^q d\mu \right)^{1/r}
\\ & \le \left( \int |f|^q d\mu \right)^{1/p} + \left( \int |f|^q d\mu \right)^{1/r}
\\ = 2
\end{align*}

General case. Let $E = \{x : f(x) \le 1\}$, $g = f \chi_{E^c} \in L^p$, and $h = f \chi_E \in L^r$. Pick simple functions $\phi_n, \psi_n$ with $\phi_n \to g$ and $\psi_n \to h$ pointwise a.e., $|\phi_n| \le |g|$, and $|\psi_n| \le |h|$.  Let $\theta_n = \phi_n + \psi_n$.  We have
\begin{align*}
\|f - \theta_n \| & = \|g + h - \phi_n - \psi_n\|
\\ & \le \|g - \phi_n\|_p + \|h - \psi_n\|_r
\\ \to 0
\end{align*} 
Hence $\|f\| \le \liminf_n \|theta_n\| + \|theta_n - f\| \le 2$. 
\end{proof}


\p{5} Suppose $0 < p < q < \infty$. Then $L^p \not\subset L^q$ iff $X$ contains sets of arbitrarily small positive measure, and $L^q \not \subset L^p$ iff $X$ contains sets of arbitrarily large finite measure. What about the case $q = \infty$? (Hint in book).
\begin{proof}
Fix $r \in \R$ with $p < r < q$.

Suppose $X$ contains sets of arbitrarily small positive measure. Pick $F_n$ with $0 < m(F_n) \le 2^{-n}$.  Let $E_n = F_n \setminus \bigcup_{j > n} F_j$. Then $0 < \mu(E_n) \le 2^{-n}$ and the $(E_n)$ are disjoint.  Let $f = \sum_n (\mu(E_n))^{-1/r} \chi_{E_n}$.  Then $\|f\|_p^p = \sum_n \mu(E_n)^{-p/r} \mu(E_n) \le \sum_n \mu(E_n)^{1-p/r} \le \sum_n 2^{-n(1-p/r)} < \infty$, but $\|f\|_q^q = \sum_n \mu(E_n)^{1-q/r} \ge \sum_n 2^{n(q/r - 1)} = \infty$.

Conversely, suppose that there exists $\delta > 0$ such that if $\mu(E) > 0$ then $\mu(E) \ge \delta$. Suppose $f \in L^p$. Since $\int_{f \le 1} |f|^q \le \int_{f \le 1} |f|^p$, WLOG $f \ge 1$.  This also implies that the support of $f$ has finite measure. 

I claim that there are only finitely many integers $n \ge 1$ such that $E_n := \{x : n \le |f|^p < n+1\}$ has nonzero measure. Suppose not. Then $\int |f|^p = \sum_n \int_{E_n} |f|^p \ge \sum_{\mu(E_n) \neq 0} \delta n = \infty$, a contradiction.

Thus $\int |f|^q = \sum_n \int_{E_n} |f|^q$ is a finite sum.  By definition $\mu(E_n)$ is finite for all $n$, and $|f|$ is bounded on each $E_n$.  Hence $f \in L^q$.

Now suppose $X$ contains sets of arbitrarily large finite measure.  Then, by disjointifying, it must contain a sequence of disjoint subsets $E_n \in \mM$ with $1 \le \mu(E_n) < \infty$. Let $f = \sum_n (n \mu(E_n))^{-1/r} \chi_{E_n}$.  Then $\|f\|_q^q = \sum_n (n \mu(E_n))^{-q/r} \mu(E_n) \le \sum_n n^{-q/r} < \infty$, but $\|f\|_p^p = \sum_n (n \mu(E_n))^{-p/r} \mu(E_n) \ge \sum_n n^{-p/r} = \infty$.  Hence $\L^q \not \subset L^p$.

Conversely, suppose there exists $K \ge 0$ such that every set $E \in \mM$ of finite measure has $\mu(E) \le K$. Let $f \in L^q$.  Let $E = \{ f \le 1 \}$.  We have $\|f\|_p^p = \int_E |f|^p d\mu + \int_{E^c} |f|^p d\mu \le \int_E |f|^p d\mu + \int_{E^c} |f|^q d\mu$.

Thus it suffices to show $\int_E |f|^p d\mu < \infty$. Let $\phi = \sum_{i=1}^n a_i \chi_{E_i}$ be a simple function with $0 \le \phi \le \chi_E |f|^p$. Then $\int \phi d\mu \le \sum_i \mu(E_i) = \mu(\bigcup_i E_i) \le K$.  Hence $\int_E |f|^p d\mu \le K$.

We now consider the case $q = \infty$. I claim $L_\infty \subset L_p$ iff $\mu(X) < \infty$. Suppose $\mu(X) < \infty$ and $f \in L_\infty$. Then $\|f\|_p^p = \int |f|^p \le \mu(X) \||f|^p\|_\infty = \mu(X) \|f\|_\infty^p$.  Conversely, suppose $\mu(X) = \infty$. Then $\chi_X \in L_\infty \setminus L_p$.
\end{proof}

\p{10} Suppose $1 \le p < \infty$. If $f_n, f \in L^p$ and $f_n \to f$ a.e., then $\|f_n - f\|_p \to 0$ iff $\|f_n\|_p \to \|f\|_p$. (Use Exercise 20 in 2.3)
\begin{proof}
Since $| \|f_n\|_p - \|f\|_p | \le \|f_n - f\|_p$, one implication is clear. For the converse, assume $\|f_n\|_p \to \|f\|_p$. We have
$\|f_n - f\|^p = \int |f_n - f|^p d\mu$  
%need to bound |f_n - f|^p by g_n integrable with g_n \to g


\end{proof}

\p{12} If $p \neq 2$, the $L^p$ norm does not arise from an inner product on $L^p(X, \mM, \mu)$, except in trivial cases when $\dim(L^p) \le 1$.
\begin{proof}}
Since $\dim(L^p) > 1$, there exist $E,F \in \mM$ with $0 < \mu(E) < \infty$, $0 < \mu(F) < \infty$, and $E \cap F = \emptyset$. Then
$\|\chi_E + \chi_F\|^2 + \|\chi_E - \chi_F\|^2 = 2 (\mu(E) + \mu(F))^{2/p}$, whereas $2\|\chi_E\|^2 + 2 \|\chi_F\|^2 = 2 \mu(E)^{2/p} + 2\mu(F)^{2/p}$. For $p < 2$, these cannot be equal by Minkowski's inequality. 

If $p > 2$, let $s = 2/p < 1$ and $\alpha = \mu(E)/(\mu(E) + \mu(F))$. Then we have $\frac{2\|\chi_E\|^2 + 2 \|\chi_F\|^2}{\|\chi_E + \chi_F\|^2 + \|\chi_E - \chi_F\|^2} = \alpha^s + (1 - \alpha)^s =: f(\alpha)$. We have $f''(\alpha) = s(s-1) (\alpha^s  + (1 - \alpha)^s) < 0$.

Suppose $f(\alpha_0) = 1$ for some $\alpha_0 \in (0,1)$.  Since $f(0) = 1 = f(1)$, the mean value theorem implies there exist $\beta_1 \in (0,\alpha_0)$ and $\beta_2 \in (\alpha_0,1)$ with $f'(\beta_1) = 0 = f'(\beta_2)$.  Applying the MVT again implies that $f''$ has a root, a contradiction.
\end{proof}

\p{13} $L^p(\R^n, m)$ is separable for $1 \le p < \infty$. However, $L^\infty(\R^n, m)$ is not separable.
\begin{proof}
To see that $L^p(\R^n, m)$ is separable for $1 \le p < \infty$, recall that the simple functions are dense in $L^p$.  Let $\phi = \sum_{i=1}^j a_i \chi_{E_i} \in L^p$ be simple. By the DCT and picking rational sequences converging to each $a_i$, WLOG $a_i \in \Q$ for all $i$. 

Since $\phi$ is bounded and supported on a set of finite measure, it suffices to approximate $\phi$ in measure. We have $m(E_i) = \inf{\bigcup_{k = 1}^\infty m(R_k) : l \in \N, R_k \text{ a rectangle with rational vertices}, E_i \subset \bigcup_k R_k \}$.  Hence, WLOG $E_i = \bigcup_{k = 1}^l R_k$ for a rational rectangle. The set of such functions $\phi$ is countable.

To see that $L^\infty(R^n, m)$ is not separable, for $i = \prod_k i_k \in \{0,1\}^\N$ let $E_i = \bigcup_{i_k = 1} [i_k, i_k + 1] \times \prod_{i=2}^n [0,1]$. Then if $i \neq j$, we have $\|\chi_{E_i} - \chi_{E_j}\|_\infty = 1$.  Since $|\{0,1\}|^\N > \N$ it follows that $L^\infty(\R^n, m)$ is not separable.
\end{proof}

\end{document}
