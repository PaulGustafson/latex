\documentclass{article}
\usepackage{../m}

\begin{document}
\noindent Paul Gustafson\\
\noindent Texas A\&M University - Math 608 \\ 
\noindent Instructor: Grigoris Paouris

%Folland, Chapter 6, Exercises: 3,4,5, 10, 12,13
\subsection*{HW 4}
\p{6.3} If $1 \le p < r \le \infty$, $L^p \cap L^r$ is a Banach space with norm $\|f\| = \|f\|_p + \|f\|_r$, and if 
$p < q< r$, the inclusion map $L^p \cap L^r \to L^q$ is continuous.
\begin{proof}
The restrictions of $\|\cdot\|_p$ and $\|\cdot\|_r$ to $L^p \cap L^r$ are norms, so their sum is a norm.  To see that $L^p \cap L^r$ is complete, suppose $\sum_n f_n$ converges absolutely with respect to $\| \cdot \|$ for $f_n \in L^p \cap L^r$.  Then the same series converges absolutely in $L^p$ and $L^q$.  Thus the pointwise limit of the series exists a.e. and lies in $L^p \cap L^q$.

To see that the inclusion map $L^p \cap L^r \to L^q$ is continuous, let $f \in L^p \cap L^r$ and pick $\lambda$ as in Prop. 6.10.  Then $\|f\|_q \le \|f\|_p^\lambda \|f\|_q^{1 - \lambda} \le \|f\|^\lambda \|f\|^{1 - \lambda} = \|f\|$.
\end{proof}

\p{4} If $1 \le p < r \le \infty$, $L^p + L^r$ is a Banach space with norm $\|f\| = \inf\{\|g\|_p + \|h\|_r : f = g + h\}$, and if
$p < q < r$, the inclusion map $L^q \to L^p + L^r$ is continuous.

\begin{proof}
To see that $\|\cdot\|$ is positive definite, we must show that $\|f\| = 0$ implies $f = 0$ a.e.  Suppose that $\mu(\{f > 0\}) > 0$.  Then there exist a measurable set $E$ and $\delta > 0$ such that $\mu(E) > 0$ and $f_{|E} \ge \delta$. Suppose $f = g + h$ for $g \in L^p$ and $h \in L^q$.  Then 
\begin{align*}
\|g\|_p + \|h\|_r &  \ge \|g_{|E}\|_p + \|h_{|E}\|_q 
\\ &  \ge \|g_{|E}\|_p + \mu(E)^{1/p - 1/q}\|h_{|E}\|_p
\\ & \ge \min(\mu(E)^{1/p - 1/q},1) (\|g_{|E}\|_p + \|h_{|E}\|_p)
\\ & \ge \min(\mu(E)^{1/p - 1/q},1) \|f_{|E}\|_p
\\  & \ge \min(\mu(E)^{1/p - 1/q},1) \delta^{1/p}
\end{align*}
This implies that $\|f\| \ge \min(\mu(E)^{1/p - 1/q},1) \delta^{1/p} > 0$.

The function $\|\cdot\|$ satisfies the homogeneity condition of a norm.  For the triangle inequality, suppose $f_1, f_2 \in L^p + L^r$.  Suppose $f_1 = g_1 + h_1$ and $f_2 = g_1 + h_2$ for some $g_1,g_2 \in L^p$ and $h_1, h_2 \in L^r$. Then $\|g_1\|_p + \|h_1\|_q +  \|g_2\|_p + \|h_2\|_q \ge \|g_1 + g_2\|_p + \|h_1 + h_2\|_q \ge \|f_1 +f_2\|$.  Thus, $\|f_1\| + \|f_2\| \ge \|f_1 + f_2\|$.

To see that $L^p + L^r$ is complete, suppose $f_n \in L^p + L^r$ and $\sum_n f_n$ converges absolutely. Pick $g_n \in L^p$ and $h_n \in L^r$ such that $\|g_n\|_p + \|h_n\|_r \le \|f\| + 2^{-n}$.  Then $\sum_n g_n$ and $\sum_n h_n$ converge absolutely in $L^p$ and $L^r$ respectively.  Let $g = \sum_n g_n$ and $h = \sum_n h_n$. Then $\sum_n f_n = g + h$ pointwise a.e.  Moreover, 
\begin{align*}
\| \sum_{n \ge N} f_n \|  & \le \sum_{n \ge N} \|f_n\|
\\ & \le \sum_{n \ge N} \|g_n\|_p + \|h_n\|_r
\\ & \underset{N \to \infty}{\to} 0,
\end{align*}
so $\sum_n f_n = g+ h$ in $L^p + L^r$.  Hence $L^p + L^r$ is complete.

%FIXME
Lemma: if $|f| \le 1$ and $1 < q < r < \infty$, then $\|f\|_r \le \|f\|_q$.


To see that the inclusion $L^q \to L^p + L^r$ is continuous, let $f \in L^q$. Let $E = \{x : f(x) > 1\}$. Then $\mu(E) < \infty$, $f\chi_E \in L^p$ and $f\chi_{E^c} \in L^r$. Hence
\begin{align*}
\|f\| & \le \|f\chi_E\|_p + \|f \chi_{E^c}\|_r
& \le \mu(E)^{1/p - 1/q} \|f \chi_E \|_q + \left( \int_{E^c} |f|^r d\mu \right)^{1/r}
& \le \mu(E)^{1/p - 1/q} \|f \|_q + 
\end{align*}



\end{proof}


\p{5} Suppose $0 < p < q < \infty$. Then $L^p \not\subset L^q$ iff $X$ contains sets of arbitrarily small positive measure, and $L^q \not \subset L^p$ iff $X$ contains sets of arbitrarily large finite measure. (Hint in book).

\p{10} Suppose $1 \le p < \infty$. If $f_n, f \in L^p$ and $f_n \to f$ a.e., then $\|f_n - f\|_p \to 0$ iff $\|f_n\|_p \to \|f\|_p$. (Use Exercise 20 in 2.3)

\p{12} If $p \neq 2$, the $L^p$ norm does not aries on $L^p$, except in trivial cases when $\dim(L^p) \le 1$.

\p{13} $L^p(\R^n, m)$ is separable for $1 \le p < \infty$. However, $L^\infty(\R^n, m)$ is not separable.

\end{document}
