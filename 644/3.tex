\documentclass{article}
\usepackage{../m}
\usepackage{tikz-cd}

\begin{document}
\noindent Paul Gustafson\\
\noindent Math 644

\subsection*{Final}
\p 1 Show that $H_c^{n+1}(X \times \R; G) \cong H_c^n(X; G)$ for all $n$.
\begin{proof}
To compute $\colim H^{n+1}(X \times \R, X \times \R - K'; G)$, it suffices to let $K'$ range over sets of the form $K' = K \times I$ for a compact $K \subset X$ and $I$ a compact interval. WLOG $I = [0,1]$.  Applying the relative Meyer-Vietoris sequence in the same way as in the proof of the suspension isomorphism gives the desired result.
\end{proof}

\p 2 Show that for any connected oriented closed manifold $M$ of dimension $n$ there is a map $f : M \to S^n$ having degree 1.
\begin{proof}
Let $B \subset M$ be an open neighborhood homeomorphic to a ball in $\R^n$.  Let $f$ be the quotient map $f: M \to M/(M-B) \simeq S^n$.
Since $M$ is orientable, we have $H^n(M) \simeq H^n(M, M - B)$ in the long exact sequence of a pair. Applying $f$ to both sides, the 
naturality of the long exact sequence gives a commutative diagram

\[
\begin{tikzcd}
H_n(M) \arrow{r} \arrow{d}{f_*} &  H^n(M, M - B) \arrow{d} \\
H_n(M/(M - B)) \arrow{r} & H_n(M/(M - B), (M - B)/(M - B))
\end{tikzcd}
\]
The rightmost arrow is an isomorphism since $(M, M-U)$ is a good pair, and the bottom arrow is an obvious isomorphism.  Thus, $f_*$ is an isomorphism, and in particular a degree $\pm 1$ map. If the degree is $-1$, compose $f$ with a reflection of $S^n$ through an equator to get a degree $1$ map.
\end{proof}

\p 3 Let $f : M \to N$ be a map between closed connected oriented manifolds of same dimension $n$.
\begin{enumerate}[a.]
\item Suppose there is a ball $B \subset N$ such that $f^{-1}(B)$ is the disjoint union of balls $B_i$ each 
mapped homeomorphically by $f$ onto $B$. Show the degree of $f$ is $\sum_i \epsilon_i$ where $\epsilon_i$
is $1$ or $-1$ according to whether $f:B_i \to B$ preserves or reverses local orientations induced from
given fundamental classes $[M]$ and $[N]$.
\begin{proof} 
Using the relative Mayer-Vietoris sequence, we have 
$$0 \to H_n(M, M- B_i -B_j) \to H_n(M, M- B_i) \oplus H_n(M, (M- B_i) \to H_n(M, (M- B_i) \cup (M - B_j))$$.
For $i \neq j$, the last term is $0$, so the middle terms are naturally isomorphic. Continuing in this way, we get
$H_n(M | \bigcup_i B_i) \cong \bigoplus_i H_n(M | B_i)$. By the naturality of the long exact sequence,
 we have $f_*:H_n(M) \cong H_n(M | \bigcup_i B_i) \to H_n( N | B) \cong H_n(N)$ is given by
$f_* = \sum_i f_i$, where $f_i: H_n(M|B_i) \to H_n(N|B)$ are the maps induced by $f$.  Since $f_{|B_i}$ is
a homeomorphism onto $B$ for each $i$, each $f_i$ is an isomorphism and the sign of $f_i$ is determined 
by whether it reverses local orientations.  
\end{proof}
\item Show that if $f$ is a $p$-sheeted covering projection then $\deg(f) = \pm p$.
\begin{proof}
It suffices to show that the set on which $f$ preserves local orientations is open. Let $x \in M$
such that $f$ preserves local orientations at $x$. Pick an evenly covered neighborhood $U$ of $f(x)$.  
Then $f$ also preserves local orientations on the component of $f^{-1}(U)$ containing $x$.
\end{proof}

\item If $M_g$ denotes the closed orientable surface of genus $g$, show that if $g \ge h$ there exists 
a map $f: M_g \to M_h$ of degree 1.
\begin{proof}
Think of $M_g$ as the connected sum of $M_h$ and $M_{g-h}$. 
Let $f$ be a map sending the $M_{g-h}$ part to a sphere and leaving the $M_h$ part alone.
 Applying part (a) to any untouched neighborhood in $M_h$ tells us that the degree of $f$ is $1$.
\end{proof}


\end{enumerate}




\p 4 If $g \geq 1$ show that for each nonzero $\alpha \in H^1(M_g; \Z)$ there exists $\beta \in H^1(M_g;\ZZ)$ with
$\alpha \smile \beta \neq 0$. Use this fact to show that $M_g$ is not homotopy equivalent to a wedge $X \vee Y$ of 
CW-complexes with non trivial reduced homology.

\begin{proof}
We proved on a homework that $H^1(M_g) = \Z^{2g}$. Hatcher's Prop. 3.38 implies that the the cup product pairing
$H^1(M_g) \times H^1(M_g) \to \Z$ given by evaluation of the cup product at the fundamental class is nonsingular. 
In particular, the first part of this problem holds.

For the second part, suppose that $M_g = X \vee Y$. Then $\Z = H_2(M_g) = H_2(X) \oplus H_2(Y)$.  Hence WLOG $H_2(X) = \Z$ and $H_2(Y) = 0$. 
Let $\alpha \in H^1(Y)$. Pick $\beta \in H^1(M_g)$ such that $\alpha \smile \beta \neq 0$.
Let $\phi \in C_2(X)$ be any chain representative of a generator of $H_2(X) = H_2(M_g)$.  Since $\alpha \in H^1(Y)$, we have
$(\alpha \smile \beta)(\phi) = 0$.  Since this holds for a representative of a generator of $H_2(M_g)$, we have $\alpha \smile \beta = 0$, a contradiction.
\end{proof}

\p 5 Let $f: M \to N$ be a map between closed connected oriented manifolds of dimensions $m$ and $n$, respectively, and let $R$ be a commutative ring with 1.
\begin{enumerate}[a.]
\item Explain how this map makes $H^*(M; R)$ into an algebra over $H^*(N; R)$.
\begin{proof}
The map $H^*(M; R) \times H^*(N; R) \to  H^*(M; R)$ defined by $(\alpha, \beta) \mapsto \alpha \smile f^*(\beta)$ gives the right action. Linearity is obvious, and associativity follows from the fact that $f^*(\beta \smile \gamma) = f^*(\alpha) \smile f^*(\gamma)$.
\end{proof}
\item Explain how to use Poincare duality to define group homomorphisms $f_!: H^i(M;R) \to H^{i+n-m}(N;R)$.
\begin{proof}
Poincare duality gives an isomorphisms $\phi:H^i(M;R) \simeq H_{m-i}(M;R)$, and 
$\psi:H_{m-i}(N;R) \simeq H^{i+n-m}(N;R)$. Let $f_! = \psi f_* \phi$.
\end{proof}
\item Show that these maps assemble to give a homomorphism $f_!:H^*(M;R) \to H^*(N;R)$ of right $H^*(N;R)$-modules.
\begin{proof}
The assembled map is clearly a morphism of abelian groups. If $\alpha \in H^i(M;R)$ and $\beta \in H^j(N;R)$,
we have $f_!(\alpha \beta)= f_!(\alpha \smile f^*(\beta)) = \psi f_* \phi (\alpha \smile f^*(\beta))
= f_!(\alpha) \smile f^*(\beta)
= f_!(\alpha) \beta,$ 
where we used the fact that $\psi, \phi$ commute with the cup product by the definition of the Poincare dual map as
the cup product with the fundamental class.
\end{proof}
\end{enumerate}

\p 6 Let $p:E \to X$ be a vector bundle of rank $n$ over a paracompact space $X$.
\begin{enumerate}[(1)]
\item Show that if $E$ has $k$ sections $s_1, \ldots, s_k$ that are linearly independent at each $x \in X$ that
it has a trivial subbundle of rank $k$.
\begin{proof}
Let $h: X \times \R^k \to E$ be defined by $h(x,t_1, \ldots, t_n) = \sum_i t_i s_i(x)$. 
Then $h$ is continuous (since it is continuous on each local trivialization), and a linear injection on each fiber since the $s_i$ are independent.  Thus, by a lemma shown in class, $h$ is a subbundle map.
\end{proof}
\item Assume that $E$ has $k$ sections $s_1, \ldots, s_k$ such that at each $x$ the elements $s_1(x), \ldots, s_k(x)$
generate the fiber $E_x$ as a vector space. Show that $E$ is a quotient of a trivial bundle of rank $k$.
\begin{proof}
Define $h:X \times \R^k \to E$ by $h(x, t_1, \ldots, t_n) = \sum_i t_i s_i(x)$.  This map is continuous and preserves fibres, hence a bundle map. It is also a linear surjection on fibers, so $E$ is a quotient of the trivial bundle of rank $k$.
\end{proof}
\item Under the assumptions of the previous question, let $x \in X$ and define $p_x: \CC^k \to E_x$ as the surjective linear map sending
$(\lambda_1, \ldots, \lambda_k)$ to $\sum_j \lambda_j s_j(x)$.  Show that the map $\phi : X \to Gr_{k-n}(\CC^k)$ sending $x \in X$ to $\ker p_x$ is a continuous map.
\begin{proof}
By restricting to a local trivialization, WLOG $E$ is a trivial bundle.  Then $\ker p_x$ is the orthogonal complement of $\spn\{s_1(x), \ldots, s_k(x)\}$. 
Since the $s_j$ are continuous and taking orthogonal complements of subspaces is a continous map from $Gr_{n-k}$ to $Gr_{k-n}$, the map $\phi$ is continuous.
\end{proof}
\item Let $q: \Q^n \to Gr_{k-n}(\CC^k)$ denote the quotient bundle $\varepsilon^k/E_{k-n}$, where $E_{k-n}$ is the tautological bundle. Show that if $\phi$ is defined in the previous section then $\phi* Q_n \cong E$.
\begin{proof}
To show that $E$ is this pullback, we need to find a map $f:  E \to Q^n$ mapping the fiber of each $x \in X$ to the fiber of $\phi(x)$ isomorphically. Define $f_x$ by $f_x(\sum_i \lambda_i s_i(x)) = \pi(\lambda_1, \ldots, \lambda_k)$, where $\pi$ is the orthogonal projection onto the orthogonal complement of $\ker p_x$ in $\CC^k$. By the rank-nullity theorem, the projection $\pi$ is an isomorphism.
\end{proof}

\end{enumerate}

\p 7 Let $X$ be a finite CW-complex with only even-dimensional cells.
\begin{enumerate}[a.]
\item Show that $K(X)$ is a free abelian group on the set of cells of $X$ and that $K(SX) = \ZZ$.  Explain why $K^{-1}(X) = 0$.
\begin{proof}
To see that $K(X)$ is free abelian on the cells of $X$, use induction on the number of cells. WLOG $X$ is connected. The base case, a point, is trivial.  For the inductive step, assume that $X$ is constructed by attaching a $k$-cell to a subcomplex $A$, for some $k$. This gives a short exact sequence
$\widetilde K^*(X/A) \to \widetilde  K^*(X) \to \widetilde  K^*(A)$.  
Since $X/A = S^k$, we have $\widetilde  K(X/A) = \ZZ$. By induction $\widetilde K(A)$ is free, hence projective, so the s.e.s splits.  Hence $\widetilde K(X)$ is free on the positive dimensional cells, so $K(X)$ is free on all the cells.


Since all the cells are even dimensional, the first term of the s.e.s. 
$K^1(X/A) \to K^1(X) \to K^1(A)$ is always 0. Thus, by induction on the number of cells, $K^1(X) = 0$. Thus $K(SX) = \Z \oplus \widetilde K(SX) = \Z \oplus K^1(X) = \Z$ and $K^{-1}(X) = K^1(X) = 0$.
\end{proof}


\item Compute $K^*(\CC\PP^n)$ and express as a module of $K^*(\text{pt})$.
\begin{proof}
The CW-complex structure of $\CC\PP^n$ has cell in each even dimension up to $2n$. So $K^0(\CC\PP^n) = \Z^{n+1}$ and $K^1(\CC\PP^n) = 0$.  Since $K^0(\text{pt}) = \Z$ and $K^1(\text{pt}) = 0$,  the action of $K^*(\text{pt})$ is given by $1 \in K^0(\text{pt}) = \Z$ maps to the $\id:K^*(\CC\PP^n) \to K^*(\CC\PP^n)$.
\end{proof}


\end{enumerate}

\end{document}
