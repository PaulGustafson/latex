\documentclass{article}
\usepackage{../m}
\usepackage{tikz-cd}

\begin{document}
\noindent Paul Gustafson\\
\noindent Math 644

\subsection*{Final}
\p 1 Show that $H_c^{n+1}(X \times \R; G) \cong H_c^n(X; G)$ for all $n$.
\begin{proof}
To compute $\colim H^{n+1}(X \times \R, X \times \R - K'; G)$, it suffices to let $K'$ range over sets of the form $K' = K \times I$ for a compact $K \subset X$ and $I$ a compact interval. WLOG $I = [0,1]$.  Applying the relative Meyer-Vietoris sequence in the same way as in the proof of the suspension isomorphism gives the desired result.
\end{proof}

\p 2 Show that for any connected oriented closed manifold $M$ of dimension $n$ there is a map $f : M \to S^n$ having degree 1.
\begin{proof}
Let $U \subset M$ be an open neighborhood homeomorphic to a ball in $\R^n$.  Let $f$ be the quotient map $f: M \to M/(M-U) \simeq S^n$.
Since $M$ is orientable, we have $H^n(M) \simeq H^n(M, M - U)$ in the long exact sequence of a pair. Applying $f$ to both sides, the 
naturality of the long exact sequence gives a commutative diagram

\[
\begin{tikzcd}
H_n(M) \arrow{r} \arrow{d}{f_*} &  H^n(M, M - U) \arrow{d} \\
H_n(M/(M - U)) \arrow{r} & H_n(M/(M - U), (M - U)/(M - U))
\end{tikzcd}
\]
The rightmost arrow and bottom arrows are isomorphisms since $(M, M-U)$ is a good pair.  Thus, $f_*$ is an isomorphism, and in particular a 
degree 1 map.
\end{proof}

\p 4 If $g \get 1$ show that for each nonzero $\alpha \in H^1(M_g; \Z)$ there exists $\beta \in H^1(M_g;\ZZ)$ with
$\alpha \smile \beta \neq 0$. Use this fact to show that $M_g$ is not homotopy equivalent to a wedge $X \vee Y$ of 
CW-complexes with non trivial reduced homology.

\begin{proof}
We proved on a homework that $H^1(M_g) = \Z^{2g}$. Hatcher's Prop. 3.38 implies that the the cup product pairing
$H^1(M_g) \times H^1(M_g) \to \Z$ given by evaluation of the cup product at the fundamental class is nonsingular. 
In particular, the first part of this problem holds.

For the second part, suppose that $M_g = X \vee Y$. Then $\Z = H_2(M_g) = H_2(X) \oplus H_2(Y)$.  Hence WLOG $H_2(X) = \Z$ and $H_2(Y) = 0$. 
Let $\alpha \in H^1(Y)$. Pick $\beta \in H^1(M_g)$ such that $\alpha \smile \beta \neq 0$.
Let $\phi \in C_2(X)$ be any chain representative of a generator of $H_2(X) = H_2(M_g)$.  Since $\alpha \in H^1(Y)$, we have
$(\alpha \smile \beta)(\phi) = 0$.  Since this holds for a representative of a generator of $H_2(M_g)$, we have $\alpha \smile \beta = 0$, a contradiction.
\end{proof}

\p 5 Let $f: M \to N$ be a map between closed connected oriented manifolds of dimensions $m$ and $n$, respectively, and let $R$ be a commutative ring with 1.
\begin{enumerate}[a.]
\item Explain how this map makes $H^*(M; R)$ into an algebra over $H^*(N; R)$.
\begin{proof}
The map $H^*(M; R) \times H^*(N; R) \to  H^*(M; R)$ defined by $(\alpha, \beta) \mapsto \alpha \smile f^*(beta)$ gives the right action. Linearity is obvious, and associativity follows from the fact that $f^*(\beta \smile \gamma) = f^*(\alpha) \smile f^*(\gamma)$.
\end{proof}
\item Explain how to use Poincare duality to define group homomorphisms $f_!: H^i(M;R) \to H^{i+n-m}(N;R)$.
\begin{proof}
Poincare duality gives an isomorphisms $\phi:H^i(M;R) \simeq H_{m-i}(M;R)$, and 
$\psi:H_{m-i}(N;R) \simeq H^{i+n-m}(N;R)$. Let $f_! = \psi f_* \phi$.
\end{proof}
\item Show that these maps assemble to give a homomorphism $f_!:H^*(M;R) \to H^*(N;R)$ of right $H^*(N;R)$-modules.
\begin{proof}
The assembled map is clearly a morphism of abelian groups. If $\alpha \in H^i(M;R)$ and $\beta \in H^j(N;R)$,
we have $f_!(\alpha \beta)= f_!(\alpha \smile f^*(beta)) = \psi f_* \phi (\alpha \smile f^*(beta))
= f_!(\alpha) \smile f^*(beta)
= f_!(\alpha) beta,$ 
where we used the fact that $\psi, \phi$ commute with the cup product by the definition of the Poincare dual map as
the cup product with the fundamental class.
\end{proof}
\end{enumerate}

\end{document}
