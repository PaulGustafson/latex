\documentclass{article}
\usepackage{../m}

\begin{document}
\noindent Paul Gustafson\\
\noindent Math 644

\subsection*{HW 1}
\p{1} To see that $(L, \phi^*)$ is compatible with the directed system, suppose $i \prec j$. Let $x \in C_i$. We have $\phi^j \circ \phi^i_j(x) = q(\phi^i_j(x)) = q(x) = \phi^i(x)$.

To see that $L$ is the colimit, suppose $(D, f^*)$ is compatible with the directed system. Let $\tilde f: \oplus_{i \in I} C_i \to D$ be the unique map such that $f^i = \tilde f \circ \iota_i$ for all $i$, where $\iota_i: C_i \to \oplus_{i \in I} C_i$ is the inclusion.   To see that $\tilde f$ descends to $L$, let $x \in C_i$ for some $i$, and suppose $i \prec j$.  Then $\tilde f(\phi^i_j(x) - x) = f_j \circ \phi^i_j (x) - f^i(x) = 0$.  Hence $\tilde f$ is $0$ on $H_\phi$, so $\tilde f$ descends to a function $f: L \to D$. The uniqueness of $f$ follows from the uniqueness of $\tilde f$ (if an $f'$ replaced $f$, then $f' \circ q$ would coincide with $\tilde f$).

For the second part of the problem, pick any $\tilde l \in q^{-1}(l)$.  Pick $j \in I$ dominating the finite support of $\tilde l$. Let $x = \sum_{i \in I} \phi^i_j(\tilde l)$ where $\phi^i_j$ is extended to be 0 outside of $C_i$. Then $x - \tilde l \in H_\phi$, so $\phi^j(x) = l$.

\p{2} Let $(L, \psi^*)$ be the colimit as defined in Problem 1.  Define $\alpha: \oplus_i C_i \to \Q$ by $\alpha \circ \iota_i(a) = a/i$. This map is a surjective homomorphism, so it suffices to show that its kernel is $H_\phi$. To see that $H_\phi \subset \ker \alpha$, suppose $x \in C_i$ and $i \prec j$.   Then $\alpha(\iota_i x - \iota_j \phi^i_j(x)) = x/i - (j/i)  (x/j) = 0$.  For the reverse inclusion, suppose $\alpha(l) = 0$.  Let $k$ dominate the the support of $l$ (i.e. $k$ is a multiple of all indices of nonzero coordinates of $l$).  We have $0 = \alpha(l) = \sum_{i \in I}  l_i/i= \sum_{i \in I}  (k/i) l_i /k$. Hence $0 = \sum_{i \in I} (k/i) l_i = \sum_{i \in I} \phi^i_k(l_i)$.  Hence $l \in H_\phi$.

\p{3} \begin{enumerate}[i.]
\item In the inclusion order, property (b) implies that the partial order is directed.  In the reverse inclusion order, property (a) implies that the empty set is in $\Phi$, which dominates every other set.

\item A closed subset of a compact Hausdorff set is compact. A finite union of compact sets is compact.

\item If $F \subset G$, we have a map of pairs $(X, X - G) \to (X, X - F)$ given by $\id_X$. Let $\phi^F_G$ be the image of this map under the contravariant functor $H^n$.  
\end{enumerate}

\p{4} \begin{enumerate}[i.]
\item Let $\phi^i: C_i \to \colim_{i \in I} C_i$ and $\psi^k: C_k \to \colim_{k \in K} C_k$ be the compatible families of morphisms satisfying the respective universal properties. Let $g: \colim_{k \in K} C_k \to \colim_{i \in I} C_i$ be the unique map such that $\phi^k = g \psi^k$ for all $k \in K$.

For each $i \in I$, pick any $k \in K$ with $i \prec k$, and define $f^i = \psi^{k} \phi^i_{k}$. To see that the particular choice of $k$ does not matter, let $k' \in K$ with $i \prec k'$. Pick any $j \in K$ with $k, k' \prec j$.  We have $\psi^{k} \phi^i_{k} = \psi^j \phi^k_j \phi^i_k = \psi^j \phi^i_j$, and the same for $k'$ in place of $k$. Hence $f^i$ is independent of the choice of $k$.

I claim $(f^i)_{i\in I}$ is compatible with the directed system $(C_i)_{i \in I}$. Indeed if $i \prec j \in I$, pick $k \in K$ such that $i,j \prec k$. Then  $f^j \phi^i_j = \psi^k \phi^j_k \phi^i_j = \psi^k \phi^i_k = f^i$.

Thus there exists a unique map $f: \colim_{i \in I} C_i \to \colim_{k \in K} C_k$ such that $f^i = f \phi^i$ for all $i \in I$.

I claim that $f$ is a left and right inverse for $g$.  For all $k \in K$, we have $fg \psi^k = f \phi^k = f^k = \psi^k$. Hence $fg = \id$, by the uniqueness of the map in the colimit universal property.  

Similarly, for all $i \in I$, we have $gf \phi^i = g f^i = g \psi^k \phi^i_k = \phi^k \phi^i_k = \phi^i$, where $k \in K$ is any index such that $i \prec k$. This implies $gf = \id$.  

Hence $f$ and $g$ are isomorphisms.

\item Any compact subset of $\R^n$ is bounded.

\item From (ii), we have $H^r_c(\R^n; R) = \colim_{i \in \N} H^r(\R^n, \R^n - B_i(0); R)$, where the $\phi^i_j$ are induced by inclusions. From the long exact sequence of a pair, we get $\tilde H^{r-1}(\R^n - B_i(0); R) \cong \tilde H^r(\R^n, \R^n - B_i(0); R)$ for all $r$.  Thus $H^r(\R^n, \R^n - B_i(0); R) = \tilde  H^r(\R^n, \R^n - B_i(0); R) = R$ if $r = n$, and $0$ otherwise.

Since the inclusion $\R^n - B_j \to \R^n - B_i$ is a homotopy equivalence for any $i < j$, the maps $\phi^i_j$ are isomorphisms. It follows that $H^r_c(\R^n; R) = \colim_{i \in \N} H^r(\R^n, \R^n - B_i(0); R) = H^r(\R^n, \R^n - B_1(0); R) = R$ if $r = n$, and $0$ otherwise.
\end{enumerate}

\p{5} Let $\Phi$ be the cofinal system of $\mathcal{K}_U$ defined by taking the complement of each neighborhood in the cofinal system mentioned in the problem. 
 
We have $H^r_c(U; R) = \colim_{F \in \Phi} H^r(U, U - F; R) = \colim_{F \in \Phi} H^r(X, X - F; R)$, where the last equality is by excision since $Y = \overline Y \subset (X - F)^\circ = X - F \subset X$. Since $X - F$ strong deformation retracts to $Y$ for all $F$, any inclusion $X - F \to X - F'$ for $F, F' \in \Phi$  maps to $\id_Y$ under the homotopy equivalences induced by the deformation retractions. Thus, $H^r_c(U; R) = \colim_{F \in \Phi} H^r(X, Y; R)$, where the morphisms of the directed system are all isomorphisms.  Thus, $H^r_c(U; R) = \colim_{F \in \Phi} H^r(X, Y; R)$.

\p{6} 3.2.2. Since $A$ and $B$ are contractible, we have an isomorphism $f^*: \tilde H^k(X;R) \to H^k(X, A;R)$ induced from a map $f:(X,a) \to (X,A)$ for some $a \in A$. Similarly, we have an isomorphism $g^*: \tilde H^l(X;R) \to H^l(X,A;R)$. Also, the cup product 
$H^k(X,A;R) \times H^l(X,B;R) \to H^{k+l}(X, A \cup B; R) = H^{k+l}(X, X; R) = 0$ must be 0.  Hence, by relative cohomology version of Proposition 3.10, if $\alpha \in H^k(X;R)$ and $\beta \in H^l(X;R)$ for $k, l > 0$, we have $\alpha \cup \beta = 0$. The same proof works for the general case.

\p{7} 3.2.3. 
\begin{enumerate}[(a)]
\item Suppose such a map $f: \R P^n \to \R P^m$ exists. We have $H^*(\R P^n;\Z_2) \cong \Z_2[\alpha]/\alpha^{n+1}$ with $|\alpha| = 1$. Hence $f^*:\Z_2[\alpha]/\alpha^{m+1} \to \Z_2[\beta]/\beta^{n+1}$ is a ring homomorphism mapping $\alpha$ to $\beta$. This implies $\beta^{m+1} = 0$, a contradiction.

In the complex case, $|alpha| = 2$, so the corresponding result would replace $H^1$ with $H^2$.

\item To see that $g$ is nonzero, use the same proof as the proof of Prop. 2B.6 on p. 175 with $g$ replacing $f$, using cohomology instead of homology.
\end{enumerate}

\p 8 3.2.7 The $\Z_2$-cohomology ring of $\R P^3$ is $Z_2[\alpha]/\alpha^4$ where $|\alpha| = 1$. The reduced $\Z_2$-cohomology ring of $\RP^2 \vee S^3$ is
$\tilde H^*(\R P^2) \oplus \tilde H^*(S^3)$. In particular, the cube of any element of $H^1(\RP^2 \vee S^3)$ is $0$, which does not hold for $H^1(\R P^3)$.

\p 9 3.2.8

\p{10} 3.2.9
\end{document}
