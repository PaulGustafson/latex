\documentclass{article}
\usepackage{../m}

\begin{document}
\noindent Paul Gustafson\\
\noindent Math 644

\subsection*{HW 1}
\p{1} Given a (left) $R$-module show:
\begin{enumerate}[i.]
\item The covariant functor $\Hom_R(M,-)$ is a left-exact functor.
\begin{proof}
Let $0 \to A \overset f \to B \overset g \to C \to 0$ be a short exact sequence. Application of the functor gives a complex
$$0 \to \Hom_R(M, A) \overset {f_*} \to \Hom_R(M, B) \overset {g_*} \to 
\Hom_R(M,C) \to 0.$$

For exactness at $\Hom_R(M, A)$, suppose $f_*(\alpha) = 0$ for some $\alpha: M \to A$.  Then $f(\alpha(m)) = 0$ for all $m \in M$.  Thus, $\alpha(m) = 0$ for all $m \in M$ since $f$ is injective.  Thus $f_*$ is injective.

For the exactness at $\Hom_R(M, B)$, suppose $g_*(\beta) = 0$ for some $\beta: M \to B$. Then $\im(\beta) \subset \ker(g)$. Since $f$ is an isomorphism from $A$ to $\im(A)$, the map $f^{-1}\beta : M \to A$ is well-defined. Thus, $\beta = f_*(f^{-1}\beta)$ is in the image of $f_*$. Thus $\ker(g_*) = \im(f_*)$.
\end{proof}

\item This functor is right-exact iff $M$ is a projective $R$-module.
\begin{proof}
In view of part (i), for the functor to be right-exact is the same as saying that $g_*$ surjects onto $\Hom_R(M, C)$ for every surjection $g: B \to C$. This is the same as saying that every map $M \to C$ lifts through every surjection $B \to C$, i.e. $M$ satisfies the definition of projective $R$-module.
\end{proof}

\end{enumerate}

\p{2} Given an $R$-module $M$ and a short exact sequence of $R$-modules
$$0 \to A \to B \to C \to 0,$$
use the previous problem to show that the sequence induces a long exact sequence:
$$0 \to \Hom_R(M, A) \to \Hom_R(M, B) \to \Hom_R(M,C) \to Ext_R^1(M, A) \to \cdots$$
\begin{proof}
Let $P_*$ be a projective resolution of $M$. Since the $P_i$ are projective we get a s.e.s. of chain complexes
$0 \to \Hom_R(P_*, A) \to \Hom_R(P_*, B) \to \Hom_R(P_*, C) \to 0.$ Applying the cohomology functor and the snake lemma gives 
the desired long exact sequence.
\end{proof}

\p{3} Regarding $\Z_2$ as a module over the ring $\Z_4$, construct a resolution of $\Z_2$ by free modules over $\Z_4$
and use this to show that $\Ext_{\Z_4}^n(\Z_2, \Z_2)$ is nonzero for all $n$.
\begin{proof}
A free resolution is the following:
$$ \cdots \overset{\times 2} \to \Z_4 \overset{\times 2} \to \Z_4 \overset{\times 2} \to \Z_4 \overset{\text{mod } 2} \to \Z_2 \to 0.$$
Applying the $\Hom_{\Z_4}(-, \Z_2)$ functor, we get
$$ \cdots \overset{0} \gets \Z_2 \overset{0} \gets  \Z_2  \overset{0} \gets \Z_2 \overset{\id} \gets \Z_2 \gets 0.$$
Thus $\Ext_{\Z_4}^n(\Z_2, \Z_2) = \Z_2$ for all $n \ge 1$.
\end{proof}


%\p{4} Complete the proof of the Comparison Theorem by showing that if $P_* \to M \to 0$ and $Q_* \to N \to 0$ are projective
%resolutions of two $R$-modules $M$ and $N$ and $f_*, g_*: P_* \to Q_*$ are two chain maps extending a homomorphism $\phi: M \to N$,
%then there is a chain homotopy $h_*: f_* \simeq g_*$.
%\begin{proof}
%Let $\alpha_* = f_* - g_*$. It suffices to show that $\alpha_* \simeq 0$. This means we need to find $h_i: P_i \to Q_{i+1}$ such that
%$\alpha_i = g_{i+1} h_i + h_{i-1} f_i$ (with the convention that $P_{-1} = Q_{-1} = M$). Let $h_{-1} = 0$. For $i = 0$, we must have
%$\alpha_0 = g_1 h_1$.
%\end{proof}

\end{document}
