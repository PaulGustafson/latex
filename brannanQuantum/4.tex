\documentclass{article}
\usepackage{../m}

\begin{document}
\noindent Paul Gustafson\\
\noindent MATH 663 - Subfactors, Knots, and Planar Algebras (Fall 2017)

\subsection*{HW 4}
\p 1 Let $(M, \tau)$ be a finite von Neumann algebra, let $1 \in N \subset M$ be a von Neumann subalgebra, and let $E: M \to N$ be the unique $\tau$-preserving conditional expectation.  Prove that $E$ is continuous and completely positive.

\begin{proof}
  The conditional expectation $E$ is a positive linear map between C-* algebras, so $E$ is bounded. Moreover, $E(x) = exe$ where $e:L^2(M) \to L^2(N)$ is the projection.  This implies that $E$ is completely bounded (by the easy converse to Stinespring's theorem).
\end{proof}

\p 2 Let $\pi, \sigma \in NC_2(2n)$.  Let $\beta \in \CC \setminus \{0\}$ and consider the canonical trace $\tau : D_n(\beta) = TL_n(\beta^{-2}) \to \CC$.  Show that $\tau(D_\sigma^* D_\pi) = \beta^{|\pi \vee \sigma|} - n$.

\p 3 Prove that the canonical trace $\tau_n$ \on $TL_n(\lambda)$ is positive semidefinite for all $\lambda \in (0, \frac{1}{4}]$.
\end{document}
