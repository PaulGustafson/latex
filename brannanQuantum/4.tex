\documentclass{article}
\usepackage{../m}

\begin{document}
\noindent Paul Gustafson\\
\noindent MATH 663 - Subfactors, Knots, and Planar Algebras (Fall 2017)

\subsection*{HW 4}
\p 1 Let $(M, \tau)$ be a finite von Neumann algebra, let $1 \in N \subset M$ be a von Neumann subalgebra, and let $E: M \to N$ be the unique $\tau$-preserving conditional expectation.  Prove that $E$ is continuous and completely positive.

\begin{proof}
  The conditional expectation $E$ is a positive linear map between C-* algebras, so $E$ is bounded. Moreover, $E(x) = exe$ where $e:L^2(M) \to L^2(N)$ is the projection.  This implies that $E$ is completely bounded (by the easy converse to Stinespring's theorem).
\end{proof}

\p 2 Let $\pi, \sigma \in NC_2(2n)$.  Let $\beta \in \CC \setminus \{0\}$ and consider the canonical trace $\tau : D_n(\beta) = TL_n(\beta^{-2}) \to \CC$.  Show that $\tau(D_\sigma^* D_\pi) = \beta^{|\pi \vee \sigma|} - n$.
\begin{proof}
  Write down the diagram for $D_\sigma^* D_\pi$ as the vertical concatenation of its two factor diagrams.  Label points on the boundary of $D_\pi$ by $1, \ldots, 2n$ in the usual way as if $D_\sigma^*$ was not there.  Label the $D_\sigma^*$ part according to the usual $D_\sigma$ labelling, i.e. reflect the usual $D_\sigma^*$ labelling through its horizontal midline. The $\sigma$-labels $n+1$ through $2n$ should agree with the $\pi$-labels. Moreover, the ``braid closure'' of $D_\sigma^* D_\pi$ connects the points labeled $1$ to $n$ for $\pi$ to the correspondingly labeled points for $\sigma$.  Thus, the connected components of the braid closure correspond to the blocks of $\sigma \vee \pi$. 
\end{proof}


\p 3 Prove that the canonical trace $\tau_n$ \on $TL_n(\lambda)$ is positive semidefinite for all $\lambda \in (0, \frac{1}{4}]$.
  \begin{proof}
    Let $$\xi_\pi = \sum_{i \in \{1,2\}^{[2n]} \prod_{r \sim_\pi s\\ r< s} F_{i(s)i(r)} e_i,$$
      where $F = \beta^{-1/2} \begin{pmatrix} 0 & q^{-1} \\ q & 0 \end{pmatrix},$ and $\beta = q^2 + q^{-2}$ with $q \in \RR$.   We have
      \begin{align*}
        \langle \xi_\pi, \xi_\sigma & = \sum_i \prod_{r \sim_\pi s\\ r < s} \prod{t \sim_\sigma u\\ t < u} F_{i(s)i(r)} F_{i(t) i(u)} \\
        & = \sum_i \prod_{b \in \pi \vee \sigma} \prod_{r<s, t<u \in b\\ r \sim_\pi s \\ t \sim_\sigma u } F_{i(s)i(r)} F_{i(t) i(u)} 
      \end{align*}

      Let $b$ be a block in $\pi \vee \sigma$, and let $x_1 \in [2n]$ be the minimal number in the block $b$.  Every element of $b$ is related to two other numbers by $\pi$ and $\sigma$ respectively, and the group generated by $\pi, \sigma \subset S_{2n}$ acts transitively on $b$.  Letting $x_{j+1} = \pj x_{j}$ if $j \ge 1$ is even and $x_{j+1} = \sigma x_j$ if $j \ge 1$ is odd, we have $b = \(x_j\)_{j = 1}^{|b|}$.  Thus,
      \begin{align*}
        \langle \xi_\pi, \xi_\sigma &  = \sum_i \prod_{b \in \pi \vee \sigma} \prod_{j = 1}^{|b|} F_{i(x_j)i(x_{j+1})}  
      \end{align*}

      For the last product to be nonvanishing, we must have $i(x_{j+1}) \neq i(x_{j})$ for all $j$.  Thus, for every block we get exactly two nonvanishing values of $i$ corresponding to the values $i(x_1)$ at the block's minimal element $x_1$.  Moreover, the map $j \mapsto x_j$ for $1 \le j \le |b| + 1$ defines a piecewise linear map $\phi: (1, |b| +1) \to \RR_{\ge 0}$ by connecting consecutive points with line segments.  It is easy to check that
      $$ \prod_{j = 1}^{|b|} F_{i(x_j)i(x_{j+1})}   = \beta^{\frac{-|b|}{2}} q^{\# (\text{local maxima of } \phi)} q^{- \# (\text{local minima of } \phi)} $$
      if $i(x_1) = 1$, and the inverse if $i(x_1) = 2$.   Since $x_1 = x_{|b| + 1}$ is minimal, the first and last local extrema of $\phi$ in $(1, |b| + 1)$ are local maxima.  Thus, the previous product is $q^2$ or $q^{-2}$, depending on $i(x_1)$.  Since the choice of $i(x_1)$ is independent for each block $b$, we have
\begin{align^*}
  \langle \xi_\pi, \xi_\sigma &  = \prod_{b \in \pi \vee \sigma} \beta^{\frac{-|b|}{2}} (q^2 + q^{-2})   \\
  & = \beta^{-n} \prod_{b \in \pi \vee \sigma} \beta
  & = \beta^{|\pi \vee \sigma| - n}
\end{align^*}

Thus, the Gram matrix for $\tau_n$ wrt to $(\D_\pi)_\pi$ is same as the Gram matrix for the vectors $(\xi_\pi)_\pi$. Thus, $\tau_n$ is positive semidefinite.
\end{proof}

\DeclareMathOperator{\slim}{s-lim}
  
  \p 4 (Exercise 10 of Speicher) Let $p,q \in B(H)$ be orthogonal projections on a separable complex Hilbert space $H$.
  \begin{enumerate}[(a)]
  \item Show that
    $$\slim_{n \to \infty} (pqp)^n = p \wedge q.$$
    \begin{proof}
      Since $pqp$ is self-adjoint, $C^*(1, pqp)$ is a unital commutative $C^*$-algebra, hence isometrically $*$-isomorphic to $C(\Spec(pqp))$ via a map $\phi$ with $\phi(pqp) = \id_{\Spec(pqp)}$.  It is easy to check that $pqp$ is a contractive positive operator. Hence, $\Spec(pqp) \subset [0,1]$.  Thus, $\phi((pqp)^n) = \id_{\Spec(pqp)}^n \to \xi_{\{1\}}$.  Thus, $pqp$ converges in norm to some projection $e$.

      It is clear that $p \wedge q \le e$.  For the opposite inequality, we have
      $\|(pqp)^n \xi\| \le \|q \xi\|$ for all $\xi \in H$, so $\|e \xi\| \le \|q \xi\|$.  The same holds for $p$.  Thus, $e \le p \wedge q$.  Thus, $e = p \wedge q$.
      
    \end{proof}
    
  \end{enumerate}
  
\end{document}
