\documentclass{article}
\usepackage{../m}
%\usepackage{amssymb}

\begin{document}
\noindent Paul Gustafson\\
\noindent MATH 663 - Subfactors, Knots, and Planar Algebras (Fall 2017)

\subsection*{HW 2}
\p 1 Let $M$ be a factor. Show that $M$ is finite if and only if every isometry $u \in M$ is unitary.
\begin{proof}
  Suppose $M$ is finite.  Let $u \in M$ be an isometry.  Then $u^* u = 1$.  Let $p = u u^*$.  We have
  $1 \sim p \le 1$.  Thus, $p = 1$, so $u$ is unitary.

  Conversely, suppose that every isometry in $M$ is unitary.  Let $p$ be a projection such that $1 \sim p \le 1$.
  Then there exists an isometry $u$ such that $u^* u = 1$ and $u u^* = p$. Since every isometry in $M$ is unitary,
  it follows that $p = 1$.
\end{proof}

\p 2 Let $\Gamma$ be a group. Prove that $L \Gamma' = R \Gamma$, where $R \Gamma \subset \mB(\ell^2(\Gamma))$ is
the von Neumann algebra generated by the right regular representation $\rho: \Gamma \to \mU(\ell^2(\Gamma))$.
\begin{proof}
  We have $L \Gamma' = J (L \Gamma) J$, where $J(x \delta_e) = x^* \delta_e$.  If $g, h \in \Gamma$, we have
  $J \lambda(g) J \delta_h = J \lambda(g) \lambda(h^{-1}) \delta_e = \lambda(h g^{-1}) \delta_e = \rho(g) \delta(h)$.
  By anti-linearity of $J$, we have $J \lambda(\CC \Gamma) J = \rho(\CC \Gamma)$.  By the continuity of $J$
  with respect to the SOT, we have $L \Gamma' = J L \Gamma J = R \Gamma$.
  %% ^ Doublecheck
\end{proof}

\p 3 Consider $M = M_n(\CC)$ equipped with its unique tracial state $\tr : M_n(\CC) \to \CC$. Let $e_{ij} \in M$ be
the standard matrix units associated to a fixed orthonormal basis $(e_i)_i$ for $\CC^n$.
\begin{enumerate}
\item Show that the map $e_{ij} \mapsto \frac{1}{\sqrt{n}} e_i \otimes \overline{e_j}$ induces a unitary identification
  $L^2(M) \cong \CC^n \otimes \overline{\CC^n}$.
  \begin{proof}
    We have
    \begin{align*}
      \langle \sum_{i,j} a_{ij} e_{ij}, \sum_{kl} b_{kl} e_{kl} \rangle & = \tr(B^* A) \\
      & = \tr(\sum_j \overline{b_{ji}} a_{jk}) \\
      & = \sum_{i,j} \overline{b_{ji}} a_{ji} \\
      & = \sum_{i,j} \overline{b_{ji}} a_{ji} \\
      & = \sum_{i,j} \langle b_{ij} (e_i \otimes \overline{e_j}),  a_{ij} (e_i \otimes \overline{e_j} \rangle \\
    \end{align*}
  \end{proof}
\item Describe how $M$ acts via the GNS representation on $\CC^n \otimes \overline{\CC^n}$.
  The image of the action of $e_{ij} \in M$ on $e_{kl} \in L^2(M)$ is $\delta_{jk} e_{il}$.  Thus,
  the image of the action of $e_{ij}$ on $e_k \otimes \overline{e_l}$ is $\delta_{jk} e_i \otimes \overline{e_l}$.
    
  \item Describe how the modular conjugation $J$ acts on $\CC^n \otimes \overline{\CC^n}$.
    The modular conjugation $J$ acts on $L^2(M)$ by $Jx \xi = x^* J \xi$, where $\xi = \sum_i e_{ii}$.  Thus, it
    acts on $\CC^n \otimes \overline{\CC^n}$ by $J x \xi = x^* J \xi$, where $\xi = \sum_i e_i \otimes \overline{e_i}$
  
  \item Describe how $M'$ acts on $\CC^n \otimes \overline{\CC^n}$.
    Since $M' = JMJ$, we have
    %% DOUBLECHECK Above, should just be acting on one factor..
\end{enumerate}

\p 4 Give an example of a group $\Gamma$ and an ergodic probability measure preserving action $\Gamma \curvearrowright (X, \Sigma, \mu)$ so that
$$L^\infty(X) \rtimes_\alpha \Gamma \cong M_n(\CC).$$
\begin{proof}
  Let $\Gamma = \ZZ_m$ and $X = \ZZ_m$ with the counting measure and left translation action $\alpha$.  This action is free and ergodic.  Thus, $\Gamma$ acts freely and ergodically on $L^\infty(X)$.  Thus, a theorem in class implies that $L^\infty(M) \rtimes_\alpha \Gamma$ is a factor.  Since this vNA is a finite dimensional factor, it is isomorphic to some $M_n(\CC)$.
\end{proof}


\p 5 A II$_1$-factor $(M, \tau)$ is said to have \emph{property Gamma} if there exists a sequence of unitaries $(u_n)_{n \in \NN} \subset M$ such that $\tau(u_n) = 0$ and
$$ \| u_nx - xu_n \|_2 \to 0 \qquad (x \in M).$$
Prove that $L(S_\infty)$ has property Gamma.
\begin{proof}
  Let $u_n = (n \quad n+1)$ be the transposition. Since $\tau(x) = \langle x \delta_e, \delta_e \rangle$, we have $\tau(u_n) = 0$.  Let $x \in S_\infty$.  Then $x \in S_m \subset S_\infty$ for some finite $m$.  By the far commutation relation, we have $u_n x = x u_n$ for $n > m$.  This implies that for all $x \in \CC S_\infty$, we have $u_n x - x u_n = 0$ for all large $n$.  The normality of $\| \cdot \|_2$ then implies that $L(S_\infty)$ has property Gamma.  
\end{proof}

\p 6 (Bonus problem) Show that $L\FF_2$ does not have property Gamma. Deduce that $L \FF_2$ is not AFD.
\begin{proof}
  See p. 485 of Effros, E. Property $\Gamma$ and inner amenability.
\end{proof}

\p 7 Let $M \subset \mB(H)$ be a von Neumann algebra and let $K$ be a Hilbert space. Consider the von
Neumann algebra $M \otimes 1 \subset \mB(H \otimes K)$. Show that $(M \otimes 1)' = M' \bar{\otimes} \mB(K)$.
(Here, $M' \bar{\otimes} \mB(K)$ is defined as the von Neumann algebra generated the algebraic
tensor product $M' \otimes \mB(K)$ inside $\mB(H \otimes K)$.
\begin{proof}
  Clearly $M' \bar\otimes \mB(K) \subset (M \otimes 1)'$.  For the reverse inclusion, suppose that $x \in M' \bar\otimes \mB(K)$.
  % https://math.stackexchange.com/questions/156204/the-commutant-of-a-tensor-product
\end{proof}


\end{document}
