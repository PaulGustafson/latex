\documentclass{article}
\usepackage{../m}

\begin{document}
\noindent Paul Gustafson\\
\noindent MATH 663 - Subfactors, Knots, and Planar Algebras (Fall 2017)

\subsection*{HW 1}
\p 1 Recall the construction of the hyperfinite II$_1$-factor $R = \pi_\infty(A_\infty)''$, where $A_\infty = \bigcup_n A_n$, $A_n = \bigotimes_n M_2(\CC)$, and $\pi_\infty$ is the GNS representation associated to the tracial state
$$ \tau_\infty = (\bigotimes_n \Tr_{M_2(\CC)}(x) \qquad (x \in A_n \subset A_\infty) $$
\begin{enumerate}[(a)]
\item Let $p \in P(R)\setminus \{0,1\}$. Explain why $pRp \cong R$.
  \begin{proof}
    If $A \subset R$ is a finite-dimensional $*$-subalgebra, then $pAp$ is a finite-dimensional $*$-subalgebra.
    Moreover, I claim that $pRp = \left( \bigcup_n pA_np \right)''$. The $\supset$ inclusion is clear.  For the
    other, suppose $x \in R$.  Then there exists a net $(x_i) \subset A_\infty$ with $x_i \to x$ in the WOT.
    Since $\langle p x_i p \xi, \eta \rangle = \langle x_i p \xi, p \eta \rangle \to \langle x p \xi, p \eta \rangle = \langle pxp \xi, \eta \rangle$, we have $px_ip \to pxp$ in the WOT.  This implies that $pRp = \left( \bigcup_n pA_np \right)''$.   Thus,
    $pRp$ is a hyperfinite II$_1$ factor, so by the uniqueness property, $pRp \cong R$.    
  \end{proof}

\item Fix $\lambda \in (0,1)$ and replace the canonical trace state $\Tr_{M_2(\CC)}$ with the state
  $$ \phi_\lambda((x_{ij})) = \frac{\lambda x_{11} + x_{22}}{1 + \lambda} $$
  Repeat the above GNS constrution for $A_\infty$ wiht $\tau_\infty$ replaced by
  the state
  $$\phi_{\lambda,\infty}:A_\infty \to \CC \quad \phi_{\lambda,\infty}(x) = \bigotimes_n\phi_\lambda(x) \quad (x \in A_n \subset A_\infty).$$
  Let $\pi_{\lambda,\infty}$ denote the corresponding GNS representation and
  let $R_\lambda := \pi_{\lambda, \infty}(A_\infty)''$.
  Show that $R_\lambda$ is AFD and does not admit any faithful normal tracial state (hence $R_\lamdba$ is a type III AFD von Neumann algebra).
\end{enumerate}

\p 2 Let $M$ be a II$_1$-factor and let $(H_i)_{i \in \NN}$ be $M$-modules. Prove that
$$\dim_M \left( \bigoplus_{i\in I} H_i \right) = \sum_i \dim_M(H_i)$$
\begin{proof}
  For each $i$, let $v_i: H_i \to L^2(M) \otimes \ell^2(\NN)$ be an isometry such that $v_i x = (x \otimes 1) v_i$ for all $x \in M$. Then
  $v := \bigoplus_i v_i: \bigoplus_i H_i \to \bigoplus_i L^2(M) \otimes \ell^2(\NN) \cong L^2(M) \otimes \ell^2(\NN)$ is an isometry such that
  $v x = (x \otimes 1) v$ for all $x \in M$.  Thus
  $\dim_M \left( \bigoplus_{i\in I} H_i \right) = \Tr(v v*) = \sum_i \Tr(v_i v_i^*) = \sum_i \dim_M(H_i)$.
\end{proof}

\p 3 Let $M \subset B(\mH)$ be a von Neumann algebra on some Hilbert space $\mH$ and let $p \in M$ be a non-zero projection.  Prove the following statements:
\begin{enumerate}[(a)]
\item We have $pMp = (M' p)'$ and $(pMp)' = M'p$ as algebras of operators on the Hilbert space $p \mH = \ran(p)$.  Thus $pMp$ and $M'p$ are both von Neumann algebras on $p \mH$
  \begin{proof}
    I claim that $(pM')' = pMp$.  Suppose $x \in M$ and $y \in M'$.  Then we have $pxp(py) = ppxpy = pypxp$.  Thus $pMp \subset (pM')'$.  On the other hand, suppose that $x \in (pM')'$.  Then, for all $y \in M'$, we have $xpy = ypx$.  Thus, $
  \end{proof}

\item If $M$ is a factor, then $pMp$ and $pM'$ are both factors on $p\mH$. Moreover, the map
  $$\Phi : M' \to M' p, \quad x \mapsto xp$$
  is a weakly continuous $*$-algebra isomorphism.

\item If $M$ is a factor and if $x \in M$ and $y \in M'$ are given, then $xy = 0$ implies that $x = 0$ or $y = 0$.

\item If $M$ is a factor, then $M \cup M'$ generates $B(\mH)$ as a von Neumann algebra.

\item If $M$ is a type II$_1$ factor, then $pMp \subset B(p \mH)$ is also a type II$_1$ factor.
\end{enumerate}

\end{document}
