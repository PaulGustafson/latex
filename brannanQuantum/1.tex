\documentclass{article}
\usepackage{../m}

\begin{document}
\noindent Paul Gustafson\\
\noindent MATH 663 - Subfactors, Knots, and Planar Algebras (Fall 2017)

\subsection*{HW 1}
\p 1 Let $\phi: \mB(H) \to \CC$ be a linear functional. Show that the following statements are equivalent:
\begin{enumerate}[(a)]
 \item There are $n \in \NN$ and $(\xi_i)_{i=1}^n, (\eta_i)_{i=1}^n \subset H$ such that
  $$ \phi(x) = \sum_{i=1}^n \langle x \xi_i | \eta_i \rangle \qquad (x \in \mB(H))$$
 \item $\phi$ is continuous with respect to the weak operator topology.
 \item $\phi$ is continuous with respect to the strong operator topology.
\end{enumerate}
\begin{proof}
  (a) $\implies$ (b):  Let $(x_\lambda)_\lambda \subset \mB(H)$ be a net such that $x_\lambda \overset{WOT}{\to} x$. Then
  \begin{align*}
  \lim_\lambda \phi(x_\lambda) & =  \sum_{i=1}^n \lim_\lambda \langle x_\lambda \xi_i | \eta_i \rangle \\
  & = \sum_{i=1}^n \langle \lim_\lambda x_\lambda \xi_i | \eta_i \rangle \\
  & = \phi(x),
  \end{align*}
  where the second equality follows from the definition of the WOT.
  
  (b) $\implies$ (c): Suppose $\phi$ is continuous wrt the WOT. Further suppose $x_\lambda \overset{SOT}{\to} x \in \mB(H)$.
  Then $x_\lambda \overset{WOT}{\to} x \in \mB(H)$, so $\phi(x_\lambda) \to \phi(x)$.

  (c) $\implies$ (a):  Suppose $\phi$ is continuous with respect to the SOT.  By the definition of the SOT, there exists an $r > 0$ and $\xi_1, \ldots, \xi_n$ such that $\|x \xi_i\| < r$ for all $i$ implies that $|\phi(x)| < 1$.   This implies that there exists $\delta$ such that $\sum_i \|x \xi_i\|^2 < \delta$ implies $|\phi(x)| < 1$.
  
  Define $i: \mB(H) \to H^{\oplus n}$ by $i(x) = \bigoplus_i x \xi_i$.  Let $K = \im(i)$. Let $\psi: K \to \CC$ be the linear functional defined by
  $$\psi(\bigoplus_i x \xi_i) = \phi(x).$$ By the Hahn-Banach theorem, we can extend $\psi$ to $\overline K$.  Hence, by the Riesz Representation Theorem, we can write
  $$\phi(x) = \sum_{i=1}^n \langle x \xi_i, \eta_i \rangle $$
for some $(\eta_i) \subset H$.
\end{proof}

\p 2 Let $H$ be an infinite dimensional Hilbert space. Show by means of explicit examples that the norm topology, the strong operator topology, and the weak operator topology are all inequivalent on $\mB(H)$.
\begin{proof}
  Define $x_n \in \mB(\ell^2(\NN))$  by $x_n(e_i) = 0$ if $i \le n$ and $x_n(e_i) = e_i$ if $i > n$. Then $x_n \to 0$ in the SOT. On the other hand, $\|x_n\| = 1$ for all $n$.

  Define $y_n \in \mB(\ell^2(\NN))$ by $y_n(e_i) = e_{i+n}$.  Then $x_n \to 0$ in the WOT.  On the other hand, $x_n(e_1) = e_n$ for all $n$, which doesn't converge.
\end{proof}

\p 3 Show that $\mB(H)$ is a factor.
\begin{frame}
  The set of bounded operators  $\mB(H)$ is obviously a von Neumann algebra (it's the commutant of the identity).  To see that it is a factor, we need to show that $\mB(H) \cap Z(\mB(H)) = \CC$.  In other words, we need to show that $Z(\mB(H)) = \CC$.

Suppose $x \in Z(\mB(H)) \setminus \CC$.  Then there exists $\xi \in H$ such that $x \xi$ is not a multiple of $\xi$.  Let $V$ be the two-dimensional Hilbert space spanned by $\xi$ and $T_\xi$.  Let $p$ be the projection onto $V$.  Then $pxp$ corresponds to a nonscalar $2\times 2$ matrix in the center of $M_2(\CC)$, a contradiction.
\end{frame}

\p 4 Let $S$ be a self-adjoint subset of $\mB(H)$. Show that $S'$ is a von Neumann algebra.
\begin{proof}
  First, I claim that $S'$ is a *-subalgebra of $\mB(H)$.  Suppose $x, y \in S'$ and $u \in S$.  Then $xyu = uxy$, and $\alpha x + \beta y) u = u (\alpha x + \beta u$ for all $\alpha, \beta \in \CC$.  Moreover,  $x^* u  = (u^* x)^* = (x u^*)^* = u x^*$.   Hence, $S'$ is a *-algebra.

  Since $S'$ obiously contains $1_{\mB(H)}$, it suffices to show that $S'$ is weakly closed.  Let $(x_\lambda) \subset S'$ be a net such that  $x_\lambda \to x \in \mB(H)$ in the weak operator topology.  Let $u \in M$ be arbitrary.  
  \begin{align*}
    0 & = \langle (x_\lambda u - u x_\lambda) \xi, \eta \rangle \\
    & = \langle x_\lambda u \xi, \eta \rangle - \langle x_\lambda \xi, u^* \eta \rangle \\
    & \to \langle x u \xi, \eta \rangle - \langle x \xi, u^* \eta \rangle \\
    & = \langle (x u - u x) \xi, \eta \rangle
  \end{align*}
  Thus, $x \in S'$.  Hence, $S'$ is weakly closed.
\end{proof}

\p 5 Let $e$ be a finite projection in a von Neumann algebra $M$. Let $f \le e$ be another projection.  Show that $f$ is also finite.
\begin{proof}
  Let $g \in P(M)$ be a projection such that $f \sim g \le f$. We have $e - f \ge 0$ and $e - f + g \le e$.  Moreover, $(e - f) \perp g$ and $(e - f) \perp f$.  Hence,
  $(e - f) + f \sim (e -f) + g \le (e-f) + f$.
  Thus, since $e$ is finite, we have $e -f + g = e$.  Thus, $f = g$.  Thus, $f$ is finite.
\end{proof}


\p 6 It is know that if $M$ is a factor, and $p,q \in P(M)$, then either $p \preceq q$ or $q \preceq p$. Using this fact, show that if $M$ is a $II_1$-factor then $p \sim q$ if and only if $\tau(p) = \tau(q)$, where $\tau$ is the unique normal faithful tracial state on $M$.
\begin{proof}
  If $p \sim q$, then there exists $u \in M$ such that $p = u^*u$ and $q = u u^*$.  Thus $\tau(p) = \tau(u^* u) = \tau(u u^*) = \tau(q)$ since $\tau$ is a trace.

  Conversely, suppose $\tau(p) = \tau(q)$.  WLOG $p \preceq q$.  Then there exists a projection $r \in P(M)$ such that $r \le q$ and $r \sim p$.  Since $r \le q$, we can write $r - q = x^* x$ for some $x \in M$.  Since $r \sim p$, the first part of this problem implies $\tau(r) = \tau(p) = \tau(q)$.  Hence, $\tau(x^*x) = \tau(q - r) = 0$. Hence, since $\tau$ is faithful, $x = 0$.  Thus, $q = r$, so $q \sim p$.
\end{proof}

\p 7 Let $M \subset \mB(H)$ be a von Neumann algebra. A vector $\xi \in H$ is called cyclic for $M$ if $H = \overline{M \xi}^{\lVert \cdot \rVert}$. We call $\xi$ separating for $M$ if for each $x \in M$, $x \xi = 0 \implies x = 0$. Show that $\xi$ is cyclic for $M$ if and only if $\xi$ is separating for $M'$.
\begin{proof}
  Suppose $\xi$ is separating for $M'$.  Let $p$ be the projection onto $\overline{M \xi}$.   Since $M$ is unital, $(p - 1) \xi = 0$.  Since $\xi$ is separating for $M'$, it is enough to show that $p - 1 \in M'$.  Or, equivalently, show that $p \in M'$.

  Suppose $x \in M$ and $v \in M \xi$.  Then $xpv = xv = pxv$.  Thus $xpv = pxv$ for all $v \in M \xi$.  Since $xp - px$ is a bounded operator, the same identity holds for all $v \in \overline{M \xi}$.  If $v \in (M \xi)^\perp$, then
  $xpv = 0$.  On the other hand, for all $w \in M \xi$, we have $\langle xv, w \rangle = \langle v, xw \rangle = 0$.  Thus, $pxv = 0$.  Thus, since $H = \overline{M \xi} \oplus (M \xi)^\perp$, we have $px = xp$.  Thus, $p \in M'$.

  Conversely suppose $\xi$ is cyclic for $M$.  Further suppose that $x \xi = 0$ for some $x \in M'$.  Then $x y \xi = y x \xi = 0$ for all $y \in M$.
  Thus, $x M \xi = 0$.  Since $x$ is bounded, this implies $0 = x \overline{M \xi} = x H$.  Thus, $x = 0$.   

\end{proof}


\p 8 Let $\Gamma$ be a group. Recall from class the definition of the (left) group von Neumann algebra $L\Gamma = \lambda(\CC \Gamma)'' \subset \mB(\ell^2\Gamma)$ and the normal tracial state $\tau : L\Gamma \to \CC$; $\tau(x) = \langle x \delta_e | \delta_e \rangle$.
\begin{enumerate}[(a)]
\item Consider the right regular representation $\rho : \CC \gamma \to \mB(\ell^2\Gamma)$; $\rho(g) \delta_h = \delta_{hg^{-1}}$, $g, h \in \Gamma$. Show that $\rho(\CC \Gamma) \subset L \Gamma'$.
  \begin{proof}
    Let $g,h,k \in G$.  Then $\rho(g) \lambda(h) \delta_k = \delta_{hkg^{-1}} = \lambda(h) \rho(g) \delta_k$.  Linearizing, we have $\rho(\CC \Gamma) \subset \lambda(\CC \Gamma)'$.
    
    Let $x \in L \Gamma'$ and $y \in \rho(\CC \Gamma)$. Then there exists a net $(x_i) \subset \lambda(\CC \Gamma)$ such that $x_i \to x$ in the WOT. Thus, for all $\xi, \eta \in \ell^2 \Gamma$, we have
\begin{align*}
    0 & = \langle (x_i y - y x_i) \xi, \eta \rangle \\
    & =  \langle x_i y \xi, \eta \rangle - \langle x_i \xi, y^* \eta \rangle \\
    & \to  \langle x  y \xi, \eta \rangle - \langle x \xi, y^* \eta \rangle \\ 
    & = \langle (x y - y x) \xi, \eta \rangle
\end{align*}
Hence, $x$ and $y$ commute.  Since $x$ and $y$ were arbitrary, this implies  $\rho(\CC \Gamma) \subset L \Gamma'$.
  \end{proof}
\item Define a linear map $\Lambda_\tau : L \Gamma \to \ell^2 \Gamma$ by $\Lambda(x) = \hat{x} = x \delta_e$. Use part (a) above to show that $\Lambda_\tau$ is injective.  Hence any $x \in L \Gamma$ is uniquely represented by a ``Fourier series $\hat{x} = \sum_{g \in \Gamma} \hat{x}(g) \delta_g \in \ell^2 \Gamma$.
  \begin{proof}
    Suppose $\Lambda_\tau(x) = 0$. Then for all $g \in \Gamma$, we have
    $0 = \rho(g) \Lambda_\tau(x) = \rho(g) x \delta_e = x \delta_g,$ where the
    last equality follows from part (b).  Thus, $x = 0$.  Thus, $\Lambda_\tau$ is injective.
  \end{proof}
\item Use the above to conclude that $\tau$ is a faithful state on $L\Gamma$.
  \begin{proof}
    Suppose $\tau(x^* x) = 0$.  Then $0 = \langle x^* x \delta_e, \delta_e \rangle = \langle x \delta_e, x \delta_e \rangle$.  Thus $x \delta_e = 0$, so part (b) implies that $x = 0$.
  \end{proof}
\item A group is said to have infinite conjugacy classes (icc) if for every $h \neq e$, the conjugacy class $C_h$ of $h$ is infinite. Show that if $x \in L \Gamma \cap L \Gamma'$, then $\hat{x}$ is constant on conjugacy classes. Conclude that if $\Gamma$ is icc, then $L\Gamma$ is a $II_1$-factor.
  \begin{proof}
    Suppose $x \in L\Gamma \cap L\Gamma'$, and $g,h \in \Gamma$.  Then
    \begin{align*}
    \hat{x}(g) & = \langle x \delta_e , \delta_g \rangle \\
    & = \langle \lambda(h) x \delta_e , \lambda(h) \delta_g \rangle \\
    & = \langle x \delta_h , \delta_{hg} \rangle \\
    & = \langle x \rho(h) \delta_{e}, \delta_{hg} \rangle \\
    & = \langle \rho(h) x \delta_{e}, \delta_{hg} \rangle \\
    & = \langle x \delta_{e}, \rho(h^{-1}) \delta_{hg} \rangle \\
    & = \langle x \delta_{e},  \delta_{hgh^{-1}} \rangle \\
    & = \hat{x}(hgh^{-1})
    \end{align*}

    Now suppose $L \Gamma$ is icc, and $x \in L \Gamma \cap L \Gamma'$.  Since $\hat{x}$ is constant on conjugacy classes, it must be zero for all non-trivial conjugacy classes (otherwise, its $\ell^2$-norm would be infinite).  Hence $L \Gamma \cap L \Gamma' = \CC$, so $L \Gamma$ is a factor. Since $\tau$ is a normal, faithful, tracial state, $L \Gamma$ is finite. Hence, since $L \Gamma$ is infinite dimensional, it is a $II_1$-factor.
\end{proof}
  
\item Conversely, show that if $\Gamma$ is not icc, then $L\Gamma \cap L\Gamma' \neq \CC 1$.
  \begin{proof}
    Let $C \subset \Gamma$ be a nontrivial, finite conjugacy class.  Then $\lambda(\delta_C) \in L\Gamma$.  Moreover, if $g \in \Gamma$, then $\lambda(g) \lambda(\delta_C) \lambda(g^{-1}) = \lambda(\delta_C)$.  Hence, by linearity, $\lambda(\delta_C) \in \CC \Gamma'$.  Moreover, if we have a net $(x_i) \subset \lambda(\CC \Gamma)$ with $x_i \to x$ in the WOT, we have
    \begin{align*}
      0 & = \langle (x_i \lambda(\delta_C) - \lambda(\delta_C) x_i) \xi, \eta \rangle \\
      & \to  \langle (x \lambda(\delta_C)  -  \lambda(\delta_C) x )\xi, \eta \rangle,
    \end{align*}
    for all $\xi, \eta \in H$.  Thus $\lambda(\delta_C) \in L\Gamma'$.
  \end{proof}
\end{enumerate}


\p 9 Consider the group $S_\infty$ given by all finite permutations of $\NN$ and the non-commutative free group $\FF_2$ on two generators. Show that both of these groups are icc.
\begin{proof}
  Let $\sigma \in S_\infty$ be a nontrivial permutation.  Then there exist $x \neq y \in \NN$ such that $\sigma(x) = y$.  For $n \in \NN$, let $\tau_n \in S_\infty$ be the transposition interchange $y$ and $n$.  Then for all $n$ greater than $x$ and $y$, we have $\tau_n \sigma \tau_n^{-1} (x) = \tau_n \sigma(x) = \tau_n y = n$.  Thus, $\tau_n \sigma \tau_n^{-1}$ are distinct for infinitely many $n$.

  Let $a,b \in \FF_2$ be the standard generators.  Let $g \in \FF_2$ be a nontrivial element. WLOG the first letter of the reduced word for $g$ is $a$.  I claim that the conjugates $g_n := b^n g b^{-n}$ are distinct for all $n \ge 0$.  This is because the reduced word for $g_n$ must start with $b^n a$ since the $b^{-n}$ can only cancel $b$'s on the right side of the this $a$.
\end{proof}
\end{document}
